\documentclass[../../Assignments.tex]{subfiles}


\begin{document}

\chapter{Solution approximation}

\section{The Bisection Method}

\begin{exercise}
    Use the Bisection method to find \(p_3\) for \(f(x) = \sqrt{x} - \cos{x}\) on \(\interval{0}{1}\).
\end{exercise}

\begin{solution}
    \(f(0) = -1\) and \(f(1) \approx \num{0.459697694}\) have the opposite
    signs, so there's a root in \(\interval{0}{1}\).

    Table of iteration for \(f(x) = \sqrt(x) - \cos{x}\) on \(\interval{0}{1}\):

    \begin{tabular}{r S[table-format=1.1] S[table-format=1.2] S[table-format=1.3] S[table-format=-1.9]}
        \\
        \toprule
        \(n\)  &  {\(a_n\)}  &  {\(b_n\)}  &  {\(p_n\)}  &  {\(f(p_n)\)}  \\
        \midrule
            1  &  0          &  1          &  0.5        &  -0.170475781  \\
            2  &  0.5        &  1          &  0.75       &   0.134336535  \\
            3  &  0.5        &  0.75       &  0.625      &  -0.020393704  \\
        \bottomrule
        \\
    \end{tabular}

    So \(p_3 = \num{0.625}\).
\end{solution}

\begin{exercise}
    Let \(f(x) = 3 (x + 1) (x - \dfrac{1}{2}) (x - 1)\). Use the bisection
    method to find \(p_3\) in the following intervals:

    \begin{enumerate}[label = (\alph*)]
        \item \(\interval{-2}{\num{1.5}}\)
        \item \(\interval{\num{-1.5}}{\num{2.5}}\)
    \end{enumerate}
\end{exercise}

\begin{solution}
    \begin{enumerate}[label = (\alph*)]
        \item \(f(-2) = \num{-22.5}\) and \(f(\num{1.5}) = \num{3.75}\) have the
            opposite signs, so there's a root in \(\interval{-2}{\num{1.5}}\).

            We have the following table:

            \begin{tabular}{r S[table-format=-1.3] S[table-format=-1.2] S[table-format=-1.4] S[table-format=-1.9]}
                \\
                \toprule
                \(n\)  &  {\(a_n\)}  &  {\(b_n\)}  &  {\(p_n\)}  &  {\(f(p_n)\)}  \\
                \midrule
                    1  &  -2         &   1.5       &  -0.25      &   2.109375     \\
                    2  &  -2         &  -0.25      &  -1.125     &  -1.294921875  \\
                    3  &  -1.125     &  -0.25      &  -0.6875    &   1.878662109  \\
                \bottomrule
                \\
            \end{tabular}

            So \(p_3 = \num{-0.6875}\).

        \item \(f(\num{-1.25}) = \num{-2.953125}\) and \(f(\num{2.5})) =
            \num{31.5}\) have the opposite signs, so there's a root in
            \(\interval{\num{-1.25}}{\num{2.5}}\).

            We have the following table:

            \begin{tabular}{r S[table-format=-1.1] S[table-format=-1.1] S[table-format=1.1] S[table-format=1.1]}
                \\
                \toprule
                \(n\)  &  {\(a_n\)}  &  {\(b_n\)}  &  {\(p_n\)}  &  {\(f(p_n)\)}  \\
                \midrule
                   1   &  -1.5       &   2.5       &   0.5       &   0            \\
                \bottomrule
                \\
            \end{tabular}

            The solution is found in the first iteration so \(p_3\) doesn't
            exist.
    \end{enumerate}
\end{solution}

\begin{exercise}
    Use the Bisection method to find solutions accurate to within \(10^{-2}\)
    for \(x^3 - 7x^2 + 14x - 6 = 0\) in the following intervals:

    \begin{enumerate}[label=(\alph*)]
        \item \(\interval{0}{1}\)
        \item \(\interval{1}{\num{3.2}}\)
        \item \(\interval{\num{3.2}}{4}\)
    \end{enumerate}
\end{exercise}

\begin{solution}
    \begin{enumerate}[label=(\alph*)]
        \item \(f(0) = -6\) and \(f(1) = 2\) have the opposite signs, so there's
            a root in \(\interval{0}{1}\).

            The number of iteration \(n\) needed to approximate \(p\) to within
            \(10^{-2}\) is:

            \[\abs{p_n - p} \leq \frac{1 - 0}{2^n} < 10^{-2} \iff n \geq 7\]

            We have the following table:

            \begin{tabular}{r S[table-format=1.6] S[table-format=1.5] S[table-format=1.7] S[table-format=-1.6]}
                \\
                \toprule
                \(n\)  &  {\(a_n\)}  &  {\(b_n\)}  &  {\(p_n\)}  &  {\(f(p_n)\)}  \\
                \midrule
                    1  &  0          &  1          &  0.5        &  -0.625        \\
                    2  &  0.5        &  1          &  0.75       &   0.984375     \\
                    3  &  0.5        &  0.75       &  0.625      &   0.259766     \\
                    4  &  0.5        &  0.625      &  0.5625     &  -0.161865     \\
                    5  &  0.5625     &  0.625      &  0.59375    &   0.054047     \\
                    6  &  0.5625     &  0.59375    &  0.578125   &  -0.052624     \\
                    7  &  0.578125   &  0.59375    &  0.5859375  &   0.001031     \\
                \bottomrule
                \\
            \end{tabular}

            So \(p \approx \num{0.5859}\).

        \item \(f(1) = 2\) and \(f(\num{3.2}) = \num{-0.112}\) have the opposite
            signs, so there's a root in \(\interval{1}{\num{3.2}}\).

            The number of iteration \(n\) needed to approximate \(p\) to within
            \(10^{-2}\) is:

            \[\abs{p_n - p} \leq \frac{\num{3.2} - 1}{2^n} < 10^{-2} \iff n \geq 8\]

            We have the following table:

            \begin{tabular}{r S[table-format=1.5] S[table-format=1.6] S[table-format=1.6] S[table-format=-1.6]}
                \\
                \toprule
                \(n\)  &  {\(a_n\)}  &  {\(b_n\)}  &  {\(p_n\)}  &  {\(f(p_n)\)}  \\
                \midrule
                    1  &  1          &  3.2        &  2.1        &   1.791        \\
                    2  &  2.1        &  3.2        &  2.65       &   0.552125     \\
                    3  &  2.65       &  3.2        &  2.925      &   0.085828     \\
                    4  &  2.925      &  3.2        &  3.0625     &  -0.054443     \\
                    5  &  2.925      &  3.0625     &  2.99375    &   0.006328     \\
                    6  &  2.99375    &  3.0625     &  3.028125   &  -0.026521     \\
                    7  &  2.99375    &  3.02813    &  3.010938   &  -0.010697     \\
                    8  &  2.99375    &  3.010938   &  3.002344   &  -0.002333     \\
                \bottomrule
                \\
            \end{tabular}

            So \(p \approx \num{3.0023}\).

        \item \(f(\num{3.2}) = \num{-0.112}\) and \(f(4) = 2\) have the opposite
            signs, so there's a root in \(\interval{\num{3.2}}{4}\).

            The number of iteration \(n\) needed to approximate \(p\) to within
            \(10^{-2}\) is:

            \[\abs{p_n - p} \leq \frac{4 - \num{3.2}}{2^n} < 10^{-2} \iff n \geq 7\]

            We have the following table:

            \begin{tabular}{r S[table-format=1.5] S[table-format=1.6] S[table-format=1.6] S[table-format=-1.6]}
                \\
                \toprule
                \(n\)  &  {\(a_n\)}  &  {\(b_n\)}  &  {\(p_n\)}  &  {\(f(p_n)\)}  \\
                \midrule
                    1  &  3.2        &  4          &  3.6        &   0.336        \\
                    2  &  3.2        &  3.6        &  3.4        &  -0.016        \\
                    3  &  3.4        &  3.6        &  3.5        &   0.125        \\
                    4  &  3.4        &  3.5        &  3.45       &   0.046125     \\
                    5  &  3.4        &  3.45       &  3.425      &   0.013016     \\
                    6  &  3.4        &  3.425      &  3.4125     &  -0.001998     \\
                    7  &  3.4125     &  3.425      &  3.41875    &   0.005382     \\
                \bottomrule
                \\
            \end{tabular}

            So \(p \approx \num{3.4188}\).
    \end{enumerate}
\end{solution}

\begin{exercise}
    Use the Bisection method to find solutions accurate to within \(10^{-2}\)
    for \(x^4 - 2x^3 - 4x^2 + 4x + 4 = 0\) for the following intervals:

    \begin{enumerate}[label=(\alph*)]
        \item \(\interval{-2}{-1}\)
        \item \(\interval{0}{2}\)
        \item \(\interval{2}{3}\)
        \item \(\interval{-1}{0}\)
    \end{enumerate}
\end{exercise}

\begin{solution}
    \begin{enumerate}[label=(\alph*)]
        \item \(f(-2) = 12\) and \(f(-1) = -1\) have the opposite signs, so
            there's a root in \(\interval{-2}{-1}\).

            The number of iteration \(n\) needed to approximate \(p\) to within
            \(10^{-2}\) is:

            \[\abs{p_n - p} \leq \frac{-1 - (-2)}{2^n} < 10^{-2} \iff n \geq 7\]

            We have the following table:

            \begin{tabular}{r S[table-format=-1.5] S[table-format=-1.5] S[table-format=-1.6] S[table-format=-1.6]}
                \\
                \toprule
                \(n\)  &  {\(a_n\)}  &  {\(b_n\)}  &  {\(p_n\)}  &  {\(f(p_n)\)}  \\
                \midrule
                    1  &  -2         &  -1         &  -1.5       &   0.8125       \\
                    2  &  -1.5       &  -1         &  -1.25      &  -0.902344     \\
                    3  &  -1.5       &  -1.25      &  -1.375     &  -0.288818     \\
                    4  &  -1.5       &  -1.375     &  -1.4375    &   0.195328     \\
                    5  &  -1.4375    &  -1.375     &  -1.40625   &  -0.062667     \\
                    6  &  -1.4375    &  -1.40625   &  -1.421875  &   0.062263     \\
                    7  &  -1.421875  &  -1.40625   &  -1.414063  &  -0.001208     \\
                \bottomrule
                \\
            \end{tabular}

            So \(p \approx \num{-1.4141}\).

        \item \(f(0) = 4\) and \(f(2) = -4\) have the opposite signs, so there's
            a root in \(\interval{0}{2}\).

            The number of iteration \(n\) needed to approximate \(p\) to within
            \(10^{-2}\) is:

            \[\abs{p_n - p} \leq \frac{2 - 0}{2^n} < 10^{-2} \iff n \geq 8\]

            We have the following table:

            \begin{tabular}{r S[table-format=1.5] S[table-format=1.6] S[table-format=1.6] S[table-format=-1.6]}
                \\
                \toprule
                \(n\)  &  {\(a_n\)}  &  {\(b_n\)}  &  {\(p_n\)}  &  {\(f(p_n)\)}  \\
                \midrule
                    1  &  0          &  2          &  1          &   3            \\
                    2  &  1          &  2          &  1.5        &  -0.6875       \\
                    3  &  1          &  1.5        &  1.25       &   1.285156     \\
                    4  &  1.25       &  1.5        &  1.375      &   0.312744     \\
                    5  &  1.375      &  1.5        &  1.4375     &  -0.186508     \\
                    6  &  1.375      &  1.4375     &  1.40625    &   0.063676     \\
                    7  &  1.40625    &  1.4375     &  1.421875   &  -0.061318     \\
                    8  &  1.40625    &  1.421875   &  1.414063   &   0.001208     \\
                \bottomrule
                \\
            \end{tabular}

            So \(p \approx \num{1.4141}\).

        \item \(f(2) = -4\) and \(f(3) = 7\) have the opposite signs, so there's
            a root in \(\interval{2}{3}\).

            The number of iteration \(n\) needed to approximate \(p\) to within
            \(10^{-2}\) is:

            \[\abs{p_n - p} \leq \frac{3 - 2}{2^n} < 10^{-2} \iff n \geq 7\]

            We have the following table:

            \begin{tabular}{r S[table-format=1.5] S[table-format=1.6] S[table-format=1.6] S[table-format=-1.6]}
                \\
                \toprule
                \(n\)  &  {\(a_n\)}  &  {\(b_n\)}  &  {\(p_n\)}  &  {\(f(p_n)\)}  \\
                \midrule
                    1  &  2          &  3          &  2.5        &  -3.1875       \\
                    2  &  2.5        &  3          &  2.75       &   0.347656     \\
                    3  &  2.5        &  2.75       &  2.625      &  -1.757568     \\
                    4  &  2.625      &  2.75       &  2.6875     &  -0.795639     \\
                    5  &  2.6875     &  2.75       &  2.71875    &  -0.247466     \\
                    6  &  2.71875    &  2.75       &  2.734375   &   0.044125     \\
                    7  &  2.71875    &  2.734375   &  2.726563   &  -0.103151     \\
                \bottomrule
                \\
            \end{tabular}

            So \(p \approx \num{2.7266}\).

        \item \(f(-1) = -1\) and \(f(0) = 4\) have the opposite signs, so
            there's a root in \(\interval{-1}{0}\).

            The number of iteration \(n\) needed to approximate \(p\) to within
            \(10^{-2}\) is:

            \[\abs{p_n - p} \leq \frac{0 - (-1)}{2^n} < 10^{-2} \iff n \geq 7\]

            We have the following table:

            \begin{tabular}{r S[table-format=-1.6] S[table-format=-1.5] S[table-format=-1.6] S[table-format=-1.6]}
                \\
                \toprule
                \(n\)  &  {\(a_n\)}  &  {\(b_n\)}  &  {\(p_n\)}  &  {\(f(p_n)\)}  \\
                \midrule
                    1  &  -1         &  0          &  -0.5       &   1.3125       \\
                    2  &  -1         &  -0.5       &  -0.75      &  -0.089844     \\
                    3  &  -0.75      &  -0.5       &  -0.625     &   0.578369     \\
                    4  &  -0.75      &  -0.625     &  -0.6875    &   0.232681     \\
                    5  &  -0.75      &  -0.6875    &  -0.71875   &   0.068086     \\
                    6  &  -0.75      &  -0.71875   &  -0.734375  &  -0.011768     \\
                    7  &  -0.734375  &  -0.71875   &  -0.726563  &   0.027943     \\
                \bottomrule
                \\
            \end{tabular}

            So \(p \approx \num{-0.7266}\).
    \end{enumerate}
\end{solution}

\begin{exercise}
    Use the Bisection method to find solutions accurate to within \(10^{-5}\)
    for the following problems:

    \begin{enumerate}[label=(\alph*)]
        \item \(x - 2^{-x} = 0\), \(x \in \interval{0}{1}\)
        \item \(e^x - x^2 + 3x - 2 = 0\), \(x \in \interval{0}{1}\)
        \item \(2 x \cos{2x} - (x + 1)^2 = 0\), \(x \in \interval{-3}{-2}\)
        \item \(x \cos{x} - 2x^2 + 3x - 1 = 0\), \(x \in \interval{\num{0.2}}{\num{0.3}}\)
    \end{enumerate}
\end{exercise}

\begin{solution}
    \begin{enumerate}[label=(\alph*)]
        \item \(f(0) = -1\) and \(f(1) = \num{0.5}\) have the opposite signs, so
            there's a root in \(\interval{0}{1}\).

            The number of iteration \(n\) needed to approximate \(p\) to within
            \(10^{-5}\) is:

            \[\abs{p_n - p} \leq \frac{1 - 0}{2^n} < 10^{-5} \iff n \geq 17\]

            We have the following table:

            % TODO: Check why longtable doesn't need \\, but tabular does
            \begin{longtable}{r S[table-format=1.9] S[table-format=1.9] S[table-format=1.9] S[table-format=-1.9]}
                \toprule
                \(n\)  &   {\(a_n\)}   &   {\(b_n\)}   &   {\(p_n\)}   &  {\(f(p_n)\)}  \\
                \midrule
                    1  &  0            &  1            &  0.5          &  -0.207106781  \\
                    2  &  0.5          &  1            &  0.75         &   0.155396442  \\
                    3  &  0.5          &  0.75         &  0.625        &  -0.023419777  \\
                    4  &  0.625        &  0.75         &  0.6875       &   0.066571094  \\
                    5  &  0.625        &  0.6875       &  0.65625      &   0.021724521  \\
                    6  &  0.625        &  0.65625      &  0.640625     &  -0.000810008  \\
                    7  &  0.640625     &  0.65625      &  0.6484375    &   0.010466611  \\
                    8  &  0.640625     &  0.6484375    &  0.64453125   &   0.004830646  \\
                    9  &  0.640625     &  0.64453125   &  0.642578125  &   0.002010906  \\
                   10  &  0.640625     &  0.642578125  &  0.641601562  &   0.000600596  \\
                   11  &  0.640625     &  0.641601562  &  0.641113281  &  -0.000104669  \\
                   12  &  0.641113281  &  0.641601562  &  0.641357422  &   0.000247972  \\
                   13  &  0.641113281  &  0.641357422  &  0.641235352  &   0.000071654  \\
                   14  &  0.641113281  &  0.641235352  &  0.641174316  &  -0.000016507  \\
                   15  &  0.641174316  &  0.641235352  &  0.641204834  &   0.000027573  \\
                   16  &  0.641174316  &  0.641204834  &  0.641189575  &   0.000005533  \\
                   17  &  0.641174316  &  0.641189575  &  0.641181946  &  -0.000005487  \\
                \bottomrule
            \end{longtable}

            So \(p \approx \num{-0.641182}\).

        \item \(f(0) = -1\) and \(f(1) = e\) have the opposite signs, so there's
            a root in \(\interval{0}{1}\).

            The number of iteration \(n\) needed to approximate \(p\) to within
            \(10^{-5}\) is:

            \[\abs{p_n - p} \leq \frac{1 - 0}{2^n} < 10^{-5} \iff n \geq 17\]

            We have the following table:

            \begin{longtable}{r S[table-format=1.9] S[table-format=1.9] S[table-format=1.9] S[table-format=-1.9]}
                \toprule
                \(n\)  &   {\(a_n\)}   &   {\(b_n\)}   &   {\(p_n\)}   &  {\(f(p_n)\)}  \\
                \midrule
                    1  &  0            &  1            &  0.5          &   0.898721271  \\
                    2  &  0            &  0.5          &  0.25         &  -0.028474583  \\
                    3  &  0.25         &  0.5          &  0.375        &   0.439366415  \\
                    4  &  0.25         &  0.375        &  0.3125       &   0.206681691  \\
                    5  &  0.25         &  0.3125       &  0.28125      &   0.089433196  \\
                    6  &  0.25         &  0.28125      &  0.265625     &   0.030564234  \\
                    7  &  0.25         &  0.265625     &  0.2578125    &   0.001066368  \\
                    8  &  0.25         &  0.2578125    &  0.25390625   &  -0.013698684  \\
                    9  &  0.25390625   &  0.2578125    &  0.255859375  &  -0.006314807  \\
                   10  &  0.255859375  &  0.2578125    &  0.256835938  &  -0.002623882  \\
                   11  &  0.256835938  &  0.2578125    &  0.257324219  &  -0.000778673  \\
                   12  &  0.257324219  &  0.2578125    &  0.257568359  &   0.000143868  \\
                   13  &  0.257324219  &  0.257568359  &  0.257446289  &  -0.000317397  \\
                   14  &  0.257446289  &  0.257568359  &  0.257507324  &  -0.000086763  \\
                   15  &  0.257507324  &  0.257568359  &  0.257537842  &   0.000028553  \\
                   16  &  0.257507324  &  0.257537842  &  0.257522583  &  -0.000029105  \\
                   17  &  0.257522583  &  0.257537842  &  0.257530212  &  -0.000000276  \\
                \bottomrule
            \end{longtable}

            So \(p \approx \num{0.25753}\).

        \item \(f(-3) \approx \num{-9.76102172}\) and \(f(-2) =
            \num{1.614574483}\) have the opposite signs, so there's a root in
            \(\interval{-3}{-2}\).

            The number of iteration \(n\) needed to approximate \(p\) to within
            \(10^{-5}\) is:

            \[\abs{p_n - p} \leq \frac{-2 - (-3)}{2^n} < 10^{-5} \iff n \geq 17\]

            We have the following table:

            \begin{longtable}{r S[table-format=-1.8] S[table-format=-1.8] S[table-format=-1.8] S[table-format=-1.9]}
                \toprule
                \(n\)  &   {\(a_n\)}   &   {\(b_n\)}   &   {\(p_n\)}   &  {\(f(p_n)\)}  \\
                \midrule
                    1  &  -3           &  -2           &  -2.5         &  -3.66831093   \\
                    2  &  -2.5         &  -2           &  -2.25        &  -0.613918903  \\
                    3  &  -2.25        &  -2           &  -2.125       &   0.630246832  \\
                    4  &  -2.25        &  -2.125       &  -2.1875      &   0.038075532  \\
                    5  &  -2.25        &  -2.1875      &  -2.21875     &  -0.280836176  \\
                    6  &  -2.21875     &  -2.1875      &  -2.203125    &  -0.119556815  \\
                    7  &  -2.203125    &  -2.1875      &  -2.1953125   &  -0.040278514  \\
                    8  &  -2.1953125   &  -2.1875      &  -2.19140625  &  -0.000985195  \\
                    9  &  -2.19140625  &  -2.1875      &  -2.18945312  &   0.018574337  \\
                   10  &  -2.19140625  &  -2.18945312  &  -2.19042969  &   0.008801851  \\
                   11  &  -2.19140625  &  -2.19042969  &  -2.19091797  &   0.003910147  \\
                   12  &  -2.19140625  &  -2.19091797  &  -2.19116211  &   0.00146293   \\
                   13  &  -2.19140625  &  -2.19116211  &  -2.19128418  &   0.000238981  \\
                   14  &  -2.19140625  &  -2.19128418  &  -2.19134521  &  -0.000373078  \\
                   15  &  -2.19134521  &  -2.19128418  &  -2.1913147   &  -0.000067041  \\
                   16  &  -2.1913147   &  -2.19128418  &  -2.19129944  &   0.000085972  \\
                \bottomrule
            \end{longtable}

            So \(p \approx \num{-2.191299}\).

        \item \(f(\num{0.2}) \approx \num{-0.283986684}\) and \(f(\num{0.3}) =
            \num{0.006600946}\) have the opposite signs, so there's a root in
            \(\interval{\num{0.2}}{\num{0.3}}\).

            The number of iteration \(n\) needed to approximate \(p\) to
            within \(10^{-5}\) is:

            \[\abs{p_n - p} \leq \frac{\num{0.3} - \num{0.2}}{2^n} < 10^{-5} \iff n \geq 14\]

            We have the following table:

            \begin{longtable}{r S[table-format=1.9] S[table-format=1.9] S[table-format=1.9] S[table-format=-1.9]}
                \toprule
                \(n\)  &   {\(a_n\)}   &   {\(b_n\)}   &   {\(p_n\)}   &  {\(f(p_n)\)}  \\
                \midrule
                    1  &  0.2          &  0.3          &  0.25         &  -0.132771895  \\
                    2  &  0.25         &  0.3          &  0.275        &  -0.061583071  \\
                    3  &  0.275        &  0.3          &  0.2875       &  -0.027112719  \\
                    4  &  0.2875       &  0.3          &  0.29375      &  -0.010160959  \\
                    5  &  0.29375      &  0.3          &  0.296875     &  -0.001756232  \\
                    6  &  0.296875     &  0.3          &  0.2984375    &   0.002428306  \\
                    7  &  0.296875     &  0.2984375    &  0.29765625   &   0.000337524  \\
                    8  &  0.296875     &  0.29765625   &  0.297265625  &  -0.000708983  \\
                    9  &  0.297265625  &  0.29765625   &  0.297460938  &  -0.000185637  \\
                   10  &  0.297460938  &  0.29765625   &  0.297558594  &   0.000075967  \\
                   11  &  0.297460938  &  0.297558594  &  0.297509766  &  -0.000054829  \\
                   12  &  0.297509766  &  0.297558594  &  0.29753418   &   0.00001057   \\
                   13  &  0.297509766  &  0.29753418   &  0.297521973  &  -0.000022129  \\
                   14  &  0.297521973  &  0.29753418   &  0.297528076  &  -0.000005779  \\
               \bottomrule
            \end{longtable}

            So \(p \approx \num{0.297528}\).
    \end{enumerate}
\end{solution}

\begin{exercise}
    Use the Bisection method to find solutions accurate to within \(10^{-5}\)
    for the following problems:

    \begin{enumerate}[label=(\alph*)]
        \item \(3x - e^x = 0\), \(x \in \interval{1}{2}\)
        \item \(2x + 3 \cos{x} -e^x = 0\), \(x \in \interval{0}{1}\)
        \item \(x^2 - 4x + 4 - \ln{x} = 0\), \(x \in \interval{1}{2}\)
        \item \(x + 1 - 2 \sin{\pi x} = 0\), \(x \in \interval{0}{\num{0.5}}\)
    \end{enumerate}
\end{exercise}

\begin{solution}
    \begin{enumerate}
        \item \(f(1) \approx \num{0.281718172}\) and \(f(2) =
            \num{-1.389056099}\) have the opposite signs, so there's a root in
            \(\interval{1}{2}\).

            The number of iteration \(n\) needed to approximate \(p\) to within
            \(10^{-5}\) is:

            \[\abs{p_n - p} \leq \frac{2 - 1}{2^n} < 10^{-5} \iff n \geq 17\]

            We have the following table:

            \begin{longtable}{r S[table-format=1.8] S[table-format=1.8] S[table-format=1.8] S[table-format=-1.9]}
                \toprule
                \(n\)  &   {\(a_n\)}   &   {\(b_n\)}   &   {\(p_n\)}   &  {\(f(p_n)\)}  \\
                \midrule
                    1  &  1            &  2            &  1.5          &   0.01831093   \\
                    2  &  1.5          &  2            &  1.75         &  -0.504602676  \\
                    3  &  1.5          &  1.75         &  1.625        &  -0.203419037  \\
                    4  &  1.5          &  1.625        &  1.5625       &  -0.083233182  \\
                    5  &  1.5          &  1.5625       &  1.53125      &  -0.030203153  \\
                    6  &  1.5          &  1.53125      &  1.515625     &  -0.005390404  \\
                    7  &  1.5          &  1.515625     &  1.5078125    &   0.006598107  \\
                    8  &  1.5078125    &  1.515625     &  1.51171875   &   0.000638447  \\
                    9  &  1.51171875   &  1.515625     &  1.51367188   &  -0.002367313  \\
                   10  &  1.51171875   &  1.51367188   &  1.51269531   &  -0.000862268  \\
                   11  &  1.51171875   &  1.51269531   &  1.51220703   &  -0.00011137   \\
                   12  &  1.51171875   &  1.51220703   &  1.51196289   &   0.000263674  \\
                   13  &  1.51196289   &  1.51220703   &  1.51208496   &   0.000076186  \\
                   14  &  1.51208496   &  1.51220703   &  1.512146     &  -0.000017584  \\
                   15  &  1.51208496   &  1.512146     &  1.51211548   &   0.000029303  \\
                   16  &  1.51211548   &  1.512146     &  1.51213074   &   0.00000586   \\
                   17  &  1.51213074   &  1.512146     &  1.51213837   &  -0.000005861  \\
                \bottomrule
            \end{longtable}

            So \(p \approx \num{1.512138}\).

        \item \(f(0) = 2\) and \(f(1) \approx \num{0.902625089}\) have the same
            sign, so there's no root in \(\interval{0}{1}\).

        \item \(f(1) = 1\) and \(f(2) = \num{-0.693147181}\) have the opposite
            signs, so there's a root in \(\interval{1}{2}\).

            The number of iteration \(n\) needed to approximate \(p\) to within
            \(10^{-5}\) is:

            \[\abs{p_n - p} \leq \frac{2 - 1}{2^n} < 10^{-5} \iff n \geq 17\]

            We have the following table:

            \begin{longtable}{r S[table-format=1.8] S[table-format=1.8] S[table-format=1.8] S[table-format=-1.9]}
                \toprule
                \(n\)  &   {\(a_n\)}   &   {\(b_n\)}   &   {\(p_n\)}   &  {\(f(p_n)\)}  \\
                \midrule
                    1  &  1            &  2            &  1.5          &  -0.155465108  \\
                    2  &  1            &  1.5          &  1.25         &   0.339356449  \\
                    3  &  1.25         &  1.5          &  1.375        &   0.072171269  \\
                    4  &  1.375        &  1.5          &  1.4375       &  -0.046499244  \\
                    5  &  1.375        &  1.4375       &  1.40625      &   0.011612476  \\
                    6  &  1.40625      &  1.4375       &  1.421875     &  -0.017747908  \\
                    7  &  1.40625      &  1.421875     &  1.4140625    &  -0.003144013  \\
                    8  &  1.40625      &  1.4140625    &  1.41015625   &   0.004215136  \\
                    9  &  1.41015625   &  1.4140625    &  1.41210938   &   0.00053079   \\
                   10  &  1.41210938   &  1.4140625    &  1.41308594   &  -0.001307804  \\
                   11  &  1.41210938   &  1.41308594   &  1.41259766   &  -0.000388805  \\
                   12  &  1.41210938   &  1.41259766   &  1.41235352   &   0.000070918  \\
                   13  &  1.41235352   &  1.41259766   &  1.41247559   &  -0.000158962  \\
                   14  &  1.41235352   &  1.41247559   &  1.41241455   &  -0.000044027  \\
                   15  &  1.41235352   &  1.41241455   &  1.41238403   &   0.000013444  \\
                   16  &  1.41238403   &  1.41241455   &  1.41239929   &  -0.000015292  \\
                   17  &  1.41238403   &  1.41239929   &  1.41239166   &  -0.000000924  \\
                \bottomrule
            \end{longtable}

            So \(p \approx \num{1.412392}\).

        \item \(f(0) = 1\) and \(f(1) = \num{-0.5}\) have the opposite signs, so
            there's a root in \(\interval{0}{\num{0.5}}\).

            The number of iteration \(n\) needed to approximate \(p\) to within
            \(10^{-5}\) is:

            \[\abs{p_n - p} \leq \frac{\num{0.5} - 0}{2^n} < 10^{-5} \iff n \geq 16\]

            We have the following table:

            \begin{longtable}{r S[table-format=1.9] S[table-format=1.9] S[table-format=1.9] S[table-format=-1.9]}
                \toprule
                \(n\)  &   {\(a_n\)}   &   {\(b_n\)}   &   {\(p_n\)}   &  {\(f(p_n)\)}  \\
                \midrule
                    1  &  0            &  0.5          &  0.25         &  -0.164213562  \\
                    2  &  0            &  0.25         &  0.125        &   0.359633135  \\
                    3  &  0.125        &  0.25         &  0.1875       &   0.076359534  \\
                    4  &  0.1875       &  0.25         &  0.21875      &  -0.050036568  \\
                    5  &  0.1875       &  0.21875      &  0.203125     &   0.011726391  \\
                    6  &  0.203125     &  0.21875      &  0.2109375    &  -0.019525681  \\
                    7  &  0.203125     &  0.2109375    &  0.20703125   &  -0.003990833  \\
                    8  &  0.203125     &  0.20703125   &  0.205078125  &   0.003845166  \\
                    9  &  0.205078125  &  0.20703125   &  0.206054688  &  -0.00007851   \\
                   10  &  0.205078125  &  0.206054688  &  0.205566406  &   0.001881912  \\
                   11  &  0.205566406  &  0.206054688  &  0.205810547  &   0.000901347  \\
                   12  &  0.205810547  &  0.206054688  &  0.205932617  &   0.00041133   \\
                   13  &  0.205932617  &  0.206054688  &  0.205993652  &   0.000166388  \\
                   14  &  0.205993652  &  0.206054688  &  0.20602417   &   0.000043934  \\
                   15  &  0.20602417   &  0.206054688  &  0.206039429  &  -0.000017289  \\
                   16  &  0.20602417   &  0.206039429  &  0.206031799  &   0.000013322  \\
                   \bottomrule
            \end{longtable}

            So \(p \approx \num{0.206032}\).
    \end{enumerate}
\end{solution}

\begin{exercise}
    \begin{enumerate}[label=(\alph*)]
        \item Sketch the graphs of \(y = x\) and \(y = 2 \sin{x}\).
        \item Use the Bisection method to find an approximation to within
            \(10^{−5}\) to the first positive value of \(x\) with \(x = 2
            \sin{x}\).
    \end{enumerate}
\end{exercise}

\begin{solution}
    \begin{enumerate}[label=(\alph*)]
        \item Graph of \(y = x\) and \(y = 2 \sin{x}\) is as follow:

            \subfile{graphics/exercise_7_graph/exercise_7_graph.tex}

        \item According to the graph, the first positive root \(p\) of \(f = x -
            2 \sin{x}\) is in \(\interval{\dfrac{\pi}{2}}{\pi}\).

            The number of iteration \(n\) needed to approximate \(p\) to within
            \(10^{-5}\) in that range is:

            \[\abs{p_n - p} \leq \frac{\pi - \dfrac{\pi}{2}}{2^n} < 10^{-5} \iff n \geq 18\]

            We have the following table:

            \begin{longtable}{r S[table-format=1.8] S[table-format=1.8] S[table-format=1.8] S[table-format=-1.9]}
                \toprule
                \(n\)  &   {\(a_n\)}   &   {\(b_n\)}   &   {\(p_n\)}   &  {\(f(p_n)\)}  \\
                \midrule
                    1  &  1.57079633   &  3.14159265   &  2.35619449   &   0.941980928  \\
                    2  &  1.57079633   &  2.35619449   &  1.96349541   &   0.115736343  \\
                    3  &  1.57079633   &  1.96349541   &  1.76714587   &  -0.194424693  \\
                    4  &  1.76714587   &  1.96349541   &  1.86532064   &  -0.048560033  \\
                    5  &  1.86532064   &  1.96349541   &  1.91440802   &   0.031319893  \\
                    6  &  1.86532064   &  1.91440802   &  1.88986433   &  -0.009192031  \\
                    7  &  1.88986433   &  1.91440802   &  1.90213618   &   0.010921526  \\
                    8  &  1.88986433   &  1.90213618   &  1.89600025   &   0.000829072  \\
                    9  &  1.88986433   &  1.89600025   &  1.89293229   &  -0.004190408  \\
                   10  &  1.89293229   &  1.89600025   &  1.89446627   &  -0.001682899  \\
                   11  &  1.89446627   &  1.89600025   &  1.89523326   &  -0.000427471  \\
                   12  &  1.89523326   &  1.89600025   &  1.89561676   &   0.000200661  \\
                   13  &  1.89523326   &  1.89561676   &  1.89542501   &  -0.00011344   \\
                   14  &  1.89542501   &  1.89561676   &  1.89552088   &   0.000043602  \\
                   15  &  1.89542501   &  1.89552088   &  1.89547295   &  -0.000034921  \\
                   16  &  1.89547295   &  1.89552088   &  1.89549692   &   0.00000434   \\
                   17  &  1.89547295   &  1.89549692   &  1.89548493   &  -0.000015291  \\
                   18  &  1.89548493   &  1.89549692   &  1.89549092   &  -0.000005476  \\
                \bottomrule
            \end{longtable}

            So \(p \approx \num{1.895491}\).

    \end{enumerate}
\end{solution}

\end{document}
