\documentclass[../../Assignments.tex]{subfiles}


\begin{document}

\chapter{Solution approximation}

\section{The Bisection Method}

\begin{exercise}
    Use the Bisection method to find \(p_3\) for \(f(x) = \sqrt{x} - \cos{x}\) on \(\interval{0}{1}\).
\end{exercise}

\begin{solution}
    \(f(0) = -1\) and \(f(1) \approx \num{0.459697694}\) have the opposite
    signs, so there's a root in \(\interval{0}{1}\).

    Applying Bisection method generates the following table:

    \begin{table}[H]    % Default table floats (i.e. [btp]). [h] keeps it here.
        % tabular must be nested inside table, otherwise manual \\ must be used
        \centering
        \begin{tabular}{r S[table-format=1.1] S[table-format=1.2] S[table-format=1.3] S[table-format=-1.9]}
            \toprule
            \(n\)  &  {\(a_n\)}  &  {\(b_n\)}  &  {\(p_n\)}  &  {\(f(p_n)\)}  \\
            \midrule
                1  &  0          &  1          &  0.5        &  -0.170475781  \\
                2  &  0.5        &  1          &  0.75       &   0.134336535  \\
                3  &  0.5        &  0.75       &  0.625      &  -0.020393704  \\
            \bottomrule
        \end{tabular}
    \end{table}

    So \(p_3 = \num{0.625}\).
\end{solution}

\begin{exercise}
    Let \(f(x) = 3 (x + 1) (x - \dfrac{1}{2}) (x - 1)\). Use the bisection
    method to find \(p_3\) in the following intervals:

    \begin{multicols}{2}
        \begin{enumerate}[label = (\alph*)]
            \item \(\interval{-2}{\num{1.5}}\)
            \item \(\interval{\num{-1.5}}{\num{2.5}}\)
        \end{enumerate}
    \end{multicols}
\end{exercise}

\begin{solution}
    \begin{enumerate}[label = (\alph*)]
        \item \(f(-2) = \num{-22.5}\) and \(f(\num{1.5}) = \num{3.75}\) have the
            opposite signs, so there's a root in \(\interval{-2}{\num{1.5}}\).

            Applying Bisection method generates the following table:

            \begin{table}[H]
                \centering
                \begin{tabular}{r S[table-format=-1.3] S[table-format=-1.2] S[table-format=-1.4] S[table-format=-1.9]}
                    \toprule
                    \(n\)  &  {\(a_n\)}  &  {\(b_n\)}  &  {\(p_n\)}  &  {\(f(p_n)\)}  \\
                    \midrule
                        1  &  -2         &   1.5       &  -0.25      &   2.109375     \\
                        2  &  -2         &  -0.25      &  -1.125     &  -1.294921875  \\
                        3  &  -1.125     &  -0.25      &  -0.6875    &   1.878662109  \\
                    \bottomrule
                \end{tabular}
            \end{table}

            So \(p_3 = \num{-0.6875}\).

        \item \(f(\num{-1.25}) = \num{-2.953125}\) and \(f(\num{2.5}) =
            \num{31.5}\) have the opposite signs, so there's a root in
            \(\interval{\num{-1.25}}{\num{2.5}}\).

            Applying Bisection method generates the following table:

            \begin{table}[H]
                \centering
                \begin{tabular}{r S[table-format=-1.1] S[table-format=1.1] S[table-format=1.1] S[table-format=1]}
                    \toprule
                    \(n\)  &  {\(a_n\)}  &  {\(b_n\)}  &  {\(p_n\)}  &  {\(f(p_n)\)}  \\
                    \midrule
                       1   &  -1.5       &   2.5       &   0.5       &   0            \\
                    \bottomrule
                \end{tabular}
            \end{table}

            The solution is found in the first iteration so \(p_3\) doesn't
            exist.
    \end{enumerate}
\end{solution}

\begin{exercise}
    Use the Bisection method to find solutions accurate to within \(10^{-2}\)
    for \(x^3 - 7x^2 + 14x - 6 = 0\) in the following intervals:

    \begin{multicols}{3}
        \begin{enumerate}[label = (\alph*)]
            \item \(\interval{0}{1}\)
            \item \(\interval{1}{\num{3.2}}\)
            \item \(\interval{\num{3.2}}{4}\)
        \end{enumerate}
    \end{multicols}
\end{exercise}

\begin{solution}
    \begin{enumerate}[label = (\alph*)]
        \item \(f(0) = -6\) and \(f(1) = 2\) have the opposite signs, so there's
            a root in \(\interval{0}{1}\).

            The number of iteration \(n\) needed to approximate \(p\) to within
            \(10^{-2}\) is:

            \[\abs{p_n - p} \leq \frac{1 - 0}{2^n} < 10^{-2} \iff n \geq 7\]

            Applying Bisection method generates the following table:

            \begin{table}[H]
                \centering
                \begin{tabular}{r S[table-format=1.6] S[table-format=1.5] S[table-format=1.7] S[table-format=-1.6]}
                    \toprule
                    \(n\)  &  {\(a_n\)}  &  {\(b_n\)}  &  {\(p_n\)}  &  {\(f(p_n)\)}  \\
                    \midrule
                        1  &  0          &  1          &  0.5        &  -0.625        \\
                        2  &  0.5        &  1          &  0.75       &   0.984375     \\
                        3  &  0.5        &  0.75       &  0.625      &   0.259766     \\
                        4  &  0.5        &  0.625      &  0.5625     &  -0.161865     \\
                        5  &  0.5625     &  0.625      &  0.59375    &   0.054047     \\
                        6  &  0.5625     &  0.59375    &  0.578125   &  -0.052624     \\
                        7  &  0.578125   &  0.59375    &  0.5859375  &   0.001031     \\
                    \bottomrule
                \end{tabular}
            \end{table}

            So \(p \approx \num{0.5859}\).

        \item \(f(1) = 2\) and \(f(\num{3.2}) = \num{-0.112}\) have the opposite
            signs, so there's a root in \(\interval{1}{\num{3.2}}\).

            The number of iteration \(n\) needed to approximate \(p\) to within
            \(10^{-2}\) is:

            \[\abs{p_n - p} \leq \frac{\num{3.2} - 1}{2^n} < 10^{-2} \iff n \geq 8\]

            Applying Bisection method generates the following table:

            \begin{table}[H]
                \centering
                \begin{tabular}{r S[table-format=1.5] S[table-format=1.6] S[table-format=1.6] S[table-format=-1.6]}
                    \toprule
                    \(n\)  &  {\(a_n\)}  &  {\(b_n\)}  &  {\(p_n\)}  &  {\(f(p_n)\)}  \\
                    \midrule
                        1  &  1          &  3.2        &  2.1        &   1.791        \\
                        2  &  2.1        &  3.2        &  2.65       &   0.552125     \\
                        3  &  2.65       &  3.2        &  2.925      &   0.085828     \\
                        4  &  2.925      &  3.2        &  3.0625     &  -0.054443     \\
                        5  &  2.925      &  3.0625     &  2.99375    &   0.006328     \\
                        6  &  2.99375    &  3.0625     &  3.028125   &  -0.026521     \\
                        7  &  2.99375    &  3.02813    &  3.010938   &  -0.010697     \\
                        8  &  2.99375    &  3.010938   &  3.002344   &  -0.002333     \\
                    \bottomrule
                \end{tabular}
            \end{table}

            So \(p \approx \num{3.0023}\).

        \item \(f(\num{3.2}) = \num{-0.112}\) and \(f(4) = 2\) have the opposite
            signs, so there's a root in \(\interval{\num{3.2}}{4}\).

            The number of iteration \(n\) needed to approximate \(p\) to within
            \(10^{-2}\) is:

            \[\abs{p_n - p} \leq \frac{4 - \num{3.2}}{2^n} < 10^{-2} \iff n \geq 7\]

            Applying Bisection method generates the following table:

            \begin{table}[H]
                \centering
                \begin{tabular}{r S[table-format=1.5] S[table-format=1.6] S[table-format=1.6] S[table-format=-1.6]}
                    \toprule
                    \(n\)  &  {\(a_n\)}  &  {\(b_n\)}  &  {\(p_n\)}  &  {\(f(p_n)\)}  \\
                    \midrule
                        1  &  3.2        &  4          &  3.6        &   0.336        \\
                        2  &  3.2        &  3.6        &  3.4        &  -0.016        \\
                        3  &  3.4        &  3.6        &  3.5        &   0.125        \\
                        4  &  3.4        &  3.5        &  3.45       &   0.046125     \\
                        5  &  3.4        &  3.45       &  3.425      &   0.013016     \\
                        6  &  3.4        &  3.425      &  3.4125     &  -0.001998     \\
                        7  &  3.4125     &  3.425      &  3.41875    &   0.005382     \\
                    \bottomrule
                \end{tabular}
            \end{table}

            So \(p \approx \num{3.4188}\).
    \end{enumerate}
\end{solution}

\begin{exercise}
    Use the Bisection method to find solutions accurate to within \(10^{-2}\)
    for \(x^4 - 2x^3 - 4x^2 + 4x + 4 = 0\) for the following intervals:

    \begin{multicols}{4}
        \begin{enumerate}[label = (\alph*)]
            \item \(\interval{-2}{-1}\)
            \item \(\interval{0}{2}\)
            \item \(\interval{2}{3}\)
            \item \(\interval{-1}{0}\)
        \end{enumerate}
    \end{multicols}
\end{exercise}

\begin{solution}
    \begin{enumerate}[label = (\alph*)]
        \item \(f(-2) = 12\) and \(f(-1) = -1\) have the opposite signs, so
            there's a root in \(\interval{-2}{-1}\).

            The number of iteration \(n\) needed to approximate \(p\) to within
            \(10^{-2}\) is:

            \[\abs{p_n - p} \leq \frac{-1 - (-2)}{2^n} < 10^{-2} \iff n \geq 7\]

            Applying Bisection method generates the following table:

            \begin{table}[H]
                \centering
                \begin{tabular}{r S[table-format=-1.6] S[table-format=-1.5] S[table-format=-1.6] S[table-format=-1.6]}
                    \toprule
                    \(n\)  &  {\(a_n\)}  &  {\(b_n\)}  &  {\(p_n\)}  &  {\(f(p_n)\)}  \\
                    \midrule
                        1  &  -2         &  -1         &  -1.5       &   0.8125       \\
                        2  &  -1.5       &  -1         &  -1.25      &  -0.902344     \\
                        3  &  -1.5       &  -1.25      &  -1.375     &  -0.288818     \\
                        4  &  -1.5       &  -1.375     &  -1.4375    &   0.195328     \\
                        5  &  -1.4375    &  -1.375     &  -1.40625   &  -0.062667     \\
                        6  &  -1.4375    &  -1.40625   &  -1.421875  &   0.062263     \\
                        7  &  -1.421875  &  -1.40625   &  -1.414063  &  -0.001208     \\
                    \bottomrule
                \end{tabular}
            \end{table}

            So \(p \approx \num{-1.4141}\).

        \item \(f(0) = 4\) and \(f(2) = -4\) have the opposite signs, so there's
            a root in \(\interval{0}{2}\).

            The number of iteration \(n\) needed to approximate \(p\) to within
            \(10^{-2}\) is:

            \[\abs{p_n - p} \leq \frac{2 - 0}{2^n} < 10^{-2} \iff n \geq 8\]

            Applying Bisection method generates the following table:

            \begin{table}[H]
                \centering
                \begin{tabular}{r S[table-format=1.5] S[table-format=1.6] S[table-format=1.6] S[table-format=-1.6]}
                    \toprule
                    \(n\)  &  {\(a_n\)}  &  {\(b_n\)}  &  {\(p_n\)}  &  {\(f(p_n)\)}  \\
                    \midrule
                        1  &  0          &  2          &  1          &   3            \\
                        2  &  1          &  2          &  1.5        &  -0.6875       \\
                        3  &  1          &  1.5        &  1.25       &   1.285156     \\
                        4  &  1.25       &  1.5        &  1.375      &   0.312744     \\
                        5  &  1.375      &  1.5        &  1.4375     &  -0.186508     \\
                        6  &  1.375      &  1.4375     &  1.40625    &   0.063676     \\
                        7  &  1.40625    &  1.4375     &  1.421875   &  -0.061318     \\
                        8  &  1.40625    &  1.421875   &  1.414063   &   0.001208     \\
                    \bottomrule
                \end{tabular}
            \end{table}

            So \(p \approx \num{1.4141}\).

        \item \(f(2) = -4\) and \(f(3) = 7\) have the opposite signs, so there's
            a root in \(\interval{2}{3}\).

            The number of iteration \(n\) needed to approximate \(p\) to within
            \(10^{-2}\) is:

            \[\abs{p_n - p} \leq \frac{3 - 2}{2^n} < 10^{-2} \iff n \geq 7\]

            Applying Bisection method generates the following table:

            \begin{table}[H]
                \centering
                \begin{tabular}{r S[table-format=1.5] S[table-format=1.6] S[table-format=1.6] S[table-format=-1.6]}
                    \toprule
                    \(n\)  &  {\(a_n\)}  &  {\(b_n\)}  &  {\(p_n\)}  &  {\(f(p_n)\)}  \\
                    \midrule
                        1  &  2          &  3          &  2.5        &  -3.1875       \\
                        2  &  2.5        &  3          &  2.75       &   0.347656     \\
                        3  &  2.5        &  2.75       &  2.625      &  -1.757568     \\
                        4  &  2.625      &  2.75       &  2.6875     &  -0.795639     \\
                        5  &  2.6875     &  2.75       &  2.71875    &  -0.247466     \\
                        6  &  2.71875    &  2.75       &  2.734375   &   0.044125     \\
                        7  &  2.71875    &  2.734375   &  2.726563   &  -0.103151     \\
                    \bottomrule
                \end{tabular}
            \end{table}

            So \(p \approx \num{2.7266}\).

        \item \(f(-1) = -1\) and \(f(0) = 4\) have the opposite signs, so
            there's a root in \(\interval{-1}{0}\).

            The number of iteration \(n\) needed to approximate \(p\) to within
            \(10^{-2}\) is:

            \[\abs{p_n - p} \leq \frac{0 - (-1)}{2^n} < 10^{-2} \iff n \geq 7\]

            Applying Bisection method generates the following table:

            \begin{table}[H]
                \centering
                \begin{tabular}{r S[table-format=-1.6] S[table-format=-1.5] S[table-format=-1.6] S[table-format=-1.6]}
                    \toprule
                    \(n\)  &  {\(a_n\)}  &  {\(b_n\)}  &  {\(p_n\)}  &  {\(f(p_n)\)}  \\
                    \midrule
                        1  &  -1         &  0          &  -0.5       &   1.3125       \\
                        2  &  -1         &  -0.5       &  -0.75      &  -0.089844     \\
                        3  &  -0.75      &  -0.5       &  -0.625     &   0.578369     \\
                        4  &  -0.75      &  -0.625     &  -0.6875    &   0.232681     \\
                        5  &  -0.75      &  -0.6875    &  -0.71875   &   0.068086     \\
                        6  &  -0.75      &  -0.71875   &  -0.734375  &  -0.011768     \\
                        7  &  -0.734375  &  -0.71875   &  -0.726563  &   0.027943     \\
                    \bottomrule
                \end{tabular}
            \end{table}

            So \(p \approx \num{-0.7266}\).
    \end{enumerate}
\end{solution}

\begin{exercise}
    Use the Bisection method to find solutions accurate to within \(10^{-5}\)
    for the following problems:

    \begin{enumerate}[label = (\alph*)]
        \item \(x - 2^{-x} = 0\), \(x \in \interval{0}{1}\)
        \item \(e^x - x^2 + 3x - 2 = 0\), \(x \in \interval{0}{1}\)
        \item \(2 x \cos{2x} - (x + 1)^2 = 0\), \(x \in \interval{-3}{-2}\)
        \item \(x \cos{x} - 2x^2 + 3x - 1 = 0\), \(x \in \interval{\num{0.2}}{\num{0.3}}\)
    \end{enumerate}
\end{exercise}

\begin{solution}
    \begin{enumerate}[label = (\alph*)]
        \item \(f(0) = -1\) and \(f(1) = \num{0.5}\) have the opposite signs, so
            there's a root in \(\interval{0}{1}\).

            The number of iteration \(n\) needed to approximate \(p\) to within
            \(10^{-5}\) is:

            \[\abs{p_n - p} \leq \frac{1 - 0}{2^n} < 10^{-5} \iff n \geq 17\]

            Applying Bisection method generates the following table:

            \begin{longtable}{r S[table-format=1.9] S[table-format=1.9] S[table-format=1.9] S[table-format=-1.9]}
                \toprule
                \(n\)  &   {\(a_n\)}   &   {\(b_n\)}   &   {\(p_n\)}   &  {\(f(p_n)\)}  \\
                \midrule
                \endfirsthead
                \(n\)  &   {\(a_n\)}   &   {\(b_n\)}   &   {\(p_n\)}   &  {\(f(p_n)\)}  \\
                \midrule
                \endhead
                    1  &  0            &  1            &  0.5          &  -0.207106781  \\
                    2  &  0.5          &  1            &  0.75         &   0.155396442  \\
                    3  &  0.5          &  0.75         &  0.625        &  -0.023419777  \\
                    4  &  0.625        &  0.75         &  0.6875       &   0.066571094  \\
                    5  &  0.625        &  0.6875       &  0.65625      &   0.021724521  \\
                    6  &  0.625        &  0.65625      &  0.640625     &  -0.000810008  \\
                    7  &  0.640625     &  0.65625      &  0.6484375    &   0.010466611  \\
                    8  &  0.640625     &  0.6484375    &  0.64453125   &   0.004830646  \\
                    9  &  0.640625     &  0.64453125   &  0.642578125  &   0.002010906  \\
                   10  &  0.640625     &  0.642578125  &  0.641601562  &   0.000600596  \\
                   11  &  0.640625     &  0.641601562  &  0.641113281  &  -0.000104669  \\
                   12  &  0.641113281  &  0.641601562  &  0.641357422  &   0.000247972  \\
                   13  &  0.641113281  &  0.641357422  &  0.641235352  &   0.000071654  \\
                   14  &  0.641113281  &  0.641235352  &  0.641174316  &  -0.000016507  \\
                   15  &  0.641174316  &  0.641235352  &  0.641204834  &   0.000027573  \\
                   16  &  0.641174316  &  0.641204834  &  0.641189575  &   0.000005533  \\
                   17  &  0.641174316  &  0.641189575  &  0.641181946  &  -0.000005487  \\
                \bottomrule
            \end{longtable}

            So \(p \approx \num{-0.641182}\).

        \item \(f(0) = -1\) and \(f(1) = e\) have the opposite signs, so there's
            a root in \(\interval{0}{1}\).

            The number of iteration \(n\) needed to approximate \(p\) to within
            \(10^{-5}\) is:

            \[\abs{p_n - p} \leq \frac{1 - 0}{2^n} < 10^{-5} \iff n \geq 17\]

            Applying Bisection method generates the following table:

            \begin{longtable}{r S[table-format=1.9] S[table-format=1.9] S[table-format=1.9] S[table-format=-1.9]}
                \toprule
                \(n\)  &   {\(a_n\)}   &   {\(b_n\)}   &   {\(p_n\)}   &  {\(f(p_n)\)}  \\
                \midrule
                \endfirsthead
                \(n\)  &   {\(a_n\)}   &   {\(b_n\)}   &   {\(p_n\)}   &  {\(f(p_n)\)}  \\
                \midrule
                \endhead
                    1  &  0            &  1            &  0.5          &   0.898721271  \\
                    2  &  0            &  0.5          &  0.25         &  -0.028474583  \\
                    3  &  0.25         &  0.5          &  0.375        &   0.439366415  \\
                    4  &  0.25         &  0.375        &  0.3125       &   0.206681691  \\
                    5  &  0.25         &  0.3125       &  0.28125      &   0.089433196  \\
                    6  &  0.25         &  0.28125      &  0.265625     &   0.030564234  \\
                    7  &  0.25         &  0.265625     &  0.2578125    &   0.001066368  \\
                    8  &  0.25         &  0.2578125    &  0.25390625   &  -0.013698684  \\
                    9  &  0.25390625   &  0.2578125    &  0.255859375  &  -0.006314807  \\
                   10  &  0.255859375  &  0.2578125    &  0.256835938  &  -0.002623882  \\
                   11  &  0.256835938  &  0.2578125    &  0.257324219  &  -0.000778673  \\
                   12  &  0.257324219  &  0.2578125    &  0.257568359  &   0.000143868  \\
                   13  &  0.257324219  &  0.257568359  &  0.257446289  &  -0.000317397  \\
                   14  &  0.257446289  &  0.257568359  &  0.257507324  &  -0.000086763  \\
                   15  &  0.257507324  &  0.257568359  &  0.257537842  &   0.000028553  \\
                   16  &  0.257507324  &  0.257537842  &  0.257522583  &  -0.000029105  \\
                   17  &  0.257522583  &  0.257537842  &  0.257530212  &  -0.000000276  \\
                \bottomrule
            \end{longtable}

            So \(p \approx \num{0.25753}\).

        \item \(f(-3) \approx \num{-9.76102172}\) and \(f(-2) \approx
            \num{1.614574483}\) have the opposite signs, so there's a root in
            \(\interval{-3}{-2}\).

            The number of iteration \(n\) needed to approximate \(p\) to within
            \(10^{-5}\) is:

            \[\abs{p_n - p} \leq \frac{-2 - (-3)}{2^n} < 10^{-5} \iff n \geq 17\]

            Applying Bisection method generates the following table:

            \begin{longtable}{r S[table-format=-1.8] S[table-format=-1.8] S[table-format=-1.8] S[table-format=-1.9]}
                \toprule
                \(n\)  &   {\(a_n\)}   &   {\(b_n\)}   &   {\(p_n\)}   &  {\(f(p_n)\)}  \\
                \midrule
                \endfirsthead
                \(n\)  &   {\(a_n\)}   &   {\(b_n\)}   &   {\(p_n\)}   &  {\(f(p_n)\)}  \\
                \midrule
                \endhead
                    1  &  -3           &  -2           &  -2.5         &  -3.66831093   \\
                    2  &  -2.5         &  -2           &  -2.25        &  -0.613918903  \\
                    3  &  -2.25        &  -2           &  -2.125       &   0.630246832  \\
                    4  &  -2.25        &  -2.125       &  -2.1875      &   0.038075532  \\
                    5  &  -2.25        &  -2.1875      &  -2.21875     &  -0.280836176  \\
                    6  &  -2.21875     &  -2.1875      &  -2.203125    &  -0.119556815  \\
                    7  &  -2.203125    &  -2.1875      &  -2.1953125   &  -0.040278514  \\
                    8  &  -2.1953125   &  -2.1875      &  -2.19140625  &  -0.000985195  \\
                    9  &  -2.19140625  &  -2.1875      &  -2.18945312  &   0.018574337  \\
                   10  &  -2.19140625  &  -2.18945312  &  -2.19042969  &   0.008801851  \\
                   11  &  -2.19140625  &  -2.19042969  &  -2.19091797  &   0.003910147  \\
                   12  &  -2.19140625  &  -2.19091797  &  -2.19116211  &   0.00146293   \\
                   13  &  -2.19140625  &  -2.19116211  &  -2.19128418  &   0.000238981  \\
                   14  &  -2.19140625  &  -2.19128418  &  -2.19134521  &  -0.000373078  \\
                   15  &  -2.19134521  &  -2.19128418  &  -2.1913147   &  -0.000067041  \\
                   16  &  -2.1913147   &  -2.19128418  &  -2.19129944  &   0.000085972  \\
                \bottomrule
            \end{longtable}

            So \(p \approx \num{-2.191299}\).

        \item \(f(\num{0.2}) \approx \num{-0.283986684}\) and \(f(\num{0.3})
            \approx \num{0.006600946}\) have the opposite signs, so there's a
            root in \(\interval{\num{0.2}}{\num{0.3}}\).

            The number of iteration \(n\) needed to approximate \(p\) to
            within \(10^{-5}\) is:

            \[\abs{p_n - p} \leq \frac{\num{0.3} - \num{0.2}}{2^n} < 10^{-5} \iff n \geq 14\]

            Applying Bisection method generates the following table:

            \begin{longtable}{r S[table-format=1.9] S[table-format=1.9] S[table-format=1.9] S[table-format=-1.9]}
                \toprule
                \(n\)  &   {\(a_n\)}   &   {\(b_n\)}   &   {\(p_n\)}   &  {\(f(p_n)\)}  \\
                \midrule
                \endfirsthead
                \(n\)  &   {\(a_n\)}   &   {\(b_n\)}   &   {\(p_n\)}   &  {\(f(p_n)\)}  \\
                \midrule
                \endhead
                    1  &  0.2          &  0.3          &  0.25         &  -0.132771895  \\
                    2  &  0.25         &  0.3          &  0.275        &  -0.061583071  \\
                    3  &  0.275        &  0.3          &  0.2875       &  -0.027112719  \\
                    4  &  0.2875       &  0.3          &  0.29375      &  -0.010160959  \\
                    5  &  0.29375      &  0.3          &  0.296875     &  -0.001756232  \\
                    6  &  0.296875     &  0.3          &  0.2984375    &   0.002428306  \\
                    7  &  0.296875     &  0.2984375    &  0.29765625   &   0.000337524  \\
                    8  &  0.296875     &  0.29765625   &  0.297265625  &  -0.000708983  \\
                    9  &  0.297265625  &  0.29765625   &  0.297460938  &  -0.000185637  \\
                   10  &  0.297460938  &  0.29765625   &  0.297558594  &   0.000075967  \\
                   11  &  0.297460938  &  0.297558594  &  0.297509766  &  -0.000054829  \\
                   12  &  0.297509766  &  0.297558594  &  0.29753418   &   0.00001057   \\
                   13  &  0.297509766  &  0.29753418   &  0.297521973  &  -0.000022129  \\
                   14  &  0.297521973  &  0.29753418   &  0.297528076  &  -0.000005779  \\
               \bottomrule
            \end{longtable}

            So \(p \approx \num{0.297528}\).
    \end{enumerate}
\end{solution}

\begin{exercise}
    Use the Bisection method to find solutions accurate to within \(10^{-5}\)
    for the following problems:

    \begin{multicols}{2}
        \begin{enumerate}[label = (\alph*)]
            \item \(3x - e^x = 0\), \(x \in \interval{1}{2}\)
            \item \(2x + 3 \cos{x} -e^x = 0\), \(x \in \interval{0}{1}\)
            \item \(x^2 - 4x + 4 - \ln{x} = 0\), \(x \in \interval{1}{2}\)
            \item \(x + 1 - 2 \sin{\pi x} = 0\), \(x \in \interval{0}{\num{0.5}}\)
        \end{enumerate}
    \end{multicols}
\end{exercise}

\begin{solution}
    \begin{enumerate}[label = (\alph*)]
        \item \(f(1) \approx \num{0.281718172}\) and \(f(2) \approx
            \num{-1.389056099}\) have the opposite signs, so there's a root in
            \(\interval{1}{2}\).

            The number of iteration \(n\) needed to approximate \(p\) to within
            \(10^{-5}\) is:

            \[\abs{p_n - p} \leq \frac{2 - 1}{2^n} < 10^{-5} \iff n \geq 17\]

            Applying Bisection method generates the following table:

            \begin{longtable}{r S[table-format=1.8] S[table-format=1.8] S[table-format=1.8] S[table-format=-1.9]}
                \toprule
                \(n\)  &   {\(a_n\)}   &   {\(b_n\)}   &   {\(p_n\)}   &  {\(f(p_n)\)}  \\
                \midrule
                \endfirsthead
                \(n\)  &   {\(a_n\)}   &   {\(b_n\)}   &   {\(p_n\)}   &  {\(f(p_n)\)}  \\
                \midrule
                \endhead
                    1  &  1            &  2            &  1.5          &   0.01831093   \\
                    2  &  1.5          &  2            &  1.75         &  -0.504602676  \\
                    3  &  1.5          &  1.75         &  1.625        &  -0.203419037  \\
                    4  &  1.5          &  1.625        &  1.5625       &  -0.083233182  \\
                    5  &  1.5          &  1.5625       &  1.53125      &  -0.030203153  \\
                    6  &  1.5          &  1.53125      &  1.515625     &  -0.005390404  \\
                    7  &  1.5          &  1.515625     &  1.5078125    &   0.006598107  \\
                    8  &  1.5078125    &  1.515625     &  1.51171875   &   0.000638447  \\
                    9  &  1.51171875   &  1.515625     &  1.51367188   &  -0.002367313  \\
                   10  &  1.51171875   &  1.51367188   &  1.51269531   &  -0.000862268  \\
                   11  &  1.51171875   &  1.51269531   &  1.51220703   &  -0.00011137   \\
                   12  &  1.51171875   &  1.51220703   &  1.51196289   &   0.000263674  \\
                   13  &  1.51196289   &  1.51220703   &  1.51208496   &   0.000076186  \\
                   14  &  1.51208496   &  1.51220703   &  1.512146     &  -0.000017584  \\
                   15  &  1.51208496   &  1.512146     &  1.51211548   &   0.000029303  \\
                   16  &  1.51211548   &  1.512146     &  1.51213074   &   0.00000586   \\
                   17  &  1.51213074   &  1.512146     &  1.51213837   &  -0.000005861  \\
                \bottomrule
            \end{longtable}

            So \(p \approx \num{1.512138}\).

        \item \(f(0) = 2\) and \(f(1) \approx \num{0.902625089}\) have the same
            sign, so there's no root in \(\interval{0}{1}\).

        \item \(f(1) = 1\) and \(f(2) = \num{-0.693147181}\) have the opposite
            signs, so there's a root in \(\interval{1}{2}\).

            The number of iteration \(n\) needed to approximate \(p\) to within
            \(10^{-5}\) is:

            \[\abs{p_n - p} \leq \frac{2 - 1}{2^n} < 10^{-5} \iff n \geq 17\]

            Applying Bisection method generates the following table:

            % \begin{longtable}{r S[table-format=1.8] S[table-format=1.8] S[table-format=1.8] S[table-format=-1.9]}
            %     \toprule
            %     \(n\)  &   {\(a_n\)}   &   {\(b_n\)}   &   {\(p_n\)}   &  {\(f(p_n)\)}  \\
            %     \midrule
            %     \endfirsthead
            %     \(n\)  &   {\(a_n\)}   &   {\(b_n\)}   &   {\(p_n\)}   &  {\(f(p_n)\)}  \\
            %     \midrule
            %     \endhead
            %         1  &  1            &  2            &  1.5          &  -0.155465108  \\
            %         2  &  1            &  1.5          &  1.25         &   0.339356449  \\
            %         3  &  1.25         &  1.5          &  1.375        &   0.072171269  \\
            %         4  &  1.375        &  1.5          &  1.4375       &  -0.046499244  \\
            %         5  &  1.375        &  1.4375       &  1.40625      &   0.011612476  \\
            %         6  &  1.40625      &  1.4375       &  1.421875     &  -0.017747908  \\
            %         7  &  1.40625      &  1.421875     &  1.4140625    &  -0.003144013  \\
            %         8  &  1.40625      &  1.4140625    &  1.41015625   &   0.004215136  \\
            %         9  &  1.41015625   &  1.4140625    &  1.41210938   &   0.00053079   \\
            %        10  &  1.41210938   &  1.4140625    &  1.41308594   &  -0.001307804  \\
            %        11  &  1.41210938   &  1.41308594   &  1.41259766   &  -0.000388805  \\
            %        12  &  1.41210938   &  1.41259766   &  1.41235352   &   0.000070918  \\
            %        13  &  1.41235352   &  1.41259766   &  1.41247559   &  -0.000158962  \\
            %        14  &  1.41235352   &  1.41247559   &  1.41241455   &  -0.000044027  \\
            %        15  &  1.41235352   &  1.41241455   &  1.41238403   &   0.000013444  \\
            %        16  &  1.41238403   &  1.41241455   &  1.41239929   &  -0.000015292  \\
            %        17  &  1.41238403   &  1.41239929   &  1.41239166   &  -0.000000924  \\
            %     \bottomrule
            % \end{longtable}

            \begin{table}[H]
                \centering
                \begin{tabular}{r S[table-format=1.8] S[table-format=1.8] S[table-format=1.8] S[table-format=-1.9]}
                    \toprule
                    \(n\)  &   {\(a_n\)}   &   {\(b_n\)}   &   {\(p_n\)}   &  {\(f(p_n)\)}  \\
                    \midrule
                        1  &  1            &  2            &  1.5          &  -0.155465108  \\
                        2  &  1            &  1.5          &  1.25         &   0.339356449  \\
                        3  &  1.25         &  1.5          &  1.375        &   0.072171269  \\
                        4  &  1.375        &  1.5          &  1.4375       &  -0.046499244  \\
                        5  &  1.375        &  1.4375       &  1.40625      &   0.011612476  \\
                        6  &  1.40625      &  1.4375       &  1.421875     &  -0.017747908  \\
                        7  &  1.40625      &  1.421875     &  1.4140625    &  -0.003144013  \\
                        8  &  1.40625      &  1.4140625    &  1.41015625   &   0.004215136  \\
                        9  &  1.41015625   &  1.4140625    &  1.41210938   &   0.00053079   \\
                       10  &  1.41210938   &  1.4140625    &  1.41308594   &  -0.001307804  \\
                       11  &  1.41210938   &  1.41308594   &  1.41259766   &  -0.000388805  \\
                       12  &  1.41210938   &  1.41259766   &  1.41235352   &   0.000070918  \\
                       13  &  1.41235352   &  1.41259766   &  1.41247559   &  -0.000158962  \\
                       14  &  1.41235352   &  1.41247559   &  1.41241455   &  -0.000044027  \\
                       15  &  1.41235352   &  1.41241455   &  1.41238403   &   0.000013444  \\
                       16  &  1.41238403   &  1.41241455   &  1.41239929   &  -0.000015292  \\
                       17  &  1.41238403   &  1.41239929   &  1.41239166   &  -0.000000924  \\
                    \bottomrule
                \end{tabular}
            \end{table}

            So \(p \approx \num{1.412392}\).

        \item \(f(0) = 1\) and \(f(1) = \num{-0.5}\) have the opposite signs, so
            there's a root in \(\interval{0}{\num{0.5}}\).

            The number of iteration \(n\) needed to approximate \(p\) to within
            \(10^{-5}\) is:

            \[\abs{p_n - p} \leq \frac{\num{0.5} - 0}{2^n} < 10^{-5} \iff n \geq 16\]

            Applying Bisection method generates the following table:

            \begin{longtable}{r S[table-format=1.9] S[table-format=1.9] S[table-format=1.9] S[table-format=-1.9]}
                \toprule
                \(n\)  &   {\(a_n\)}   &   {\(b_n\)}   &   {\(p_n\)}   &  {\(f(p_n)\)}  \\
                \midrule
                \endfirsthead
                \(n\)  &   {\(a_n\)}   &   {\(b_n\)}   &   {\(p_n\)}   &  {\(f(p_n)\)}  \\
                \midrule
                \endhead
                    1  &  0            &  0.5          &  0.25         &  -0.164213562  \\
                    2  &  0            &  0.25         &  0.125        &   0.359633135  \\
                    3  &  0.125        &  0.25         &  0.1875       &   0.076359534  \\
                    4  &  0.1875       &  0.25         &  0.21875      &  -0.050036568  \\
                    5  &  0.1875       &  0.21875      &  0.203125     &   0.011726391  \\
                    6  &  0.203125     &  0.21875      &  0.2109375    &  -0.019525681  \\
                    7  &  0.203125     &  0.2109375    &  0.20703125   &  -0.003990833  \\
                    8  &  0.203125     &  0.20703125   &  0.205078125  &   0.003845166  \\
                    9  &  0.205078125  &  0.20703125   &  0.206054688  &  -0.00007851   \\
                    10  &  0.205078125  &  0.206054688  &  0.205566406  &   0.001881912  \\
                    11  &  0.205566406  &  0.206054688  &  0.205810547  &   0.000901347  \\
                    12  &  0.205810547  &  0.206054688  &  0.205932617  &   0.00041133   \\
                    13  &  0.205932617  &  0.206054688  &  0.205993652  &   0.000166388  \\
                    14  &  0.205993652  &  0.206054688  &  0.20602417   &   0.000043934  \\
                    15  &  0.20602417   &  0.206054688  &  0.206039429  &  -0.000017289  \\
                    16  &  0.20602417   &  0.206039429  &  0.206031799  &   0.000013322  \\
                    \bottomrule
            \end{longtable}

            So \(p \approx \num{0.206032}\).
    \end{enumerate}
\end{solution}

\begin{exercise}
    \begin{enumerate}[label = (\alph*)]
        \item Sketch the graphs of \(y = x\) and \(y = 2 \sin{x}\).
        \item Use the Bisection method to find an approximation to within
            \(10^{−5}\) to the first positive value of \(x\) with \(x = 2
            \sin{x}\).
    \end{enumerate}
\end{exercise}

\begin{solution}
    \begin{enumerate}[label = (\alph*)]
        \item Graph of \(y = x\) and \(y = 2 \sin{x}\) is as follow:

            \begin{center}
                \subfile{graphics/exercise_7_graph/exercise_7_graph.tex}
            \end{center}

        \item According to the graph, the first positive root \(p\) of \(f = x -
            2 \sin{x}\) is in \(\interval{\dfrac{\pi}{2}}{\pi}\).

            The number of iteration \(n\) needed to approximate \(p\) to within
            \(10^{-5}\) in that interval is:

            \[\abs{p_n - p} \leq \frac{\pi - \dfrac{\pi}{2}}{2^n} < 10^{-5} \iff n \geq 18\]

            Applying Bisection method generates the following table:

            \begin{longtable}{r S[table-format=1.8] S[table-format=1.8] S[table-format=1.8] S[table-format=-1.9]}
                \toprule
                \(n\)  &   {\(a_n\)}   &   {\(b_n\)}   &   {\(p_n\)}   &  {\(f(p_n)\)}  \\
                \midrule
                \endfirsthead
                \(n\)  &   {\(a_n\)}   &   {\(b_n\)}   &   {\(p_n\)}   &  {\(f(p_n)\)}  \\
                \midrule
                \endhead
                    1  &  1.57079633   &  3.14159265   &  2.35619449   &   0.941980928  \\
                    2  &  1.57079633   &  2.35619449   &  1.96349541   &   0.115736343  \\
                    3  &  1.57079633   &  1.96349541   &  1.76714587   &  -0.194424693  \\
                    4  &  1.76714587   &  1.96349541   &  1.86532064   &  -0.048560033  \\
                    5  &  1.86532064   &  1.96349541   &  1.91440802   &   0.031319893  \\
                    6  &  1.86532064   &  1.91440802   &  1.88986433   &  -0.009192031  \\
                    7  &  1.88986433   &  1.91440802   &  1.90213618   &   0.010921526  \\
                    8  &  1.88986433   &  1.90213618   &  1.89600025   &   0.000829072  \\
                    9  &  1.88986433   &  1.89600025   &  1.89293229   &  -0.004190408  \\
                   10  &  1.89293229   &  1.89600025   &  1.89446627   &  -0.001682899  \\
                   11  &  1.89446627   &  1.89600025   &  1.89523326   &  -0.000427471  \\
                   12  &  1.89523326   &  1.89600025   &  1.89561676   &   0.000200661  \\
                   13  &  1.89523326   &  1.89561676   &  1.89542501   &  -0.00011344   \\
                   14  &  1.89542501   &  1.89561676   &  1.89552088   &   0.000043602  \\
                   15  &  1.89542501   &  1.89552088   &  1.89547295   &  -0.000034921  \\
                   16  &  1.89547295   &  1.89552088   &  1.89549692   &   0.00000434   \\
                   17  &  1.89547295   &  1.89549692   &  1.89548493   &  -0.000015291  \\
                   18  &  1.89548493   &  1.89549692   &  1.89549092   &  -0.000005476  \\
                \bottomrule
            \end{longtable}

            So \(p \approx \num{1.895491}\).
    \end{enumerate}
\end{solution}

\begin{exercise}
    \begin{enumerate}[label = (\alph*)]
        \item Sketch the graphs of \(y = x\) and \(y = \tan{x}\).
        \item Use the Bisection method to find an approximation to within
            \(10^{-5}\) to the first positive value of \(x\) with \(y =
            \tan{x}\).
    \end{enumerate}
\end{exercise}

\begin{solution}
    \begin{enumerate}[label = (\alph*)]
        \item Graph of \(y = x\) and \(y = \tan{x}\) is as follow:

            \begin{center}
                \subfile{graphics/exercise_8_graph/exercise_8_graph.tex}
            \end{center}

        \item According to the graph, the first positive root \(p\) of \(f = x -
            \tan{x}\) is in \(\interval{\pi}{\dfrac{3 \pi}{2}}\).

            The number of iteration \(n\) needed to approximate \(p\) to within
            \(10^{-5}\) in that interval is:

            \[\abs{p_n - p} \leq \frac{\frac{3 \pi}{2} - \pi}{2^n} < 10^{-5} \iff n \geq 18\]

            Applying Bisection method generates the following table:

            \begin{longtable}{r S[table-format=1.8] S[table-format=1.8] S[table-format=1.8] S[table-format=-1.9]}
                \toprule
                \(n\)  &   {\(a_n\)}   &   {\(b_n\)}   &   {\(p_n\)}   &  {\(f(p_n)\)}  \\
                \midrule
                \endfirsthead
                \(n\)  &   {\(a_n\)}   &   {\(b_n\)}   &   {\(p_n\)}   &  {\(f(p_n)\)}  \\
                \midrule
                \endhead
                    1  &  3.14159265   &  4.71238898   &  3.92699082   &   2.92699082   \\
                    2  &  3.92699082   &  4.71238898   &  4.3196899    &   1.90547634   \\
                    3  &  4.3196899    &  4.71238898   &  4.51603944   &  -0.511300053  \\
                    4  &  4.3196899    &  4.51603944   &  4.41786467   &   1.12130646   \\
                    5  &  4.41786467   &  4.51603944   &  4.46695205   &   0.474728271  \\
                    6  &  4.46695205   &  4.51603944   &  4.49149575   &   0.038293523  \\
                    7  &  4.49149575   &  4.51603944   &  4.50376759   &  -0.219861735  \\
                    8  &  4.49149575   &  4.50376759   &  4.49763167   &  -0.086980389  \\
                    9  &  4.49149575   &  4.49763167   &  4.49456371   &  -0.023432692  \\
                   10  &  4.49149575   &  4.49456371   &  4.49302973   &   0.007653323  \\
                   11  &  4.49302973   &  4.49456371   &  4.49379672   &  -0.007833371  \\
                   12  &  4.49302973   &  4.49379672   &  4.49341322   &  -0.00007602   \\
                   13  &  4.49302973   &  4.49341322   &  4.49322148   &   0.003792144  \\
                   14  &  4.49322148   &  4.49341322   &  4.49331735   &   0.001858936  \\
                   15  &  4.49331735   &  4.49341322   &  4.49336529   &   0.000891677  \\
                   16  &  4.49336529   &  4.49341322   &  4.49338925   &   0.000407883  \\
                   17  &  4.49338925   &  4.49341322   &  4.49340124   &   0.000165946  \\
                   18  &  4.49340124   &  4.49341322   &  4.49340723   &   0.000044966  \\
                \bottomrule
            \end{longtable}

            So \(p \approx \num{4.493407}\).
    \end{enumerate}
\end{solution}

\begin{exercise}
    \begin{enumerate}[label = (\alph*)]
        \item Sketch the graphs of \(y = e^x - 2\) and \(y = \cos{e^x - 2}\).
        \item Use the Bisection method to find an approximation to within
            \(10^{-5}\) to a value in \(\interval{\num{0.5}}{\num{1.5}}\) with
            \(e^x - 2 = \cos{e^x - 2}\).
    \end{enumerate}
\end{exercise}

\begin{solution}
    \begin{enumerate}[label = (\alph*)]
        \item The graphs of the 2 functions are as follow:

            \begin{center}
                \subfile{graphics/exercise_9_graph/exercise_9_graph.tex}
            \end{center}

        \item Let \(f = e^x - 2 - \cos{e^x - 2}\). \(f(\num{0.5}) \approx
            \num{-1.290212} \) and \(f(\num{1.5}) \approx \num{3.27174}\) have
            the opposite signs, so there's a root \(p\) of \(f\) in
            \(\interval{\num{0.5}}{\num{1.5}}\).

            The number of iteration \(n\) needed to approximate \(p\) to within
            \(10^{-5}\) in that interval is:

            \[\abs{p_n - p} \leq \frac{\num{1.5} - \num{0.5}}{2^n} < 10^{-5} \iff n \geq 17\]

            Applying Bisection method generates the following table:

            \begin{longtable}{r S[table-format=1.8] S[table-format=1.8] S[table-format=1.8] S[table-format=-1.9]}
                \toprule
                \(n\)  &   {\(a_n\)}   &   {\(b_n\)}   &   {\(p_n\)}   &  {\(f(p_n)\)}  \\
                \midrule
                \endfirsthead
                \(n\)  &   {\(a_n\)}   &   {\(b_n\)}   &   {\(p_n\)}   &  {\(f(p_n)\)}  \\
                \midrule
                \endhead
                    1  &  0.5          &  1.5          &  1            &  -0.034655726  \\
                    2  &  1            &  1.5          &  1.25         &   1.40997635   \\
                    3  &  1            &  1.25         &  1.125        &   0.609079747  \\
                    4  &  1            &  1.125        &  1.0625       &   0.266982288  \\
                    5  &  1            &  1.0625       &  1.03125      &   0.111147764  \\
                    6  &  1            &  1.03125      &  1.015625     &   0.037002875  \\
                    7  &  1            &  1.015625     &  1.0078125    &   0.000864425  \\
                    8  &  1            &  1.0078125    &  1.00390625   &  -0.016972716  \\
                    9  &  1.00390625   &  1.0078125    &  1.00585938   &  -0.00807344   \\
                   10  &  1.00585938   &  1.0078125    &  1.00683594   &  -0.003609335  \\
                   11  &  1.00683594   &  1.0078125    &  1.00732422   &  -0.001373662  \\
                   12  &  1.00732422   &  1.0078125    &  1.00756836   &  -0.00025492   \\
                   13  &  1.00756836   &  1.0078125    &  1.00769043   &   0.000304677  \\
                   14  &  1.00756836   &  1.00769043   &  1.00762939   &   0.000024859  \\
                   15  &  1.00756836   &  1.00762939   &  1.00759888   &  -0.000115035  \\
                   16  &  1.00759888   &  1.00762939   &  1.00761414   &  -0.000045089  \\
                \bottomrule
            \end{longtable}

            So \(p \approx \num{1.007614}\).
    \end{enumerate}
\end{solution}

\begin{exercise}
    Let \(f(x) = (x + 2) (x+1)^2 x (x - 1)^3 (x - 2)\). To which zero of \(f\)
    does the Bisection method converge when applied on the following intervals?

    \begin{multicols}{4}
        \begin{enumerate}[label = (\alph*)]
            \item \(\interval{\num{-1.5}}{\num{2.5}}\)
            \item \(\interval{\num{-0.5}}{\num{2.4}}\)
            \item \(\interval{\num{-0.5}}{3}\)
            \item \(\interval{-3}{\num{-0.5}}\)
        \end{enumerate}
    \end{multicols}
\end{exercise}

\begin{solution}
    \(f\) has 5 zeros: \(\pm 2\), \(\pm 1\), \(0\).

    \begin{enumerate}[label = (\alph*)]
        \item Applying Bisection method generates the following table:

            \begin{table}[H]
                \centering
                \begin{tabular}{r S[table-format=-1.1] S[table-format=1.1] S[table-format=-1.1] S[table-format=-1.8]}
                    \toprule
                    \(n\)  &   {\(a_n\)}   &   {\(b_n\)}   &   {\(p_n\)}   &  {\(f(p_n)\)}  \\
                    \midrule
                        1  &  -1.5         &  2.5          &   0.5         &   0.52734375   \\
                        2  &  -1.5         &  0.5          &  -0.5         &  -1.58203125   \\
                        3  &  -0.5         &  0.5          &   0           &   0            \\
                    \bottomrule
                \end{tabular}
            \end{table}

            So when applied on \(\interval{\num{-1.5}}{\num{2.5}}\), the
            Bisection method gives \(0\).

        \item Applying Bisection method generates the following table:

            \begin{table}[H]
                \centering
                \begin{tabular}{r S[table-format=-1.1] S[table-format=1.2] S[table-format=1.3] S[table-format=1.9]}
                    \toprule
                    \(n\)  &   {\(a_n\)}   &   {\(b_n\)}   &   {\(p_n\)}   &  {\(f(p_n)\)}  \\
                    \midrule
                        1  &  -0.5         &  2.4          &  0.95         &  0.001398666   \\
                        2  &  -0.5         &  0.95         &  0.225        &  0.62070919    \\
                    \bottomrule
                \end{tabular}
            \end{table}

            At \(n = 2\), the interval shrinks to
            \(\interval{\num{-0.5}}{\num{0.95}}\). So when applied on
            \(\interval{\num{-0.5}}{\num{2.4}}\), the Bisection method gives
            \(0\).

        \item Applying Bisection method generates the following table:

            \begin{table}[H]
                \centering
                \begin{tabular}{r S[table-format=-1.2] S[table-format=1] S[table-format=1.3] S[table-format=-1.9]}
                    \toprule
                    \(n\)  &   {\(a_n\)}   &   {\(b_n\)}   &   {\(p_n\)}   &  {\(f(p_n)\)}  \\
                    \midrule
                        1  &  -0.5         &  3            &  1.25         &  -0.241012573  \\
                        2  &   1.25        &  3            &  2.125        &  15.2352825    \\
                    \bottomrule
                \end{tabular}
            \end{table}

            At \(n = 2\), the interval shrinks to \(\interval{\num{1.25}}{3}\).
            So when applied on \(\interval{\num{-0.5}}{3}\), the Bisection
            method gives \(2\).

        \item Applying Bisection method generates the following table:

            \begin{table}[H]
                \centering
                \begin{tabular}{r S[table-format=-1] S[table-format=-1.2] S[table-format=-1.3] S[table-format=-2.7]}
                    \toprule
                    \(n\)  &   {\(a_n\)}   &   {\(b_n\)}   &   {\(p_n\)}   &  {\(f(p_n)\)}  \\
                    \midrule
                        1  &  -3           &  -0.5         &  -1.75        &  -19.1924286   \\
                        2  &  -3           &  -1.75        &  -2.375       &  283.204185    \\
                    \bottomrule
                \end{tabular}
            \end{table}

            At \(n = 2\), the interval shrinks to
            \(\interval{\-3}{\num{-1.75}}\). So when applied on
            \(\interval{-3}{\num{-0.5}}\), the Bisection method gives \(-2\).
    \end{enumerate}
\end{solution}

\begin{exercise}
    Let \(f(x) = (x + 2) (x+1) x (x - 1)^3 (x - 2)\). To which zero of \(f\)
    does the Bisection method converge when applied on the following intervals?

    \begin{multicols}{2}
        \begin{enumerate}[label = (\alph*)]
            \item \(\interval{-3}{\num{2.5}}\)
            \item \(\interval{\num{-2.5}}{3}\)
            \item \(\interval{\num{-1.75}}{\num{1.5}}\)
            \item \(\interval{\num{-1.5}}{\num{-1.75}}\)
        \end{enumerate}
    \end{multicols}
\end{exercise}

\begin{solution}
    \(f\) has 5 zeros: \(\pm 2\), \(\pm 1\), \(0\).

    \begin{enumerate}[label = (\alph*)]
        \item Applying Bisection method generates the following table:

            \begin{table}[H]
                \centering
                \begin{tabular}{r S[table-format=-1.3] S[table-format=1.1] S[table-format=-1.4] S[table-format=-1.9]}
                    \toprule
                    \(n\)  &   {\(a_n\)}   &   {\(b_n\)}   &   {\(p_n\)}   &  {\(f(p_n)\)}  \\
                    \midrule
                        1  &  -3           &  2.5          &  -0.25        &  -1.44195557   \\
                        2  &  -0.25        &  2.5          &   1.125       &  -0.012767315  \\
                        3  &   1.125       &  2.5          &   1.8125      &  -1.95457248   \\
                    \bottomrule
                \end{tabular}
            \end{table}

            At \(n = 3\), the interval shrinks to
            \(\interval{\num{1.125}}{\num{2.5}}\). So when applied on
            \(\interval{-3}{\num{2.5}}\), the Bisection method gives \(2\).

        \item Applying Bisection method generates the following table:

            \begin{table}[H]
                \centering
                \begin{tabular}{r S[table-format=-1.1] S[table-format=-1.3] S[table-format=-1.4] S[table-format=2.9]}
                    \toprule
                    \(n\)  &   {\(a_n\)}   &   {\(b_n\)}   &   {\(p_n\)}   &  {\(f(p_n)\)}  \\
                    \midrule
                        1  &  -2.5         &   3           &   0.25        &    0.519104004 \\
                        2  &  -2.5         &   0.25        &  -1.125       &    3.68975401  \\
                        3  &  -2.5         &  -1.125       &  -1.8125      &   23.4201732   \\
                    \bottomrule
                \end{tabular}
            \end{table}

            At \(n = 3\), the interval shrinks to
            \(\interval{\num{-2.5}}{\num{-1.125}}\). So when applied on
            \(\interval{\num{-2.5}}{3}\), the Bisection method gives \(-2\).

        \item Applying Bisection method generates the following table:

            \begin{table}[H]
                \centering
                \begin{tabular}{r S[table-format=-1.2] S[table-format=-1.3] S[table-format=-1.4] S[table-format=-1.9]}
                    \toprule
                    \(n\)  &   {\(a_n\)}   &   {\(b_n\)}   &   {\(p_n\)}   &  {\(f(p_n)\)}  \\
                    \midrule
                         1 &  -1.75        &   1.5         &  -0.125       & -0.620491505    \\
                         2 &  -1.75        &  -0.125       &  -0.9375      & -1.33009678     \\
                    \bottomrule
                \end{tabular}
            \end{table}

            At \(n = 2\), the interval shrinks to
            \(\interval{\num{-1.75}}{\num{-0.125}}\). So when applied on
            \(\interval{\num{-1.75}}{\num{1.5}}\), the Bisection method gives
            \(-1\).

        \item Applying Bisection method generates the following table:

            \begin{table}[H]
                \centering
                \begin{tabular}{r S[table-format=-1.3] S[table-format=1.2] S[table-format=1.4] S[table-format=1.9]}
                    \toprule
                    \(n\)  &   {\(a_n\)}   &   {\(b_n\)}   &   {\(p_n\)}   &  {\(f(p_n)\)}  \\
                    \midrule
                        1  &  -1.5         & 1.75          &  0.125        &  0.375359058   \\
                        2  &   0.125       & 1.75          &  0.9375       &  0.001384076   \\
                    \bottomrule
                \end{tabular}
            \end{table}

            At \(n = 2\), the interval shrinks to
            \(\interval{\num{0.125}}{\num{1.75}}\). So when applied on
            \(\interval{\num{-1.5}}{\num{1.75}}\), the Bisection method gives
            \(1\).
    \end{enumerate}
\end{solution}

\begin{exercise}
    Find an approximation to \(\sqrt{3}\) correct to within \(10^{−4}\) using
    the Bisection Algorithm.
\end{exercise}

\begin{solution}
    Let \(f(x) = x^2 - 3\). The positive zero of \(f\) is \(\sqrt{3}\), so by
    approximate that positive zero, we get an approximation of \(\sqrt{3}\).

    The positive zero of \(f\) clearly is inside \(\interval{1}{2}\). Using
    Bisection, the number of iteration \(n\) needed to approximate \(\sqrt{3}\)
    to within \(10^{-4}\) in that interval is:

    \[\frac{2 - 1}{2^n} < 10^{-4} \iff n \geq 14\]

    Applying Bisection method generates the following table:

    \begin{longtable}{r S[table-format=1.8] S[table-format=1.8] S[table-format=1.8] S[table-format=-1.9]}
        \toprule
        \(n\)  &   {\(a_n\)}   &   {\(b_n\)}   &   {\(p_n\)}   &  {\(f(p_n)\)}  \\
        \midrule
        \endfirsthead
        \(n\)  &   {\(a_n\)}   &   {\(b_n\)}   &   {\(p_n\)}   &  {\(f(p_n)\)}  \\
        \midrule
        \endhead
            1  &  1            &  2            &  1.5          &  -0.75         \\
            2  &  1.5          &  2            &  1.75         &   0.0625       \\
            3  &  1.5          &  1.75         &  1.625        &  -0.359375     \\
            4  &  1.625        &  1.75         &  1.6875       &  -0.15234375   \\
            5  &  1.6875       &  1.75         &  1.71875      &  -0.045898438  \\
            6  &  1.71875      &  1.75         &  1.734375     &   0.008056641  \\
            7  &  1.71875      &  1.734375     &  1.7265625    &  -0.018981934  \\
            8  &  1.7265625    &  1.734375     &  1.73046875   &  -0.005477905  \\
            9  &  1.73046875   &  1.734375     &  1.73242188   &   0.001285553  \\
           10  &  1.73046875   &  1.73242188   &  1.73144531   &  -0.00209713   \\
           11  &  1.73144531   &  1.73242188   &  1.73193359   &  -0.000406027  \\
           12  &  1.73193359   &  1.73242188   &  1.73217773   &   0.000439703  \\
           13  &  1.73193359   &  1.73217773   &  1.73205566   &   0.000016823  \\
           14  &  1.73193359   &  1.73205566   &  1.73199463   &  -0.000194605  \\
       \bottomrule
    \end{longtable}

    So \(\sqrt{3} \approx \num{1.73199}\).
\end{solution}

\begin{exercise}
    Find an approximation to \(\sqrt[3]{25}\) correct to within \(10^{−4}\)
    using the Bisection Algorithm.
\end{exercise}

\begin{solution}
    Let \(f(x) = x^3 - 25\). The zero of \(f\) is \(\sqrt[3]{25}\), so by
    approximate that positive zero, we get an approximation of \(\sqrt[3]{25}\).

    The positive zero of \(f\) clearly is inside \(\interval{2}{3}\). Using
    Bisection, the number of iteration \(n\) needed to approximate
    \(\sqrt[3]{25}\) to within \(10^{-4}\) in that interval is:

    \[\frac{3 - 2}{2^n} < 10^{-4} \iff n \geq 14\]

    Applying Bisection method generates the following table:

    \begin{longtable}{r S[table-format=1.8] S[table-format=1.8] S[table-format=1.8] S[table-format=-1.9]}
        \toprule
        \(n\)  &   {\(a_n\)}   &   {\(b_n\)}   &   {\(p_n\)}   &  {\(f(p_n)\)}  \\
        \midrule
        \endfirsthead
        \(n\)  &   {\(a_n\)}   &   {\(b_n\)}   &   {\(p_n\)}   &  {\(f(p_n)\)}  \\
        \midrule
        \endhead
            1  &  2            &  3            &  2.5          &  -9.375        \\
            2  &  2.5          &  3            &  2.75         &  -4.203125     \\
            3  &  2.75         &  3            &  2.875        &  -1.23632812   \\
            4  &  2.875        &  3            &  2.9375       &   0.347412109  \\
            5  &  2.875        &  2.9375       &  2.90625      &  -0.452972412  \\
            6  &  2.90625      &  2.9375       &  2.921875     &  -0.054920197  \\
            7  &  2.921875     &  2.9375       &  2.9296875    &   0.145709515  \\
            8  &  2.921875     &  2.9296875    &  2.92578125   &   0.045260727  \\
            9  &  2.921875     &  2.92578125   &  2.92382812   &  -0.004863195  \\
           10  &  2.92382812   &  2.92578125   &  2.92480469   &   0.020190398  \\
           11  &  2.92382812   &  2.92480469   &  2.92431641   &   0.00766151   \\
           12  &  2.92382812   &  2.92431641   &  2.92407227   &   0.001398635  \\
           13  &  2.92382812   &  2.92407227   &  2.9239502    &  -0.001732411  \\
           14  &  2.9239502    &  2.92407227   &  2.92401123   &  -0.000166921  \\
       \bottomrule
    \end{longtable}

    So \(\sqrt[3]{25} \approx \num{2.92401}\).
\end{solution}

\begin{exercise}
    Use Theorem 2.1 (\emph{Định lí 2.2} in the Lectures.pdf of the project) to
    find a bound for the number of iterations needed to achieve an approximation
    with accuracy \(10^{-3}\) to the solution of \(x^3 + x − 4 = 0\) lying in
    the interval \(\interval{1}{4}\). Find an approximation to the root with
    this degree of accuracy.
\end{exercise}

\begin{solution}
    Let \(f(x) = x^3 + x − 4\). \(f(1) = -2\) and \(f(4) = 64\) have the
    opposite signs, so there's a root \(p\) of \(f\) in \(\interval{1}{4}\).

    The number of iteration \(n\) needed to approximate \(p\) to within
    \(10^{-3}\) in that interval is:

    \[\abs{p_n - p} \leq \frac{4 - 1}{2^n} < 10^{-3} \iff n \geq 12\]

    Applying Bisection method generates the following table:

    \begin{longtable}{r S[table-format=1.8] S[table-format=1.8] S[table-format=1.8] S[table-format=-1.9]}
        \toprule
        \(n\)  &   {\(a_n\)}   &   {\(b_n\)}   &   {\(p_n\)}   &  {\(f(p_n)\)}  \\
        \midrule
        \endfirsthead
        \(n\)  &   {\(a_n\)}   &   {\(b_n\)}   &   {\(p_n\)}   &  {\(f(p_n)\)}  \\
        \midrule
        \endhead
            1  &  1            &  4            &  2.5          &  14.125        \\
            2  &  1            &  2.5          &  1.75         &   3.109375     \\
            3  &  1            &  1.75         &  1.375        &  -0.025390625  \\
            4  &  1.375        &  1.75         &  1.5625       &   1.37719727   \\
            5  &  1.375        &  1.5625       &  1.46875      &   0.637176514  \\
            6  &  1.375        &  1.46875      &  1.421875     &   0.296520233  \\
            7  &  1.375        &  1.421875     &  1.3984375    &   0.13326025   \\
            8  &  1.375        &  1.3984375    &  1.38671875   &   0.053363502  \\
            9  &  1.375        &  1.38671875   &  1.38085938   &   0.013844214  \\
           10  &  1.375        &  1.38085938   &  1.37792969   &  -0.005808686  \\
           11  &  1.37792969   &  1.38085938   &  1.37939453   &   0.004008885  \\
           12  &  1.37792969   &  1.37939453   &  1.37866211   &  -0.000902119  \\
        \bottomrule
    \end{longtable}

    So \(p \approx \num{1.3787}\).
\end{solution}

\begin{exercise}
    Use Theorem 2.1 (\emph{Định lí 2.2} in the Lectures.pdf of the project) to
    find a bound for the number of iterations needed to achieve an approximation
    with accuracy \(10^{-4}\) to the solution of \(x^3 - x − 1 = 0\) lying in
    the interval \(\interval{1}{2}\). Find an approximation to the root with
    this degree of accuracy.
\end{exercise}

\begin{solution}
    Let \(f(x) = x^3 - x − 1\). \(f(1) = -2\) and \(f(4) = 64\) have the
    opposite signs, so there's a root \(p\) of \(f\) in \(\interval{1}{2}\).

    The number of iteration \(n\) needed to approximate \(p\) to within
    \(10^{-4}\) in that interval is:

    \[\abs{p_n - p} \leq \frac{2 - 1}{2^n} < 10^{-4} \iff n \geq 14\]

    Applying Bisection method generates the following table:

    \begin{longtable}{r S[table-format=1.8] S[table-format=1.8] S[table-format=1.8] S[table-format=-1.9]}
        \toprule
        \(n\)  &   {\(a_n\)}   &   {\(b_n\)}   &   {\(p_n\)}   &  {\(f(p_n)\)}  \\
        \midrule
        \endfirsthead
        \(n\)  &   {\(a_n\)}   &   {\(b_n\)}   &   {\(p_n\)}   &  {\(f(p_n)\)}  \\
        \midrule
        \endhead
            1  &  1            &  2            &  1.5          &   0.875        \\
            2  &  1            &  1.5          &  1.25         &  -0.296875     \\
            3  &  1.25         &  1.5          &  1.375        &   0.224609375  \\
            4  &  1.25         &  1.375        &  1.3125       &  -0.051513672  \\
            5  &  1.3125       &  1.375        &  1.34375      &   0.082611084  \\
            6  &  1.3125       &  1.34375      &  1.328125     &   0.014575958  \\
            7  &  1.3125       &  1.328125     &  1.3203125    &  -0.018710613  \\
            8  &  1.3203125    &  1.328125     &  1.32421875   &  -0.002127945  \\
            9  &  1.32421875   &  1.328125     &  1.32617188   &   0.00620883   \\
           10  &  1.32421875   &  1.32617188   &  1.32519531   &   0.002036651  \\
           11  &  1.32421875   &  1.32519531   &  1.32470703   &  -0.000046595  \\
           12  &  1.32470703   &  1.32519531   &  1.32495117   &   0.000994791  \\
           13  &  1.32470703   &  1.32495117   &  1.3248291    &   0.000474039  \\
           14  &  1.32470703   &  1.3248291    &  1.32476807   &   0.000213707  \\
        \bottomrule
    \end{longtable}

    So \(p \approx \num{1.32477}\).
\end{solution}

\begin{exercise}
    Let \(f(x) = (x − 1)^{10}\), \(p = 1\), and \(p_n = 1 + \frac{1}{n}\). Show
    that \(\abs{f(p_n)} < 10^{-3}\) whenever \(n > 1\) but that \(\abs{p - p_n}
    < 10^{-3}\) requires that \(n > 1000\).
\end{exercise}

\begin{solution}
    For \(f(p_n) < 10^{-3}\), it is required that \(n > 1\) as:

    \begin{align*}
        &\qquad&         f(p_n) & < 10^{-3} \\
        \iff&&   (p_n - 1)^{10} & < 10^{-3} \\
        \iff&& \frac{1}{n^{10}} & < 10^{-3} \\
        \iff&&                n & > 1
    \end{align*}

    For \(\abs{p - p_n} < 10^{-3}\), it is required that \(n > 1000\) as:

    \begin{align*}
        &\qquad& \abs{p - p_n} & < 10^{-3} \\
        \iff&&     \frac{1}{n} & < 10^{-3} \\
        \iff&&               n & > 1000
    \end{align*}

    \qed
\end{solution}

\begin{exercise}
    Let \(\{p_n\}\) be the sequence defined by \(p_n = \sum_{k = 1}^{n}
    \frac{1}{k}\). Show that \(\{p_n\}\) diverges even though \(\lim_{n \to
    \infty} (p_n - p_{n - 1}) = 0\).
\end{exercise}

\begin{solution}
    It's clear that the difference of 2 consecutive terms goes to zero:

    \[\lim_{n \to \infty} (p_n - p_{n - 1}) = \lim_{n \to \infty} \frac{1}{n} = 0\]

    However, the sequence diverges as:

    \[\begin{aligned}
        p_n & = \sum_{k = 1}^{n} \frac{1}{k} \\
            & = 1 + \frac{1}{2} + \frac{1}{3} + \frac{1}{4} + \ldots \\
            & > 1 + (\frac{1}{2}) + (\frac{1}{4} + \frac{1}{4}) + \ldots \\
            & = 1 + \frac{1}{2} + \frac{1}{2} + \ldots \\
            & = \infty
    \end{aligned}\]
\end{solution}

\begin{exercise}
    The function defined by \(f(x) = \sin{\pi x}\) has zeros at every integer.
    Show that when \(−1 < a < 0\) and \(2 < b < 3\), the Bisection method
    converges to

    \begin{multicols}{3}
        \begin{enumerate}[label = (\alph*)]
            \item \(0\) if \(a + b < 2\)
            \item \(2\) if \(a + b > 2\)
            \item \(1\) if \(a + b = 2\)
        \end{enumerate}
    \end{multicols}
\end{exercise}

\begin{solution}
    Let \(p\) be the zero converged by Bisection.

    With \(-1 < a < 0\) and \(2 < b < 3\):

    \begin{gather*}
        \sin{\pi a} < 0 \\
        \sin{\pi b} > 0 \\
        1 < a + b < 3
    \end{gather*}

    \begin{enumerate}[label = (\alph*)]
        \item If \(a + b < 2\), then \(\num{0.5} < p_1 = \frac{a + b}{2} < 1\).
            Then \(\sin{p_1} > 0\), and the interval shrinks to
            \(\interval{a}{p_1}\). \(0\) is the only zero in that interval, so
            \(p = 0\).

        \item If \(a + b > 2\), then \(1 < p_1 = \frac{a + b}{2} < \num{1.5}\).
            Then \(\sin{p_1} < 0\), and the interval shrinks to
            \(\interval{p_1}{b}\). \(2\) is the only zero in that interval, so
            \(p = 0\).

        \item If \(a + b = 2\), then \(p_1 = \frac{a + b}{2} = 1\). Then
            \(\sin{p_1} = 0\), and a zero \(p = 1\) is found.
    \end{enumerate}
\end{solution}

\begin{exercise}
    A trough of length \(L\) has a cross section in the shape of a semicircle
    with radius \(r\). When filled with water to within a distance \(h\) of the
    top, the volume \(V\) of water is:

    \[V = L [0.5 \pi r^2 - r^2 \arcsin{\frac{h}{r}} - h \sqrt{r^2 - h^2}]\]

    Suppose \(L = \SI{10}{ft}\), \(r = \SI{1}{ft}\), and \(V =
    \SI{12.4}{ft^3}\). Find the depth of water in the trough to within
    \(\SI{0.01}{ft}\).
\end{exercise}

\begin{solution}
    Let \(d\) be the depth of the water, so \(d = r - h\). Let

    \[f(h) = 10 (\num{0.5} \pi - \arcsin(h) - h \sqrt{1 - h^2}) - \num{12.4}\]

    Instead of finding \(d\) directly, we find \(h\), also to within
    \(\SI{0.01}{ft}\). The number of iteration \(n\) needed to approximate \(h\)
    to within \(\num{0.01}\) in \(\interval{0}{r}\) is:

    \[\abs{h - h_n} < \frac{1 - 0}{2^n} < 0.01 \iff n \geq 7\]

    Applying Bisection method generates the following table:

    \begin{longtable}{r S[table-format=1.5] S[table-format=1.6] S[table-format=1.7] S[table-format=-1.9]}
        \toprule
        \(n\)  &   {\(a_n\)}   &   {\(b_n\)}   &   {\(p_n\)}   &  {\(f(p_n)\)}  \\
        \midrule
        \endfirsthead
        \(n\)  &   {\(a_n\)}   &   {\(b_n\)}   &   {\(p_n\)}   &  {\(f(p_n)\)}  \\
        \midrule
        \endhead
            1  &  0            &  1            &  0.5          &  -6.25815151   \\
            2  &  0            &  0.5          &  0.25         &  -1.63945387   \\
            3  &  0            &  0.25         &  0.125        &   0.814489029  \\
            4  &  0.125        &  0.25         &  0.1875       &  -0.419946724  \\
            5  &  0.125        &  0.1875       &  0.15625      &   0.195725903  \\
            6  &  0.15625      &  0.1875       &  0.171875     &  -0.112536394  \\
            7  &  0.15625      &  0.171875     &  0.1640625    &   0.041493241  \\
        \bottomrule
    \end{longtable}

    So \(h \approx \num{0.1641}\), hence \(d = r - h \approx \num{0.8359}\).
\end{solution}



\end{document}
