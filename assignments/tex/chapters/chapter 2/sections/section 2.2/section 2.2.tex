\documentclass[../../../../Assignments]{subfiles}


\begin{document}

\section{Fixed-Point Iteration}

\begin{exercise}\label{exer:2.2.1}
    Use algebraic manipulation to show that each of the following functions has
    a fixed-point at \(p\) precisely when \(f(p) = 0\), where \(f(x) = x^4 +
    2x^2 - x - 3\).

\begin{tasks}(2)
    \task \(g_1(x) = (3 + x - 2x^2)^{\sfrac{1}{4}}\)
    \task \(g_2(x) = \left(\dfrac{x + 3 - x^4}{2}\right)^{\sfrac{1}{2}}\)
    \task \(g_3(x) = \left(\dfrac{x + 3}{x^2 + 2}\right)^{\sfrac{1}{2}}\)
    \task \(g_4(x) = \dfrac{3x^4 + 2x^2 + 3}{4x^3 + 4x - 1}\)
\end{tasks}
\end{exercise}

\begin{solution}
    \begin{enumerate}[label = \alph*)]
        \item For \(x = p\):

            \[g_1(p) = (3 + p - 2p^2)^{\frac{1}{4}} = (p^4 - f(p))^{\sfrac{1}{4}} = \abs{p}\]

            So \(p\) is a fixed-point of \(g_1\).

        \item For \(x = p\):

            \begin{align*}
                g_2(p) &= \left(\dfrac{p + 3 - p^4}{2}\right)^{\sfrac{1}{2}} \\
                       &= \left(\dfrac{2p^2}{2}\right)^{\frac{1}{2}} \\
                       &= \abs{p}
            \end{align*}

            So \(p\) is a fixed-point of \(g_2\).

        \item For \(x = p\):

            \begin{align*}
                g_3(p) &= \left(\dfrac{p + 3}{p^2 + 2}\right)^{\sfrac{1}{2}} \\
                       &= \left(\dfrac{p^4 + 2p^2}{p^2 + 2}\right)^{\sfrac{1}{2}} \\
                       &= \abs{p}
            \end{align*}

            So \(p\) is a fixed-point of \(g_3\).

        \item For \(x = p\):

            \begin{align*}
                g_4(p) &= \dfrac{3p^4 + 2p^2 + 3}{4p^3 + 4p - 1} \\
                       &= \dfrac{4p^4 - (3 + p - 2p^2) + 2p^2 + 3}{4p^3 + 4p - 1} \\
                       &= \dfrac{4p^4 + 4p^2 -p}{4p^3 + 4p - 1} \\
                       &= p
            \end{align*}

            So \(p\) is a fixed-point of \(g_4\).
    \end{enumerate}
\end{solution}

\begin{exercise}
    \begin{tasks}
        \task Perform four iterations, if possible, on each of the functions
            \(g\) defined in \hyperref[exer:2.2.1]{Exercise 1}. Let \(p_0 = 1\)
            and \(p_{n + 1} = g(p_n)\), for \(n = 0, 1, 2, 3\).

        \task Which function do you think gives the best approximation to the
            solution?
    \end{tasks}
\end{exercise}

\begin{solution}
    \begin{enumerate}[label = \alph*)]
        \item Applying fixed-point method on the four functions \(g\) generates
            the following table:

            \begin{table}[H]
                \centering
                \begin{tabular}{r S[table-format=1.9] S[table-format=1.9] S[table-format=1.9] S[table-format=1.9]}
                    \toprule
                    \(n\)  & {\(p_n\) by \(g_1\)} & {\(p_n\) by \(g_2\)} & {\(p_n\) by \(g_3\)} & {\(p_n\) by \(g_4\)} \\
                    \midrule
                        0  &  1                   &  1                   &  1                   &  1                   \\
                        1  &  1.189207115         &  1.224744871         &  1.154700538         &  1.142857143         \\
                        2  &  1.080057753         &  0.993666159         &  1.11642741          &  1.12448169          \\
                        3  &  1.149671431         &  1.228568645         &  1.126052233         &  1.124123164         \\
                        4  &  1.107820053         &  0.987506429         &  1.123638885         &  1.12412303          \\
                    \bottomrule
                \end{tabular}
            \end{table}

        \item \(g_4\) gives the best approximation as it generates the smallest
            difference between \(p_3\) and \(p_4\): \({\abs{p_4 - p_3} =
            \num{-134e-7}}\).
    \end{enumerate}
\end{solution}

\begin{exercise}
    The following four methods are proposed to compute \(21^{\sfrac{1}{3}}\).
    Rank them in order, based on their apparent speed of convergence, assuming
    \(p_0 = 1\).

    \begin{tasks}(2)
        \task \(p_n = \dfrac{20 p_{n - 1} + \sfrac{21}{p_{n - 1}^2}}{21}\)               \label{exer:2.2.3:a}
        \task \(p_n = p_{n - 1} - \dfrac{p_{n - 1}^3 - 21}{3 p_{n - 1}^2}\)              \label{exer:2.2.3:b}
        \task \(p_n = p_{n - 1} - \dfrac{p_{n - 1}^4 - 21 p_{n - 1}}{p_{n - 1}^2 - 21}\) \label{exer:2.2.3:c}
        \task \(p_n = \left(\dfrac{21}{p_{n-1}}\right)^{\sfrac{1}{2}}\)                  \label{exer:2.2.3:d}
    \end{tasks}
\end{exercise}

\begin{solution}
    Applying fixed-point method on the four sequences generate the following
    table:

    \begin{longtable}{r S[table-format=1.9] S[table-format=1.9] S[table-format=1.9] S[table-format=1.9]}
        \toprule
        \(n\)  & {\hyperref[exer:2.2.3:a]{a)}} & {\hyperref[exer:2.2.3:b]{b)}} & {\hyperref[exer:2.2.3:c]{c)}} & {\hyperref[exer:2.2.3:d]{d)}} \\
        \midrule
        \endfirsthead
        \(n\)  & {\hyperref[exer:2.2.3:a]{a)}} & {\hyperref[exer:2.2.3:b]{b)}} & {\hyperref[exer:2.2.3:c]{c)}} & {\hyperref[exer:2.2.3:d]{d)}} \\
        \midrule
        \endhead
            0  &  1            &  1            &  1            &  1            \\
            1  &  1.952380952  &  7.666666667  &  0            &  4.582575695  \\
            2  &  2.121754174  &  5.230203739  &  0            &  2.140695143  \\
            3  &  2.242849692  &  3.742696919  &               &  3.132075595  \\
            4  &  2.334839673  &  2.994853568  &               &  2.589366527  \\
            5  &  2.40109338   &  2.777022226  &               &  2.847822274  \\
            6  &  2.465059288  &  2.759041866  &               &  2.715521253  \\
            7  &  2.512243463  &  2.758924181  &               &  2.780885095  \\
            8  &  2.551057096  &  2.758924176  &               &  2.748008838  \\
            9  &  2.583237767  &  2.758924176  &               &  2.764398093  \\
           10  &  2.610081445  &               &               &  2.756191284  \\
           11  &  2.632580301  &               &               &  2.760291639  \\
           12  &  2.651509504  &               &               &  2.758240699  \\
           13  &  2.667484488  &               &               &  2.759265978  \\
           14  &  2.681000202  &               &               &  2.758753291  \\
           15  &  2.692458887  &               &               &  2.759009623  \\
           16  &  2.702190249  &               &               &  2.758881454  \\
           17  &  2.710466453  &               &               &  2.758945538  \\
           18  &  2.717513483  &               &               &  2.758913496  \\
           19  &  2.723519902  &               &               &  2.758929517  \\
        \bottomrule
    \end{longtable}

    Apparently, the speed of convergence is ranked in descending order as
    follow: \hyperref[exer:2.2.3:b]{b)}, \hyperref[exer:2.2.3:d]{d)},
    \hyperref[exer:2.2.3:a]{a)}. \hyperref[exer:2.2.3:c]{c)} does not converge.
\end{solution}

\begin{exercise}
    The following four methods are proposed to compute \(7^{\sfrac{1}{5}}\).
    Rank them in order, based on their apparent speed of convergence, assuming
    \(p_0 = 1\).

    \begin{tasks}(2)
        \task \(p_n = p_{n - 1} - \left(1 + \frac{7 - p_{n - 1}^5}{p_{n - 1}^2}\right)^3\) \label{exer:2.2.4:a}
        \task \(p_n = p_{n - 1} - \dfrac{p_{n - 1}^5 - 7}{p_{n - 1}^2}\)                   \label{exer:2.2.4:b}
        \task \(p_n = p_{n - 1} - \dfrac{p_{n - 1}^5 - 7}{5 p_{n - 1}^4}\)                 \label{exer:2.2.4:c}
        \task \(p_n = p_{n - 1} - \dfrac{p_{n - 1}^5 - 7}{12}\)                            \label{exer:2.2.4:d}
    \end{tasks}
\end{exercise}

\begin{solution}
    Applying fixed-point method on the four sequences generate the following
    table:

    \begin{longtable}{r c c S[table-format=1.9] S[table-format=1.9]}
        \toprule
        \(n\)  & {\hyperref[exer:2.2.4:a]{a)}} & {\hyperref[exer:2.2.4:b]{b)}} & {\hyperref[exer:2.2.4:c]{c)}} & {\hyperref[exer:2.2.4:d]{d)}} \\
        \midrule
        \endfirsthead
        \(n\)  & {\hyperref[exer:2.2.4:a]{a)}} & {\hyperref[exer:2.2.4:b]{b)}} & {\hyperref[exer:2.2.4:c]{c)}} & {\hyperref[exer:2.2.4:d]{d)}} \\
        \midrule
        \endhead
            0  &         1        &         1        &  2.2          &  1            \\
            1  &        343       &         7        &  1.819763677  &  1.5          \\
            2  &  \num{-2.25e25}  &  \num{-335.857}  &  1.58347483   &  1.450520833  \\
            3  &                  &  \num{37884356}  &  1.489460974  &  1.498749661  \\
            4  &                  &                  &  1.476022436  &  1.451903535  \\
            5  &                  &                  &  1.475773246  &  1.497577067  \\
            6  &                  &                  &  1.475773162  &  1.45319229   \\
            7  &                  &                  &  1.475773162  &  1.496475364  \\
            9  &                  &                  &               &  1.454396119  \\
            8  &                  &                  &               &  1.495438587  \\
           10  &                  &                  &               &  1.45552281   \\
           11  &                  &                  &               &  1.494461513  \\
           12  &                  &                  &               &  1.456579138  \\
           13  &                  &                  &               &  1.493539533  \\
           14  &                  &                  &               &  1.457571031  \\
           15  &                  &                  &               &  1.49266856   \\
           16  &                  &                  &               &  1.458803715  \\
           17  &                  &                  &               &  1.491844948  \\
           18  &                  &                  &               &  1.459381814  \\
           19  &                  &                  &               &  1.491065425  \\
        \bottomrule
    \end{longtable}

    Apparently, the speed of convergence is ranked in descending order as
    follow: \hyperref[exer:2.2.4:c]{c)}, \hyperref[exer:2.2.4:d]{d)}.
    \hyperref[exer:2.2.4:a]{a)} and \hyperref[exer:2.2.4:b]{b)} do not converge.
\end{solution}

\begin{exercise}
    Use a fixed-point iteration method to determine a solution accurate to
    within \(10^{-2}\) for \(x^4 - 3x^2 - 3 = 0\) on \(\interval{1}{2}\). Use
    \(p_0 = 1\).
\end{exercise}

\begin{solution}
    Let \(f(x) = x^4 - 3x^2 - 3\). Let \(p\) be the root of \(f\) in
    \(\interval{1}{2}\). We need to find a function \(g\) for which \(p = g(p)\)
    to perform the fixed-point method.

    Extract \(p\) to RHS gives:

    \[p^4 = 3p^2 + 3 \iff \abs{p} = (3p^2 + 3)^{\sfrac{1}{4}}\]

    Then \(g\) is chosen as:

    \[g(x) = (3x^2 + 3)^{\sfrac{1}{4}}\]

    Applying fixed-point method on \(g\) generate the following table:

    \begin{table}[H]
        \centering
        \begin{tabular}{r S[table-format=1.9] r S[table-format=1.9]}
            \toprule
            \(n\)  &   {\(p_n\)}   &  \(n\)  &   {\(p_n\)}   \\
            \midrule
                0  &  1            &      4  &  1.922847844  \\
                1  &  1.56508458   &      5  &  1.93750754   \\
                2  &  1.793572879  &      6  &  1.94331693   \\
                3  &  1.885943743  &         &               \\
            \bottomrule
        \end{tabular}
    \end{table}

    We can try the other obvious option

    \[g(x) = \left(\frac{x^4 - 3}{3}\right)^{\num{0.5}}\]

    \noindent which fails on the first iteration. A reasonable explanation for
    the choice of \(g\) is that we need \(\abs{g'}\) to be as small as possible.
    On \(\interval{1}{2}\), the \(O(x^{\num{0.5}})\) of the first choice clearly
    has an advantage over \(O(x^2)\) of the second choice of \(g\).

    We conclude that \(p \approx \num{1.943}\).
\end{solution}

\begin{exercise}
    Use a fixed-point iteration method to determine a solution accurate to
    within \(10^{-2}\) for \(x^3 - x - 1 = 0\) on \(\interval{1}{2}\). Use \(p_0
    = 1\).
\end{exercise}

\begin{solution}
    Let \(f(x) = x^3 - x - 1 = 0\). Let \(p\) be the root of \(f\) in
    \(\interval{1}{2}\). We need to find a function \(g\) for which \(p = g(p)\)
    to perform the fixed-point method.

    Extract \(p\) to RHS gives:

    \[p^3 = p + 1 \iff p = (p + 1)^{\sfrac{1}{3}}\]

    Then \(g\) is chosen as:

    \[g(x) = (p + 1)^{\sfrac{1}{3}}\]

    Applying fixed-point method on \(g\) generates the following table:

    \begin{table}[H]
        \centering
        \begin{tabular}{r S[table-format=1.9] r S[table-format=1.9]}
            \toprule
            \(n\)  &   {\(p_n\)}   &  \(n\)  &   {\(p_n\)}   \\
            \midrule
                0  &  1            &      3  &  1.322353819  \\
                1  &  1.25992105   &      4  &  1.324268745  \\
                2  &  1.312293837  &         &               \\
            \bottomrule
        \end{tabular}
    \end{table}

    We conclude that \(p \approx \num{1.324}\).
\end{solution}

\begin{exercise}
    Use Theorem 2.3 (Định lý 2.3 in the accompanying Lectures.pdf) to show that
    \(g(x) = \pi + \num{0.5} \sin{\num{0.5} x}\) has a unique fixed point on
    \(\interval{0}{2 \pi}\). Use fixed-point iteration to find an approximation
    to the fixed point that is accurate to within \(10^{-2}\). Use Corollary 2.5
    (Hệ quả 2.1) to estimate the number of iterations required to achieve
    \(10^{-2}\) accuracy, and compare this theoretical estimate to the number
    actually needed.
\end{exercise}

\begin{solution}
    From the formula of \(g\):

    \begin{align*}
                    g(x) & = \pi + \num{0.5} \sin{\num{0.5} x} \\
        \Rightarrow g(x) & \in \interval{\pi - \num{0.5}}{\pi + \num{0.5}} \, \forall x
    \end{align*}

    Consider the interval \(I = \interval{\pi - \num{0.5}}{\pi + \num{0.5}} \in
    \interval{0}{2 \pi}\). From the above equations, we know that:

    \begin{itemize}
        \item \(g \in C I\)
        \item \(g(x) \in I \, \forall x \in I\)
    \end{itemize}

    According to Theorem 2.3, there exists a fixed point of \(g\) on \(I\).

    Differentiating \(g\) gives:

    \[g'(x) = -\num{0.25} \cos{\num{0.5} x} \Rightarrow \abs{g'(x)} \leq k = \num{0.25} < 1 \, \forall x\]

    Again, according to Theorem 2.3, there exists one and only one fixed point
    of \(g\) on \(I\).

    Applying fixed-point method on \(g\), with \(p_0 = \pi\), generates the
    following table:

    \begin{table}[H]
        \centering
        \begin{tabular}{r S[table-format=1.9] r S[table-format=1.9]}
            \toprule
            \(n\)  &   {\(p_n\)}   &  \(n\)  &   {\(p_n\)}   \\
            \midrule
                0  &  3.141592654  &      2  &  3.626048864  \\
                1  &  3.641592654  &      3  &  3.626995622  \\
            \bottomrule
        \end{tabular}
    \end{table}

    Using corollary 2.5, the number of iterations \(n\) required to achieve
    \(10^{-2}\) accuracy is

    \[\abs{p_n - p} \leq k^n \num{0.5} < 10^{-2} \iff n \geq 3\]

    \noindent which is in line with the number of iteration actually performed.
\end{solution}

\begin{exercise}
    Use Theorem 2.3 (Định lý 2.3 in the accompanying Lectures.pdf) to show that
    \(g(x) = 2^{-x}\) has a unique fixed point on \(\interval{\frac{1}{3}}{1}\).
    Use fixed-point iteration to find an approximation to the fixed point that
    is accurate to within \(10^{-4}\). Use Corollary 2.5 (Hệ quả 2.1) to
    estimate the number of iterations required to achieve \(10^{-4}\) accuracy,
    and compare this theoretical estimate to the number actually needed.
\end{exercise}

\begin{solution}
    From the formula of \(g\):

    \begin{align*}
                     g(x) & = 2^{-x} \\
        \Rightarrow g'(x) & = - 2^{-x} \ln{2}
    \end{align*}

    It is clear that \(g \in C^1 R\).

    Consider the interval \(I = \interval{\frac{1}{3}}{1}\), \(I_{open} =
    \interval[open]{\frac{1}{3}}{1}\):

    \begin{align*}
        &g'(x) < 0 \forall x \in I \\
        &\Rightarrow 1 > g(\frac{1}{3}) = 2^{-\sfrac{1}{3}} \geq g(x) \geq g(1) = 2^{-1} > \frac{1}{3} \\
        &\Rightarrow g(x) \in I \, \forall x \in I
    \end{align*}

    So far, we know that:

    \begin{itemize}
        \item \(g \in C I\) (\(g \in C R\) even)
        \item \(g(x) \in I \, \forall x \in I\)
    \end{itemize}

    According to Theorem 2.3, there exists a fixed point of \(g\) on \(I\).

    Consider \(g'\):

    \begin{gather*}
        -1 < - \ln{2} \leq g'(x) \leq - \frac{1}{3} \ln{2} < 0 \, \forall x \in I \\
        \Rightarrow \abs{g'(x)} \leq k = \ln{2} < 1 \, \forall x \in I
    \end{gather*}

    Again, according to Theorem 2.3, there exists one and only one fixed point
    of \(g\) on \(I\).

    Applying fixed-point method on \(g\), with \(p_0 = \frac{2}{3}\), generates
    the following table:

    \begin{table}[H]
        \centering
        \begin{tabular}{r S[table-format=1.9] r S[table-format=1.9]}
            \toprule
            \(n\)  &   {\(p_n\)}   &  \(n\)  &   {\(p_n\)}   \\
            \midrule
                0  &  0.666666667  &      5  &  0.640746653  \\
                1  &  0.629960525  &      6  &  0.641380922  \\
                2  &  0.646194096  &      7  &  0.641099006  \\
                3  &  0.638963711  &      8  &  0.641224295  \\
                4  &  0.642174057  &      9  &  0.641168611  \\
            \bottomrule
        \end{tabular}
    \end{table}

    Using Corollary 2.5, the number of iterations \(n\) required to achieve
    \(10^{-4}\) accuracy is

    \[\abs{p_n - p} \leq k^n \frac{1}{3} < 10^{-4} \iff n \geq 23\]

    \noindent which is quit a bit higher than the number of iteration actually
    performed.
\end{solution}

\begin{exercise}
    % TODO: link Exercise 12
    Use a fixed-point iteration method to find an approximation to \(\sqrt{3}\)
    that is accurate to within \(10^{-4}\). Compare your result and the number
    of iterations required with the answer obtained in Exercise 12 of Section
    2.1.
\end{exercise}

\begin{solution}
    Let \(f(x) = x^2 - 3\), \(p > 0\) is a zero of \(f\). Then \(p = \sqrt{3}\),
    and an approximation of \(p\) is an approximation of \(\sqrt{3}\).

    Consider \(g(x) = \frac{3}{x}\). It is clear that this is a bad choice, as
    applying \(g\) on any \(p_0\) generates a sequence that jumps between
    \(p_0\) and \(\frac{3}{p_0}\).

    From the textbook examples, we choose \(g(x) = x - \frac{x^2 - 3}{x^2}\).
    Applying fixed-point method on \(g\) with \(p_0 = \num{1.5}\) generates the
    following table:

    \begin{table}[H]
        \centering
        \begin{tabular}{r S[table-format=1.8] r S[table-format=1.8]}
            \toprule
            \(n\)  &   {\(p_n\)}   &  \(n\)  &   {\(p_n\)}   \\
            \midrule
                0  &  1.5          &      4  &  1.73189858   \\
                1  &  1.83333333   &      5  &  1.73207438   \\
                2  &  1.72589532   &      6  &  1.73204716   \\
                3  &  1.73304114   &         &               \\
            \bottomrule
        \end{tabular}
    \end{table}

    We conclude that \(\sqrt{3} \approx \num{1.73205}\). In exercise 12 of
    section 2.1, 14 iteration is needed, much higher than that of this method.
\end{solution}

\begin{exercise}
    % TODO: link Exercise 12
    Use a fixed-point iteration method to find an approximation to
    \(\sqrt[3]{25}\) that is accurate to within \(10^{-4}\). Compare your result
    and the number of iterations required with the answer obtained in Exercise
    13 of Section 2.1.
\end{exercise}

\begin{solution}
    Let \(f(x) = x^3 - 25\), \(p > 0\) is a zero of \(f\). Then \(p =
    \sqrt[3]{25}\), and an approximation of \(p\) is an approximation of
    \(\sqrt[3]{25}\).

    We choose \(g(x) = x - \frac{x^3 - 25}{x^3}\). Applying fixed-point method
    on \(g\) with \(p_0 = \num{2.5}\) generates the following table:

    \begin{table}[H]
        \centering
        \begin{tabular}{r S[table-format=1.8] r S[table-format=1.8]}
            \toprule
            \(n\)  &   {\(p_n\)}   &  \(n\)  &   {\(p_n\)}   \\
            \midrule
                0  &  2.5          &      3  &  2.92378369   \\
                1  &  3.1          &      4  &  2.92402386   \\
                2  &  2.93917962   &      5  &  2.92401758   \\
            \bottomrule
        \end{tabular}
    \end{table}

    We conclude that \(\sqrt[3]{25} \approx \num{2.92402}\). In exercise 13 of
    section 2.1, 14 iteration is needed, much higher than that of this method.
\end{solution}

\begin{exercise}
    For each of the following equations, determine an interval
    \(\interval{a}{b}\) on which fixed-point iteration converges. Estimate the
    number of iterations necessary to obtain approximations accurate to within
    \(10^{-5}\), and perform the calculations.

    \begin{tasks}(2)
        \task \(x = \dfrac{2 - e^x + x^2}{3}\)
        \task \(x = \dfrac{5}{x^2} + 2\)
        \task \(x = (\sfrac{e^x}{3})^{\sfrac{1}{2}}\)
        \task \(x = 5^{-x}\)
        \task \(x = 6^{-x}\)
        \task \(x = \num{0.5}(\sin{x} + \cos{x})\)
    \end{tasks}
\end{exercise}

\begin{solution}
    \begin{enumerate}[label = \alph*)]
        \item Let

            \begin{align*}
                &\quad&         g(x) &= \frac{2 - e^x + x^2}{3} \\
                \Rightarrow&&  g'(x) &= \frac{2x - e^x}{3} \\
                \Rightarrow&& g''(x) &= \frac{2 - e^x}{3}
            \end{align*}

            It is clear that \(g\) is continuous in \(\mathbb{R}\).

            Consider \(g''\):

            \begin{itemize}
                \item \(g''(x) > 0 \iff x < \ln{2}\)
                \item \(g''(x) = 0 \iff x = \ln{2}\)
                \item \(g''(x) < 0 \iff x > \ln{2}\)
            \end{itemize}

            So, \(\max g'(x) = g'(\ln{2}) = \dfrac{\ln{4} - 2}{3} < 0\). So \(g\)
            is monotonically decreasing in \(\mathbb{R}\).

            Consider the interval \(I = \interval{0}{1}\):

            \[\]
            \begin{align*}
                1 > g(0) = \frac{1}{3} > &g(x) > g(1) = \frac{3 - e}{3} > 0 \, \forall x \in I \\
                             \Rightarrow &g(x) \in I \, \forall x \in I
            \end{align*}

            So, \(I\) is an interval in which a fixed point \(p\) of \(g\)
            exists. Applying fixed-point method on \(g\) with \(p_0 =
            \num{0.5}\) generates the following table:

            \begin{table}[H]
                \centering
                \begin{tabular}{r S[table-format=1.9] r S[table-format=1.9]}
                    \toprule
                    \(n\)  &   {\(p_n\)}   &  \(n\)  &   {\(p_n\)}   \\
                    \midrule
                        0  &  0.5          &      5  &  0.257265636  \\
                        1  &  0.200426243  &      6  &  0.257598985  \\
                        2  &  0.272749065  &      7  &  0.257512455  \\
                        3  &  0.253607157  &      8  &  0.257534914  \\
                        4  &  0.258550376  &      9  &  0.257529084  \\
                    \bottomrule
                \end{tabular}
            \end{table}

            We conclude that the fixed point \(p \approx \num{0.257529}\).

        \item Let

            \[g = \frac{5}{x^2} + 2\]

            Consider the interval \(I = \interval{\num{2.5}}{3}\). \(0 \notin
            I\), so \(g\) is continuous in \(I\).

            \(x^2\) is monotonically increasing in \(I\), so \(g\) is
            monotonically decreasing in \(I\). So that:

            \begin{align*}
                3 > g(\num{2.5}) = \num{2.8} > &g(x) > g(3) = \sfrac{23}{9} > \num{2.5} \, \forall x \in I \\
                                   \Rightarrow &g(x) \in I \, \forall x \in I
            \end{align*}

            So, \(I\) is an interval in which a fixed point \(p\) of \(g\)
            exists. Applying fixed-point method on \(g\) with \(p_0 =
            \num{2.75}\) generates the following table:

            \begin{table}[H]
                \centering
                \begin{tabular}{r S[table-format=1.8] r S[table-format=1.8] r S[table-format=1.8]}
                    \toprule
                    \(n\)  &   {\(p_n\)}   &  \(n\)  &   {\(p_n\)}   &  \(n\)  &   {\(p_n\)}   \\
                    \midrule
                        0  &  2.75         &      6  &  2.69171092   &     12  &  2.69066691   \\
                        1  &  2.66115702   &      7  &  2.69010182   &     13  &  2.69063746   \\
                        2  &  2.7060395    &      8  &  2.69092764   &     14  &  2.69065258   \\
                        3  &  2.68281293   &      9  &  2.69050363   &     15  &  2.69064482   \\
                        4  &  2.69468708   &     10  &  2.69072129   &         &               \\
                        5  &  2.68857829   &     11  &  2.69060954   &         &               \\
                    \bottomrule
                \end{tabular}
            \end{table}

            We conclude that the fixed point \(p \approx \num{2.690645}\).

        \item Let

            \[g(x) = \left(\frac{e^x}{3}\right)^{\sfrac{1}{2}}\]

            It is clear that \(g\) is continuous in \(\mathbb{R}\).

            \(g\) is monotonically increasing in \(\mathbb{R}\). Consider the
            interval \(I = \interval{0}{1}\):

            \begin{align*}
                0 < g(0) = \dfrac{1}{\sqrt{3}} < &g(x) < g(1) = \sqrt{\dfrac{e}{3}} < 1 \\
                                     \Rightarrow &g(x) \in I \, \forall x \in I
            \end{align*}

            So, \(I\) is an interval in which a fixed point \(p\) of \(g\)
            exists. Applying fixed-point method on \(g\) with \(p_0 =
            \num{0.5}\) generates the following table:

            \begin{table}[H]
                \centering
                \begin{tabular}{r S[table-format=1.9] r S[table-format=1.9] r S[table-format=1.9]}
                    \toprule
                    \(n\)  &   {\(p_n\)}   &  \(n\)  &   {\(p_n\)}   &  \(n\)  &   {\(p_n\)}   \\
                    \midrule
                        0  &  0.5          &      5  &  0.903281143  &     10  &  0.909876791  \\
                        1  &  0.74133242   &      6  &  0.906952163  &     11  &  0.909948068  \\
                        2  &  0.836407007  &      7  &  0.908618411  &     12  &  0.909980498  \\
                        3  &  0.87712774   &      8  &  0.909375718  &     13  &  0.909995254  \\
                        4  &  0.895169428  &      9  &  0.909720122  &     14  &  0.910001967  \\
                    \bottomrule
                \end{tabular}
            \end{table}

            We conclude that the fixed point \(p \approx \num{0.910002}\).

        \item Let \(g(x) = 5^{-x}\). It is clear that \(g\) is continuous in
            \(\mathbb{R}\).

            \(5^x\) is monotonically increasing in \(\mathbb{R}\), so \(g\) is
            monotonically decreasing in \(\mathbb{R}\).

            Consider the interval \(I = \interval{0}{1}\):

            \begin{align*}
                0 < g(1) = \num{0.2} < &g(x) < g(0) = 1 \\
                           \Rightarrow &g(x) \in I \, \forall x \in I
            \end{align*}

            So, \(I\) is an interval in which a fixed point \(p\) of \(g\)
            exists. Applying fixed-point method on \(g\) with \(p_0 =
            \num{0.5}\) generates the following table:

            \begin{longtable}{r S[table-format=1.9] r S[table-format=1.9] r S[table-format=1.9]}
                \toprule
                \(n\)  &   {\(p_n\)}   &  \(n\)  &   {\(p_n\)}   &  \(n\)  &   {\(p_n\)}   \\
                \midrule
                \endfirsthead
                \(n\)  &   {\(p_n\)}   &  \(n\)  &   {\(p_n\)}   &  \(n\)  &   {\(p_n\)}   \\
                \midrule
                \endhead
                    0  &  0.5          &     11  &  0.468245559  &     22  &  0.469685261  \\
                    1  &  0.447213595  &     12  &  0.470663369  &     23  &  0.469574052  \\
                    2  &  0.486867866  &     13  &  0.468835429  &     24  &  0.469658106  \\
                    3  &  0.456766207  &     14  &  0.470216753  &     25  &  0.469594575  \\
                    4  &  0.479439843  &     15  &  0.469172549  &     26  &  0.469642593  \\
                    5  &  0.462259591  &     16  &  0.469961695  &     27  &  0.4696063    \\
                    6  &  0.475219673  &     17  &  0.469365184  &     28  &  0.469633731  \\
                    7  &  0.465409992  &     18  &  0.469816013  &     29  &  0.469612998  \\
                    8  &  0.47281623   &     19  &  0.469475247  &     30  &  0.469628669  \\
                    9  &  0.467213774  &     20  &  0.469732798  &     31  &  0.469616824  \\
                   10  &  0.4714456    &     21  &  0.469538128  &     32  &  0.469625777  \\
                \bottomrule
            \end{longtable}

            We conclude that the fixed point \(p \approx \num{0.469626}\).

        \item Let \(g(x) = 6^{-x}\). It is clear that \(g\) is continuous in
            \(\mathbb{R}\).

            \(6^x\) is monotonically increasing in \(\mathbb{R}\), so \(g\) is
            monotonically decreasing in \(\mathbb{R}\).

            Consider the interval \(I = \interval{0}{1}\):

            \begin{align*}
                0 < g(1) = \frac{1}{6} < &g(x) < g(0) = 1 \\
                             \Rightarrow &g(x) \in I \, \forall x \in I
            \end{align*}

            So, \(I\) is an interval in which a fixed point \(p\) of \(g\)
            exists. Applying fixed-point method on \(g\) with \(p_0 =
            \num{0.5}\) generates the following table:

            \begin{longtable}{r S[table-format=1.9] r S[table-format=1.9] r S[table-format=1.9]}
                \toprule
                \(n\)  &   {\(p_n\)}   &  \(n\)  &   {\(p_n\)}   &  \(n\)  &   {\(p_n\)}   \\
                \midrule
                \endfirsthead
                \(n\)  &   {\(p_n\)}   &  \(n\)  &   {\(p_n\)}   &  \(n\)  &   {\(p_n\)}   \\
                \midrule
                \endhead
                    0  &  0.5          &     15  &  0.446190464  &     30  &  0.448132603  \\
                    1  &  0.40824829   &     16  &  0.449568975  &     31  &  0.448007263  \\
                    2  &  0.481194974  &     17  &  0.446855739  &     32  &  0.448107887  \\
                    3  &  0.422238208  &     18  &  0.449033402  &     33  &  0.448027103  \\
                    4  &  0.469282988  &     19  &  0.447284756  &     34  &  0.448091958  \\
                    5  &  0.431347074  &     20  &  0.448688365  &     35  &  0.448039891  \\
                    6  &  0.461686032  &     21  &  0.447561363  &     36  &  0.448081691  \\
                    7  &  0.437258678  &     22  &  0.448466044  &     37  &  0.448048133  \\
                    8  &  0.456821582  &     23  &  0.447739682  &     38  &  0.448075074  \\
                    9  &  0.441086448  &     24  &  0.44832278   &     39  &  0.448053445  \\
                   10  &  0.453699216  &     25  &  0.44785463   &     40  &  0.448070809  \\
                   11  &  0.443561035  &     26  &  0.448230453  &     41  &  0.448056869  \\
                   12  &  0.451692029  &     27  &  0.447928723  &     42  &  0.44806806   \\
                   13  &  0.445159128  &     28  &  0.448170951  &     43  &  0.448059076  \\
                   14  &  0.450400504  &     29  &  0.447976481  &         &               \\
                \bottomrule
            \end{longtable}

            We conclude that the fixed point \(p \approx \num{0.448059}\).

        \item Let \(g(x) = \num{0.5}(\sin{x} + \cos{x})\). It is clear that \(g\)
            is continuous in \(\mathbb{R}\).

            Manipulating \(g\) gives:

            \begin{align*}
                \sin{x} + \cos{x} &= \sqrt{2} \left(\frac{1}{\sqrt{2}} \sin{x} + \frac{1}{\sqrt{2}} \cos{x}\right) \\
                                  &= \sqrt{2} \left(\cos{\frac{\pi}{4}} \sin{x} + \sin{\frac{\pi}{4}} \cos{x}\right) \\
                                  &= \sqrt{2} \sin \left(x + \frac{\pi}{4}\right) \\
                \Rightarrow  g(x) &= \num{0.5}(\sin{x} + \cos{x}) \\
                                  &= \frac{1}{\sqrt{2}} \sin \left(x + \frac{\pi}{4}\right)
            \end{align*}

            Consider the interval \(I = \interval{0}{\frac{\pi}{4}}\).
            \(sin{x}\) is monotonically increasing in
            \(\interval{0}{\frac{\pi}{2}}\), so \(\sin{x + \frac{\pi}{4}}\) also
            is monotonically increasing in \(I\). It follows that:

            \begin{align*}
                0 < g(0) = \num{0.5} < g(x) < &g(\frac{\pi}{4}) = \frac{1}{\sqrt{2}} < \frac{\pi}{4} \\
                                  \Rightarrow &g(x) \in I \, \forall x \in I
            \end{align*}

            So, \(I\) is an interval in which a fixed point \(p\) of \(g\)
            exists. Applying fixed-point method on \(g\) with \(p_0 =
            \frac{\pi}{8}\) generates the following table:

            \begin{table}[H]
                \centering
                \begin{tabular}{r S[table-format=1.9] r S[table-format=1.9]}
                    \toprule
                    \(n\)  &   {\(p_n\)}   &  \(n\)  &   {\(p_n\)}   \\
                    \midrule
                        0  &  0.392699082  &      4  &  0.704799153  \\
                        1  &  0.653281482  &      5  &  0.704811271  \\
                        2  &  0.700944543  &      6  &  0.70481196   \\
                        3  &  0.70458659   &         &               \\
                    \bottomrule
                \end{tabular}
            \end{table}

            We conclude that the fixed point \(p \approx \num{0.704812}\).
    \end{enumerate}
\end{solution}

\begin{exercise}
    For each of the following equations, use the given interval or determine an
    interval \(\interval{a}{b}\) on which fixed-point iteration will converge.
    Estimate the number of iterations necessary to obtain approximations
    accurate to within \(10^{-5}\), and perform the calculations.

    \begin{tasks}(2)
        \task \(2 + \sin{x} - x = 0\) on \(\interval{2}{3}\)
        \task \(x^3 - 3x - 5 = 0\) on \(\interval{2}{3}\)
        \task \(3x^2 - e^x = 0\)
        \task \(x - \cos{x} = 0\)
    \end{tasks}
\end{exercise}

\begin{solution}
    \begin{enumerate}[label = \alph*)]
        \item Let \(I = \interval{2}{3}\) and

            \begin{align*}
                             g(x) &= \sin{x} + 2 \\
                \Rightarrow g'(x) &= \cos{x}
            \end{align*}

            A fixed point \(p\) of \(g\) is also a root of the problem.

            Consider \(g\). It is clear that \(g\) is continuous on
            \(\mathbb{R}\). \(\sin{x}\) is monotonically decreasing in \(I\), so
            that:

            \[2 < g(3) = \sin{3} + 2 < g(x) < g(2) = \sin{2} + 2 < 3\]

            Consider \(g'\). \(\cos{x}\) is monotonically decreasing in \(I\),
            so that:

            \begin{gather*}
                \cos{3} \leq g'(x) \leq \cos{2} < 0 \, \forall x \in I \\
                \Rightarrow \abs{g'(x)} \leq k = - \cos{3} < 1
            \end{gather*}

            Therefore, all the conditions in Corollary 2.5 hold. Using Corollary
            2.5, with \(p_0 = \num{2.5}\), the number of iteration \(n\)
            required to obtain approximations accurate to within \(10^{-5}\) is:

            \[\abs{p_n - p} \leq k^n \num{0.5} < 10^{-5} \iff n \geq 1076\]

            Applying fixed-point method on \(g\) generates the following table:

            \begin{longtable}{r S[table-format=1.8] r S[table-format=1.8] r S[table-format=1.8]}
                \toprule
                \(n\)  &   {\(p_n\)}   &  \(n\)  &   {\(p_n\)}   &  \(n\)  &   {\(p_n\)}   \\
                \midrule
                \endfirsthead
                \(n\)  &   {\(p_n\)}   &  \(n\)  &   {\(p_n\)}   &  \(n\)  &   {\(p_n\)}   \\
                \midrule
                \endhead
                    0  &  2.5          &     18  &  2.55222543   &     36  &  2.55412346   \\
                    1  &  2.59847214   &     19  &  2.55583511   &     37  &  2.55425629   \\
                    2  &  2.51680997   &     20  &  2.5528308    &     38  &  2.55414573   \\
                    3  &  2.58492102   &     21  &  2.55533177   &     39  &  2.55423776   \\
                    4  &  2.52836328   &     22  &  2.55325015   &     40  &  2.55416115   \\
                    5  &  2.57551141   &     23  &  2.55498297   &     41  &  2.55422492   \\
                    6  &  2.5363287    &     24  &  2.55354068   &     42  &  2.55417184   \\
                    7  &  2.56897915   &     25  &  2.55474128   &     43  &  2.55421602   \\
                    8  &  2.54183051   &     26  &  2.55374195   &     44  &  2.55417925   \\
                    9  &  2.56444615   &     27  &  2.5545738    &     45  &  2.55420986   \\
                   10  &  2.54563487   &     28  &  2.5538814    &     46  &  2.55418438   \\
                   11  &  2.56130168   &     29  &  2.55445776   &     47  &  2.55420559   \\
                   12  &  2.5482673    &     30  &  2.55397801   &     48  &  2.55418793   \\
                   13  &  2.55912111   &     31  &  2.55437735   &     49  &  2.55420263   \\
                   14  &  2.55008961   &     32  &  2.55404495   &     50  &  2.5541904    \\
                   15  &  2.55760933   &     33  &  2.55432164   &     51  &  2.55420058   \\
                   16  &  2.55135148   &     34  &  2.55409133   &     52  &  2.5541921    \\
                   17  &  2.55656141   &     35  &  2.55428304   &         &               \\
               \bottomrule
            \end{longtable}

            So one solution of the problem is \(p \approx \num{2.554192}\).

        \item Let \(I = \interval{2}{3}\) and

            \begin{align*}
                             g(x) &= \sqrt[3]{2x + 5} \\
                \Rightarrow g'(x) &= \dfrac{2}{3} (2x + 5)^{- \sfrac{2}{3}}
            \end{align*}

            A fixed point \(p\) of \(g\) is also a solution of the problem.

            Consider \(g\). It is clear that \(g\) is continuous and
            monotonically increasing on \(\mathbb{R}\), so that:

            \begin{align*}
                2 < g(2) = \sqrt[3]{9} < &g(x) < g(3) = \sqrt[3]{11} < 3 \\
                             \Rightarrow &g(x) \in I \, \forall x \in I
            \end{align*}

            Consider \(g'\). Since \(- \sfrac{2}{3} < 0\) and \(I > 0\),
            \(g'(x)\) is monotonically decreasing in \(I\), so that:

            \begin{gather*}
                g'(2) = \frac{2}{9 \sqrt[3]{3}} \geq g'(x) \geq g'(3) = \frac{2}{3 \sqrt[3]{121}} \\
                \Rightarrow \abs{g'(x)} \leq k = \frac{2}{9 \sqrt[3]{3}} < 1
            \end{gather*}

            Therefore, all the conditions in Corollary 2.5 hold. Using Corollary
            2.5, with \(p_0 = \num{2.5}\), the number of iteration \(n\)
            required to obtain approximations accurate to within \(10^{-5}\) is:

            \[\abs{p_n - p} \leq k^n \num{0.5} < 10^{-5} \iff n \geq 6\]

            Applying fixed-point method on \(g\) generates the following table:

            \begin{table}[H]
                \centering
                \begin{tabular}{r S[table-format=1.8] r S[table-format=1.8]}
                    \toprule
                    \(n\)  &   {\(p_n\)}   &  \(n\)  &   {\(p_n\)}   \\
                    \midrule
                        0  &  2.5          &      4  &  2.09476055   \\
                        1  &  2.15443469   &      5  &  2.09458325   \\
                        2  &  2.10361203   &      6  &  2.09455631   \\
                        3  &  2.09592741   &      7  &  2.09455222   \\
                    \bottomrule
                \end{tabular}
            \end{table}

            So one solution of the problem is \(p \approx \num{2.094552}\).

        \item Let \(I = \interval{3}{4}\) and

            \begin{align*}
                             g(x) &= \ln{3x^2} = 2 \ln{x} + \ln{3} \\
                \Rightarrow g'(x) &= \frac{2}{x}
            \end{align*}

            A fixed point \(p\) of \(g\) is also a solution of the problem.

            Consider \(g\). It is clear that \(g\) is continuous and
            monotonically increasing on \(I\), so that:

            \begin{align*}
                3 < g(3) = \ln{27} < &g(x) < g(4) = \ln{48} < 4 \\
                         \Rightarrow &g(x) \in I \, \forall x \in I
            \end{align*}

            Consider \(g'\). Since \(I > 0\), \(g'(x)\) is monotonically
            decreasing in \(I\), so that:

            \begin{gather*}
                g'(3) = \frac{2}{3} \geq g'(x) \geq g'(4) = \frac{1}{2} \\
                \Rightarrow \abs{g'(x)} \leq k = \frac{2}{3} < 1
            \end{gather*}

            Therefore, all the conditions in Corollary 2.5 hold. Using Corollary
            2.5, with \(p_0 = \num{3.5}\), the number of iteration \(n\)
            required to obtain approximations accurate to within \(10^{-5}\) is:

            \[\abs{p_n - p} \leq k^n \num{0.5} < 10^{-5} \iff n \geq 27\]

            Applying fixed-point method on \(g\) generates the following table:

            \begin{table}[H]
                \centering
                \begin{tabular}{r S[table-format=1.8] r S[table-format=1.8] r S[table-format=1.8]}
                    \toprule
                    \(n\)  &   {\(p_n\)}   &  \(n\)  &   {\(p_n\)}   &  \(n\)  &   {\(p_n\)}   \\
                    \midrule
                        0  &  3.5          &      6  &  3.72717712   &     12  &  3.73293923   \\
                        1  &  3.60413823   &      7  &  3.72991458   &     13  &  3.73300413   \\
                        2  &  3.66277767   &      8  &  3.73138295   &     14  &  3.7330389    \\
                        3  &  3.69505586   &      9  &  3.73217015   &     15  &  3.73305753   \\
                        4  &  3.71260363   &     10  &  3.73259204   &     16  &  3.73306751   \\
                        5  &  3.72207913   &     11  &  3.7328181    &         &               \\
                    \bottomrule
                \end{tabular}
            \end{table}

            So one solution of the problem is \(p \approx \num{3.733068}\).

        \item Let \(I = \interval{0}{1}\) and

            \begin{align*}
                             g(x) &= \cos{x} \\
                \Rightarrow g'(x) &= - \sin{x}
            \end{align*}

            A fixed point \(p\) of \(g\) is also a solution of the problem.

            Consider \(g\). It is clear that \(g\) is continuous and
            monotonically decreasing on \(I\), so that:

            \begin{align*}
                1 = g(0) \geq &g(x) \geq g(1) = \cos{1} > 0  \\
                  \Rightarrow &g(x) \in I \, \forall x \in I
            \end{align*}

            Consider \(g'\). Since \(I > 0\), \(g'(x)\) is monotonically
            decreasing in \(I\), so that:

            \begin{gather*}
                g'(0) = 0 \geq g'(x) \geq g'(1) = - \sin{1} \\
                \Rightarrow \abs{g'(x)} \leq k = \sin{1} < 1
            \end{gather*}

            Therefore, all the conditions in Corollary 2.5 hold. Using Corollary
            2.5, with \(p_0 = \num{0.5}\), the number of iteration \(n\)
            required to obtain approximations accurate to within \(10^{-5}\) is:

            \[\abs{p_n - p} \leq k^n \num{0.5} < 10^{-5} \iff n \geq 63\]

            Applying fixed-point method on \(g\) generates the following table:

            \begin{longtable}{r S[table-format=1.9] r S[table-format=1.9] r S[table-format=1.9]}
                \toprule
                \(n\)  &   {\(p_n\)}   &  \(n\)  &   {\(p_n\)}   &  \(n\)  &   {\(p_n\)}   \\
                \midrule
                \endfirsthead
                \(n\)  &   {\(p_n\)}   &  \(n\)  &   {\(p_n\)}   &  \(n\)  &   {\(p_n\)}   \\
                \midrule
                \endhead
                    0  &  0.5          &     10  &  0.735006309  &     20  & 0.73900678    \\
                    1  &  0.877582562  &     11  &  0.741826523  &     21  & 0.739137911   \\
                    2  &  0.639012494  &     12  &  0.737235725  &     22  & 0.739049581   \\
                    3  &  0.802685101  &     13  &  0.740329652  &     23  & 0.739109081   \\
                    4  &  0.694778027  &     14  &  0.738246238  &     24  & 0.739069001   \\
                    5  &  0.768195831  &     15  &  0.739649963  &     25  & 0.739096      \\
                    6  &  0.719165446  &     16  &  0.738704539  &     26  & 0.739077813   \\
                    7  &  0.752355759  &     17  &  0.739341452  &     27  & 0.739090064   \\
                    8  &  0.730081063  &     18  &  0.738912449  &     28  & 0.739081812   \\
                    9  &  0.745120341  &     19  &  0.739201444  &         &               \\
               \bottomrule
            \end{longtable}

            So one root of the problem is \(p \approx \num{0.739082}\).
    \end{enumerate}
\end{solution}

\begin{exercise}
    Find all the zeros of \(f(x) = x^2 + 10 \cos{x}\) by using the fixed-point
    iteration method for an appropriate iteration function \(g\). Find the zeros
    accurate to within \(10^{-4}\).
\end{exercise}

\begin{solution}
    Consider \(f = 0\). Since \(x^2 \geq 0\), \(\cos{x}\) must be negative for
    the equation to hold, so that:

    \begin{equation*}\label{eq:exer:2.2.13:bound_x_by_cos}
        x \in I_k = \interval{\frac{\pi}{2} + k 2 \pi}{\frac{3 \pi}{2} + k 2 \pi} \, \forall k \in \mathbb{N} \tag{1}
    \end{equation*}

    Also, since \(10 \cos{x} \in \interval{-10}{0}\):

    \begin{equation*}\label{eq:exer:2.2.13:bound_x_by_x^2}
        x \in \interval{- \sqrt{10}}{\sqrt{10}} \tag{2}
    \end{equation*}

    Combining \eqref{eq:exer:2.2.13:bound_x_by_cos} and
    \eqref{eq:exer:2.2.13:bound_x_by_x^2} gives:

    \[x \in I = I_a \cup I_b \text{ where } I_a = \interval{- \sqrt{10}}{- \frac{\pi}{2}} \text{ and } I_b = \interval{\frac{\pi}{2}}{\sqrt{10}}\]

    As \(x^2\) and \(\cos{x}\) take \(Oy\) as a symmetry axis, each zero \(z_b\)
    of \(f\) in \(I_b\) results in another zero \(z_a = - z_b\) in \(I_a\).
    Hence, from now on, we just need to examine on \(I_b\).

    Differentiating \(f\) gives:

    \[f'(x) = 2x - 10 \sin{x}\]

    \(x\) is monotonically increasing on \(I_b\), \(\sin{x}\) is monotonically
    decreasing on \(I_b\). It follows that \(f'\) is monotonically increasing on
    \(I_b\), which means:

    \[f'(\frac{\pi}{2}) = \pi - 10 \leq f'(x) \leq f'(\sqrt{10}) = 2 \sqrt{10} - 10 \sin{\sqrt{10}}\]

    Combining with the fact that \(f'\) is continuous on \(I_b\), according to
    Intermediate Value Theorem, \(f'\) has one zero in \(I_b\). It follows that
    \(f\) has at most two zeros in \(I_b\).

    Let

    \[g(x) = x - \frac{-10 \cos{x}}{x^2} + 1 = x + \frac{10 \cos{x}}{x^2} + 1\]

    A fixed point of \(g\) is also a zero of \(f\). \emph{Try} applying
    fixed-point method on \(g\) with several \(p_0\), we found two fixed points:

    \begin{itemize}
        \item \(p_0 = \frac{\pi}{2}\) generates the following table:

            \begin{table}[H]
                \centering
                \begin{tabular}{r S[table-format=1.8] r S[table-format=1.8] r S[table-format=1.8]}
                    \toprule
                    \(n\)  &   {\(p_n\)}   &  \(n\)  &   {\(p_n\)}   &  \(n\)  &   {\(p_n\)}   \\
                    \midrule
                        0  &  1.57079633   &      4  &  1.95354867   &      8  &  1.96859328   \\
                        1  &  2.57079633   &      5  &  1.9749308    &      9  &  1.96897439   \\
                        2  &  2.29757529   &      6  &  1.96675733   &     10  &  1.96883622   \\
                        3  &  2.03884343   &      7  &  1.96964871   &     11  &  1.96888624   \\
                    \bottomrule
                \end{tabular}
            \end{table}

        \item \(p_0 = -\sqrt{10}\) generates the following table:

            \begin{table}[H]
                \centering
                \begin{tabular}{r S[table-format=-1.8]}
                    \toprule
                    \(n\)  &   {\(p_n\)}   \\
                    \midrule
                        0  &  -3.16227766  \\
                        1  &  -3.16206373  \\
                        2  &  -3.16198949  \\
                    \bottomrule
                \end{tabular}
            \end{table}
    \end{itemize}

    The second fixed point is interesting. It is indeed a fixed point of \(g\),
    a zero of \(f\), but it belongs to \(I_a\). Due to the symmetry property, we
    conclude that \(f\) has 4 zeros: \(\pm \num{1.96889}\) and \(\pm
    \num{3.16199}\).
\end{solution}

\begin{exercise}
    Use a fixed-point iteration method to determine a solution accurate to
    within \(10^{-4}\) for \(x = \tan{x}\), for \(x \in \interval{4}{5}\).
\end{exercise}

\begin{solution}
    Let

    \[g(x) = x - \sqrt[3]{\frac{\tan{x}}{x}} + 1\]

    A fixed point \(p\) of \(g\) is also a solution of the problem. Applying
    fixed-point method on \(g\) generates the following table:

    \begin{table}[H]
        \centering
        \begin{tabular}{r S[table-format=1.8] r S[table-format=1.8] r S[table-format=1.8]}
            \toprule
            \(n\)  &   {\(p_n\)}   &  \(n\)  &   {\(p_n\)}   &  \(n\)  &   {\(p_n\)}   \\
            \midrule
                0  &  4            &      4  &  4.49534411   &      8  &  4.49352955   \\
                1  &  4.33850407   &      5  &  4.49242947   &      9  &  4.49334961   \\
                2  &  4.50097594   &      6  &  4.49389301   &     10  &  4.49343923   \\
                3  &  4.48937873   &      7  &  4.4931677    &         &               \\
            \bottomrule
        \end{tabular}
    \end{table}

    So \(p \approx \num{4.49344}\) is a solution of the problem in
    \(\interval{4}{5}\).
\end{solution}

\begin{exercise}
    Use a fixed-point iteration method to determine a solution accurate to
    within \(10^{-2}\) for \(2 \sin{\pi x} + x = 0\) on \(\interval{1}{2}\). Use
    \(p_0 = 1\).
\end{exercise}

\begin{solution}
    Consider \(f\):

    \begin{align*}
        &\quad&         f(x) &= 0 \\
        \iff&& 2 \sin{\pi x} &= -x \\
        \iff&&         \pi x &= \arcsin{\num{-0.5} x} + k 2 \pi \text{ (\(k \in \mathbb{N}\))} \\
        \iff&&             x &= \frac{\arcsin{\num{-0.5} x}}{\pi} + 2k
    \end{align*}

    Let

    \[g(x) = \frac{\arcsin\num{-0.5} x}{\pi} + 2\]

    \(\arcsin\) is chosen as it ``behaves'' nicer than normal \(\sin\). Since
    \(\arcsin\) returns values in principal branch
    \(\interval{-\frac{\pi}{2}}{\frac{\pi}{2}}\), we need to use \(k = 1\) to
    shift the value to cover \(\interval{1}{2}\).

    A fixed point \(p\) of \(g\) is also a solution of the problem. Applying
    fixed-point method on \(g\) generates the following table:

    \begin{table}[H]
        \centering
        \begin{tabular}{r S[table-format=1.8] r S[table-format=1.8]}
            \toprule
            \(n\)  &   {\(p_n\)}   &  \(n\)  &   {\(p_n\)}   \\
            \midrule
                0  &  1            &      3  &  1.696498     \\
                1  &  1.83333333   &      4  &  1.67765706   \\
                2  &  1.63086925   &      5  &  1.68324099   \\
            \bottomrule
        \end{tabular}
    \end{table}

    So \(p \approx \num{1.683}\) is a solution of the problem in
    \(\interval{1}{2}\).

\end{solution}

\begin{exercise}
    Let \(A\) be a given positive constant and \(g(x) = 2x - Ax^2\).

    \begin{tasks}
        \task Show that if fixed-point iteration converges to a nonzero limit,
            then the limit is \(p = \sfrac{1}{A}\), so the inverse of a number
            can be found using only multiplications and subtractions.

        \task Find an interval about \(\sfrac{1}{A}\) for which fixed-point
            iteration converges, provided \(p_0\) is in that interval.
    \end{tasks}
\end{exercise}

\begin{solution}
    \begin{enumerate}[label = \alph*)]
        \item If fixed-point iteration converges to a nonzero limit \(p\), then:

            \begin{align*}
                     p &= \lim_{n \to \infty} p_n \\
                       &= \lim_{n \to \infty} g(p_{n - 1}) \\
                       &= \lim_{n \to \infty} \left(2p_{n - 1} - Ap_{n - 1}^2\right) \\
                       &= 2p - Ap^2 \\
                \iff p &= Ap^2 \iff  p = \frac{1}{A}
            \end{align*}

        \item We try to find \(\delta > 0\) such that fixed-point method
            converges on \(I = \interval{\sfrac{1}{A} - \delta}{\sfrac{1}{A} +
            \delta}\) using Fixed Point Theorem.

            The condition that \(g\) is continuous on \(I\) is satisfied with
            any \(\delta\).

            Consider \(g\):

            \[g(x) = -Ax^2 + 2x = -A \left(x - \frac{1}{A}\right)^2 + \frac{1}{A}\]

            So \(x = \frac{1}{A}\) is the axis of symmetry for \(g\).

            Differentiating \(g\) gives:

            \[g'(x) = 2 - 2Ax\]

            It follows that:

            \begin{itemize}
                \item \(g'(x) < 0 \iff x > \frac{1}{A}\)
                \item \(g'(x) = 0 \iff x = \frac{1}{A}\)
                \item \(g'(x) > 0 \iff x < \frac{1}{A}\)
            \end{itemize}

            Combining with the fact that \(x = \frac{1}{A}\) is the symmetry
            axis of \(g\) gives:

            \begin{gather*}
                g \left(\frac{1}{A} + \delta\right) = g \left(\frac{1}{A} - \delta\right) = g \left(\frac{1}{A} \pm \delta\right) \leq g(x) \leq g \left(\frac{1}{A}\right) \, \forall x \in I \\
                \iff \frac{2}{A} - A \delta^2 \leq g(x) \leq \frac{1}{A}
            \end{gather*}

            Then, to satisfy the condition that \(g(x) \in I \, \forall x \in
            I\), \(\delta\) must satisfy the following:

            \begin{align*}
                &\quad&   \frac{2}{A} - A \delta^2 &\geq \frac{1}{A} - \delta \\
                \iff&& (A \delta)^2 - A \delta - 1 &\leq 0 \\
                \iff&&                  0 < \delta &\leq \frac{1 + \sqrt{5}}{2A} \text{ (as \(\delta > 0\))} \tag{1}\label{eq:exer:2.2.17:1}
            \end{align*}

            Consider \(g'\). \(g'\) is monotonically decreasing on \(\mathbb{R}\), so:

            \begin{gather*}
                g' \left(\frac{1}{A} - \delta\right) = 2 A \delta \geq g'(x) \geq g' \left(\frac{1}{A} - \delta\right) = -2 A \delta \\
                \iff \abs{g'(x)} \leq 2 A \delta \text{ (equal sign only at either end)} \tag{2}\label{eq:exer:2.2.17:2}
            \end{gather*}

            Then, to satisfy the condition that \(\abs{g'(x)} < 1 \, \forall x
            \in I_{open} = \interval[open]{\sfrac{1}{A} - \delta}{\sfrac{1}{A} +
            \delta}\), \(\delta\) must satisfy the following:

            \[2 A \delta \leq 1 \iff \delta \leq \frac{1}{2A}\]

            From \eqref{eq:exer:2.2.17:1} and \eqref{eq:exer:2.2.17:2}:

            \[0 < \delta < \frac{1}{2A}\]

            As all the conditions needed for Fixed Point Theorem hold, we
            conclude that for any \(\delta \in \interval[open
            left]{0}{\frac{1}{2A}}\), applying fixed-point method on \(g\) with
            \(p_0 \in I\) converges to the fixed point.
    \end{enumerate}
\end{solution}

\begin{exercise}
    Find a function \(g\) defined on \(\interval{0}{1}\) that satisfies none of
    the hypotheses of Theorem 2.3 but still has a unique fixed point on
    \(\interval{0}{1}\).
\end{exercise}

\begin{solution}
    Let \(I = \interval{0}{1}\), \(g = \dfrac{1}{x + \num{0.5}}\).

    Consider \(g\). \(g\) is defined on \(\mathbb{R} \setminus \{\num{-0.5}\}\),
    so it is defined on \(I\).

    \(g(x) > 1 \, \forall x \in \interval{\num{-0.5}}{\num{0.5}}\), so the
    condition that \(g(x) \in I \, \forall x \in I\) does not hold.

    Differentiating \(g\) gives:

    \[g'(x) = - \dfrac{1}{(x + \num{0.5})^2} < -1 \iff x \in \interval[open]{\num{-1.5}}{\num{0.5}} \setminus \{\num{-0.5}\}\]

    So the condition that \(\abs{g'(x)} < 1 \, \forall x \in I\) does not hold.

    Yet, \(g\) has a fixed point at \(x = \dfrac{\sqrt{17} - 1}{4}\).
\end{solution}

\begin{exercise}
    \begin{tasks}
        \task Show that Theorem 2.2 is true if the inequality \(\abs{g'(x)} \leq
            k\) is replaced by \(g'(x) \leq k\), for all \(x \in
            \interval[open]{a}{b}\). [Hint: Only uniqueness is in question.]

        \task Show that Theorem 2.3 may not hold if inequality \(\abs{g'(x)}
            \leq k\) is replaced by \(g'(x) \leq k\).
    \end{tasks}
\end{exercise}

\begin{solution}
    \begin{enumerate}[label = \alph*)]
        \item Where the fuck is Theorem 2.2 in the fucking book?
        \item In the proof of Theorem 2.3, if \(\abs{g'(x) \leq k}\) is replaced
            with \(g'(x) \leq k\), then there is a chance that \(g'(\xi) = -1\).
            In that case, the assumption is no longer a contradiction, therefore
            the proof is invalid, and the theorem doesn't hold.
    \end{enumerate}
\end{solution}

\begin{exercise}
    \begin{tasks}
        \task Use Theorem 2.4 (Định lí 2.5 in the accompanying Lectures.pdf) to
            show that the sequence defined by:

            \[x_n = \frac{1}{2} x_{n - 1} + \frac{1}{x_{n - 1}} \text{, for \(n \leq 1\)}\]

            converges to \(\sqrt{2}\) whenever \(x_0 > \sqrt{2}\).

        \task Use the fact that \(0 < (x_0 - \sqrt{2})^2\) whenever \(x_0 \neq
            \sqrt{2}\) to show that if \(0 < x_0 < \sqrt{2}\), then \(x_1 >
            \sqrt{2}\).

        \task Use the above results to show that the sequence in (a) converges
            to \(\sqrt{2}\) whenever \(x_0 > 0\).
    \end{tasks}
\end{exercise}

\begin{solution}
    \begin{enumerate}[label = \alph*)]
        \item Let \(g\) be the function that generates the sequence \(\{x_n\}\):

            \begin{align*}
                             g(x) &= \frac{x}{2} + \frac{1}{x} = \frac{x^2 + 2}{2x} \\
                \Rightarrow g'(x) &= \frac{1}{2} - \frac{1}{x^2} = \frac{x^2 - 2}{2x^2}
            \end{align*}

            Consider \(I = \interval{\sqrt{2}}{b}\), for any \(b > \sqrt{2}\).
            It is clear that \(g\) and \(g'\) exists on \(I\). Since \(g'(x) \leq
            0 \, \forall x \in I\), \(g\) is monotonically increasing on \(I\).

            Consider \(g'\). \(x^2\) is strictly increasing on \(I\), so \(g'\)
            is strictly decreasing on \(I\), therefore:

            \begin{gather*}
                \frac{1}{2} > g'(x) \leq g'(\sqrt{2}) = 0 \, \forall x \in I \\
                \Rightarrow \abs{g'(x)} < 1 \, \forall x \in I
            \end{gather*}

            Let

            \[f(x) = g(x) - x = \frac{1}{x} - \frac{x}{2}\]

            \(\sfrac{1}{x}\) is strictly decreasing on \(I\), and so is \(-x\).
            Therefore, \(f\) is strictly decreasing on \(I\), so:

            \[f(\sqrt{2}) = 0 \leq f(x) \, \forall x \in I\]

            In other words, \(g(x) \leq x \, \forall x \in I\). It means that
            for any \(b\), \(g(b) < b\). Combining with the fact that
            \(g(\sqrt{2}) = \sqrt{2}\), it is guaranteed that:

            \[g(x) \in I \, \forall x \in I\]

            All the conditions of Theorem 2.4 hold, so we can apply it here: for
            any \(x_0 \in I\), applying fixed-point method on \(g\) converges to
            the unique fixed point in \(I\), using any \(x_0 \in I\).

            Trivially, \(\sqrt{2}\) is a fixed point of \(g\), therefore it must
            be the unique fixed point on \(I\).

            We can conclude that for any \(x_0 > \sqrt{2}\), the sequence
            converges to \(\sqrt{2}\).

        \item When \(0 < x < \sqrt{2}\), \(g'(x) < 0\), which means \(g\) is
            monotonically decreasing. Applying this on \(0 < x_0 < \sqrt{2}\)
            gives:

            \[x_1 = g(x_0) > g(\sqrt{2}) = \sqrt{2}\]

        \item We have:

            \begin{itemize}
                \item If \(x_0 > \sqrt{2}\): proven.
                \item If \(x_0 = \sqrt{2}\): it is exactly the fixed point.
                \item If \(0 < x_0 < \sqrt{2}\): \(x_1 = g(x_0) > \sqrt{2}\),
                    then from \(x_1\) onwards, the sequence converges to
                    \(\sqrt{2}\), as proven with the case \(x_0 > \sqrt{2}\).
            \end{itemize}

            Therefore, we can conclude that the sequence converges to
            \(\sqrt{2}\) whenever \(x_0 > 0\).
    \end{enumerate}
\end{solution}

\begin{exercise}
    \begin{tasks}
        \task Show that if A is any positive number, then the sequence defined
            by

            \[x_n = \frac{1}{2} x_{n - 1} + \frac{A}{2x_{n - 1}} \text{, for } n \geq 1\]

            converges to \(\sqrt{A}\) whenever \(x_0 > 0\).

        \task What happens if \(x_0 < 0\)?
    \end{tasks}
\end{exercise}

\begin{solution}
    \begin{enumerate}[label = \alph*)]
        \item Let

            \begin{align*}
                             g(x) &= \frac{x}{2} + \frac{A}{2x} = \frac{x^2 + A}{2x}\\
                \Rightarrow g'(x) &= \frac{1}{2} - \frac{A}{2x^2} = \frac{x^2 - A}{2x^2}
            \end{align*}

            Trivially, we can find out that \(\sqrt{A}\) is a fixed point of
            \(g\).

            Let

            \begin{align*}
                             f(x) &= g(x) - x = \frac{A}{2x} - \frac{x}{2} = \frac{A - x^2}{2x} \\
                \Rightarrow f'(x) &= - \frac{A}{2x^2} - \frac{1}{2} = - \frac{x^2 + A}{2x^2}
            \end{align*}

            Since \(f'(x) < 0 \, \forall x \neq 0\), \(f(x)\) is monotonically
            increasing when \(x > 0\).

            Consider the sign of \(g'\):

            \begin{itemize}
                \item \(g'(x) < 0 \iff \abs{x} < \sqrt{A}\)
                \item \(g'(x) = 0 \iff \abs{x} = \sqrt{A}\)
                \item \(g'(x) > 0 \iff \abs{x} > \sqrt{A}\)
            \end{itemize}

            If \(x > \sqrt{A}\), then:

            \begin{itemize}
                \item \(g' > 0\), which means \(g\) is monotonically increasing.
                    It follows that:

                    \[g(x) > g(\sqrt{A}) = \sqrt{A}\]

                \item \(f(x) < f(\sqrt{A}) = 0\), which means \(g(x) < x\),
                    making \(\{x_n\}\) a decreasing sequence.
            \end{itemize}

            From both of the above, we know that \(\{x_n\}\) is a lower-bounded
            decreasing sequence, and therefore must converge:

            \begin{align*}
                     x &= \lim_{n \to \infty} x_n \\
                       &= \lim_{n \to \infty} g(x_{n - 1}) \\
                       &= \lim_{n \to \infty} \frac{x_{n - 1}}{2} + \frac{A}{2 x_{n - 1}} \\
                       &= \frac{x}{2} + \frac{A}{2x} \\
                \iff x &= \sqrt{A}
            \end{align*}

            So, for all \(x_0 > \sqrt{A}\), the sequence converges to
            \(\sqrt{A}\).

            If \(x = \sqrt{A}\), then \(g(x) = x = \sqrt{A}\). Hence \(x_n =
            \sqrt{A} \, \forall n \geq 0\). So, for \(x_0 = \sqrt{A}\), the
            sequence converges to \(\sqrt{A}\).

            If \(0 < x < \sqrt{A}\), then \(g' < 0\), which means \(g\) is
            monotonically decreasing. It follows that:

            \[g(x) > g(\sqrt{A}) = \sqrt{A}\]

            So, for \(0 < x_0 < \sqrt{A}\), \(x_1 = g(x_0) > \sqrt{A}\), then
            from \(x_1\) onwards, the sequence converges to \(\sqrt{A}\), as
            proven with the case \(x_0 > \sqrt{A}\).

            We can conclude that the sequence \(\{x_n\}\) converges to
            \(\sqrt{2}\) whenever \(x_0 > 0\).

        \item If \(x_0 < 0\), then similar to the above proof, we conclude that
            the sequence converges to \(-\sqrt{A}\).
    \end{enumerate}
\end{solution}

\begin{exercise}
    Replace the assumption in Theorem 2.4 that ``a positive number \(k < 1\)
    exists with \(\abs{g(x)} \leq k\)'' with ``g satisfies a Lipschitz condition
    on the interval \(\interval{a}{b}\) with Lipschitz constant \(L < 1\)'' (See
    Exercise 27, Section 1.1.) Show that the conclusions of this theorem are
    still valid.
\end{exercise}

\begin{solution}
    \(g\) satisfies a Lipschitz condition on the interval \(\interval{a}{b}\)
    with Lipschitz constant \(L < 1\) means that:

    \[\frac{g(x_1) - g(x_2)}{x_1 - x_2} \leq L \, \forall x_1, x_2 \in \interval{a}{b} \tag{*}\label{eq:exer:2.2.21}\]

    In the proof of Theorem 2.4, we see that:

    \[\abs{p - p_n} = \abs{g(p) - g(p_{n - 1})}\]

    From the previous section of the proof, we already proved that \(p\) and
    \(p_{n - 1}\) is in \(\interval{a}{b}\). Applying \eqref{eq:exer:2.2.21}
    with \(x_1 = p\), \(x_2 = p_{n - 1}\) gives:

    \[\abs{p - p_n} = \abs{g(p) - g(p_{n - 1})} \leq L \abs{p - p_{n - 1}}\]

    Then the proof proceeds normally, replacing \(k\) with \(L\).
\end{solution}

\begin{exercise}
    Suppose that \(g\) is continuously differentiable on some interval
    \(\interval[open]{c}{d}\) that contains the fixed point \(p\) of \(g\). Show
    that if \(\abs{g'(p)} < 1\), then there exists a \(\delta > 0\) such that if
    \(\abs{p_0 - p} \leq \delta\), then the fixed-point iteration converges.
\end{exercise}

\begin{solution}\label{exer:2.2.22}
    Since \(p\) is a fixed point in \(\interval[open]{c}{d}\) of \(g\),
    \(g(p) = p\).

    Since \(g'\) is continuous at \(p\), according to the definition of
    continuity and limit, for every \(\varepsilon > 0\), there exist \(\delta >
    0\) such that:

    \begin{gather*}
        \abs{g'(x) - g'(p)} < \varepsilon \, \forall x \in D = \interval{p - \delta}{p + \delta} \\
        \iff g'(x) \in E = \interval{g'(p) - \varepsilon}{g'(p) + \varepsilon} \, \forall x \in D
    \end{gather*}

    We can always choose a \(\varepsilon\) such that \(E \subset
    \interval[open]{-1}{1}\). Then the proof proceeds normally, replacing
    \(\interval{a}{b}\) with \(E\).
\end{solution}

\begin{exercise}
    An object falling vertically through the air is subjected to viscous
    resistance as well as to the force of gravity. Assume that an object with
    mass \(m\) is dropped from a height \(s_0\) and that the height of the
    object after \(t\) seconds is:

    \[s(t) = s_0 - \frac{mg}{k} t + \frac{m^2g}{k^2} (1 - e^{- \sfrac{kt}{m}})\]

    \noindent where \(g = \SI{32.17}{ft \per \second \squared}\) and \(k\)
    represents the coefficient of air resistance in lb/s. Suppose \(s_0 =
    \SI{300}{ft}\), \(m = \SI{0.25}{lb}\), and \(k = \SI{0.1}{lb \per
    \second}\). Find, to within \(\SI{0.01}{\second}\), the time it takes this
    quarter-pounder to hit the ground.
\end{exercise}

\begin{solution}
    Replacing symbols in \(s(t)\) with number gives:

    \[s(t) = \num{501.0625} - \num{80.425} t - \num{201.0625} e^{\num{-0.4} t}\]

    Let

    \[g(t) = \frac{1}{80.425} (501.0625 - \num{201.0625} e^{\num{-0.4} t})\]

    A fixed point \(p\) of \(g\) is also a root of \(s(t) = 0\), which is the
    time it takes the quarter-pounder to hit the ground.

    Applying fixed-point method on \(g\) with \(p_0 = 3\) generates the
    following table:

    \begin{table}[H]
        \centering
        \begin{tabular}{r S[table-format=1.8] r S[table-format=1.8]}
            \toprule
            \(n\)  &   {\(p_n\)}   &  \(n\)  &   {\(p_n\)}   \\
            \midrule
                0  &  3            &      3  &  5.99886594   \\
                1  &  5.47719787   &      4  &  6.00328561   \\
                2  &  5.9506374    &         &               \\
            \bottomrule
        \end{tabular}
    \end{table}

    We conclude that it takes approximately \SI{6.003}{\second} for the
    quarter-pounder to hit the ground.
\end{solution}

\begin{exercise}
    Let \(g \in C^1 \interval{a}{b}\) and \(p\) be in \(\interval[open]{a}{b}\)
    with \(g(p) = p\) and \(\abs{g'(p)} > 1\). Show that there exists a \(\delta
    > 0\) such that if \(0 < \abs{p_0 - p} < \delta\), then \(\abs{p_0 - p} <
    \abs{p_1 - p}\). Thus, no matter how close the initial approximation \(p_0\)
    is to \(p\), the next iterate \(p_1\) is farther away, so the fixed-point
    iteration does not converge if \(p_0 \neq p\).
\end{exercise}

\begin{solution}
    This problem is similar to \hyperref[exer:2.2.22]{Exercise 22}.

    Since \(g'\) is continuous at \(p\), according to the definition of
    continuity and limit, for every \(\varepsilon > 0\), there exist \(\delta >
    0\) such that:

    \begin{gather*}
        \abs{g'(x) - g'(p)} < \varepsilon \, \forall x \in D = \interval{p - \delta}{p + \delta} \\
        \iff g'(x) \in E = \interval{g'(p) - \varepsilon}{g'(p) + \varepsilon} \, \forall x \in D
    \end{gather*}

    We can always choose a \(\varepsilon\) such that \(E \subset
    \interval[open]{1}{\infty}\).

    If \(p_0 \in D\), then according to Mean Value Theorem, there exist a \(\xi
    \in D\) such that:

    \[\abs{p_1 - p} = \abs{g(p_0) - g(p)} = \abs{g'(\xi)} \abs{p_0 - p} > \abs{p_0 - p}\]
\end{solution}

\end{document}
