\documentclass[../../../../Assignments]{subfiles}


\begin{document}

\section{Newton's Method and Its Extensions}

\begin{exercise}
    Let \(f(x) = x^2 - 6\) and \(p_0 = 1\). Use Newton's method to find \(p_2\).
\end{exercise}

\begin{solution}
    \(f'(x) = 2x\). Therefore, \(p_1 = \num{3.5}\), \(p_2 = \num{2.607142}\).
\end{solution}

\begin{exercise}
    Let \(f(x) = -x^3 - \cos{x}\) and \(p_0 = -1\). Use Newton's method to find
    \(p_2\). Could \(p_0 = 0\) be used?
\end{exercise}

\begin{solution}
    \(f'(x) = -3x^2 + \sin{x}\). Therefore, \(p_1 = \num{-0.880333}\), \(p_2 =
    \num{-0.865684}\).

    \(p_0 = 0\) can't be used, as \(f'(p_0) = 0\), therefore \(p_1\) can't be
    calculated.
\end{solution}

\begin{exercise}
    Let \(f(x) = x^2 - 6\). With \(p_0 = 3\) and \(p_1 = 2\), find \(p_3\).

    \begin{tasks}
        \task Use the Secant method.
        \task Use the method of False Position.
        \task Which of the above is closer to \(\sqrt{6}\)?
    \end{tasks}
\end{exercise}

\begin{solution}
    \begin{enumerate}[label = \alph*)]
        \item Applying Secant method generates the following table:

            \begin{table}[H]
                \centering
                \begin{tabular}{r S[table-format=1.6] S[table-format=-1.6]}
                    \toprule
                    \(n\)  &   {\(p_n\)}   &  {\(f(p_n)\)}  \\
                    \midrule
                        0  &  3            &   3            \\
                        1  &  2            &  -2            \\
                        2  &  2.4          &  -0.24         \\
                        3  &  2.454545     &   0.024793     \\
                    \bottomrule
                \end{tabular}
            \end{table}

            So \(p_3 = \num{2.454545}\).

        \item Applying False Position method generates the following table:

            \begin{table}[H]
                \centering
                \begin{tabular}{r S[table-format=1.6] S[table-format=-1.6]}
                    \toprule
                    \(n\)  &   {\(p_n\)}   &  {\(f(p_n)\)}  \\
                    \midrule
                        0  &  3            &   3            \\
                        1  &  2            &  -2            \\
                        2  &  2.4          &  -0.24         \\
                        3  &  2.454545     &   2.444444     \\
                    \bottomrule
                \end{tabular}
            \end{table}

            So \(p_3 = \num{2.444444}\).

        \item \(p_3\) produced by Secant method is better.
    \end{enumerate}
\end{solution}

\begin{exercise}
    Let \(f(x) = -x^3 - \cos{x}\). With \(p_0 = -1\) and \(p_1 = 0\), find
    \(p_3\).

    \begin{tasks}(2)
        \task Use the Secant method.
        \task Use the method of False Position.
    \end{tasks}
\end{exercise}

\begin{solution}
    \begin{enumerate}[label = \alph*)]
        \item Applying Secant method generates the following table:

            \begin{table}[H]
                \centering
                \begin{tabular}{r S[table-format=-1.9] S[table-format=-1.9]}
                    \toprule
                    \(n\)  &    {\(p_n\)}   &  {\(f(p_n)\)}  \\
                    \midrule
                        0  &  -1            &   0.459697694  \\
                        1  &   0            &  -1            \\
                        2  &  -0.685073357  &  -0.452850234  \\
                        3  &  -1.252076489  &   1.649523592  \\
                    \bottomrule
                \end{tabular}
            \end{table}

            So \(p_3 = \num{-1.252076}\).

        \item Applying False Position method generates the following table:

            \begin{table}[H]
                \centering
                \begin{tabular}{r S[table-format=-1.9] S[table-format=-1.9]}
                    \toprule
                    \(n\)  &    {\(p_n\)}   &  {\(f(p_n)\)}  \\
                    \midrule
                        0  &  -1            &   0.459697694  \\
                        1  &   0            &  -1            \\
                        2  &  -0.685073357  &  -0.452850234  \\
                        3  &  -0.841355126  &  -0.070875968  \\
                    \bottomrule
                \end{tabular}
            \end{table}

            So \(p_3 = \num{-0.841355}\).

    \end{enumerate}
\end{solution}

\begin{exercise}\label{exer:2.3.5}
    Use Newton's method to find solutions accurate to within \(10^{-4}\) for the
    following problems.

    \begin{tasks}
        \task \(x^3 - 2x^2 - 5 = 0\) in \(\interval{1}{4}\)
        \task \(x^3 + 3x^2 - 1 = 0\) in \(\interval{-3}{-2}\)
        \task \(x - \cos{x} = 0\) in \(\interval{0}{\sfrac{\pi}{2}}\)
        \task \(x - \num{0.8} - \num{0.2} \sin{x} = 0\) in \(\interval{0}{\sfrac{\pi}{2}}\)
    \end{tasks}
\end{exercise}

\begin{solution}
    \begin{enumerate}[label = \alph*)]
        \item Let

            \begin{align*}
                             f(x) &= x^3 - 2x^2 - 5 \\
                \Rightarrow f'(x) &= 3x^2 - 4x
            \end{align*}

            Applying Newton's method on \(f\) with \(p_0 = \num{2.5}\) gives:

            \begin{table}[H]
                \centering
                \begin{tabular}{r S[table-format=1.9] S[table-format=-1.9] S[table-format=2.9]}
                    \toprule
                    \(n\)  &   {\(p_n\)}   &  {\(f(p_n)\)}  &  {\(f'(p_n)\)}  \\
                    \midrule
                        0  &  2.5          &  -1.875        &   8.75          \\
                        1  &  2.714285714  &   0.262390671  &  11.24489796    \\
                        2  &  2.690951571  &   0.003331987  &  10.95985413    \\
                        3  &  2.690647499  &   0.000000561  &  10.9561619     \\
                        4  &  2.690647448  &   0            &  10.95616128    \\
                    \bottomrule
                \end{tabular}
            \end{table}

            We conclude that \(p \approx \num{2.69065}\) is a solution of the
            problem.

        \item Let

            \begin{align*}
                             f(x) &= x^3 + 3x^2 - 1 \\
                \Rightarrow f'(x) &= 3x^2 + 6x
            \end{align*}

            Applying Newton's method on \(f\) with \(p_0 = \num{-2.5}\) gives:

            \begin{longtable}{r S[table-format=-1.9] S[table-format=-1.9] S[table-format=2.9]}
                \toprule
                \(n\)  &    {\(p_n\)}   &  {\(f(p_n)\)}  &  {\(f'(p_n)\)}  \\
                \midrule
                \endfirsthead
                \(n\)  &    {\(p_n\)}   &  {\(f(p_n)\)}  &  {\(f'(p_n)\)}  \\
                \midrule
                \endhead
                    0  &  -2.5          &   2.125        &  3.75           \\
                    1  &  -3.06666667   &  -1.626962963  &  9.81333333     \\
                    2  &  -2.900875604  &  -0.165860349  &  7.839984184    \\
                    3  &  -2.879719904  &  -0.002542819  &  7.600040757    \\
                    4  &  -2.879385325  &  -0.000000631  &  7.596267596    \\
                    5  &  -2.879385242  &   0            &  7.596266659    \\
                \bottomrule
            \end{longtable}

            We conclude that \(p \approx \num{2.69065}\) is a solution of the
            problem.

        \item Let

            \begin{align*}
                             f(x) &= x - \cos{x} \\
                \Rightarrow f'(x) &= 1 + \sin{x}
            \end{align*}

            Applying Newton's method on \(f\) with \(p_0 = \num{0.739}\) gives:

            \begin{table}[H]
                \centering
                \begin{tabular}{r S[table-format=1.9] S[table-format=-1.9] S[table-format=1.9]}
                    \toprule
                    \(n\)  &   {\(p_n\)}   &  {\(f(p_n)\)}  &  {\(f'(p_n)\)}  \\
                    \midrule
                        0  &  0.739        &  -0.000142477  &  1.673549106    \\
                        1  &  0.739085135  &   0.000000002  &  1.67361203     \\
                    \bottomrule
                \end{tabular}
            \end{table}

            We conclude that \(p \approx \num{0.73909}\) is a solution of the
            problem.

        \item Let

            \begin{align*}
                             f(x) &= x - \num{0.8} - \num{0.2} \sin{x} \\
                \Rightarrow f'(x) &= 1 - \num{0.2} \cos{x}
            \end{align*}

            Applying Newton's method on \(f\) with \(p_0 = \num{0.964}\) gives:

            \begin{table}[H]
                \centering
                \begin{tabular}{r S[table-format=1.9] S[table-format=-1.9] S[table-format=1.9]}
                    \toprule
                    \(n\)  &   {\(p_n\)}   &  {\(f(p_n)\)}  &  {\(f'(p_n)\)}  \\
                    \midrule
                        0  &  0.964        &  -0.000295817  &  0.885952272    \\
                        1  &  0.964333898  &  -0.000000009  &  0.886007136    \\
                        2  &  0.964333888  &   0            &  0.886007135    \\
                    \bottomrule
                \end{tabular}
            \end{table}

            We conclude that \(p \approx \num{0.96433}\) is a solution of the
            problem.
    \end{enumerate}
\end{solution}

\begin{exercise}\label{exer:2.3.6}
    Use Newton's method to find solutions accurate to within \(10^{-5}\) for the
    following problems.

    \begin{tasks}
        \task \(e^x + 2^{-x} + 2 \cos{x} - 6 = 0\) for \(x \in \interval{1}{2}\)
        \task \(\ln(x - 1) + \cos(x - 1) = 0\) for \(x \in
            \interval{\num{1.3}}{2}\)
        \task \(2 x \cos(2x) - (x - 2)^2 = 0\) for \(x \in \interval{2}{3}\) and
            \(x \in \interval{3}{4}\)
        \task \((x - 2)^2 - \ln{x} = 0\) for \(x \in \interval{1}{2}\) and \(x
            \in \interval{e}{4}\)
        \task \(e^x - 3x^2 = 0\) for \(x \in \interval{0}{1}\) and \(x \in
            \interval{3}{5}\)
        \task \(\sin{x} - e^x = 0\) for \(x \in \interval{0}{1}\), \(x \in
            \interval{3}{4}\) and \(x \in \interval{6}{7}\)
    \end{tasks}
\end{exercise}

\begin{solution}
    \begin{enumerate}[label = \alph*)]
        \item Let

            \begin{align*}
                             f(x) &= e^x + 2^{-x} + 2 \cos{x} - 6 \\
                \Rightarrow f'(x) &= e^x - \ln{2} \cdot 2^{-x} - 2 \sin{x}
            \end{align*}

            Applying Newton's method on \(f\) with \(p_0 = \num{1.829}\) gives:

            \begin{table}[H]
                \centering
                \begin{tabular}{r S[table-format=1.9] S[table-format=-1.9] S[table-format=1.9]}
                    \toprule
                    \(n\)  &   {\(p_n\)}   &  {\(f(p_n)\)}  &  {\(f'(p_n)\)}  \\
                    \midrule
                        0  &  1.829        &  -0.001572837  &  4.098862489    \\
                        1  &  1.829383725  &   0.000000506  &  4.101500646    \\
                        2  &  1.829383602  &   0            &  4.101499798    \\
                    \bottomrule
                \end{tabular}
            \end{table}

            We conclude that \(p \approx \num{1.829384}\) is a solution of the
            problem.

        \item Let

            \begin{align*}
                             f(x) &= \ln(x - 1) + \cos(x - 1) \\
                \Rightarrow f'(x) &= \frac{1}{x - 1} - \sin(x - 1)
            \end{align*}

            Applying Newton's method on \(f\) with \(p_0 = \num{1.398}\) gives:

            \begin{table}[H]
                \centering
                \begin{tabular}{r S[table-format=1.9] S[table-format=-1.9] S[table-format=1.9]}
                    \toprule
                    \(n\)  &   {\(p_n\)}   &  {\(f(p_n)\)}  &  {\(f'(p_n)\)}  \\
                    \midrule
                        0  &  1.398        &   0.000534714  &  1.527454989    \\
                        1  &  1.397649931  &  -0.00020962   &  1.52972716     \\
                    \bottomrule
                \end{tabular}
            \end{table}

            We conclude that \(p \approx \num{1.39765}\) is a solution of the
            problem.

        \item Let

            \begin{align*}
                             f(x) &= 2 x \cos(2x) - (x - 2)^2 \\
                \Rightarrow f'(x) &= 2(\cos{x} - x \sin(2x) 2) - 2(x - 2) \\
                                  &= 2(\cos{x} - 2 x \sin(2x) - x + 2)
            \end{align*}

            Applying Newton's method on \(f\) with \(p_0 = \num{2.371}\) gives:

            \begin{table}[H]
                \centering
                \begin{tabular}{r S[table-format=1.9] S[table-format=-1.9] S[table-format=1.9]}
                    \toprule
                    \(n\)  &   {\(p_n\)}   &   {\(f(p_n)\)}  &  {\(f'(p_n)\)}  \\
                    \midrule
                        0  &  2.371        &   0.002753936  &  7.30284651      \\
                        1  &  2.3706229    &  -0.000563086  &  7.30282746      \\
                        2  &  2.3707       &   0.000115071  &  7.30283178      \\
                        3  &  2.37068424   &  -0.000023518  &  7.30283091      \\
                    \bottomrule
                \end{tabular}
            \end{table}

            Applying Newton's method on \(f\) with \(p_0 = \num{3.722}\) gives:

            \begin{table}[H]
                \centering
                \begin{tabular}{r S[table-format=1.9] S[table-format=1.9] S[table-format=-2.8]}
                    \toprule
                    \(n\)  &   {\(p_n\)}   &  {\(f(p_n)\)}  &  {\(f'(p_n)\)}  \\
                    \midrule
                        0  &  3.722        &  0.001838451   &  -18.77068249   \\
                        1  &  3.722097943  &  0.000241783   &  -18.77229246   \\
                        2  &  3.722110823  &  0.000031801   &  -18.77250414   \\
                        3  &  3.722112517  &  0.000004182   &  -18.77253198   \\
                    \bottomrule
                \end{tabular}
            \end{table}

            We conclude that \(p \approx \num{2.370684}\) and \(p \approx
            \num{3.722113}\) are solutions of the problem.

        \item Let

            \begin{align*}
                             f(x) &= (x - 2)^2 - \ln{x} \\
                \Rightarrow f'(x) &= 2(x - 2) - \frac{1}{x}
            \end{align*}

            Applying Newton's method on \(f\) with \(p_0 = \num{1.412}\) gives:

            \begin{table}[H]
                \centering
                \begin{tabular}{r S[table-format=1.9] S[table-format=1.9] S[table-format=-1.9]}
                    \toprule
                    \(n\)  &   {\(p_n\)}   &  {\(f(p_n)\)}  &  {\(f'(p_n)\)}  \\
                    \midrule
                        0  &  1.412        &  0.00073686   &  -1.884215297    \\
                        1  &  1.41239107   &  0.000000191  &  -1.883237062    \\
                        2  &  1.412391172  &  0            &  -1.883236808    \\
                    \bottomrule
                \end{tabular}
            \end{table}

            Applying Newton's method on \(f\) with \(p_0 = \num{3.057}\) gives:

            \begin{table}[H]
                \centering
                \begin{tabular}{r S[table-format=1.9] S[table-format=-1.9] S[table-format=1.9]}
                    \toprule
                    \(n\)  &   {\(p_n\)}   &   {\(f(p_n)\)}  &  {\(f'(p_n)\)}  \\
                    \midrule
                        0  &  3.057        &  -0.000185043   &  1.78688191     \\
                        1  &  3.05710356   &   0.000000011   &  1.7871001      \\
                        2  &  3.05710355   &   0             &  1.78710009     \\
                    \bottomrule
                \end{tabular}
            \end{table}

            We conclude that \(p \approx \num{1.412391}\) and \(p \approx
            \num{3.057104}\) are solutions of the problem.

        \item Let

            \begin{align*}
                             f(x) &= e^x - 3x^2 \\
                \Rightarrow f'(x) &= e^x - 6x
            \end{align*}

            Applying Newton's method on \(f\) with \(p_0 = \num{0.91}\) gives:

            \begin{table}[H]
                \centering
                \begin{tabular}{r S[table-format=1.9] S[table-format=1.9] S[table-format=-1.9]}
                    \toprule
                    \(n\)  &   {\(p_n\)}   &  {\(f(p_n)\)}  &  {\(f'(p_n)\)}  \\
                    \midrule
                        0  &  0.91         &   0.000022533  &  -2.97567747    \\
                        1  &  0.910007573  &   0            &  -2.97570409    \\
                    \bottomrule
                \end{tabular}
            \end{table}

            Applying Newton's method on \(f\) with \(p_0 = \num{3.733}\) gives:

            \begin{table}[H]
                \centering
                \begin{tabular}{r S[table-format=1.9] S[table-format=-1.9] S[table-format=2.9]}
                    \toprule
                    \(n\)  &   {\(p_n\)}   &   {\(f(p_n)\)}  &  {\(f'(p_n)\)}  \\
                    \midrule
                        0  &  3.733        &  -0.001533768   &  19.4063332     \\
                        1  &  3.73307903   &   0.000000112   &  19.4091631     \\
                        2  &  3.73307903   &   0             &  19.4091629     \\
                    \bottomrule
                \end{tabular}
            \end{table}

            We conclude that \(p \approx \num{0.910008}\) and \(p \approx
            \num{3.733079}\) are solutions of the problem.

        \item Let

            \begin{align*}
                             f(x) &= \sin{x} - e^{-x} \\
                \Rightarrow f'(x) &= \cos{x} + e^{-x}
            \end{align*}

            Applying Newton's method on \(f\) with \(p_0 = \num{0.588}\) gives:

            \begin{table}[H]
                \centering
                \begin{tabular}{r S[table-format=1.9] S[table-format=-1.9] S[table-format=1.8]}
                    \toprule
                    \(n\)  &   {\(p_n\)}   &  {\(f(p_n)\)}  &  {\(f'(p_n)\)}  \\
                    \midrule
                        0  &  0.588        &  -0.000739019  &  1.38748879     \\
                        1  &  0.58853263   &  -0.000000157  &  1.38689746     \\
                        2  &  0.588532744  &   0            &  1.38689733     \\
                    \bottomrule
                \end{tabular}
            \end{table}

            Applying Newton's method on \(f\) with \(p_0 = \num{3.096}\) gives:

            \begin{table}[H]
                \centering
                \begin{tabular}{r S[table-format=1.8] S[table-format=-1.9] S[table-format=-1.9]}
                    \toprule
                    \(n\)  &   {\(p_n\)}   &  {\(f(p_n)\)}  &  {\(f'(p_n)\)}  \\
                    \midrule
                        0  &  3.096        &   0.0003471    &  -0.953731075   \\
                        1  &  3.09636394   &  -0.000000601  &  -0.953764054   \\
                        2  &  3.09636393   &   0            &  -0.953764053   \\
                    \bottomrule
                \end{tabular}
            \end{table}

            Applying Newton's method on \(f\) with \(p_0 = \num{6.285}\) gives:

            \begin{table}[H]
                \centering
                \begin{tabular}{r S[table-format=1.8] S[table-format=-1.9] S[table-format=1.8]}
                    \toprule
                    \(n\)  &   {\(p_n\)}   &  {\(f(p_n)\)}  &  {\(f'(p_n)\)}  \\
                    \midrule
                        0  &  6.285        &  -0.000049365 &  1.00186241      \\
                        1  &  6.28504927   &   0           &  1.00186223      \\
                        2  &  6.28504927   &   0           &  1.00186223      \\
                    \bottomrule
                \end{tabular}
            \end{table}

            We conclude that \(p \approx \num{0.58853}\), \(p \approx
            \num{3.09636}\) and \(p = 6.285049\) are solutions of the problem.
        \end{enumerate}
\end{solution}

\begin{exercise}
    Repeat \hyperref[exer:2.3.5]{Exercise 5} using the Secant method.
\end{exercise}

\begin{solution}
    \begin{enumerate}[label= \alph*)]
        \item Applying Secant method with \(p_0 = \num{2.6}\) and \(p_1 =
            \num{2.7}\) generates the following table:

            \begin{table}[H]
                \centering
                \begin{tabular}{r S[table-format=1.9] S[table-format=-1.9]}
                    \toprule
                    \(n\)  &   {\(p_n\)}   &  {\(f(p_n)\)}  \\
                    \midrule
                        0  &  2.6          &  -0.944        \\
                        1  &  2.7          &   0.103        \\
                        2  &  2.690162369  &  -0.005313179  \\
                        3  &  2.690644942  &  -0.000027451  \\
                        4  &  2.690647449  &   0.000000007  \\
                    \bottomrule
                \end{tabular}
            \end{table}

            We conclude that \(p \approx \num{2.69065}\) is a solution of the
            problem.

        \item Applying Secant method with \(p_0 = \num{-2.8}\) and \(p_1 =
            \num{-2.9}\) generates the following table:

            \begin{table}[H]
                \centering
                \begin{tabular}{r S[table-format=-1.9] S[table-format=-1.9]}
                    \toprule
                    \(n\)  &    {\(p_n\)}   &  {\(f(p_n)\)}  \\
                    \midrule
                        0  &  -2.8          &   0.568        \\
                        1  &  -2.9          &  -0.159        \\
                        2  &  -2.878129298  &   0.009531586  \\
                        3  &  -2.879366233  &   0.000144394  \\
                        4  &  -2.879385259  &  -0.000000134  \\
                    \bottomrule
                \end{tabular}
            \end{table}

            We conclude that \(p \approx \num{-2.87939}\) is a solution of the
            problem.

        \item Applying Secant method with \(p_0 = \num{0.73}\) and \(p_1 =
            \num{0.74}\) generates the following table:

            \begin{table}[H]
                \centering
                \begin{tabular}{r S[table-format=1.9] S[table-format=-1.9]}
                    \toprule
                    \(n\)  &    {\(p_n\)}   &  {\(f(p_n)\)}  \\
                    \midrule
                        0  &   0.73         &  -0.015174402  \\
                        1  &   0.74         &   0.001531441  \\
                        2  &   0.73908329   &  -0.000003084  \\
                        3  &   0.739085133  &   0            \\
                    \bottomrule
                \end{tabular}
            \end{table}

            We conclude that \(p \approx \num{0.73909}\) is a solution of the
            problem.

        \item Applying Secant method with \(p_0 = \num{0.96}\) and \(p_1 =
            \num{0.97}\) generates the following table:

            \begin{table}[H]
                \centering
                \begin{tabular}{r S[table-format=1.9] S[table-format=-1.9]}
                    \toprule
                    \(n\)  &   {\(p_n\)}   &  {\(f(p_n)\)}  \\
                    \midrule
                        0  &  0.96         &  -0.003838313  \\
                        1  &  0.97         &  -0.005022857  \\
                        2  &  0.96433161   &  -0.000002018  \\
                        3  &  0.964333887  &  -0.000000001  \\
                    \bottomrule
                \end{tabular}
            \end{table}

            We conclude that \(p \approx \num{0.96433}\) is a solution of the
            problem.
    \end{enumerate}
\end{solution}

\begin{exercise}
    Repeat \hyperref[exer:2.3.6]{Exercise 6} using the Secant method.
\end{exercise}

\begin{solution}
    \begin{enumerate}[label = \alph*)]
        \item Applying Secant method with \(p_0 = \num{1.82}\) and \(p_1 =
            \num{1.83}\) generates the following table:

            \begin{table}[H]
                \centering
                \begin{tabular}{r S[table-format=1.9] S[table-format=-1.9]}
                    \toprule
                    \(n\)  &   {\(p_n\)}   &  {\(f(p_n)\)}  \\
                    \midrule
                        0  &  1.82         &  -0.038185199  \\
                        1  &  1.83         &   0.002529463  \\
                        2  &  1.829378734  &  -0.000019965  \\
                        3  &  1.829383599  &   0.000000001  \\
                    \bottomrule
                \end{tabular}
            \end{table}

            We conclude that \(p \approx \num{1.829384}\) is a solution of the
            problem.

        \item Applying Secant method with \(p_0 = \num{1.39}\) and \(p_1 =
            \num{1.4}\) generates the following table:

            \begin{table}[H]
                \centering
                \begin{tabular}{r S[table-format=1.9] S[table-format=-1.9]}
                    \toprule
                    \(n\)  &   {\(p_n\)}   &  {\(f(p_n)\)}  \\
                    \midrule
                        0  &  1.39         &  -0.01669948   \\
                        1  &  1.4          &   0.004770262  \\
                        2  &  1.397778147  &   0.0000631    \\
                        3  &  1.397748362  &  -0.000000242  \\
                        4  &  1.397748476  &   0            \\
                    \bottomrule
                \end{tabular}
            \end{table}

            We conclude that \(p \approx \num{1.397748}\) is a solution of the
            problem.

        \item Applying Secant method with \(p_0 = \num{2.37}\) and \(p_1 =
            \num{2.375}\) generates the following table:

            \begin{table}[H]
                \centering
                \begin{tabular}{r S[table-format=1.9] S[table-format=-1.9]}
                    \toprule
                    \(n\)  &   {\(p_n\)}   &  {\(f(p_n)\)}  \\
                    \midrule
                        0  &  2.37         &  -0.006040395  \\
                        1  &  2.375        &   0.037985226  \\
                        2  &  2.370686009  &  -0.00000799   \\
                        3  &  2.370686916  &  -0.000000001  \\
                    \bottomrule
                \end{tabular}
            \end{table}

            Applying Secant method with \(p_0 = \num{3.72}\) and \(p_1 =
            \num{3.73}\) generates the following table:

            \begin{table}[H]
                \centering
                \begin{tabular}{r S[table-format=1.9] S[table-format=-1.9]}
                    \toprule
                    \(n\)  &   {\(p_n\)}   &  {\(f(p_n)\)}  \\
                    \midrule
                        0  &  3.72         &   0.034398018  \\
                        1  &  3.73         &  -0.129244414  \\
                        2  &  3.722102023  &   0.000175259  \\
                        3  &  3.722112719  &   0.000000889  \\
                        4  &  3.722112773  &   0            \\
                    \bottomrule
                \end{tabular}
            \end{table}

            We conclude that \(p \approx \num{2.37069}\) and \(p \approx
            \num{3.722113}\) are solutions of the problem.

        \item Applying Secant method with \(p_0 = \num{1.41}\) and \(p_1 =
            \num{1.42}\) generates the following table:

            \begin{table}[H]
                \centering
                \begin{tabular}{r S[table-format=1.8] S[table-format=-1.9]}
                    \toprule
                    \(n\)  &   {\(p_n\)}   &  {\(f(p_n)\)}  \\
                    \midrule
                        0  &  1.41         &   0.004510296  \\
                        1  &  1.42         &  -0.014256872  \\
                        2  &  1.41240329   &  -0.000022822  \\
                        3  &  1.41239111   &   0.000000116  \\
                        4  &  1.41239117   &   0            \\
                    \bottomrule
                \end{tabular}
            \end{table}

            Applying Secant method with \(p_0 = \num{3.05}\) and \(p_1 =
            \num{3.06}\) generates the following table:

            \begin{table}[H]
                \centering
                \begin{tabular}{r S[table-format=1.8] S[table-format=-1.9]}
                    \toprule
                    \(n\)  &   {\(p_n\)}   &  {\(f(p_n)\)}  \\
                    \midrule
                        0  &  3.05         &  -0.012641591  \\
                        1  &  3.06         &   0.005185084  \\
                        2  &  3.05709139   &  -0.000021731  \\
                        3  &  3.05710353   &  -0.000000037  \\
                        4  &  3.05710355   &   0            \\
                    \bottomrule
                \end{tabular}
            \end{table}

            We conclude that \(p \approx \num{1.412391}\) and \(p \approx
            \num{3.057104}\) are solutions of the problem.

        \item Applying Secant method with \(p_0 = \num{0.91}\) and \(p_1 =
            \num{0.92}\) generates the following table:

            \begin{table}[H]
                \centering
                \begin{tabular}{r S[table-format=1.9] S[table-format=-1.9]}
                    \toprule
                    \(n\)  &   {\(p_n\)}   &  {\(f(p_n)\)}  \\
                    \midrule
                        0  &  0.91         &   0.000022533  \\
                        1  &  0.92         &  -0.02990961   \\
                        2  &  0.910007528  &   0.000000132  \\
                        3  &  0.910007572  &   0            \\
                    \bottomrule
                \end{tabular}
            \end{table}

            Applying Secant method with \(p_0 = \num{3.73}\) and \(p_1 =
            \num{3.74}\) generates the following table:

            \begin{table}[H]
                \centering
                \begin{tabular}{r S[table-format=1.8] S[table-format=-1.9]}
                    \toprule
                    \(n\)  &   {\(p_n\)}   &  {\(f(p_n)\)}  \\
                    \midrule
                        0  &  3.73         &  -0.059591836  \\
                        1  &  3.74         &   0.135190165  \\
                        2  &  3.73305941   &  -0.000380739  \\
                        3  &  3.7330789    &  -0.000002422  \\
                        4  &  3.73307903   &   0            \\
                    \bottomrule
                \end{tabular}
            \end{table}

            We conclude that \(p \approx \num{0.910008}\) and \(p \approx
            \num{3.733079}\) are solutions of the problem.

        \item Applying Secant method with \(p_0 = \num{0.58}\) and \(p_1 =
            \num{0.59}\) generates the following table:

            \begin{table}[H]
                \centering
                \begin{tabular}{r S[table-format=1.9] S[table-format=-1.9]}
                    \toprule
                    \(n\)  &   {\(p_n\)}   &  {\(f(p_n)\)}  \\
                    \midrule
                        0  &  0.58         &  -0.01187443   \\
                        1  &  0.59         &   0.002033738  \\
                        2  &  0.588537738  &   0.000006927  \\
                        3  &  0.588532741  &  -0.000000004  \\
                    \bottomrule
                \end{tabular}
            \end{table}

            Applying Secant method with \(p_0 = \num{3.09}\) and \(p_1 =
            \num{3.1}\) generates the following table:

            \begin{table}[H]
                \centering
                \begin{tabular}{r S[table-format=1.8] S[table-format=-1.9]}
                    \toprule
                    \(n\)  &   {\(p_n\)}   &  {\(f(p_n)\)}  \\
                    \midrule
                        0  &  3.09         &   0.006067814  \\
                        1  &  3.1          &  -0.00346854   \\
                        2  &  3.09636282   &   0.000001057  \\
                        3  &  3.09636393   &   0            \\
                    \bottomrule
                \end{tabular}
            \end{table}

            Applying Secant method with \(p_0 = \num{6.28}\) and \(p_1 =
            \num{6.29}\) generates the following table:

            \begin{table}[H]
                \centering
                \begin{tabular}{r S[table-format=1.8] S[table-format=-1.9]}
                    \toprule
                    \(n\)  &   {\(p_n\)}   &  {\(f(p_n)\)}  \\
                    \midrule
                        0  &  6.28         &  -0.005058702  \\
                        1  &  6.29         &   0.00495988   \\
                        2  &  6.28504932   &   0.000000046  \\
                        3  &  6.28504927   &   0            \\
                    \bottomrule
                \end{tabular}
            \end{table}

            We conclude that \(p \approx \num{0.588533}\), \(p \approx
            \num{3.096364}\) and \(p \approx \num{6.285049}\) are solutions of
            the problem.
    \end{enumerate}
\end{solution}

\begin{exercise}
    Repeat \hyperref[exer:2.3.5]{Exercise 5} using the method of False Position.
\end{exercise}

\begin{solution}
    \begin{enumerate}[label= \alph*)]
        \item Applying False Position method with \(p_0 = \num{2.6}\) and \(p_1
            = \num{2.7}\) generates the following table:

            \begin{table}[H]
                \centering
                \begin{tabular}{r S[table-format=1.9] S[table-format=-1.9]}
                    \toprule
                    \(n\)  &   {\(p_n\)}   &  {\(f(p_n)\)}  \\
                    \midrule
                        0  &  2.6          &  -0.944        \\
                        1  &  2.7          &   0.103        \\
                        2  &  2.690162369  &  -0.005313179  \\
                        3  &  2.690644942  &  -0.000027451  \\
                        4  &  2.690647435  &  -0.000000141  \\
                    \bottomrule
                \end{tabular}
            \end{table}

            We conclude that \(p \approx \num{2.690647}\) is a solution of the
            problem.

        \item Applying False Position method with \(p_0 = \num{-2.8}\) and \(p_1
            = \num{-2.9}\) generates the following table:

            \begin{table}[H]
                \centering
                \begin{tabular}{r S[table-format=-1.9] S[table-format=-1.9]}
                    \toprule
                    \(n\)  &    {\(p_n\)}   &  {\(f(p_n)\)}  \\
                    \midrule
                        0  &  -2.8          &   0.568        \\
                        1  &  -2.9          &  -0.159        \\
                        2  &  -2.878129298  &   0.009531586  \\
                        3  &  -2.879366233  &   0.000144394  \\
                        4  &  -2.87938526   &  -0.000000135  \\
                        \bottomrule
                \end{tabular}
            \end{table}

            We conclude that \(p \approx \num{-2.87939}\) is a solution of the
            problem.

        \item Applying False Position method with \(p_0 = \num{0.73}\) and \(p_1
            = \num{0.74}\) generates the following table:

            \begin{table}[H]
                \centering
                \begin{tabular}{r S[table-format=1.9] S[table-format=-1.9]}
                    \toprule
                    \(n\)  &    {\(p_n\)}   &  {\(f(p_n)\)}  \\
                    \midrule
                        0  &   0.73         &  -0.015174402  \\
                        1  &   0.74         &   0.001531441  \\
                        2  &   0.73908329   &  -0.000003084  \\
                        3  &   0.739085133  &   0            \\
                    \bottomrule
                \end{tabular}
            \end{table}

            We conclude that \(p \approx \num{0.73909}\) is a solution of the
            problem.

        \item Applying False Position method with \(p_0 = \num{0.96}\) and \(p_1
            = \num{0.97}\) generates the following table:

            \begin{table}[H]
                \centering
                \begin{tabular}{r S[table-format=1.9] S[table-format=-1.9]}
                    \toprule
                    \(n\)  &   {\(p_n\)}   &  {\(f(p_n)\)}  \\
                    \midrule
                        0  &  0.96         &  -0.003838313  \\
                        1  &  0.97         &  -0.005022857  \\
                        2  &  0.96433161   &  -0.000002018  \\
                        3  &  0.964333887  &  -0.000000001  \\
                    \bottomrule
                \end{tabular}
            \end{table}

            We conclude that \(p \approx \num{0.96433}\) is a solution of the
            problem.
    \end{enumerate}
\end{solution}

\begin{exercise}
    Repeat \hyperref[exer:2.3.6]{Exercise 6} using the False Position method.
\end{exercise}

\begin{solution}
    \begin{enumerate}[label = \alph*)]
        \item Applying False Position method with \(p_0 = \num{1.82}\) and \(p_1
            = \num{1.83}\) generates the following table:

            \begin{table}[H]
                \centering
                \begin{tabular}{r S[table-format=1.9] S[table-format=-1.9]}
                    \toprule
                    \(n\)  &   {\(p_n\)}   &  {\(f(p_n)\)}  \\
                    \midrule
                        0  &  1.82         &  -0.038185199  \\
                        1  &  1.83         &   0.002529463  \\
                        2  &  1.829378734  &  -0.000019965  \\
                        3  &  1.829383599  &   0.000000001  \\
                    \bottomrule
                \end{tabular}
            \end{table}

            We conclude that \(p \approx \num{1.829384}\) is a solution of the
            problem.

        \item Applying False Position method with \(p_0 = \num{1.39}\) and \(p_1
            = \num{1.4}\) generates the following table:

            \begin{table}[H]
                \centering
                \begin{tabular}{r S[table-format=1.8] S[table-format=-1.9]}
                    \toprule
                    \(n\)  &   {\(p_n\)}   &  {\(f(p_n)\)}  \\
                    \midrule
                        0  &  1.39         &  -0.01669948   \\
                        1  &  1.4          &   0.004770262  \\
                        2  &  1.39777815   &   0.0000631    \\
                        3  &  1.39774887   &   0.000000831  \\
                        4  &  1.39774848   &   0.000000001  \\
                    \bottomrule
                \end{tabular}
            \end{table}

            We conclude that \(p \approx \num{1.397748}\) is a solution of the
            problem.

        \item Applying False Position method with \(p_0 = \num{2.37}\) and \(p_1
            = \num{2.375}\) generates the following table:

            \begin{table}[H]
                \centering
                \begin{tabular}{r S[table-format=1.9] S[table-format=-1.9]}
                    \toprule
                    \(n\)  &   {\(p_n\)}   &  {\(f(p_n)\)}  \\
                    \midrule
                        0  &  2.37         &  -0.006040395  \\
                        1  &  2.375        &   0.037985226  \\
                        2  &  2.370686009  &  -0.00000799   \\
                        3  &  2.370686916  &  -0.000000001  \\
                    \bottomrule
                \end{tabular}
            \end{table}

            Applying False Position method with \(p_0 = \num{3.72}\) and \(p_1 =
            \num{3.73}\) generates the following table:

            \begin{table}[H]
                \centering
                \begin{tabular}{r S[table-format=1.9] S[table-format=-1.9]}
                    \toprule
                    \(n\)  &   {\(p_n\)}   &  {\(f(p_n)\)}  \\
                    \midrule
                        0  &  3.72         &   0.034398018  \\
                        1  &  3.73         &  -0.129244414  \\
                        2  &  3.722102023  &   0.000175259  \\
                        3  &  3.722112719  &   0.000000889  \\
                        4  &  3.72211277   &   0.000000001  \\
                    \bottomrule
                \end{tabular}
            \end{table}

            We conclude that \(p \approx \num{2.37069}\) and \(p \approx
            \num{3.722113}\) are solutions of the problem.

        \item Applying False Position method with \(p_0 = \num{1.41}\) and \(p_1
            = \num{1.42}\) generates the following table:

            \begin{table}[H]
                \centering
                \begin{tabular}{r S[table-format=1.8] S[table-format=-1.9]}
                    \toprule
                    \(n\)  &   {\(p_n\)}   &  {\(f(p_n)\)}  \\
                    \midrule
                        0  &  1.41         &   0.004510296  \\
                        1  &  1.42         &  -0.014256872  \\
                        2  &  1.41240329   &  -0.000022822  \\
                        3  &  1.41239119   &  -0.000000036  \\
                        4  &  1.41239117   &   0            \\
                        \bottomrule
                \end{tabular}
            \end{table}

            Applying False Position method with \(p_0 = \num{3.05}\) and \(p_1 =
            \num{3.06}\) generates the following table:

            \begin{table}[H]
                \centering
                \begin{tabular}{r S[table-format=1.8] S[table-format=-1.9]}
                    \toprule
                    \(n\)  &   {\(p_n\)}   &  {\(f(p_n)\)}  \\
                    \midrule
                        0  &  3.05         &  -0.012641591  \\
                        1  &  3.06         &   0.005185084  \\
                        2  &  3.05709139   &  -0.000021731  \\
                        3  &  3.05710353   &  -0.000000037  \\
                        4  &  3.05710355   &   0            \\
                    \bottomrule
                \end{tabular}
            \end{table}

            We conclude that \(p \approx \num{1.412391}\) and \(p \approx
            \num{3.057104}\) are solutions of the problem.

        \item Applying False Position method with \(p_0 = \num{0.91}\) and \(p_1
            = \num{0.92}\) generates the following table:

            \begin{table}[H]
                \centering
                \begin{tabular}{r S[table-format=1.9] S[table-format=-1.9]}
                    \toprule
                    \(n\)  &   {\(p_n\)}   &  {\(f(p_n)\)}  \\
                    \midrule
                        0  &  0.91         &   0.000022533  \\
                        1  &  0.92         &  -0.02990961   \\
                        2  &  0.910007528  &   0.000000132  \\
                        3  &  0.910007572  &   0            \\
                    \bottomrule
                \end{tabular}
            \end{table}

            Applying False Position method with \(p_0 = \num{3.73}\) and \(p_1 =
            \num{3.74}\) generates the following table:

            \begin{table}[H]
                \centering
                \begin{tabular}{r S[table-format=1.8] S[table-format=-1.9]}
                    \toprule
                    \(n\)  &   {\(p_n\)}   &  {\(f(p_n)\)}  \\
                    \midrule
                        0  &  3.73         &  -0.059591836  \\
                        1  &  3.74         &   0.135190165  \\
                        2  &  3.73305941   &  -0.000380739  \\
                        3  &  3.7330789    &  -0.000002422  \\
                        4  &  3.73307903   &  -0.000000015  \\
                    \bottomrule
                \end{tabular}
            \end{table}

            We conclude that \(p \approx \num{0.910008}\) and \(p \approx
            \num{3.733079}\) are solutions of the problem.

        \item Applying False Position method with \(p_0 = \num{0.58}\) and \(p_1
            = \num{0.59}\) generates the following table:

            \begin{table}[H]
                \centering
                \begin{tabular}{r S[table-format=1.9] S[table-format=-1.9]}
                    \toprule
                    \(n\)  &   {\(p_n\)}   &  {\(f(p_n)\)}  \\
                    \midrule
                        0  &  0.58         &  -0.01187443   \\
                        1  &  0.59         &   0.002033738  \\
                        2  &  0.588537738  &   0.000006927  \\
                        3  &  0.588532761  &   0.000000024  \\
                        \bottomrule
                \end{tabular}
            \end{table}

            Applying False Position method with \(p_0 = \num{3.09}\) and \(p_1 =
            \num{3.1}\) generates the following table:

            \begin{table}[H]
                \centering
                \begin{tabular}{r S[table-format=1.8] S[table-format=-1.9]}
                    \toprule
                    \(n\)  &   {\(p_n\)}   &  {\(f(p_n)\)}  \\
                    \midrule
                        0  &  3.09         &   0.006067814  \\
                        1  &  3.1          &  -0.00346854   \\
                        2  &  3.09636282   &   0.000001057  \\
                        3  &  3.09636393   &   0            \\
                    \bottomrule
                \end{tabular}
            \end{table}

            Applying False Position method with \(p_0 = \num{6.28}\) and \(p_1 =
            \num{6.29}\) generates the following table:

            \begin{table}[H]
                \centering
                \begin{tabular}{r S[table-format=1.8] S[table-format=-1.9]}
                    \toprule
                    \(n\)  &   {\(p_n\)}   &  {\(f(p_n)\)}  \\
                    \midrule
                        0  &  6.28         &  -0.005058702  \\
                        1  &  6.29         &   0.00495988   \\
                        2  &  6.28504932   &   0.000000046  \\
                        3  &  6.28504927   &   0            \\
                    \bottomrule
                \end{tabular}
            \end{table}

            We conclude that \(p \approx \num{0.588533}\), \(p \approx
            \num{3.096364}\) and \(p \approx \num{6.285049}\) are solutions of
            the problem.
    \end{enumerate}
\end{solution}

\begin{exercise}
    Use all three methods in this Section to find solutions to within
    \(10^{-5}\) for the following problems.

    \begin{tasks}
        \task \(3xe^x = 0\) for \(x \in \interval{1}{2}\)
        \task \(2x + 3 \cos{x} - e^x\) for \(x \in \interval{0}{1}\)
    \end{tasks}
\end{exercise}

\begin{solution}
    \begin{enumerate}[label = \alph*)]
        \item Such math... much difficult...
        \item Let

            \begin{align*}
                             f(x) &= 2x + 3 \cos{x} - e^x \\
                \Rightarrow f'(x) &= 2 - 3 \sin{x} - e^x
            \end{align*}

            \(\sin{x}\) and \(e^x\) are both monotonically increasing in \(I =
            \interval{0}{1}\), therefore \(f'(x)\) is monotonically decreasing
            \(I\). It follows that

            \[f'(0) = 2 \geq f'(x) \geq f'(1) \approx \num{-0.5244129544}\]

            \noindent and that \(f'(x)\) has exactly one zero \(p\) in \(I\).
            Since the sign of \(f'(x)\) changes from positive to negative as
            \(x\) passes \(p\), the local maximum of \(f\) in \(I\) is at \(p\).
            Then the minimum value of \(f\) in \(I\) is achieved at either end:

            \[f(x) \geq \min \{f(0), f(1)\} \approx \num{0.9026250891} > 0\]

            Then \(f\) has no zero in \(I\).
    \end{enumerate}
\end{solution}

\begin{exercise}
    Use all three methods in this Section to find solutions to within
    \(10^{-7}\) for the following problems.

    \begin{tasks}
        \task \(x^2 - 4x + 4 - \ln{x} = 0\) for \(x \in \interval{1}{2}\) and
            \(x \in \interval{2}{4}\)
        \task \(x + 1 - 2 \sin{\pi x} = 0\) for \(x \in
            \interval{0}{\sfrac{1}{2}}\) and \(x \in
            \interval{\sfrac{1}{2}}{1}\)
    \end{tasks}
\end{exercise}

\begin{solution}
    \begin{enumerate}[label = \alph*)]
        \item Let

            \begin{align*}
                            f(x) &= x^2 - 4x + 4 - \ln{x} \\
                \Rightarrow f'(x) &= 2x - 4 - \frac{1}{x}
            \end{align*}

            Applying Newton's method on \(f\) with \(p_0 = \num{1.41}\)
            generates the following table:

            \begin{table}[H]
                \centering
                \begin{tabular}{r S[table-format=1.11] S[table-format=1.11] S[table-format=-1.11]}
                    \toprule
                    \(n\)  &    {\(p_n\)}    &   {\(f(p_n)\)}  &  {\(f'(p_n)\)}  \\
                    \midrule
                        0  &  1.41           &  0.00451029561  &  -1.88921985816  \\
                        1  &  1.41238738524  &  0.00000713142  &  -1.88324627986  \\
                        2  &  1.41239117201  &  0.00000000002  &  -1.88323680804  \\
                        3  &  1.41239117202  &  0              &  -1.88323680802  \\
                    \bottomrule
                \end{tabular}
            \end{table}

            Applying Newton's method on \(f\) with \(p_0 = \num{3.05}\)
            generates the following table:

            \begin{table}[H]
                \centering
                \begin{tabular}{r S[table-format=1.11] S[table-format=-1.11] S[table-format=1.11]}
                    \toprule
                    \(n\)  &    {\(p_n\)}    &   {\(f(p_n)\)}   &  {\(f'(p_n)\)}  \\
                    \midrule
                        0  &  3.05           &  -0.01264159062  &  1.77213114754  \\
                        1  &  3.05713355252  &   0.00005361847  &  1.78716330575  \\
                        2  &  3.05710355053  &   0.00000000095  &  1.7871000916   \\
                        3  &  3.05710354999  &   0              &  1.78710009048  \\
                    \bottomrule
                \end{tabular}
            \end{table}

            Applying Secant method with \(p_0 = \num{1.41}\) and \(p_1 =
            \num{1.42}\) generates the following table:

            \begin{table}[H]
                \centering
                \begin{tabular}{r S[table-format=1.11] S[table-format=-1.11]}
                    \toprule
                    \(n\)  &    {\(p_n\)}    &   {\(f(p_n)\)}   \\
                    \midrule
                        0  &  1.41           &   0.00451029561  \\
                        1  &  1.42           &  -0.01425687161  \\
                        2  &  1.41240329057  &  -0.00002282192  \\
                        3  &  1.41239111052  &   0.00000011582  \\
                        4  &  1.41239117202  &   0              \\
                    \bottomrule
                \end{tabular}
            \end{table}

            Applying Secant method with \(p_0 = \num{3.05}\) and \(p_1 =
            \num{3.06}\) generates the following table:

            \begin{table}[H]
                \centering
                \begin{tabular}{r S[table-format=1.11] S[table-format=-1.11]}
                    \toprule
                    \(n\)  &    {\(p_n\)}    &   {\(f(p_n)\)}   \\
                    \midrule
                        0  &  3.05           &  -0.01264159062  \\
                        1  &  3.06           &   0.00518508404  \\
                        2  &  3.05709139021  &  -0.00002173059  \\
                        3  &  3.05710352927  &  -0.00000003704  \\
                        4  &  3.05710354999  &   0              \\
                    \bottomrule
                \end{tabular}
            \end{table}

            Applying False Position method with \(p_0 = \num{1.41}\) and \(p_1 =
            \num{1.42}\) generates the following table:

            \begin{table}[H]
                \centering
                \begin{tabular}{r S[table-format=1.11] S[table-format=-1.11]}
                    \toprule
                    \(n\)  &    {\(p_n\)}    &   {\(f(p_n)\)}   \\
                    \midrule
                        0  &  1.41           &   0.00451029561  \\
                        1  &  1.42           &  -0.01425687161  \\
                        2  &  1.41240329057  &  -0.00002282192  \\
                        3  &  1.41239119124  &  -0.00000003619  \\
                        4  &  1.41239117205  &  -0.00000000006  \\
                    \bottomrule
                \end{tabular}
            \end{table}

            Applying False Position method with \(p_0 = \num{3.05}\) and \(p_1 =
            \num{3.06}\) generates the following table:

            \begin{table}[H]
                \centering
                \begin{tabular}{r S[table-format=1.11] S[table-format=-1.11]}
                    \toprule
                    \(n\)  &    {\(p_n\)}    &   {\(f(p_n)\)}   \\
                    \midrule
                        0  &  3.05           &  -0.01264159062  \\
                        1  &  3.06           &   0.00518508404  \\
                        2  &  3.05709139021  &  -0.00002173059  \\
                        3  &  3.05710352927  &  -0.00000003704  \\
                        4  &  3.05710354996  &   0              \\
                    \bottomrule
                \end{tabular}
            \end{table}

        \item Let

            \begin{align*}
                            f(x) &= x + 1 - 2 \sin{\pi x} \\
                \Rightarrow f'(x) &= 1 - 2 \pi \cos{\pi x}
            \end{align*}

            Applying Newton's method on \(f\) with \(p_0 = \num{0.21}\)
            generates the following table:

            \begin{table}[H]
                \centering
                \begin{tabular}{r S[table-format=1.11] S[table-format=-1.11] S[table-format=-1.11]}
                    \toprule
                    \(n\)  &    {\(p_n\)}    &   {\(f(p_n)\)}   &   {\(f'(p_n)\)}  \\
                    \midrule
                        0  &  0.21           &  -0.01581410731  &  -3.96469036415  \\
                        1  &  0.20601126296  &   0.0000957226   &  -4.01255625306  \\
                        2  &  0.20603511873  &   0.00000000339  &  -4.01227230982  \\
                        3  &  0.20603511957  &   0              &  -4.01227229977  \\
                    \bottomrule
                \end{tabular}
            \end{table}

            Applying Newton's method on \(f\) with \(p_0 = \num{0.68}\)
            generates the following table:

            \begin{table}[H]
                \centering
                \begin{tabular}{r S[table-format=1.11] S[table-format=-1.11] S[table-format=1.11]}
                    \toprule
                    \(n\)  &    {\(p_n\)}    &   {\(f(p_n)\)}   &  {\(f'(p_n)\)}  \\
                    \midrule
                        0  &  0.68           &  -0.008655851    &  4.36669904541  \\
                        1  &  0.68198224126  &   0.00003270017  &  4.39967030778  \\
                        2  &  0.68197480884  &   0.00000000046  &  4.39954692747  \\
                        3  &  0.68197480874  &   0              &  4.39954692574  \\
                    \bottomrule
                \end{tabular}
            \end{table}

            Applying Secant method with \(p_0 = \num{0.21}\) and \(p_1 =
            \num{0.22}\) generates the following table:

            \begin{longtable}{r S[table-format=1.11] S[table-format=-1.11]}
                \toprule
                \(n\)  &    {\(p_n\)}    &   {\(f(p_n)\)}   \\
                \midrule
                \endfirsthead
                \(n\)  &    {\(p_n\)}    &   {\(f(p_n)\)}   \\
                \midrule
                \endhead
                    0  &  0.21           &  -0.01581410731  \\
                    1  &  0.22           &  -0.0548479795   \\
                    2  &  0.20594861939  &   0.00034710682  \\
                    3  &  0.20603698468  &  -0.0000074833   \\
                    4  &  0.20603511981  &  -0.00000000096  \\
                    5  &  0.20603511957  &   0              \\
                \bottomrule
            \end{longtable}

            Applying Secant method with \(p_0 = \num{0.68}\) and \(p_1 =
            \num{0.69}\) generates the following table:

            \begin{longtable}{r S[table-format=1.11] S[table-format=-1.11]}
                \toprule
                \(n\)  &    {\(p_n\)}    &   {\(f(p_n)\)}   \\
                \midrule
                \endfirsthead
                \(n\)  &    {\(p_n\)}    &   {\(f(p_n)\)}   \\
                \midrule
                \endhead
                    0  &  0.68           &  -0.008655851    \\
                    1  &  0.69           &   0.03583885145  \\
                    2  &  0.68194536665  &  -0.00012952468  \\
                    3  &  0.68197437195  &  -0.00000192166  \\
                    4  &  0.68197480876  &   0.00000000107  \\
                    5  &  0.68197480874  &   0              \\
                \bottomrule
            \end{longtable}

            Applying False Position method with \(p_0 = \num{0.21}\) and \(p_1 =
            \num{0.22}\) generates the following table:

            \begin{longtable}{r S[table-format=1.11] S[table-format=-1.11]}
                \toprule
                \(n\)  &    {\(p_n\)}    &   {\(f(p_n)\)}   \\
                \midrule
                \endfirsthead
                \(n\)  &    {\(p_n\)}    &   {\(f(p_n)\)}   \\
                \midrule
                \endhead
                    0  &  0.21           &  -0.01581410731  \\
                    1  &  0.22           &  -0.0548479795   \\
                    2  &  0.20594861939  &   0.00034710682  \\
                    3  &  0.20603698468  &  -0.0000074833   \\
                    4  &  0.20603511981  &  -0.00000000096  \\
                    5  &  0.20603511957  &   0              \\
                \bottomrule
            \end{longtable}

            Applying False Position method with \(p_0 = \num{0.68}\) and \(p_1 =
            \num{0.69}\) generates the following table:

            \begin{longtable}{r S[table-format=1.11] S[table-format=-1.11]}
                \toprule
                \(n\)  &    {\(p_n\)}    &   {\(f(p_n)\)}   \\
                \midrule
                \endfirsthead
                \(n\)  &    {\(p_n\)}    &   {\(f(p_n)\)}   \\
                \midrule
                \endhead
                    0  &  0.68           &  -0.008655851    \\
                    1  &  0.69           &   0.03583885145  \\
                    2  &  0.68194536665  &  -0.00012952467  \\
                    3  &  0.68197437195  &  -0.00000192166  \\
                    4  &  0.68197480226  &  -0.00000002851  \\
                    5  &  0.68197480864  &  -0.00000000042  \\
                \bottomrule
            \end{longtable}
    \end{enumerate}
\end{solution}

\begin{exercise}
    Use Newton's method to approximate, to within \(10^{-4}\), the value of
    \(x\) that produces the point on the graph of \(y = x^2\) that is closest to
    \((1, 0)\).
\end{exercise}

\begin{solution}
    Let \(d\) be the squared distance between the point \((x, x^2)\) of the
    graph and \((1, 0)\).

    \begin{align*}
                      d(x) &= (x - 1)^2 + x^4 \\
        \Rightarrow  d'(x) &= 4x^3 + 2(x - 1) \\
        \Rightarrow d''(x) &= 12x^2 + 2
    \end{align*}

    We need to find \(x\) that minimizes \(d\). First we have to examine \(d'\).
    As \(d''(x) \geq 2 > 0 \, \forall x \in \mathbb{R}\), \(d'\) is
    monotonically increasing in \(\mathbb{R}\). It follows that \(d'\) has at
    most one zero in \(\mathbb{R}\).

    Applying Newton's method on \(d'\) with \(p_0 = 0.59\) generates the
    following table:

    \begin{table}[H]
        \centering
        \begin{tabular}{r S[table-format=1.9] S[table-format=1.9] S[table-format=1.8]}
            \toprule
            \(n\)  &   {\(p_n\)}   &  {\(d'(p_n)\)}  &  {\(d''(p_n)\)}  \\
            \midrule
                0  &  0.59         &  0.001516       &  6.1772          \\
                1  &  0.589754581  &  0.000000426    &  6.17372559      \\
                2  &  0.589754512  &  0              &  6.17372462      \\
            \bottomrule
        \end{tabular}
    \end{table}

    Then \(p \approx \num{0.58975}\) is the only zero of \(d'\). Since the sign
    of \(d'\) changes from negative to positive as \(x\) passes \(p\), the
    global minimum of \(d\) is achieved at \(p\).

    We conclude that \(x \approx \num{0.58975}\) produces the point on the graph
    of \(y = x^2\) that is closest to \((1, 0)\).
\end{solution}

\begin{exercise}
    Use Newton's method to approximate, to within \(10^{-4}\), the value of
    \(x\) that produces the point on the graph of \(y = \frac{1}{x}\) that is
    closest to \((2, 1)\).
\end{exercise}

\begin{solution}
    Let \(d\) be the squared distance between the point \((x, \frac{1}{x})\) of
    the graph and \((2, 1)\).

    \begin{align*}
                      d(x) &= (x - 2)^2 + \left(\frac{1}{x} - 1\right)^2 \\
        \Rightarrow  d'(x) &= 2(x - 2) - 2 \left(\frac{1}{x} - 1\right) \frac{1}{x^2} = \frac{2(x^4 - 2x^3 + x - 1)}{x^3} \\
        \Rightarrow d''(x) &= 2 \left(\frac{3}{x} - 2\right) \frac{1}{x^3} + 2 = \frac{2(x^4 - 2x + 3)}{x^4}
    \end{align*}

    Let

    \begin{align*}
                     f(x) &= x^4 - 2x + 3 \\
        \Rightarrow f'(x) &= 4x^3 - 2
    \end{align*}

    \(f'\) has exactly one zero at \(\num{0.5}^{\sfrac{1}{3}}\). Since \(f'\) is
    monotonically increasing in \(\mathbb{R}\), the sign of \(f'\) changes from
    negative to positive as \(x\) passes \(\num{0.5}^{\sfrac{1}{3}}\). It
    follows that the global minimum of \(f\) is achieved at
    \(\num{0.5}^{\sfrac{1}{3}}\):

    \[f(x) \geq f(\num{0.5}^{\sfrac{1}{3}}) \approx \num{1.809449211} > 0\]

    Then, \(d''(x) > 0 \, \forall x \in \mathbb{R} \setminus {0}\). It follows
    that \(d'\) is monotonically increasing in \(D^+ = \mathbb{R}_{> 0}\) and
    \(D^- = \mathbb{R}_{< 0}\), which means it has at most one zero in \(D^+\)
    and \(D^-\) alike.

    Let

    \begin{align*}
                     g(x) &= x^4 - 2x^3 + x - 1 \\
        \Rightarrow g'(x) &= 4x^3 - 6x^2 + 1
    \end{align*}

    Every zero of \(g\) is also a zero of \(d'\). Applying Newton's method on
    \(g\) with \(p_0 = \num{1.86}\) generates the following table:

    \begin{table}[H]
        \centering
        \begin{tabular}{r S[table-format=1.8] S[table-format=-1.9] S[table-format=1.8]}
            \toprule
            \(n\)  &   {\(p_n\)}   &  {\(g(p_n)\)}  &  {\(g'(p_n)\)}  \\
            \midrule
                0  &  1.86         &  -0.04087984   &  5.981824       \\
                1  &  1.86683401   &   0.000449982  &  6.11376765     \\
                2  &  1.86676041   &   0.000000053  &  6.11233849     \\
            \bottomrule
        \end{tabular}
    \end{table}

    Applying Newton's method on \(g\) with \(p_0 = \num{-0.86}\) generates the
    following table:

    \begin{table}[H]
        \centering
        \begin{tabular}{r S[table-format=-1.9] S[table-format=-1.9] S[table-format=-1.8]}
            \toprule
            \(n\)  &    {\(p_n\)}   &  {\(g(p_n)\)}  &  {\(g'(p_n)\)}  \\
            \midrule
                0  &  -0.86         &  -0.04087984   &  -5.981824      \\
                1  &  -0.866834009  &   0.000449982  &  -6.11376765    \\
                2  &  -0.866760408  &   0.000000053  &  -6.11233849    \\
            \bottomrule
        \end{tabular}
    \end{table}

    We conclude that \(x \approx \num{1.86676}\) and \(x \approx
    \num{-0.86676}\) produce the points on the graph of \(y = x^2\) that are
    closest to \((1, 0)\).
\end{solution}

\begin{exercise}
    The following describes Newton's method graphically:

    Suppose that \(f'(x)\) exists on \(\interval{a}{b}\) and that \(f'(x) \neq 0
    \, \forall x \in \interval{a}{b}\). Further, suppose there exists one \(p
    \in \interval{a}{b}\) such that \(f(p) = 0\).

    Let \(p_0 \in \interval{a}{b}\) be arbitrary. Let \(p_1\) be the point at
    which the tangent line to \(f\) at \((p_0, f(p_0))\) crosses the x-axis. For
    each \(n \geq 1\), let \(p_n\) be the x-intercept of the line tangent to
    \(f\) at \((p_{n - 1}, f(p_{n - 1}))\). Derive the formula describing this
    method.
\end{exercise}

\begin{solution}
    The equation of the line tangent to \(f\) at \((p_{n - 1}, f(p_{n - 1}))\)
    is:

    \[y = f'(p_{n - 1}) (x - p_{n - 1}) + f(p_{n - 1})\]

    Then its x-intercept is:

    \[x = p_{n - 1} - \frac{f(p_{n - 1})}{f'(p_{n - 1})}\]

    Then the formula describing the sequence generated by the procedure is:

    \[\{p_n\} \mid p_n = p_{n - 1} - \frac{f(p_{n - 1})}{f'(p_{n - 1})}\]
\end{solution}

\begin{exercise}
    Use Newton's method to solve the equation

    \[0 = \frac{1}{2} + \frac{1}{4} x^2 - x \sin{x} - \frac{1}{2} \cos{2x} \text{ with } p_0 = \frac{\pi}{2}\]

    Iterate using Newton's method until an accuracy of \(10^{-5}\) is obtained.
    Explain why the result seems unusual for Newton's method. Also, solve the
    equation with \(p_0 = 5 \pi\) and \(p_0 = 10 \pi\).
\end{exercise}

\begin{solution}
    Let

    \begin{align*}
                     f(x) &= \frac{1}{2} + \frac{1}{4} x^2 - x \sin{x} - \frac{1}{2} \cos{2x} \\
        \Rightarrow f'(x) &= \frac{1}{2} x - \sin{x} + x \cos{x} + \sin{2x}
    \end{align*}

    Applying Newton's method on \(f\) with \(p_0 = \frac{\pi}{2}\) generates the
    following table:

    \begin{longtable}{r S[table-format=1.8] S[table-format=1.9] S[table-format=-1.9]}
        \toprule
        \(n\)  &   {\(p_n\)}   &  {\(f(p_n)\)}  &  {\(f'(p_n)\)}  \\
        \midrule
        \endfirsthead
        \(n\)  &   {\(p_n\)}   &  {\(f(p_n)\)}  &  {\(f'(p_n)\)}  \\
        \midrule
        \endhead
            0  &  1.57079633   &  0.046053948    &  -0.214601837  \\
            1  &  1.78539816   &  0.007116978    &  -0.120293455  \\
            2  &  1.84456163   &  0.001638544    &  -0.062366566  \\
            3  &  1.87083442   &  0.000396329    &  -0.031675918  \\
            4  &  1.88334643   &  0.000097601    &  -0.015954846  \\
            5  &  1.88946376   &  0.000024225    &  -0.008005932  \\
            6  &  1.89248962   &  0.000006035    &  -0.004010008  \\
            7  &  1.89399457   &  0.000001506    &  -0.002006754  \\
            8  &  1.89474507   &  0.000000376    &  -0.001003813  \\
            9  &  1.89511983   &  0.000000094    &  -0.000502015  \\
           10  &  1.89530709   &  0.000000023    &  -0.000251035  \\
           11  &  1.89540069   &  0.000000006    &  -0.000125524  \\
           12  &  1.89544748   &  0.000000001    &  -0.000062764  \\
           13  &  1.89547087   &  0              &  -0.000031382  \\
           14  &  1.89548257   &  0              &  -0.000015691  \\
           15  &  1.89548842   &  0              &  -0.000007846  \\
        \bottomrule
    \end{longtable}

    It's clear that the number of iteration is unusually large.

    Applying Newton's method on \(f\) with \(p_0 = 5 \pi\) generates the
    following table:

    \begin{longtable}{r S[table-format=2.8] S[table-format=3.9] S[table-format=-1.9]}
        \toprule
        \(n\)  &   {\(p_n\)}   &   {\(f(p_n)\)}  &  {\(f'(p_n)\)}  \\
        \midrule
        \endfirsthead
        \(n\)  &   {\(p_n\)}   &   {\(f(p_n)\)}  &  {\(f'(p_n)\)}  \\
        \midrule
        \endhead
            0  &  15.7079633   &   61.6850275    &  23.5619449     \\
            1  &  13.0899694   &   36.54184      &  -4.42523593    \\
            2  &  21.347572    &  101.479949     &  26.1907751     \\
            3  &  17.4729273   &   94.4331539    &   5.96762372    \\
            4  &   1.64867992  &    0.029800649  &  -0.199491346   \\
            5  &   1.79806309  &    0.005663214  &  -0.109166251   \\
            6  &   1.84994006  &    0.001319265  &  -0.056337315   \\
            7  &   1.87335731  &    0.000320334  &  -0.028563789   \\
            8  &   1.884572    &    0.000079014  &  -0.014376187   \\
            9  &   1.89006817  &    0.000019626  &  -0.007211151   \\
           10  &   1.8927898   &    0.00000489   &  -0.003611278   \\
           11  &   1.89414416  &    0.00000122   &  -0.001807057   \\
           12  &   1.89481974  &    0.000000305  &  -0.000903882   \\
           13  &   1.89515714  &    0.000000076  &  -0.000452029   \\
           14  &   1.89532573  &    0.000000019  &  -0.000226037   \\
           15  &   1.89541001  &    0.000000005  &  -0.000113024   \\
           16  &   1.89545214  &    0.000000001  &  -0.000056513   \\
           17  &   1.8954732   &    0            &  -0.000028257   \\
           18  &   1.89548374  &    0            &  -0.000014129   \\
           19  &   1.895489    &    0            &  -0.000007064   \\
        \bottomrule
    \end{longtable}

    For \(p_0 = 10 \pi\), the sequence converges and diverges back and forth,
    then finally stops at \(p_{154} \approx \num{-0.000006}\).
\end{solution}

\begin{exercise}
    The fourth-degree polynomial

    \[f(x) = 230x^4 + 18x^3 + 9x^2 - 221x - 9\]

    \noindent has two real zeros, one in \(\interval{-1}{0}\) and the other in
    \(\interval{0}{1}\). Attempt to approximate these zeros to within
    \(10^{-6}\) using the

    \begin{tasks}
        \task Method of False Position
        \task Secant method
        \task Newton's method
    \end{tasks}

    Use the endpoints of each interval as the initial approximations in a) and
    b) and the midpoints as the initial approximation in c).
\end{exercise}

\begin{solution}
    \begin{enumerate}[label = \alph*)]
        \item Applying False Position method with \(p_0 = -1\) and \(p_1 = 0\)
            generates the following table:

            \begin{longtable}{r S[table-format=-1.9] S[table-format=3.9]}
                \toprule
                \(n\)  &    {\(p_n\)}    &   {\(f(p_n)\)}   \\
                \midrule
                \endfirsthead
                \(n\)  &    {\(p_n\)}    &   {\(f(p_n)\)}   \\
                \midrule
                \endhead
                    0  &  -1             &  433             \\
                    1  &   0             &   -9             \\
                    2  &  -0.020361991   &   -4.49638093    \\
                    3  &  -0.030430247   &   -2.26689137    \\
                    4  &  -0.035479814   &   -1.14807119    \\
                    5  &  -0.038030414   &   -0.58277074    \\
                    6  &  -0.03932338    &   -0.296160751   \\
                    7  &  -0.039980008   &   -0.150595231   \\
                    8  &  -0.040313782   &   -0.076599144   \\
                    9  &  -0.040483524   &   -0.038967468   \\
                   10  &  -0.040569867   &   -0.019825027   \\
                   11  &  -0.040613793   &   -0.010086543   \\
                   12  &  -0.040636141   &   -0.005131916   \\
                   13  &  -0.040647511   &   -0.002611086   \\
                   14  &  -0.040653296   &   -0.00132851    \\
                   15  &  -0.04065624    &   -0.000675943   \\
                   16  &  -0.040657737   &   -0.000343918   \\
                   17  &  -0.040658499   &   -0.000174985   \\
                \bottomrule
            \end{longtable}

            Applying False Position method with \(p_0 = 0\) and \(p_1 = 1\)
            generates the following table:

            \begin{longtable}{r S[table-format=1.9] S[table-format=-2.9]}
                \toprule
                \(n\)  &    {\(p_n\)}    &   {\(f(p_n)\)}   \\
                \midrule
                \endfirsthead
                \(n\)  &    {\(p_n\)}    &   {\(f(p_n)\)}   \\
                \midrule
                \endhead
                    0  &  0              &   -9             \\
                    1  &  1              &   27             \\
                    2  &  0.25           &  -62.5078125     \\
                    3  &  0.773762765    &  -83.8305203     \\
                    4  &  0.944885169    &  -11.2651302     \\
                    5  &  0.961110797    &   -0.855867823   \\
                    6  &  0.962305662    &   -0.061802369   \\
                    7  &  0.962391747    &   -0.004446181   \\
                    8  &  0.962397939    &   -0.000319781   \\
                    9  &  0.962398384    &   -0.000022999   \\
                \bottomrule
            \end{longtable}

        \item Applying Secant method with \(p_0 = -1\) and \(p_1 = 0\) generates
            the following table:

            \begin{longtable}{r S[table-format=-1.9] S[table-format=3.9]}
                \toprule
                \(n\)  &    {\(p_n\)}    &   {\(f(p_n)\)}   \\
                \midrule
                \endfirsthead
                \(n\)  &    {\(p_n\)}    &   {\(f(p_n)\)}   \\
                \midrule
                \endhead
                    0  &  -1             &  433             \\
                    1  &   0             &   -9             \\
                    2  &  -0.020361991   &   -4.49638093    \\
                    3  &  -0.040691256   &    0.007087483   \\
                    4  &  -0.040659263   &   -0.000005706   \\
                    5  &  -0.040659288   &    0             \\
                \bottomrule
            \end{longtable}

            Applying Secant method with \(p_0 = 0\) and \(p_1 = 1\) generates
            the following table:

            \begin{longtable}{r S[table-format=-1.9] S[table-format=-3.9]}
                \toprule
                \(n\)  &    {\(p_n\)}    &   {\(f(p_n)\)}   \\
                \midrule
                \endfirsthead
                \(n\)  &    {\(p_n\)}    &   {\(f(p_n)\)}   \\
                \midrule
                \endhead
                    0  &   0             &    -9            \\
                    1  &   1             &    27            \\
                    2  &   0.25          &   -62.5078125    \\
                    3  &   0.773762765   &   -83.8305203    \\
                    4  &  -1.28541778    &   879.638986     \\
                    5  &   0.59459552    &  -104.691389     \\
                    6  &   0.394641105   &   -88.1289404    \\
                    7  &  -0.669318136   &   183.71316      \\
                    8  &   0.049714398   &   -19.9610216    \\
                    9  &  -0.020754151   &    -4.40957429   \\
                   10  &  -0.040735333   &     0.016859473  \\
                   11  &  -0.040659228   &    -0.000013318  \\
                   12  &  -0.040659288   &     0            \\
                \bottomrule
            \end{longtable}

        \item Applying Newton's method with \(p_0 = \num{-0.5}\) generates the
            following table:

            \begin{table}[H]
                \centering
                \begin{tabular}{r S[table-format=-1.9] S[table-format=3.9] S[table-format=-3.6]}
                    \toprule
                    \(n\)  &    {\(p_n\)}   &   {\(g(p_n)\)}  &  {\(g'(p_n)\)}  \\
                    \midrule
                        0  &  -0.5          &  115.875        &  -331.5         \\
                        1  &  -0.150452489  &   24.510271     &  -225.618988    \\
                        2  &  -0.041816814  &    0.256640771  &  -221.725549    \\
                        3  &  -0.040659344  &    0.000012234  &  -221.704436    \\
                        4  &  -0.040659288  &    0            &  -221.704435    \\
                    \bottomrule
                \end{tabular}
            \end{table}

            Applying Newton's method with \(p_0 = \num{0.5}\) generates the
            following table:

            \begin{longtable}{r S[table-format=-1.9] S[table-format=-3.9] S[table-format=-3.6]}
                \toprule
                \(n\)  &    {\(p_n\)}   &   {\(g(p_n)\)}   &  {\(g'(p_n)\)}  \\
                \midrule
                \endfirsthead
                \(n\)  &    {\(p_n\)}   &   {\(g(p_n)\)}   &  {\(g'(p_n)\)}  \\
                \midrule
                \endhead
                    0  &   0.5          &  -100.625        &   -83.5         \\
                    1  &  -0.70508982   &   201.836304     &  -529.339073    \\
                    2  &  -0.323791114  &    65.4184267    &  -252.397607    \\
                    3  &  -0.064603131  &     5.31400707   &  -222.185539    \\
                    4  &  -0.040686151  &     0.005955616  &  -221.704923    \\
                    5  &  -0.040659288  &     0.000000007  &  -221.704435    \\
                    6  &  -0.040659288  &     0            &  -221.704435    \\
                \bottomrule
            \end{longtable}
    \end{enumerate}
\end{solution}

\begin{exercise}
    The function \(f(x) = \tan{\pi x} - 6\) has a zero at
    \(\frac{\arctan(6)}{\pi} \approx \num{0.447431543}\). Let \(p_0 = 0\) and
    \(p_1 = \num{0.48}\), and use ten iterations of each of the following
    methods to approximate this root. Which method is most successful and why?

    \begin{tasks}(3)
        \task Bisection
        \task False Position
        \task Secant
    \end{tasks}
\end{exercise}

\begin{solution}
    \begin{enumerate}[label = \alph*)]
        \item Applying Bisection method on \(f\) with \(a = 0\), \(b =
            \num{0.48}\) generates the following table:

            \begin{longtable}{r S[table-format=1.7] S[table-format=1.2] S[table-format=1.8] S[table-format=-2.7]}
                \toprule
                \(n\)  &   {\(a_n\)}   &   {\(b_n\)}   &   {\(p_n\)}   &  {\(f(p_n)\)}  \\
                \midrule
                \endfirsthead
                \(n\)  &   {\(a_n\)}   &   {\(b_n\)}   &   {\(p_n\)}   &  {\(f(p_n)\)}  \\
                \midrule
                \endhead
                    1  &  0            &  0.48         &  0.24         &  -60.5096832   \\
                    2  &  0.24         &  0.48         &  0.36         &  -82.6906752   \\
                    3  &  0.36         &  0.48         &  0.42         &  -91.7419152   \\
                    4  &  0.42         &  0.48         &  0.45         &  -95.5558125   \\
                    5  &  0.45         &  0.48         &  0.465        &  -97.2559241   \\
                    6  &  0.465        &  0.48         &  0.4725       &  -98.0504281   \\
                    7  &  0.4725       &  0.48         &  0.47625      &  -98.4332975   \\
                    8  &  0.47625      &  0.48         &  0.478125     &  -98.6210739   \\
                    9  &  0.478125     &  0.48         &  0.4790625    &  -98.7140395   \\
                   10  &  0.4790625    &  0.48         &  0.47953125   &  -98.7602908   \\
                \bottomrule
            \end{longtable}

            The method indeed does not produce the root in this case, as
            \(f(a_1)\) and \(f(b_1)\) have the same sign.

        \item Applying method of False Position on \(f\) with \(p_0 = 0\) and
            \(p_1 = \num{0.48}\) generates the following table:

            \begin{longtable}{r S[table-format=-1.9] S[table-format=-2.9]}
                \toprule
                \(n\)  &    {\(p_n\)}    &   {\(f(p_n)\)}   \\
                \midrule
                \endfirsthead
                \(n\)  &    {\(p_n\)}    &   {\(f(p_n)\)}   \\
                \midrule
                \endhead
                    0  &   0             &   -9             \\
                    1  &   0.48          &  -98.8063872     \\
                    2  &  -0.048103483   &    1.65092314    \\
                    3  &  -0.03942459    &   -0.273724354   \\
                    4  &  -0.040658906   &   -0.000084697   \\
                    5  &  -0.040659288   &   -0.000000026   \\
                \bottomrule
            \end{longtable}

        \item Applying Secant method on \(f\) with \(p_0 = 0\) and \(p_1 =
            \num{0.48}\) generates the following table:

            \begin{longtable}{r S[table-format=-1.9] S[table-format=-2.9]}
                \toprule
                \(n\)  &    {\(p_n\)}    &   {\(f(p_n)\)}   \\
                \midrule
                \endfirsthead
                \(n\)  &    {\(p_n\)}    &   {\(f(p_n)\)}   \\
                \midrule
                \endhead
                    0  &   0             &   -9             \\
                    1  &   0.48          &  -98.8063872     \\
                    2  &  -0.048103483   &    1.65092314    \\
                    3  &  -0.03942459    &   -0.273724354   \\
                    4  &  -0.040658906   &   -0.000084697   \\
                    5  &  -0.040659288   &    0.000000004   \\
                \bottomrule
            \end{longtable}
    \end{enumerate}

    Clearly, Secant method is the most successful one in this case.
\end{solution}

\begin{exercise}
    The iteration equation for the Secant method can be written in the simpler
    form:

    \[p_n = \frac{f(p_{n - 1}) p_{n - 2} - f(p_{n - 2}) p_{n - 1}}{f(p_{n - 1}) - f(p_{n - 2})}\]

    Explain why, in general, this iteration equation is likely to be less
    accurate than the one given in the text book.
\end{exercise}

\begin{solution}
    In both formulas, the denominator is close to \(0\) as consecutive \(p_n\)
    is close to each other.

    In the above formula, the numerator is also close to \(0\) for the same
    reason. Therefore, both numerator and denominator are close to \(0\), which
    can lead to losing digits.

    The formula provided in the text book circumvents this situation by having
    the difference of 2 consecutive \(p_n\) multiplied with \(f\) \emph{before}
    dividing.

    As a consequence, the formula should be written in the specific way that it
    is printed in the text book, as it implies the multiplication should be done
    before division.
\end{solution}

\begin{exercise}
    The equation \(x^2 - 10 \cos{x} = 0\) has two solutions, \(\pm
    \num{1.3793646}\). Use Newton's method to approximate the solutions to
    within \(10^{-5}\) with the following values of \(p_0\).

    \begin{tasks}(3)
        \task \(p_0 = -100\)
        \task \(p_0 = -50\)
        \task \(p_0 = -25\)
        \task \(p_0 = 25\)
        \task \(p_0 = 50\)
        \task \(p_0 = 100\)
    \end{tasks}
\end{exercise}

\begin{solution}
    Let

    \begin{align*}
                     f(x) &= x^2 - 10 \cos{x} \\
        \Rightarrow f'(x) &= 2x + 10 \sin{x}
    \end{align*}

    \begin{enumerate}[label = \alph*)]
        \item Applying Newton's method with \(p_0 = -100\) generates the
            following table:

            \begin{longtable}{r S[table-format=-3.10] S[table-format=4.10] S[table-format=-3.10]}
                \toprule
                \(n\)  &     {\(p_n\)}     &    {\(f(p_n)\)}   &   {\(f'(p_n)\)}   \\
                \midrule
                \endfirsthead
                \(n\)  &     {\(p_n\)}     &    {\(f(p_n)\)}   &   {\(f'(p_n)\)}   \\
                \midrule
                \endhead
                    0  &  -100             &  9991.3768112771  &  -194.9363435889  \\
                    1  &   -48.7454384989  &  2375.6104686195  &   -87.503753248   \\
                    2  &   -21.596769094   &   475.6527869722  &   -47.0358919679  \\
                    3  &   -11.4842195691  &   127.1929976708  &   -14.1387429948  \\
                    4  &    -2.4881583409  &    14.1309390157  &   -11.0554850027  \\
                    5  &    -1.2099747957  &    -2.0663908208  &   -11.7760206276  \\
                    6  &    -1.3854492523  &     0.076592885   &   -12.5996219873  \\
                    7  &    -1.3793702695  &     0.0000713728  &   -12.5760796699  \\
                    8  &    -1.3793645942  &     0.0000000001  &   -12.5760575214  \\
                \bottomrule
            \end{longtable}

        \item Applying Newton's method with \(p_0 = -50\) generates the
            following table:

            \begin{longtable}{r S[table-format=-2.10] S[table-format=4.10] S[table-format=-2.10]}
                \toprule
                \(n\)  &     {\(p_n\)}    &    {\(f(p_n)\)}   &   {\(f'(p_n)\)}  \\
                \midrule
                \endfirsthead
                \(n\)  &     {\(p_n\)}    &    {\(f(p_n)\)}   &   {\(f'(p_n)\)}  \\
                \midrule
                \endhead
                    0  &  -50             &  2490.3503397151  &  -97.376251463   \\
                    1  &  -24.4254856569  &   589.0028702885  &  -42.3534708223  \\
                    2  &  -10.5186473541  &   115.2324542098  &  -12.1531966041  \\
                    3  &   -1.0369893209  &    -4.0127969624  &  -10.6827411852  \\
                    4  &   -1.4126229615  &     0.4203572492  &  -12.7004124469  \\
                    5  &   -1.3795250404  &     0.0020178304  &  -12.5766835597  \\
                    6  &   -1.3793645982  &     0.0000000502  &  -12.576057537   \\
                    7  &   -1.3793645942  &     0             &  -12.5760575214  \\
                \bottomrule
            \end{longtable}

        \item Applying Newton's method with \(p_0 = -25\) generates the
            following table:

            \begin{longtable}{r S[table-format=-2.10] S[table-format=3.10] S[table-format=-2.10]}
                \toprule
                \(n\)  &     {\(p_n\)}    &    {\(f(p_n)\)}   &   {\(f'(p_n)\)}  \\
                \midrule
                \endfirsthead
                \(n\)  &     {\(p_n\)}    &    {\(f(p_n)\)}   &   {\(f'(p_n)\)}  \\
                \midrule
                \endhead
                    0  &  -25             &  615.0879718814  &  -48.676482499   \\
                    1  &  -12.3637547271  &  143.0669956648  &  -22.7151855357  \\
                    2  &   -6.0654572538  &   27.0258643344  &   -9.9707957587  \\
                    3  &   -3.3549550042  &   21.0289678026  &   -4.5924380275  \\
                    4  &    1.2240872555  &   -1.8996558667  &   11.8531352735  \\
                    5  &    1.3843533642  &    0.0627874198  &   12.5954047231  \\
                    6  &    1.3793684177  &    0.0000480838  &   12.5760724428  \\
                    7  &    1.3793645942  &    0             &   12.5760575214  \\
                \bottomrule
            \end{longtable}

        \item Applying Newton's method with \(p_0 = 25\) generates the following
            table:

            \begin{longtable}{r S[table-format=-1.10] S[table-format=3.10] S[table-format=-2.10]}
                \toprule
                \(n\)  &    {\(p_n\)}    &   {\(f(p_n)\)}   &   {\(f'(p_n)\)}  \\
                \midrule
                \endfirsthead
                \(n\)  &    {\(p_n\)}    &   {\(f(p_n)\)}   &   {\(f'(p_n)\)}  \\
                \midrule
                \endhead
                    0  &  25             &  615.0879718814  &   48.676482499   \\
                    1  &  12.3637547271  &  143.0669956648  &   22.7151855357  \\
                    2  &   6.0654572538  &   27.0258643344  &    9.9707957587  \\
                    3  &   3.3549550042  &   21.0289678026  &    4.5924380275  \\
                    4  &  -1.2240872555  &   -1.8996558667  &  -11.8531352735  \\
                    5  &  -1.3843533642  &    0.0627874198  &  -12.5954047231  \\
                    6  &  -1.3793684177  &    0.0000480838  &  -12.5760724428  \\
                    7  &  -1.3793645942  &    0             &  -12.5760575214  \\
                \bottomrule
            \end{longtable}

        \item Applying Newton's method with \(p_0 = 50\) generates the following
            table:

            \begin{longtable}{r S[table-format=2.10] S[table-format=4.10] S[table-format=2.10]}
                \toprule
                \(n\)  &    {\(p_n\)}    &    {\(f(p_n)\)}   &  {\(f'(p_n)\)}  \\
                \midrule
                \endfirsthead
                \(n\)  &    {\(p_n\)}    &    {\(f(p_n)\)}   &  {\(f'(p_n)\)}  \\
                \midrule
                \endhead
                    0  &  50             &  2490.3503397151  &  97.376251463   \\
                    1  &  24.4254856569  &   589.0028702885  &  42.3534708223  \\
                    2  &  10.5186473541  &   115.2324542098  &  12.1531966041  \\
                    3  &   1.0369893209  &    -4.0127969624  &  10.6827411852  \\
                    4  &   1.4126229615  &     0.4203572492  &  12.7004124469  \\
                    5  &   1.3795250404  &     0.0020178304  &  12.5766835597  \\
                    6  &   1.3793645982  &     0.0000000502  &  12.576057537   \\
                    7  &   1.3793645942  &     0             &  12.5760575214  \\
                \bottomrule
            \end{longtable}

        \item Applying Newton's method with \(p_0 = 100\) generates the
            following table:

            \begin{longtable}{r S[table-format=3.10] S[table-format=4.10] S[table-format=3.10]}
                \toprule
                \(n\)  &     {\(p_n\)}    &    {\(f(p_n)\)}   &   {\(f'(p_n)\)}  \\
                \midrule
                \endfirsthead
                \(n\)  &     {\(p_n\)}    &    {\(f(p_n)\)}   &   {\(f'(p_n)\)}  \\
                \midrule
                \endhead
                    0  &  100             &  9991.3768112771  &  194.9363435889  \\
                    1  &   48.7454384989  &  2375.6104686195  &   87.503753248   \\
                    2  &   21.596769094   &   475.6527869722  &   47.0358919679  \\
                    3  &   11.4842195691  &   127.1929976708  &   14.1387429948  \\
                    4  &    2.4881583409  &    14.1309390157  &   11.0554850027  \\
                    5  &    1.2099747957  &    -2.0663908208  &   11.7760206276  \\
                    6  &    1.3854492523  &     0.076592885   &   12.5996219873  \\
                    7  &    1.3793702695  &     0.0000713728  &   12.5760796699  \\
                    8  &    1.3793645942  &     0.0000000001  &   12.5760575214  \\
                \bottomrule
            \end{longtable}
    \end{enumerate}
\end{solution}

\begin{exercise}
    The equation \(4x^2 - e^x - e^{-x} = 0\) has two positive solutions \(x_1\)
    and \(x_2\). Use Newton's method to approximate the solution to within
    \(10^{-5}\) with the following values of \(p_0\).

    \begin{tasks}(3)
        \task \(p_0 = -10\)
        \task \(p_0 = -5\)
        \task \(p_0 = -3\)
        \task \(p_0 = -1\)
        \task \(p_0 = 0\)
        \task \(p_0 = 1\)
        \task \(p_0 = 3\)
        \task \(p_0 = 5\)
        \task \(p_0 = 10\)
    \end{tasks}
\end{exercise}

\begin{solution}
    Let

    \begin{align*}
                     f(x) &= 4x^2 - e^x - e^{-x} \\
        \Rightarrow f'(x) &= 8x - e^x + e^{-x}
    \end{align*}

    \begin{enumerate}[label = \alph*)]
        \item Applying Newton's method with \(p_0 = -10\) generates the
            following table:

            \begin{longtable}{r S[table-format=-2.10] S[table-format=-5.10] S[table-format=5.10]}
                \toprule
                \(n\)  &     {\(p_n\)}    &     {\(f(p_n)\)}    &    {\(f'(p_n)\)}   \\
                \midrule
                \endfirsthead
                \(n\)  &     {\(p_n\)}    &     {\(f(p_n)\)}    &    {\(f'(p_n)\)}   \\
                \midrule
                \endhead
                    0  &  -10             &  -21626.4658402066  &  21946.4657494068  \\
                    1  &   -9.0145809313  &   -7897.0494558112  &   8149.9832425813  \\
                    2  &   -8.0456158156  &   -2861.1584947403  &   3055.7206626145  \\
                    3  &   -7.1092872664  &   -1021.1083215684  &   1166.4002502262  \\
                    4  &   -6.2338516504  &    -354.2732875489  &    459.8421761797  \\
                    5  &   -5.4634280009  &    -116.5127783823  &    192.1930584606  \\
                    6  &   -4.8572001833  &     -34.3016609642  &     89.7980895533  \\
                    7  &   -4.4752136496  &      -7.7145986461  &     52.0002627102  \\
                    8  &   -4.3268567329  &      -0.8324004204  &     41.0778853008  \\
                    9  &   -4.3065927778  &      -0.0137992441  &     39.7210636401  \\
                   10  &   -4.3062453741  &      -0.0000039943  &     39.6980697257  \\
                   11  &   -4.3062452735  &       0             &     39.6980630673  \\
                \bottomrule
            \end{longtable}

        \item Applying Newton's method with \(p_0 = -5\) generates the following
            table:

            \begin{longtable}{r S[table-format=-1.10] S[table-format=-2.10] S[table-format=3.10]}
                \toprule
                \(n\)  &     {\(p_n\)}    &    {\(f(p_n)\)}   &   {\(f'(p_n)\)}  \\
                \midrule
                \endfirsthead
                \(n\)  &     {\(p_n\)}    &    {\(f(p_n)\)}   &   {\(f'(p_n)\)}  \\
                \midrule
                \endhead
                    0  &  -5             &  -48.4198970496  &  108.4064211556  \\
                    1  &  -4.5533484407  &  -12.0284142159  &   58.5124910196  \\
                    2  &  -4.3477784161  &   -1.7067559697  &   42.5113662274  \\
                    3  &  -4.3076301894  &   -0.0550419721  &   39.7897810066  \\
                    4  &  -4.3062468701  &   -0.0000633809  &   39.6981687205  \\
                    5  &  -4.3062452735  &   -0.0000000001  &   39.6980630674  \\
                \bottomrule
            \end{longtable}

        \item Applying Newton's method with \(p_0 = -3\) generates the following
            table:

            \begin{longtable}{r S[table-format=-1.10] S[table-format=2.10] S[table-format=-1.10]}
                \toprule
                \(n\)  &    {\(p_n\)}    &   {\(f(p_n)\)}  &  {\(f'(p_n)\)}  \\
                \midrule
                \endfirsthead
                \(n\)  &    {\(p_n\)}    &   {\(f(p_n)\)}  &  {\(f'(p_n)\)}  \\
                \midrule
                \endhead
                    0  &  -3             &  15.8646760084  &  -3.9642501452  \\
                    1  &   1.0019361613  &   0.9247864701  &   5.6591071879  \\
                    2  &   0.8385205483  &   0.0671745913  &   4.82757152    \\
                    3  &   0.8246057692  &   0.0005095513  &   4.754272591   \\
                    4  &   0.8244985917  &   0.0000000303  &   4.7537066175  \\
                    5  &   0.8244985853  &   0             &   4.7537065838  \\
                \bottomrule
            \end{longtable}

        \item Applying Newton's method with \(p_0 = -1\) generates the following
            table:

            \begin{longtable}{r S[table-format=-1.10] S[table-format=1.10] S[table-format=-1.10]}
                \toprule
                \(n\)  &    {\(p_n\)}    &  {\(f(p_n)\)}  &  {\(f'(p_n)\)}  \\
                \midrule
                \endfirsthead
                \(n\)  &    {\(p_n\)}    &  {\(f(p_n)\)}  &  {\(f'(p_n)\)}  \\
                \midrule
                \endhead
                    0  &  -1             &  0.9138387304  &  -5.6495976127  \\
                    1  &  -0.8382471119  &  0.065854754   &  -4.8261346213  \\
                    2  &  -0.824601667   &  0.0004900484  &  -4.7542509289  \\
                    3  &  -0.8244985912  &  0.0000000281  &  -4.753706615   \\
                    4  &  -0.8244985853  &  0             &  -4.7537065838  \\
                \bottomrule
            \end{longtable}

        \item The method fails in this case as \(f'(0) = 0\).

        \item Applying Newton's method with \(p_0 = 1\) generates the following
            table:

            \begin{longtable}{r S[table-format=1.10] S[table-format=1.10] S[table-format=1.10]}
                \toprule
                \(n\)  &    {\(p_n\)}   &   {\(f(p_n)\)}  &  {\(f'(p_n)\)}  \\
                \midrule
                \endfirsthead
                \(n\)  &    {\(p_n\)}   &   {\(f(p_n)\)}  &  {\(f'(p_n)\)}  \\
                \midrule
                \endhead
                    0  &  1             &   0.9138387304  &  5.6495976127   \\
                    1  &  0.8382471119  &   0.065854754   &  4.8261346213   \\
                    2  &  0.824601667   &   0.0004900484  &  4.7542509289   \\
                    3  &  0.8244985912  &   0.0000000281  &  4.753706615    \\
                    4  &  0.8244985853  &   0             &  4.7537065838   \\
                \bottomrule
            \end{longtable}

        \item Applying Newton's method with \(p_0 = 3\) generates the following
            table:

            \begin{longtable}{r S[table-format=-1.10] S[table-format=2.10] S[table-format=-1.10]}
                \toprule
                \(n\)  &    {\(p_n\)}    &   {\(f(p_n)\)}  &  {\(f'(p_n)\)}  \\
                \midrule
                \endfirsthead
                \(n\)  &    {\(p_n\)}    &   {\(f(p_n)\)}  &  {\(f'(p_n)\)}  \\
                \midrule
                \endhead
                    0  &   3             &  15.8646760084  &   3.9642501452  \\
                    1  &  -1.0019361613  &   0.9247864701  &  -5.6591071879  \\
                    2  &  -0.8385205483  &   0.0671745913  &  -4.82757152    \\
                    3  &  -0.8246057692  &   0.0005095513  &  -4.754272591   \\
                    4  &  -0.8244985917  &   0.0000000303  &  -4.7537066175  \\
                    5  &  -0.8244985853  &   0             &  -4.7537065838  \\
                \bottomrule
            \end{longtable}

        \item Applying Newton's method with \(p_0 = 5\) generates the following
            table:

            \begin{longtable}{r S[table-format=1.10] S[table-format=-2.10] S[table-format=-3.10]}
                \toprule
                \(n\)  &    {\(p_n\)}   &   {\(f(p_n)\)}   &   {\(f'(p_n)\)}   \\
                \midrule
                \endfirsthead
                \(n\)  &    {\(p_n\)}   &   {\(f(p_n)\)}   &   {\(f'(p_n)\)}   \\
                \midrule
                \endhead
                    0  &  5             &  -48.4198970496  &  -108.4064211556  \\
                    1  &  4.5533484407  &  -12.0284142159  &   -58.5124910196  \\
                    2  &  4.3477784161  &   -1.7067559697  &   -42.5113662274  \\
                    3  &  4.3076301894  &   -0.0550419721  &   -39.7897810066  \\
                    4  &  4.3062468701  &   -0.0000633809  &   -39.6981687205  \\
                    5  &  4.3062452735  &   -0.0000000001  &   -39.6980630674  \\
                \bottomrule
            \end{longtable}

        \item Applying Newton's method with \(p_0 = 10\) generates the following
            table:

            \begin{longtable}{r S[table-format=2.10] S[table-format=-5.10] S[table-format=-5.10]}
                \toprule
                \(n\)  &    {\(p_n\)}    &     {\(f(p_n)\)}    &    {\(f'(p_n)\)}    \\
                \midrule
                \endfirsthead
                \(n\)  &    {\(p_n\)}    &     {\(f(p_n)\)}    &    {\(f'(p_n)\)}    \\
                \midrule
                \endhead
                    0  &  10             &  -21626.4658402066  &  -21946.4657494068  \\
                    1  &   9.0145809313  &   -7897.0494558112  &   -8149.9832425813  \\
                    2  &   8.0456158156  &   -2861.1584947403  &   -3055.7206626145  \\
                    3  &   7.1092872664  &   -1021.1083215684  &   -1166.4002502262  \\
                    4  &   6.2338516504  &    -354.2732875489  &    -459.8421761797  \\
                    5  &   5.4634280009  &    -116.5127783823  &    -192.1930584606  \\
                    6  &   4.8572001833  &     -34.3016609642  &     -89.7980895533  \\
                    7  &   4.4752136496  &      -7.7145986461  &     -52.0002627102  \\
                    8  &   4.3268567329  &      -0.8324004204  &     -41.0778853008  \\
                    9  &   4.3065927778  &      -0.0137992441  &     -39.7210636401  \\
                   10  &   4.3062453741  &      -0.0000039943  &     -39.6980697257  \\
                   11  &   4.3062452735  &       0             &     -39.6980630673  \\
                \bottomrule
            \end{longtable}
    \end{enumerate}
\end{solution}

\begin{exercise}
    Use Maple to determine how many iterations of Newton's method with \(p_0 =
    \sfrac{\pi}{4}\) are needed to find a root of \(f(x) = \cos{x} - x\) to
    within \(10^{-100}\).
\end{exercise}

\begin{solution}
    Python FTW: 51 iterations.
\end{solution}

\begin{exercise}
    The function described by \(f(x) = \ln(x^2 + 1) - e^{\num{0.4} x} \cos{\pi
    x}\) has an infinite number of zeros.

    \begin{tasks}
        \task Determine, within \(10^{-6}\), the only negative zero.
        \task Determine, within \(10^{-6}\), the four smallest positive zeros.

        \task Determine a reasonable initial approximation to find the
            \(n^{th}\) smallest positive zero of \(f\). [Hint: Sketch an
            approximate graph of \(f\).]
        \task Use part c) to determine, within \(10^{-6}\), the \(25^{th}\)
            smallest positive zero of \(f\).
    \end{tasks}
\end{exercise}

\begin{solution}
    Differentiating \(f\) gives:

    \[f'(x) = \frac{2x}{x^2 + 1} - e^{\num{0.4} x} (\num{0.4} \cos{\pi x} - \pi \sin{\pi x})\]

    Consider each term of \(f\):

    \begin{itemize}
        \item \(\ln(x^2 + 1) \geq 0 \, \forall x \in \mathbb{R}\)
        \item \(e^{\num{0.4} x} > 0 \, \forall x \in \mathbb{R}\)
        \item \(\cos{\pi x} > 0 \iff \num{-0.5} + 2k < x < \num{0.5} + 2k\), with \(k \in \mathbb{N}\)
    \end{itemize}

    \noindent which means that every zero of \(f\) must be in \(\interval{2k -
    \num{0.5}}{2k + \num{0.5}}\), \(k \in \mathbb{N}\).

    \begin{enumerate}[label = \alph*)]
        \item \(e^x\) is monotonically increasing in \(\mathbb{R}\). It follows
            that:

            \[0 < e^{\num{0.4} x} \cos{\pi x} \leq e^{\num{0.4} x} 1 < e^{\num{0.4} \cdot 0} = 1 \, \forall x < 0\]

            \(\ln{x}\) is monotonically increasing in \(\mathbb{R}_{> 0}\).
            Therefore \(\ln(x^2 + 1)\) is monotonically decreasing in
            \(\mathbb{R}_{< 0}\). Also, \(e^{x}\) is monotonically increasing in
            \(\mathbb{R}\). Therefore, if \(f\) has a negative zero, it must
            satisfy:

            \[\ln(x^2 + 1) < 1 \iff - \sqrt{e - 1} \approx \num{-1.310832494} < x < 0\]

            Combining the above points, it is clear that if \(f\) has a negative
            zero, it must be in \(D_1 = \interval{\num{-0.5}}{0}\).

            As \(\ln(x^2 + 1)\) is monotonically decreasing in \(D_1\), it
            follows that:

            \[\ln(\num{-0.5}^2 + 1) \geq \ln(x^2 + 1) \geq \ln{1} = 0 \, \forall x \in D_1\]

            As both \(e^x\) and \(\cos{\pi x}\) is monotonically increasing in
            \(D_1\), it follows that:

            \[0 \leq e^{\num{0.4} x} \cos{\pi x} \leq 1 \, \forall x \in D_1\]

            From the above points, there must be exactly one zero of \(f\) in
            \(D_1\).

            Applying Newton method on \(f\) with \(p_0 = \num{-0.25}\) generates
            the following table:

            \begin{longtable}{r S[table-format=-1.9] S[table-format=-1.9] S[table-format=-1.9]}
                \toprule
                \(n\)  &    {\(p_n\)}   &  {\(f(p_n)\)}  &  {\(f'(p_n)\)}  \\
                \midrule
                \endfirsthead
                \(n\)  &    {\(p_n\)}   &  {\(f(p_n)\)}  &  {\(f'(p_n)\)}  \\
                \midrule
                \endhead
                    0  &  -0.25         &  -0.579192052  &  -2.797220033   \\
                    1  &  -0.457059883  &   0.077693927  &  -3.74279653    \\
                    2  &  -0.436301627  &   0.007306593  &  -3.691332860   \\
                    3  &  -0.434322236  &   0.000606405  &  -3.685958212   \\
                    4  &  -0.434157718  &   0.000049647  &  -3.685507782   \\
                    5  &  -0.434144247  &   0.00000406   &  -3.685470876   \\
                    6  &  -0.434143145  &   0.000000332  &  -3.685467857   \\
                    7  &  -0.434143055  &   0.000000027  &  -3.68546761    \\
                \bottomrule
            \end{longtable}

            We conclude that the sole negative zero of \(f\) is \(p \approx
            \num{-0.4341431}\).
    \end{enumerate}

    \begin{center}
        \textbf{not yet finished}
    \end{center}
\end{solution}

\begin{exercise}
    Find an approximation for \(\lambda\), accurate to within \(10^{-4}\), for
    the population equation

    \[\num{1564000} = \num{1000000} e^\lambda + \frac{\num{435000}}{\lambda} (e^\lambda - 1)\]

    \noindent discussed in the introduction to this chapter. Use this value to
    predict the population at the end of the second year, assuming that the
    immigration rate during this year remains at \num{435000} individuals per
    year.
\end{exercise}

\begin{solution}
    Let

    \begin{align*}
                     f(x) &= \num{1000} e^\lambda + \frac{\num{435}}{\lambda} (e^\lambda - 1) - \num{1564} \\
        \Rightarrow f'(x) &= \num{1000} e^\lambda + \num{435} \left(\frac{1 - e^\lambda}{\lambda^2} + \frac{e^\lambda}{\lambda}\right)
    \end{align*}

    Applying Newton's method on \(f\) with \(p_0 = \num{0.1}\) generates the
    following table:

    \begin{table}[H]
        \centering
        \begin{tabular}{r S[table-format=1.10] S[table-format=-1.10] S[table-format=4.10]}
            \toprule
            \(n\)  &    {\(p_n\)}   &   {\(f(p_n)\)}  &   {\(f'(p_n)\)}   \\
            \midrule
                0  &  0.1           &  -1.3355882953  &  1337.729475414   \\
                1  &  0.1009983994  &   0.000628932   &  1338.9895592632  \\
                2  &  0.1009979297  &   0.0000000001  &  1338.988966158   \\
            \bottomrule
        \end{tabular}
    \end{table}

    So \(\lambda \approx \num{0.1009979}\).

    Since

    \[N(t) = N_0 e^{\lambda t} + \frac{v}{\lambda} (e^{\lambda t} - 1)\]

    \noindent then the population predicted at the end of the second year \(N(2)
    \approx \num{2187.938632} \cdot 1000 = \num{2187938.632}\).
\end{solution}

\begin{exercise}
    The sum of two numbers is \(20\). If each number is added to its square
    root, the product of the two sums is \num{155.55}. Determine the two numbers
    to within \(10^{-4}\).
\end{exercise}

\begin{solution}
    Let one number is \(x \in \interval{0}{20}\), and the other is \(20 - x\).
    We have:

    \[(x + \sqrt{x}) (20 - x + \sqrt{20 - x}) = \num{155.55}\]

    Let

    \begin{align*}
                     f(x) &= (x + \sqrt{x}) (20 - x + \sqrt{20 - x}) - \num{155.55} \\
        \Rightarrow f'(x) &= \frac{2 \sqrt{x} + 1}{2 \sqrt{x}} (20 - x + \sqrt{20 - x}) - \frac{2 \sqrt{20 - x} + 1}{2 \sqrt{20 - x}} (x + \sqrt{x})
    \end{align*}

    Applying Newton's method on \(f\) with \(p_0 = \num{6.5}\) generates the
    following table:

    \begin{table}[H]
        \centering
        \begin{tabular}{r S[table-format=1.10] S[table-format=-1.10] S[table-format=2.10]}
            \toprule
            \(n\)  &    {\(p_n\)}   &   {\(f(p_n)\)}  &  {\(f'(p_n)\)}  \\
            \midrule
                0  &  6.5           &  -0.1315962935  &  10.261387078   \\
                1  &  6.5128244157  &  -0.0002485155  &  10.2226328622  \\
                2  &  6.512848726   &  -0.0000000009  &  10.2225594124  \\
            \bottomrule
        \end{tabular}
    \end{table}

    We conclude that the two numbers are approximately \num{6.51285} and
    \num{13.48715}.
\end{solution}

\begin{exercise}
    The accumulated value of a savings account based on regular periodic
    payments can be determined from the \emph{annuity due equation}:

    \[A = \frac{P}{i} [(1 + i)^n - 1]\]

    In this equation, \(A\) is the amount in the account, \(P\) is the amount
    regularly deposited, and \(i\) is the rate of interest per period for the
    \(n\) deposit periods. An engineer would like to have a savings account
    valued at \$\num{750000} upon retirement in 20 years and can afford to put
    \$\num{1500} per month toward this goal. What is the minimal interest rate
    at which this amount can be invested, assuming that the interest is
    compounded monthly?
\end{exercise}

\begin{solution}
    Replacing symbols with numbers gives:

    \[A = \frac{\num{1500}}{i} [(1 + i)^{20 \cdot 12} - 1]\]

    Find the minimal interest rate is finding \(i > 0\) such that \(A \geq
    \num{750000}\):

    \begin{gather*}
        \frac{\num{1500}}{i} [(1 + i)^{240} - 1] \geq \num{750000} \\
        \iff \num{1500} (1 + i)^{240} - \num{750000} i - \num{1500} \geq 0 \tag{*}\label{eq:exer:2.3.26}
    \end{gather*}

    Let

    \begin{align*}
                     f(x) &= (1 + x)^{240} - 500x - 1 \\
        \Rightarrow f'(x) &= 240(x + 1)^{239} - 500
    \end{align*}

    Consider \(f'\).

    \[f'(x) = 0 \iff x = A = \sqrt[239]{\frac{25}{12}} - 1\]

    As \(f'\) is monotonically increasing in \(\mathbb{R}^+\), it follows that:

    \begin{itemize}
        \item \(f\) is monotonically decreasing in \(D_1 = \mathbb{R}_{\leq A} \cap \mathbb{R}^+\)
        \item \(f\) is monotonically increasing in \(\mathbb{R}_{\geq A}\)
    \end{itemize}

    Consider the set \(D_1\).

    \[f(0) = 0 > f(x) \, \forall x \in D_1\]

    Therefore, \eqref{eq:exer:2.3.26} has no positive zero in \(D_1\).

    Consider the set \(\mathbb{R}_{\geq A}\).

    \[f(A) \approx \num{-0.448119} \leq f(x) \, \forall x \in \mathbb{R}_{\geq A}\]

    Therefore, \(f\) has at most one zero in \(\mathbb{R}_{\geq A}\). Applying
    Newton's method on \(f\) with \(p_0 = \num{0.005}\) generates the following
    table:

    \begin{longtable}{r S[table-format=1.13] S[table-format=-1.13] S[table-format=3.13]}
        \toprule
        \(n\)  &     {\(p_n\)}     &    {\(f(p_n)\)}    &    {\(f'(p_n)\)}    \\
        \midrule
        \endfirsthead
        \(n\)  &     {\(p_n\)}     &    {\(f(p_n)\)}    &    {\(f'(p_n)\)}    \\
        \midrule
        \endhead
            0  &  0.005            &  -0.1897955241926  &  290.4965912375794  \\
            1  &  0.0056533485415  &   0.0422743720995  &  423.3277805212566  \\
            2  &  0.0055534865101  &   0.0010855795042  &  401.6714997843162  \\
            3  &  0.0055507838551  &   0.0000007825278  &  401.0924808210714  \\
            4  &  0.0055507819041  &   0.0000000000003  &  401.092062972948   \\
            5  &  0.0055507819041  &   0.0000000000001  &  401.0920629728054  \\
        \bottomrule
    \end{longtable}

    We conclude that the minimal monthly interest rate (assuming that the
    interest is compounded monthly) is approximately \SI{0.555078}{\percent}.
\end{solution}

\begin{exercise}
    Problems involving the amount of money required to pay off a mortgage over a
    fixed period of time involve the formula

    \[A = \frac{P}{i} [1 - (1 + i)^{-n}]\]

    \noindent known as an \emph{ordinary annuity equation}. In this equation,
    \(A\) is the amount of the mortgage, \(P\) is the amount of each payment,
    and \(i\) is the interest rate per period for the \(n\) payment periods.
    Suppose that a 30-year home mortgage in the amount of \$\num{135000} is
    needed and that the borrower can afford house payments of at most
    \$\num{1000} per month. What is the maximal interest rate the borrower can
    afford to pay?
\end{exercise}

\begin{solution}
    Replacing symbols with numbers gives:

    \[A = \frac{\num{1000}}{i} [1 - (1 + i)^{-(30 \cdot 12)}]\]

    Find the maximal interest rate is finding \(i\) such that \(A \leq
    \num{135000}\):

    \begin{gather*}
        \frac{\num{1000}}{i} [1 - (1 + i)^{-360}] \leq \num{135000} \\
        \iff \num{1000} [1 - (1 + i)^{-360}] - \num{135000} i \leq 0 \tag{*}\label{eq:exer:2.3.27}
    \end{gather*}

    Let

    \begin{align*}
                     f(x) &= 1 - (1 + x)^{-360} - 135x \\
        \Rightarrow f'(x) &= 360(x + 1)^{-361} - 135
    \end{align*}

    Consider \(f'\).

    \[f'(x) = 0 \iff x = A = \sqrt[-361]{\num{0.375}} - 1\]

    As \(f'\) is monotonically decreasing in \(\mathbb{R}^+\), it follows that:

    \begin{itemize}
        \item \(f\) is monotonically increasing in \(D_1 = \mathbb{R}_{\leq A} \cap \mathbb{R}^+\)
        \item \(f\) is monotonically decreasing in \(\mathbb{R}_{\geq A}\)
    \end{itemize}

    Consider the set \(D_1\).

    \[f(0) = 0 < f(x) \, \forall x \in D_1\]

    Therefore, \eqref{eq:exer:2.3.27} has no positive zero in \(D_1\).

    Consider the set \(\mathbb{R}_{\geq A}\).

    \[f(A) \approx \num{0.256689} \geq f(x) \, \forall x \in \mathbb{R}_{\geq A}\]

    Therefore, \(f\) has at most one zero in \(\mathbb{R}_{\geq A}\). Applying
    Newton's method on \(f\) with \(p_0 = \num{0.0067}\) generates the following
    table:

    \begin{table}[H]
        \centering
        \begin{tabular}{r S[table-format=1.13] S[table-format=-1.13] S[table-format=-3.13]}
            \toprule
            \(n\)  &     {\(p_n\)}     &    {\(f(p_n)\)}    &     {\(f'(p_n)\)}    \\
            \midrule
                0  &  0.0067           &   0.0051401919049  &  -102.6869664108261  \\
                1  &  0.0067500569068  &  -0.0000144304894  &  -103.2618053134924  \\
                2  &  0.0067499171601  &  -0.0000000001111  &  -103.2602148635103  \\
            \bottomrule
        \end{tabular}
    \end{table}

    We conclude that the maximal monthly interest rate is approximately
    \SI{0.674992}{\percent}.
\end{solution}

\begin{exercise}
    A drug administered to a patient produces a concentration in the blood
    stream given by \(c(t) = A t e^{\frac{-t}{3}}\) milligrams per milliliter,
    \(t\) hours after \(A\) units have been injected. The maximum safe
    concentration is \SI{1}{\milli\gram\per\milli\liter}.

    \begin{tasks}
        \task What amount should be injected to reach this maximum safe
            concentration, and when does this maximum occur?

        \task An additional amount of this drug is to be administered to the
            patient after the concentration falls to
            \SI{0.25}{\milli\gram\per\milli\liter}. Determine, to the nearest
            minute, when this second injection should be given.

        \task Assume that the concentration from consecutive injections is
            additive and that \SI{75}{\percent} of the amount originally
            injected is administered in the second injection. When is it time
            for the third injection?
    \end{tasks}
\end{exercise}

\begin{solution}
    \begin{enumerate}[label = \alph*)]
        \item Let

            \begin{align*}
                             f(x) &= xe^{\frac{-x}{3}} \\
                \Rightarrow f'(x) &= \left(1 - \frac{x}{3}\right) e^{\frac{-x}{3}}
            \end{align*}

            Consider \(f'\).

            \[f'(x) = 0 \iff x = 3\]

            It's clear that \(f'\) is monotonically decreasing in
            \(\mathbb{R}\). It follows that:

            \begin{itemize}
                \item \(f\) is monotonically increasing in \(\mathbb{R}_{\leq 3}\)
                \item \(f\) is monotonically decreasing in \(\mathbb{R}_{\geq 3}\)
                \item \(f\) has a global maximum at \num{3}
            \end{itemize}

            We now know that \(\max{f} = \frac{3}{e}\) is achieved at \num{3}.
            In other words, the maximum concentration of any injection is
            reached \num{3} hours later, regardless of the amount administered.

            To reach the maximum safe concentration of
            \SI{1}{\milli\gram\per\milli\liter}, the amount should be injected
            is:

            \[A \frac{3}{e} = 1 \iff A = \frac{e}{3} \approx \num{0.9060939428}\]

            We conclude that to reach the maximum safe concentration,
            approximately \num{0.9060939428} unit should be injected, and the
            concentration reaches its highest 3 hours after injection.

        \item Let

            \begin{align*}
                             g(t) &= Ate^{\frac{-t}{3}} - 0.25 \\
                \Rightarrow g'(t) &= A\left(1 - \frac{t}{3}\right) e^{\frac{-t}{3}}
            \end{align*}

            \noindent with \(A = \frac{e}{3}\).

            We want to inject after the concentration of the first injection
            already reached its highest, therefore the second injection should
            be no sooner than \num{3} hours since the first one.

            Applying Newton's method on \(g\) with \(p_0 = \num{11.08}\)
            generates the following table:

            \begin{table}[H]
                \centering
                \begin{tabular}{r S[table-format=2.9] S[table-format=-1.9] S[table-format=-1.9]}
                    \toprule
                    \(n\)  &    {\(p_n\)}   &  {\(g(p_n)\)}  &  {\(g'(p_n)\)}  \\
                    \midrule
                        0  &  11.08         &  -0.000127362  &  -0.060739197   \\
                        1  &  11.077903126  &   0.000000028  &  -0.060765892   \\
                        2  &  11.077903587  &   0            &  -0.060765887   \\
                    \bottomrule
                \end{tabular}
            \end{table}

            We conclude that after about 11 hours and 5 minutes since the first
            injection, the second one can be administered.

        \item Let

            \begin{align*}
                             c_n(t) &= \sum_{i = 1}^{n} A_i (t - t_i) e^{\frac{-(t - t_i)}{3}} \\
                \Rightarrow c_n'(t) &= \sum_{i = 1}^{n} A_i \left(1 - \frac{t - t_i}{3}\right) e^{\frac{-(t - t_i)}{3}}
            \end{align*}

            \noindent be the function of concentration \(t \geq t_n\) hours
            since the first injection \emph{and} during that time window another
            \(n - 1\) shots are administered. \(t_n\) is the number of hours
            between the first injection and the \(n^{th}\) one, and clearly
            \(t_1 = 0\).

            From the above parts, we know that \(A_1 = \frac{e}{3}\), \(A_2 =
            \num{0.75} A_1 = \frac{e}{4}\), \(t_2 = \num{11.077903587}\).

            Consider \(c_2\).

            \begin{gather*}
                c_2(t) = 0 \\
                \iff (1 - \frac{t}{3}) + \num{0.75} (1 - \frac{t - t_2}{3}) B = 0 \text{ with } B = e^{\frac{t_2}{3}} \\
                \begin{aligned}
                    \iff t - 3 &= \num{2.25} (3 - t + t_2) B \\
                    \iff     t &= \frac{\num{2.25}(t_2 + 3)B}{1 + \num{2.25} B} \approx \num{13.92377483}
                \end{aligned}
            \end{gather*}

            We want to inject after the total concentration from the previous
            injections already reached its highest, therefore the third
            injection should be no sooner than \num{13.92377483} hours since the
            first one.

            Applying Newton's method on \(h_2 = c_2 - 0.25\) with \(p_0 =
            \num{21.25}\) generates the following table:

            \begin{table}[H]
                \centering
                \begin{tabular}{r S[table-format=2.13] S[table-format=-1.13] S[table-format=-1.13]}
                    \toprule
                    \(n\)  &    {\(p_n\)}   &   {\(h_2(p_n)\)}   &   {\(h_2'(p_n)\)}  \\
                    \midrule
                    0  &  21.25             &  -0.0009922998726  &  -0.0593509605878  \\
                    1  &  21.2332808119236  &   0.0000016642222  &  -0.0595501020878  \\
                    2  &  21.2333087585113  &   0.0000000000047  &  -0.0595497689062  \\
                    3  &  21.2333087585895  &   0                &  -0.0595497689052  \\
                    \bottomrule
                \end{tabular}
            \end{table}

            We conclude that after about 21 hours and 14 minutes since the first
            injection, the third one can be administered.
    \end{enumerate}
\end{solution}

\begin{exercise}\label{exer:2.3.29}
    Let

    \[f(x) = 3^{3x + 1} - 7 \cdot 5^{2x}\]

    \begin{tasks}
        \task Use the Maple commands \lstinline|solve| and \lstinline|fsolve| to
            try to find all roots of \(f\).
        \task Plot \(f\) to find initial approximations to roots of \(f\).
        \task Use Newton's method to find roots of \(f\) to within \(10^{-16}\).
        \task Find the exact solutions of \(f(x) = 0\) without using Maple.
    \end{tasks}
\end{exercise}

\begin{solution}
    \begin{enumerate}
        \item Opps, can't help without Maple license.

        \item The graph of \(f\) is as follow:

            \begin{figure}[H]
                \centering
                \begingroup
                    \tikzset{every picture/.style={scale=0.9}}%
                    \subfile{graphics/exercise_29_graph/exercise_29_graph}
                \endgroup
            \end{figure}

            No useful initial point found, every where: MATLAB, Maple, gnuplot,...

        \item Let:

            \begin{align*}
                             f(x) &= 3^{3x + 1} - 7 \cdot 5^{2x} \\
                \Rightarrow f'(x) &= 3 (\ln{3}) 3^{3x + 1} - 14 (\ln{5}) 5^{2x}
            \end{align*}

            Applying Newton's method on \(f\) with \(p_0 = 11\) generates the
            following table:

            \begin{longtable}{r S[table-format=2.18] S[table-format=-14] S[table-format=16]}
                \toprule
                \(n\)  &        {\(p_n\)}        &    {\(f(p_n)\)}   &    {\(f'(p_n)\)}   \\
                \midrule
                \endfirsthead
                \(n\)  &        {\(p_n\)}        &    {\(f(p_n)\)}   &    {\(f'(p_n)\)}   \\
                \midrule
                \endhead
                    0  &  11                     &  -12118837442806  &  1244484233952568  \\
                    1  &  11.00973804015525026   &     396801311654  &  1326632411906544  \\
                    2  &  11.009438935966258555  &        386222634  &  1324050511461616  \\
                    3  &  11.009438644268449536  &              370  &  1324047995335120  \\
                    4  &  11.009438644268170648  &              -38  &  1324047995332592  \\
                    5  &  11.00943864426819907   &                4  &  1324047995332848  \\
                    6  &  11.009438644268195517  &               66  &  1324047995333032  \\
                    7  &  11.009438644268145779  &                0  &  1324047995332608  \\
                \bottomrule
            \end{longtable}

            So \(p \approx \num{11.009438644268145779}\).

        \item Manipulating \(f = 0\) gives:

            \begin{align*}
                                  f(x) &= 0 \\
                \iff 3    \cdot 3^{3x} &= 7 \cdot 5^{2x} \\
                \iff \frac{27^x}{25^x} &= \frac{7}{3} \\
                \iff                 x &= \log_{\sfrac{27}{25}}{\frac{7}{3}}
            \end{align*}
    \end{enumerate}
\end{solution}

\begin{exercise}
    Repeat \hyperref[exer:2.3.29]{Exercise 29} using \(f(x) = 2^{x^2} - 3 \cdot
    7^{x + 1}\).
\end{exercise}

\begin{solution}
    \begin{enumerate}[label = \alph*)]
        \item Opps, can't help without Maple license.

        \item The graph of \(f\) is as follow:

            \begin{figure}[H]
                \centering
                \begingroup
                    \tikzset{every picture/.style={scale=0.9}}%
                    \subfile{graphics/exercise_30_graph/exercise_30_graph}
                \endgroup
            \end{figure}

        \item Let:

            \begin{align*}
                             f(x) &= 2^{x^2} - 3 \cdot 7^{x + 1} \\
                \Rightarrow f'(x) &= (\ln{2}) 2x 2^{x^2} - 21 (\ln{7}) 7^x
            \end{align*}

            Applying Newton's method on \(f\) with \(p_0 = \num{3.92}\)
            generates the following table:

            \begin{longtable}{r S[table-format=1.18] S[table-format=-3.12] S[table-format=6.12]}
                \toprule
                \(n\)  &        {\(p_n\)}       &     {\(f(p_n)\)}    &     {\(f'(p_n)\)}     \\
                \midrule
                \endfirsthead
                \(n\)  &        {\(p_n\)}       &     {\(f(p_n)\)}    &     {\(f'(p_n)\)}     \\
                \midrule
                \endhead
                    0  &  3.919999999999999929  &  -909.989020751884  &  145585.672581531893  \\
                    1  &  3.92625053966242632   &    22.625719019627  &  152874.530827350565  \\
                    2  &  3.926102537775538082  &     0.013028085261  &  152698.506017085223  \\
                    3  &  3.926102452456528891  &     0.000000004293  &  152698.404592337756  \\
                    4  &  3.926102452456500913  &     0.000000000095  &  152698.404592304723  \\
                    5  &  3.926102452456500469  &    -0.000000000015  &  152698.404592304141  \\
                \bottomrule
            \end{longtable}

            So \(p \approx \num{3.926102452456500469}\).

        \item Manipulating \(f = 0\) gives:

            \begin{gather*}
                f(x) = 0 \\
                \begin{aligned}
                    \iff 2^{x^2} &= 21 \cdot 7^x \\
                    \iff     x^2 &= \log_{2}(21 \cdot 7^x) \\
                                 &= \log_{2}{21} + x \log_{2}{7} \\
                \end{aligned}\\
                \iff x^2 - \log_{2}{7} x - \log_{2}{21} = 0 \\
                \iff x = \frac{\log_{2}{7} \pm \sqrt{\Delta}}{2} \text{ with } \Delta = (\log_{2} 7)^2 + 4 * \log_{2} 21 = \log_{2} \num{9529569}
            \end{gather*}
    \end{enumerate}
\end{solution}

\begin{exercise}\label{exer:2.3.31}
    The logistic population growth model is described by an equation of the form

    \[P(t) = \frac{P_L}{1 - ce^{-kt}}\]

    \noindent where \(P_L\), \(c\), and \(k > 0\) are constants, and \(P(t)\) is
    the population at time \(t\). \(P_L\) represents the limiting value of the
    population since \(\lim_{t \to \infty} P(t) = P_L\). Use the census data for
    the years 1950, 1960, and 1970 listed in the table on page 105 to determine
    the constants \(P_L\), \(c\), and \(k\) for a logistic growth model. Use the
    logistic model to predict the population of the United States in 1980 and in
    2010, assuming \(t = 0\) at 1950. Compare the 1980 prediction to the actual
    value.
\end{exercise}

\begin{solution}
    We have:

    \begin{align*}
         P(0) = \frac{P_L}{1 - ce^{-k0}}  = P_1 &\iff ce^0 = 1 - \frac{P_L}{P_1}    \tag{1}\label{eq:exer:2.3.31:1} \\
        P(10) = \frac{P_L}{1 - ce^{-k10}} = P_2 &\iff ce^{-10k} = 1 - \frac{P_L}{P_2} \tag{2}\label{eq:exer:2.3.31:2} \\
        P(20) = \frac{P_L}{1 - ce^{-k20}} = P_3 &\iff ce^{-20k} = 1 - \frac{P_L}{P_3} \tag{3}\label{eq:exer:2.3.31:3} \\
    \end{align*}

    Divide \eqref{eq:exer:2.3.31:1} by \eqref{eq:exer:2.3.31:2} and
    \eqref{eq:exer:2.3.31:2} by \eqref{eq:exer:2.3.31:3} gives:

    \begin{gather*}
        e^{10k} = \frac{A - P_2 P_L}{A - P_1 P_L} \text{ with } A = P_1 P_2 \\
        e^{10k} = \frac{B - P_3 P_L}{B - P_2 P_L} \text{ with } B = P_2 P_3
    \end{gather*}

    Combining both above equations gives:

    \begin{gather*}
        \frac{A - P_2 P_L}{A - P_1 P_L} = \frac{B - P_3 P_L}{B - P_2 P_L} \\
        \iff (A - P_6 P_L)(B - P_6 P_L) = (A - P_5 P_L) (B - P_7 P_L) \\
        \iff (P_6^2 - P_5 P_7) P_L^2 + (-AP_6 - BP_6 + AP_7 + BP_5) P_L = 0 \\
        \iff P_L = \frac{A(P_7 - P_6) + B(P_5 - P_6)}{P_5 P_7 - P_6^2} \approx \num{265816.4151}
    \end{gather*}

    It follows that \(k \approx \num{0.04501750225}\), and \(c \approx
    \num{-0.7565812558}\).

    We now predict the US population in 1980 and 2010:

    \begin{gather*}
        P_{1980} = P(30) \approx \num{222248.3277} \\
        P_{2010} = P(60) \approx \num{252967.4246}
    \end{gather*}

    It is predicted, using the above model, that the US population in 1980 is
    \num{222248323} and in 2010 is \num{252967425}. However, the actual
    population in 1980 is larger, so the 1980 prediction undershoots.
\end{solution}

\begin{exercise}
    The Gompertz population growth model is described by

    \[P(t) = P_L e^{-ce^{-kt}}\]

    \noindent where \(P_L\), \(c\), and \(k > 0\) are constants, and \(P(t)\) is
    the population at time \(t\). Repeat \hyperref[exer:2.3.31]{Exercise 31}
    using the Gompertz growth model in place of the logistic model.
\end{exercise}

\begin{solution}
    We have:

    \begin{align*}
         P(0) = P_L e^{-ce^{-k0}}  = P_1 &\iff e^{-k0}  = \log_{d} \frac{P_1}{P_L} \tag{1}\label{eq:exer:2.3.32:1} \\
        P(10) = P_L e^{-ce^{-k10}} = P_2 &\iff e^{-k10} = \log_{d} \frac{P_2}{P_L} \tag{2}\label{eq:exer:2.3.32:2} \\
        P(20) = P_L e^{-ce^{-k20}} = P_3 &\iff e^{-k20} = \log_{d} \frac{P_3}{P_L} \tag{3}\label{eq:exer:2.3.32:3}
    \end{align*}

    \noindent with \(d = e^{-c}\).

    From \eqref{eq:exer:2.3.32:1}, we know that:

    \[e^{-k0} = 1 = \log_{d} \frac{P_1}{P_L} \iff d = \frac{P_1}{P_L}\]

    Divide \eqref{eq:exer:2.3.32:1} by \eqref{eq:exer:2.3.32:2} and
    \eqref{eq:exer:2.3.32:2} by \eqref{eq:exer:2.3.32:3} gives:

    \begin{gather*}
        e^{10k} = \frac{\log_{d} \frac{P_1}{P_L}}{\log_{d} \frac{P_2}{P_L}} = \frac{\log_{d} P_1 - \log_{d} P_L}{\log_{d} P_2 - \log_{d} P_L} = \frac{\ln P_1 - \ln P_L}{\ln P_2 - \ln P_L} \\
        e^{10k} = \frac{\log_{d} \frac{P_2}{P_L}}{\log_{d} \frac{P_3}{P_L}} = \frac{\log_{d} P_2 - \log_{d} P_L}{\log_{d} P_3 - \log_{d} P_L} = \frac{\ln P_2 - \ln P_L}{\ln P_3 - \ln P_L}
    \end{gather*}

    Combining both above equations gives:

    \begin{align*}
        \frac{\ln P_1 - \ln P_L}{\ln P_2 - \ln P_L} &= \frac{\ln P_2 - \ln P_L}{\ln P_3 - \ln P_L} \\
        \iff                  (\ln P_2 - \ln P_L)^2 &= (\ln P_1 - \ln P_L)(\ln P_3 - \ln P_L) \\
        \iff        (\ln P_2)^2 - 2 \ln P_2 \ln P_L &= \ln P_1 \ln P_3 - \ln(P_1 P_3) \ln P_L \\
        \iff                                \ln P_L &= \frac{(\ln P_2)^2 - \ln P_1 \ln P_3}{2 \ln P_2 - \ln(P_1 P_3)} \\
        \iff                                    P_L &\approx \num{290227.8618}
    \end{align*}

    It follows that \(k \approx \num{0.0302002813}\), \(d =
    \num{0.5214041101}\), \(c = \num{0.6512298947}\).

    We now predict the US population in 1980 and 2010:

    \begin{gather*}
        P_{1980} = P(30) \approx \num{223069.2173} \\
        P_{2010} = P(60) \approx \num{260943.6839}
    \end{gather*}

    It is predicted, using the above model, that the US population in 1980 is
    \num{223069217} and in 2010 is \num{260943684}. However, the actual
    population in 1980 is larger, so the 1980 prediction undershoots.
\end{solution}

\begin{exercise}
    Player A will shut out (win by a score of 21-0) player B in a game of
    racquetball with probability

    \[P = \frac{1 + p}{2} \left(\frac{p}{1 - p + p^2}\right)^{21}\]

    \noindent where \(p\) denotes the probability A will win any specific rally
    (independent of the server). Determine, to within \(10^{-3}\), the minimal
    value of \(p\) that will ensure that A will shut out B in at least half the
    matches they play.
\end{exercise}

\begin{solution}
    Let

    \begin{gather*}
        \begin{aligned}
                         g(x) &= \frac{x}{1 - x + x^2} \\
            \Rightarrow g'(x) &= \frac{1 - x^2}{(1 - x + x^2)^2} \\
        \end{aligned} \\
        \begin{aligned}
                         f(x) &= \frac{1 + x}{2} \left(\frac{x}{1 - x + x^2}\right)^{21} \\
            \Rightarrow f'(x) &= \frac{1}{2} \left(\frac{x}{1 - x + x^2}\right)^{21} + \frac{1 + x}{2} 21 \left(\frac{x}{1 - x + x^2}\right)^{20} \frac{1 - x^2}{(1 - x + x^2)^2} \\
                              &= \frac{1}{2} \left(\frac{x}{1 - x + x^2}\right)^{20} \left[\frac{x}{1 - x + x^2} + \frac{21(1 + x)(1 - x^2)}{(1 - x + x^2)^2}\right] \\
                              &= \frac{1}{2} \left(\frac{x}{1 - x + x^2}\right)^{20} \frac{-20x^3 - 22x^2 + 22x + 21}{(1 - x + x^2)^2}
        \end{aligned}
    \end{gather*}

    Finding the minimal value of \(p\) that will ensure that A will shut out B
    in at least half the matches they play is finding the minimal \(x \in D =
    \interval{0}{1}\) such that \(f(x) \geq \num{0.5}\).

    Consider \(g'\).

    \begin{gather*}
        g'(x) = 0 \iff x = \pm 1 \\
        x^2 - x + 1 = x^2 - 2x\num{0.5} + \num{0.5}^2 + \num{0.75} \geq \num{0.75} > 0 \, \forall x \in \mathbb{R}
    \end{gather*}

    It follows that the sign of \(g'\) is the sign of \(1 - x^2\). Therefore, in
    \(D\), \(g' \geq 0\). Therefore, \(g\) and then \(f\) are monotonically
    increasing in \(D\):

    \[f(0) = 0 \leq f(x) \leq f(1) = 1 \, \forall x \in D\]

    It's clear that \(f(x) \geq \num{0.5}\) is guaranteed to have solution in
    \(D\).

    Applying Newton's method on \(h = f - \num{0.5}\) with \(p_0 = \num{0.84}\)
    generates the following table:

    \begin{table}[H]
        \centering
        \begin{tabular}{r S[table-format=1.18] S[table-format=-1.18] S[table-format=1.18]}
            \toprule
            \(n\)  &        {\(p_n\)}       &       {\(h(p_n)\)}      &      {\(h'(p_n)\)}     \\
            \midrule
                0  &  0.84                  &  -0.010231745763236211  &  4.430566512699972925  \\
                1  &  0.842309353834076791  &   0.000020294149810418  &  4.44775767420762147   \\
                2  &  0.842304791051817325  &   0.000000000072282402  &  4.447725988980080203  \\
                3  &  0.84230479103556577   &   0.000000000000000888  &  4.447725988867216707  \\
                4  &  0.842304791035565548  &  -0.000000000000000444  &  4.447725988867211377  \\
            \bottomrule
        \end{tabular}
    \end{table}

    We conclude that \(p \geq \num{0.842304791035565548}\) will ensure that A
    will shut out B in at least half the matches they play.
\end{solution}

\begin{exercise}
    In the design of all-terrain vehicles, it is necessary to consider the
    failure of the vehicle when attempting to negotiate two types of obstacles.
    One type of failure is called \emph{hang-up failure} and occurs when the
    vehicle attempts to cross an obstacle that causes the bottom of the vehicle
    to touch the ground. The other type of failure is called \emph{nose-in
    failure} and occurs when the vehicle descends into a ditch and its nose
    touches the ground.

    The accompanying figure shows the components associated with the nose-in
    failure of a vehicle. It is shown that the maximum angle \(\alpha\) that can
    be negotiated by a vehicle when \(\beta\) is the maximum angle at which
    hang-up failure does \emph{not} occur satisfies the equation

    \[A \sin{\alpha} \cos{\alpha} + B \sin^{2}{\alpha} - C \cos{\alpha} - E \sin{\alpha} = 0\]

    \noindent where

    \[\begin{cases}
        D: \text{wheel diameter} \\
        A = l \sin{\beta_1} \\
        B = l \cos{\beta_1} \\
        C = (h + \num{0.5} D) \sin{\beta_1} - \num{0.5} D \tan{\beta_1} \\
        E = (h + \num{0.5} D) \cos{\beta_1} - \num{0.5} D
    \end{cases}\]

    \begin{tasks}
        \task It is stated that when \(l = \SI{89}{in}\), \(h = \SI{49}{in}\),
            \(D = \SI{55}{in}\), and \(\beta_1 = \SI{11.5}{\degree}\), angle
            \(\alpha\) is approximately \SI{33}{\degree}. Verify this result.

        \task Find \(\alpha\) for the situation when \(l\), \(h\), and
            \(\beta_1\) are the same as in part a) but \(D = \SI{30}{in}\).
    \end{tasks}
\end{exercise}

\begin{solution}
    Let

    \begin{align*}
                     f(x) &= A \sin{x} \cos{x} + B \sin^{2}{x} - C \cos{x} - E \sin{x} \\
        \Rightarrow f'(x) &= A (\cos^{2}{x} - \sin^{2}{x}) + 2 B \sin{x} \cos{x} + C \sin{x} - E \cos{x}
    \end{align*}

    \begin{enumerate}[label = \alph*)]
        \item Applying Newton's method on \(f\) with \(p_0 = \SI{33}{\degree}
            \approx \num{0.57595865315813}\) generates the following table:

            \begin{table}[H]
                \centering
                \begin{tabular}{r S[table-format=1.14] S[table-format=1.14] S[table-format=2.14]}
                    \toprule
                    \(n\)  &      {\(p_n\)}     &    {\(g(p_n)\)}    &    {\(g'(p_n)\)}    \\
                    \midrule
                        0  &  0.57595865315813  &  0.02541130581159  &  52.34290413106125  \\
                        1  &  0.5754731755899   &  0.00000854683891  &  52.30768181120521  \\
                        2  &  0.57547301219442  &  0.00000000000097  &  52.30766994413587  \\
                        3  &  0.5754730121944   &  0                 &  52.30766994413455  \\
                    \bottomrule
                \end{tabular}
            \end{table}

            So \(\alpha \approx \num{0.5754730121944} \approx
            \SI{32.97217482}{\degree}\), which is indeed close to
            \SI{33}{\degree}.

        \item Applying Newton's method on \(f\) with \(p_0 = \SI{33}{\degree}
            \approx \num{0.57595865315813}\) generates the following table:

            \begin{table}[H]
                \centering
                \begin{tabular}{r S[table-format=1.14] S[table-format=-1.14] S[table-format=2.14]}
                    \toprule
                    \(n\)  &      {\(p_n\)}     &     {\(f(p_n)\)}    &    {\(f'(p_n)\)}    \\
                    \midrule
                        0  &  0.57595865315813  &  -0.15407902197157  &  52.16025344654213  \\
                        1  &  0.57891260778432  &   0.00031564555417  &  52.37350858776342  \\
                        2  &  0.57890658096727  &   0.00000000130272  &  52.37307627539987  \\
                        3  &  0.5789065809424   &   0.00000000000001  &  52.37307627361562  \\
                    \bottomrule
                \end{tabular}
            \end{table}

            So \(\alpha \approx \num{0.5789065809424} \approx
            \SI{33.16890382}{\degree}\).
    \end{enumerate}
\end{solution}

\end{document}
