\documentclass[../../../../Assignments]{subfiles}


\begin{document}

\section{Gauss elimination}

\begin{exercise}
    For each of the following linear systems, obtain a solution by graphical
    methods, if possible. Explain the results from a geometrical standpoint.

    \begin{tasks}(2)
        \task
            \begin{alignat*}{5}
                &x_1 &{}+{}& 2&x_2 &{}= 3 \\
                &x_1 &{}-{}&  &x_2 &{}= 0
            \end{alignat*}
        \task
            \begin{alignat*}{5}
                 &x_1 &{}+{}& 2&x_2 &{}= 3 \\
                2&x_1 &{}+{}& 4&x_2 &{}= 6
            \end{alignat*}
        \task
            \begin{alignat*}{5}
                 &x_1 &{}+{}& 2&x_2 &{}= 0 \\
                2&x_1 &{}+{}& 4&x_2 &{}= 0
            \end{alignat*}
        \task
            \begin{alignat*}{5}
                2&x_1 &{}+{}& 2&x_2 &{}= -1 \\
                4&x_1 &{}+{}& 2&x_2 &{}= -2 \\
                 &x_1 &{}-{}& 3&x_2 &{}=  5
            \end{alignat*}
    \end{tasks}
\end{exercise}

\begin{solution}
    \begin{enumerate}[label = \alph*)]
        \item The graphs of the equations are as follow:
            \begin{figure}[H]
                \centering
                \begingroup
                    \tikzset{every picture/.style={scale=0.9}}%
                    \subfile{graphics/exercise_1_graphs/exercise_1a_graph}
                \endgroup
            \end{figure}

            The solution is \(x_1 = 1\), \(x_2 = 1\) as the lines intersect at
            \((1, 1)\).

        \item The graphs of the equations are as follow:
            \begin{figure}[H]
                \centering
                \begingroup
                    \tikzset{every picture/.style={scale=0.9}}%
                    \subfile{graphics/exercise_1_graphs/exercise_1b_graph}
                \endgroup
            \end{figure}

            The system of equation has an infinite number of solutions, as the
            line coincide.

        \item The graphs of the equations are as follow:
            \begin{figure}[H]
                \centering
                \begingroup
                    \tikzset{every picture/.style={scale=0.9}}%
                    \subfile{graphics/exercise_1_graphs/exercise_1c_graph}
                \endgroup
            \end{figure}

            The system of equation has an infinite number of solutions, as the
            lines coincide.

        \item The graphs of the equations are as follow:
            \begin{figure}[H]
                \centering
                \begingroup
                    \tikzset{every picture/.style={scale=0.9}}%
                    \subfile{graphics/exercise_1_graphs/exercise_1d_graph}
                \endgroup
            \end{figure}

            The solution is \(x_1 = -\frac{11}{7}\), \(x_2 = \frac{2}{7}\) as
            the lines intersect at \((\frac{2}{7}, -\frac{11}{7})\).
    \end{enumerate}
\end{solution}

\begin{exercise}
    For each of the following linear systems, obtain a solution by graphical
    methods, if possible. Explain the results from a geometrical standpoint.
\end{exercise}

\begin{solution}
    \begin{tasks}(2)
        \task
            \begin{alignat*}{5}
                &x_1 &{}+{}& 2&x_2 &{}= 0 \\
                &x_1 &{}-{}&  &x_2 &{}= 0
            \end{alignat*}
        \task
            \begin{alignat*}{5}
                  &x_1 &{}+{}& 2&x_2 &{}= 3 \\
                -2&x_1 &{}-{}& 4&x_2 &{}= 6
            \end{alignat*}
        \task
            \begin{alignat*}{5}
                2&x_1 &{}+{}&  &x_2 &{}= -1 \\
                 &x_1 &{}+{}&  &x_2 &{}=  2 \\
                 &x_1 &{}-{}& 3&x_2 &{}=  5
            \end{alignat*}
        \task
            \begin{alignat*}{5}
                2&x_1 &{}+{}&  &x_2 &{}+{}& x_3 &{}=  1 \\
                2&x_1 &{}+{}& 4&x_2 &{}-{}& x_3 &{}= -1
            \end{alignat*}
    \end{tasks}
\end{solution}

\begin{solution}
    \begin{enumerate}[label = \alph*)]
        \item The graphs of the equations are as follow:
            \begin{figure}[H]
                \centering
                \begingroup
                    \tikzset{every picture/.style={scale=0.9}}%
                    \subfile{graphics/exercise_2_graphs/exercise_2a_graph}
                \endgroup
            \end{figure}

            The solution is \(x_1 = 0\), \(x_2 = 0\) as the lines intersect at
            \((0, 0)\).

        \item The graphs of the equations are as follow:
            \begin{figure}[H]
                \centering
                \begingroup
                    \tikzset{every picture/.style={scale=0.9}}%
                    \subfile{graphics/exercise_2_graphs/exercise_2b_graph}
                \endgroup
            \end{figure}

            The system of equation has no solution, as the lines are parallel to
            each other.

        \item The graphs of the equations are as follow:
            \begin{figure}[H]
                \centering
                \begingroup
                    \tikzset{every picture/.style={scale=0.9}}%
                    \subfile{graphics/exercise_2_graphs/exercise_2c_graph}
                \endgroup
            \end{figure}

            The system of equation has no solution, as the lines do not
            intersect.
    \end{enumerate}
\end{solution}

\begin{exercise}\label{exer:3.1.3}
    Use Gaussian elimination with backward substitution and two-digit rounding
    arithmetic to solve the following linear systems. Do not reorder the
    equations.

    \begin{tasks}(2)
        \task
            \begin{alignat*}{9}
                4&x_1 &{}-{}&  &x_2 &{}+{}&  &x_3 &{}={}&  8 \\
                2&x_1 &{}+{}& 5&x_2 &{}-{}& 2&x_3 &{}={}&  3 \\
                 &x_1 &{}+{}& 2&x_2 &{}-{}& 4&x_3 &{}={}& 11
            \end{alignat*}
        \task
            \begin{alignat*}{9}
                4&x_1 &{}+{}&  &x_2 &{}+{}& 2&x_3 &{}={}&  9 \\
                2&x_1 &{}+{}& 4&x_2 &{}-{}& 1&x_3 &{}={}& \num{-5} \\    % somehow -5 is rendered as - 5 without \num{}
                 &x_1 &{}+{}&  &x_2 &{}-{}& 3&x_3 &{}={}& \num{-9}
            \end{alignat*}
    \end{tasks}
\end{exercise}

\begin{solution}
    \begin{enumerate}[label = \alph*)]
        \item Let
            \[
                \bm{\tilde{A}} = \bm{\tilde{A}}^{(1)} =
                    \begin{pNiceArray}{ S[table-format=1] S[table-format=-1] S[table-format=1] : S[table-format=2] }
                        4  &  -1  &  1  &   8  \\
                        2  &   5  &  2  &   3  \\
                        1  &   2  &  4  &  11
                    \end{pNiceArray}
            \]

            Eliminating \(x_1\) by these transformation
            \[E_2 \coloneqq E_2 - \num{0.5} E_1; \, E_3 \coloneqq E_3 - \num{0.25}E_1\]
            gives:
            \[
                \bm{\tilde{A}}^{(2)} =
                    \begin{pNiceArray}{ S[table-format=1] S[table-format=-1.2] S[table-format=1.2] : S[table-format=-1] }
                        4  &  -1     &  1     &   8  \\
                        0  &   5.5   &  1.5   &  -1  \\
                        0  &   2.25  &  3.75  &   9
                    \end{pNiceArray}
            \]

            Eliminating \(x_2\) by these transformation
            \[E_3 \coloneqq E_3 - \frac{9}{22} E_2\]
            gives:
            \[
                \bm{\tilde{A}}^{(3)} =
                    \begin{pNiceArray}{ S[table-format=1] S[table-format=-1.1] S[table-format=1.5] : S[table-format=-1.5] }
                        4  &  -1    &  1        &   8        \\
                        0  &   5.5  &  1.5      &  -1        \\
                        0  &   0    &  3.13636  &   9.40909
                    \end{pNiceArray}
            \]

            The solution is \(x_3 \approx 3\), \(x_2 \approx -1\), \(x_1 \approx
            1\).

        \item Let
            \[
                \bm{\tilde{A}} = \bm{\tilde{A}}^{(1)} =
                    \begin{pNiceArray}{ S[table-format=1] S[table-format=1] S[table-format=-1] : S[table-format=-1] }
                        4  &  1  &   2  &   9  \\
                        2  &  4  &  -1  &  -5  \\
                        1  &  1  &  -3  &  -9
                    \end{pNiceArray}
            \]

            Eliminating \(x_1\) by these transformation
            \[E_2 \coloneqq E_2 - \num{0.5} E_1; \, E_3 \coloneqq E_3 - \num{0.25}E_1\]
            gives:
            \[
                \bm{\tilde{A}}^{(2)} =
                    \begin{pNiceArray}{ S[table-format=1] S[table-format=1.2] S[table-format=-1.1] : S[table-format=-2.2] }
                        4  &  1     &   2    &    9     \\
                        0  &  3.5   &  -2    &   -9.5   \\
                        0  &  0.75  &  -3.5  &  -11.25
                    \end{pNiceArray}
            \]

            Eliminating \(x_2\) by these transformation
            \[E_3 \coloneqq E_3 - \frac{3}{14} E_2\]
            gives:
            \[
                \bm{\tilde{A}}^{(3)} =
                    \begin{pNiceArray}{ S[table-format=1] S[table-format=1.1] S[table-format=-1.5] : S[table-format=-1.5] }
                        4  &  1    &   2        &   9        \\
                        0  &  3.5  &  -2        &  -9.5      \\
                        0  &  0    &  -3.07143  &  -9.21429
                    \end{pNiceArray}
            \]

            The solution is \(x_3 \approx 3\), \(x_2 \approx -1\), \(x_1 \approx
            1\).
    \end{enumerate}
\end{solution}

\begin{exercise}
    Use Gaussian elimination with backward substitution and two-digit rounding
    arithmetic to solve the following linear systems. Do not reorder the
    equations.

    \begin{tasks}(2)
        \task
            \begin{alignat*}{9}
                         -1&x_1 &{}+{}&           4&x_2 &{}+{}&            &x_3 &{}={}&  8 \\
                \frac{5}{3}&x_1 &{}+{}& \frac{2}{3}&x_2 &{}+{}& \frac{2}{3}&x_3 &{}={}&  1 \\
                          2&x_1 &{}+{}&            &x_2 &{}+{}&           4&x_3 &{}={}& 11
            \end{alignat*}
        \task
            \begin{alignat*}{9}
                          4&x_1 &{}+{}&           2&x_2 &{}-{}&            &x_3 &{}={}& \num{-5} \\
                \frac{1}{9}&x_1 &{}+{}& \frac{1}{9}&x_2 &{}-{}& \frac{1}{3}&x_3 &{}={}& \num{-1} \\
                          1&x_1 &{}+{}&           4&x_2 &{}+{}&           2&x_3 &{}={}&  9
            \end{alignat*}
    \end{tasks}
\end{exercise}

\begin{solution}
    \begin{enumerate}[label = \alph*)]
        \item Let
            \[
                \bm{\tilde{A}} = \bm{\tilde{A}}^{(1)} =
                    \begin{pNiceArray}{ S[table-format=-1.5] S[table-format=1.5] S[table-format=1.5] : S[table-format=2] }
                        -1        &  4        &  1        &   8  \\
                         1.66667  &  0.66667  &  0.66667  &   1  \\
                         2        &  1        &  4        &  11
                    \end{pNiceArray}
            \]

            Eliminating \(x_1\) by these transformation
            \[E_2 \coloneqq E_2 - (\num{-1.6667}) E_1; \, E_3 \coloneqq E_3 - (-2)E_1\]
            gives:
            \[
                \bm{\tilde{A}}^{(2)} =
                    \begin{pNiceArray}{ S[table-format=-1] S[table-format=1.5] S[table-format=1.5] : S[table-format=2.5] }
                        -1  &  4        &  1        &   8        \\
                         0  &  7.33333  &  2.33333  &  14.33333  \\
                         0  &  9        &  6        &  27
                    \end{pNiceArray}
            \]

            Eliminating \(x_2\) by these transformation
            \[E_3 \coloneqq E_3 - \num{1.22727} E_2\]
            gives:
            \[
                \bm{\tilde{A}}^{(3)} =
                    \begin{pNiceArray}{ S[table-format=-1] S[table-format=1.5] S[table-format=1.5] : S[table-format=2.5] }
                        -1  &  4        &  1        &   8        \\
                         0  &  7.33333  &  2.33333  &  14.33333  \\
                         0  &  0        &  3.13636  &   9.40909
                    \end{pNiceArray}
            \]

            The solution is \(x_3 \approx 3\), \(x_2 \approx 1\), \(x_1 \approx
            -1\).

        \item Let
            \[
                \bm{\tilde{A}} = \bm{\tilde{A}}^{(1)} =
                    \begin{pNiceArray}{ S[table-format=1.5] S[table-format=1.5] S[table-format=-1.5] : S[table-format=-1] }
                        4        &  2        &  -1        &  -5  \\
                        0.11111  &  0.11111  &  -0.33333  &  -1  \\
                        1        &  4        &   2        &   9
                    \end{pNiceArray}
            \]

            Eliminating \(x_1\) by these transformation
            \[E_2 \coloneqq E_2 - \num{0.02778} E_1; \, E_3 \coloneqq E_3 - \num{0.25} E_1\]
            gives:
            \[
                \bm{\tilde{A}}^{(2)} =
                    \begin{pNiceArray}{ S[table-format=1] S[table-format=1.5] S[table-format=-1.5] : S[table-format=-1.5] }
                        4  &  2        &  -1        &  -5        \\
                        0  &  0.05556  &  -0.30555  &  -0.86111  \\
                        0  &  3.5      &   2.25     &  10.25
                    \end{pNiceArray}
            \]

            Eliminating \(x_2\) by these transformation
            \[E_3 \coloneqq E_3 - \num{63.00063} E_2\]
            gives:
            \[
                \bm{\tilde{A}}^{(3)} =
                    \begin{pNiceArray}{ S[table-format=1] S[table-format=1.5] S[table-format=-1.5] : S[table-format=-1.5] }
                        4  &  2        &  -1        &  -5        \\
                        0  &  0.05556  &  -0.30555  &  -0.86111  \\
                        0  &  0        &  21.5      &  64.50063
                    \end{pNiceArray}
            \]

            The solution is \(x_3 \approx 3\), \(x_2 \approx 1\), \(x_1 \approx
            -1\).
    \end{enumerate}
\end{solution}

\begin{exercise}
    Use the Gaussian Elimination Algorithm to solve the following linear
    systems, if possible, and determine whether row interchanges are necessary:

    \begin{tasks}(2)
        \task
            \begin{alignat*}{9}
                 &x_1 &{}-{}& 1&x_2 &{}+{}& 3&x_3 &{}={}&  2 \\
                3&x_1 &{}-{}& 3&x_2 &{}+{}& 1&x_3 &{}={}& -1 \\
                 &x_1 &{}+{}& 1&x_2 &{}-{}&  &    &{}={}&  3
            \end{alignat*}
        \task
            \begin{alignat*}{9}
                 2&x_1 &{}-{}& \num{1.5}&x_2 &{}+{}& 3&x_3 &{}={}&  1 \\
                -1&x_1 &{} {}&          &    &{}+{}& 2&x_3 &{}={}&  3 \\
                 4&x_1 &{}-{}& \num{4.5}&x_2 &{}+{}& 5&x_3 &{}={}&  1
            \end{alignat*}
        \task
            \begin{alignat*}{12}
                2&x_1 &{} {}&          &    &{} {}&          &    &{} {}&  &    &{}={}&       3     \\
                 &x_1 &{}+{}& \num{1.5}&x_2 &{} {}&          &    &{} {}&  &    &{}={}&  \num{4.5}  \\
                 &    &{} {}&  \num{-3}&x_2 &{}+{}& \num{0.5}&x_3 &{} {}&  &    &{}={}& \num{-6.6}  \\
                2&x_1 &{}-{}&         2&x_2 &{}+{}&          &x_3 &{}+{}&  &x_4 &{}={}&  \num{0.8}
            \end{alignat*}
        \task
            \begin{alignat*}{12}
                 &x_1 &{}+{}&  &x_2 &{} {}&  &    &{}+{}&  &x_4 &{}={}&  2 \\
                2&x_1 &{}+{}&  &x_2 &{}-{}&  &x_3 &{}+{}&  &x_4 &{}={}&  1 \\
                4&    &{}-{}&  &x_2 &{}-{}& 2&x_3 &{}+{}& 2&    &{}={}&  0 \\
                3&x_1 &{}-{}&  &x_2 &{}-{}&  &x_3 &{}+{}& 2&x_4 &{}={}& -3
            \end{alignat*}
    \end{tasks}
\end{exercise}

\begin{solution}
    \begin{enumerate}[label = \alph*)]
        \item Let
            \[
                \bm{\tilde{A}} = \bm{\tilde{A}}^{(1)} =
                    \begin{pNiceArray}{ S[table-format=1] S[table-format=-1] S[table-format=1] : S[table-format=-1] }
                        1  &  -1  &  3  &   2  \\
                        3  &  -3  &  1  &  -1  \\
                        1  &   1  &  0  &   3  \\
                    \end{pNiceArray}
            \]

            Eliminating \(x_1\) by these transformation
            \[E_2 \coloneqq E_2 - 3E_1; \, E_3 \coloneqq E_3 - 1E_1\]
            gives:
            \[
                \bm{\tilde{A}}^{(2)} =
                    \begin{pNiceArray}{ S[table-format=1] S[table-format=-1] S[table-format=-1] : S[table-format=-1] }
                        1  &  -1  &   3  &   2  \\
                        0  &   0  &  -8  &  -7  \\
                        0  &   2  &  -3  &   1  \\
                    \end{pNiceArray}
            \]

            As \(a_{22}^{(2)} = 0\), we have to swap row 2 and 3. Eliminating
            \(x_2\) by these transformation
            \[E_3 \coloneqq E_3 - \num{63.00063} E_2\]
            gives:
            \[
                \bm{\tilde{A}}^{(3)} =
                    \begin{pNiceArray}{ S[table-format=1] S[table-format=-1] S[table-format=-1] : S[table-format=-1] }
                        1  &  -1  &   3  &   2  \\
                        0  &   2  &  -3  &   1  \\
                        0  &   0  &  -8  &  -7  \\
                    \end{pNiceArray}
            \]

            The solution is \(x_3 = \num{0.875}\), \(x_2 = \num{1.8125}\), \(x_1
            = \num{1.1875}\).

        \item Let
            \[
                \bm{\tilde{A}} = \bm{\tilde{A}}^{(1)} =
                    \begin{pNiceArray}{ S[table-format=-1] S[table-format=-1.1] S[table-format=1] : S[table-format=1] }
                         2  &  -1.5  &  3  &  1  \\
                        -1  &   0    &  2  &  3  \\
                         4  &  -4.5  &  5  &  1  \\
                    \end{pNiceArray}
            \]

            Eliminating \(x_1\) by these transformation
            \[E_2 \coloneqq E_2 - (\num{-0.5}) E_1; \, E_3 \coloneqq E_3 - 2E_1\]
            gives:
            \[
                \bm{\tilde{A}}^{(2)} =
                    \begin{pNiceArray}{ S[table-format=1] S[table-format=-1.2] S[table-format=-1.1] : S[table-format=-1.1] }
                        2  &  -1.5   &   3    &   1    \\
                        0  &  -0.75  &   3.5  &   3.5  \\
                        0  &  -1.5   &  -1    &  -1    \\
                    \end{pNiceArray}
            \]

            Eliminating \(x_2\) by these transformation
            \[E_3 \coloneqq E_3 - 2E_2\]
            gives:
            \[
                \bm{\tilde{A}}^{(3)} =
                    \begin{pNiceArray}{ S[table-format=1] S[table-format=-1.2] S[table-format=-1.1] : S[table-format=-1.1] }
                        2  &  -1.5   &   3    &   1    \\
                        0  &  -0.75  &   3.5  &   3.5  \\
                        0  &   0     &  -8    &  -8    \\
                    \end{pNiceArray}
            \]

            The solution is \(x_3 = 1\), \(x_2 = 0\), \(x_1 = -1\).

        \item Let
            \[
                \bm{\tilde{A}} = \bm{\tilde{A}}^{(1)} =
                    \begin{pNiceArray}{ S[table-format=1] S[table-format=-1.1] S[table-format=1.1] S[table-format=1] : S[table-format=-1.1] }
                        2  &   0    &  0    &  0  &   3    \\
                        1  &   1.5  &  0    &  0  &   4.5  \\
                        0  &  -3    &  0.5  &  0  &  -6.6  \\
                        2  &  -2    &  1    &  1  &   0.8  \\
                    \end{pNiceArray}
            \]

            Eliminating \(x_1\) by these transformation
            \[E_2 \coloneqq E_2 - 0.5 E_1; \, E_3 \coloneqq E_3 - 0E_1; \, E_4 \coloneqq E_4 - 1E_1\]
            gives:
            \[
                \bm{\tilde{A}}^{(2)} =
                \begin{pNiceArray}{ S[table-format=1] S[table-format=-1.1] S[table-format=1.1] S[table-format=1] : S[table-format=-1.1] }
                        2  &   0    &  0    &  0  &   3    \\
                        0  &   1.5  &  0    &  0  &   3    \\
                        0  &  -3    &  0.5  &  0  &  -6.6  \\
                        0  &  -2    &  1    &  1  &  -2.2  \\
                    \end{pNiceArray}
            \]

            Eliminating \(x_2\) by these transformation
            \[E_3 \coloneqq E_3 - (-2) E_2; \, E_4 \coloneqq E_4 - (\num{-1.33333}) E_2\]
            gives:
            \[
                \bm{\tilde{A}}^{(3)} =
                \begin{pNiceArray}{ S[table-format=1] S[table-format=1.1] S[table-format=1.1] S[table-format=1] : S[table-format=-1.1] }
                        2  &  0    &  0    &  0  &   3    \\
                        0  &  1.5  &  0    &  0  &   3    \\
                        0  &  0    &  0.5  &  0  &  -0.6  \\
                        0  &  0    &  1    &  1  &   1.8  \\
                    \end{pNiceArray}
            \]

            Eliminating \(x_3\) by these transformation
            \[E_4 \coloneqq E_4 - 2E_3\]
            gives:
            \[
                \bm{\tilde{A}}^{(4)} =
                \begin{pNiceArray}{ S[table-format=1] S[table-format=1.1] S[table-format=1.1] S[table-format=1] : S[table-format=-1.1] }
                        2  &  0    &  0    &  0  &   3    \\
                        0  &  1.5  &  0    &  0  &   3    \\
                        0  &  0    &  0.5  &  0  &  -0.6  \\
                        0  &  0    &  0    &  1  &   3    \\
                    \end{pNiceArray}
            \]

            The solution is \(x_4 = 3\), \(x_3 = \num{-1.2}\), \(x_2 = 2\),
            \(x_1 = \num{1.5}\).

        \item Let
            \[
                \bm{\tilde{A}} = \bm{\tilde{A}}^{(1)} =
                    \begin{pNiceArray}{ S[table-format=1] S[table-format=-1] S[table-format=-1] S[table-format=1] : S[table-format=-1] }
                        1  &   1  &   0  &  1  &   2  \\
                        2  &   1  &  -1  &  1  &   1  \\
                        4  &  -1  &  -2  &  2  &   0  \\
                        3  &  -1  &  -1  &  2  &  -3  \\
                    \end{pNiceArray}
            \]

            Eliminating \(x_1\) by these transformation
            \[E_2 \coloneqq E_2 - 2E_1; \, E_3 \coloneqq E_3 - 4E_1; \, E_4 \coloneqq E_4 - 3E_1\]
            gives:
            \[
                \bm{\tilde{A}}^{(2)} =
                \begin{pNiceArray}{ S[table-format=1] *{3}{ S[table-format=-1] } : S[table-format=-1] }
                        1  &   1  &   0  &   1  &   2  \\
                        0  &  -1  &  -1  &  -1  &  -3  \\
                        0  &  -5  &  -2  &  -2  &  -8  \\
                        0  &  -4  &  -1  &  -1  &  -9  \\
                    \end{pNiceArray}
            \]

            Eliminating \(x_2\) by these transformation
            \[E_3 \coloneqq E_3 - 5E_2; \, E_4 \coloneqq E_4 - 4E_2\]
            gives:
            \[
                \bm{\tilde{A}}^{(3)} =
                    \begin{pNiceArray}{ S[table-format=1] *{3}{ S[table-format=-1] } : S[table-format=-1] }
                        1  &   1  &   0  &   1  &   2  \\
                        0  &  -1  &  -1  &  -1  &  -3  \\
                        0  &   0  &   3  &   3  &   7  \\
                        0  &   0  &   3  &   3  &   3  \\
                    \end{pNiceArray}
            \]

            Eliminating \(x_3\) by these transformation
            \[E_4 \coloneqq E_4 - E_3\]
            gives:
            \[
                \bm{\tilde{A}}^{(4)} =
                    \begin{pNiceArray}{ S[table-format=1] *{3}{ S[table-format=-1] } : S[table-format=-1] }
                        1  &   1  &   0  &   1  &   2  \\
                        0  &  -1  &  -1  &  -1  &  -3  \\
                        0  &   0  &   3  &   3  &   7  \\
                        0  &   0  &   0  &   0  &  -4  \\
                    \end{pNiceArray}
            \]

            The system has no unique solution.
    \end{enumerate}
\end{solution}


\begin{exercise}
    Use the Gaussian Elimination Algorithm to solve the following linear
    systems, if possible, and determine whether row interchanges are necessary:

    \begin{tasks}(2)
        \task
            \begin{alignat*}{9}
                &    &{} {}&  &x_2 &{}-{}& 2&x_3 &{}={}& 4 \\
                &x_1 &{}-{}& 3&x_2 &{}+{}&  &x_3 &{}={}& 6 \\
                &x_1 &{} {}&  &    &{}-{}&  &x_3 &{}={}& 2
            \end{alignat*}
        \task
           \begin{alignat*}{12}
                 &x_1 &{}-{}& \num{0.5}&    &{}+{}&          &x_3 &{} {}&  &    &{}={}& 4 \\
                2&x_1 &{}-{}&          &x_2 &{}-{}&          &x_3 &{}+{}&  &x_4 &{}={}& 5 \\
                 &x_1 &{}+{}&          &x_2 &{}+{}& \num{0.5}&x_3 &{} {}&  &    &{}={}& 2 \\
                 &x_1 &{}-{}& \num{0.5}&x_2 &{}+{}&          &x_3 &{}+{}&  &x_4 &{}={}& 5
            \end{alignat*}
        \task
            \begin{alignat*}{12}
                2&x_1 &{} {}& -{}&x_2 &{}+{}&  &x_3 &{}-{}&  &x_4 &{}={}& 6 \\
                 &    &{} {}&    &x_2 &{}-{}&  &x_3 &{}+{}&  &x_4 &{}={}& 5 \\
                 &    &{} {}&    &    &{} {}&  &    &{} {}&  &x_4 &{}={}& 5 \\
                 &    &{} {}&    &    &{} {}&  &x_3 &{}-{}&  &x_4 &{}={}& 3
            \end{alignat*}
        \task
            \begin{alignat*}{12}
                  &x_1 &{}+{}&  &x_2 &{} {}&  &    &{}+{}&  &x_4 &{}={}&  2 \\
                 2&x_1 &{}+{}&  &x_2 &{}-{}&  &x_3 &{}+{}&  &x_4 &{}={}&  1 \\
                -1&x_1 &{}+{}& 2&x_2 &{}+{}& 3&x_3 &{}-{}&  &x_4 &{}={}&  4 \\
                 3&x_1 &{}-{}&  &x_2 &{}-{}&  &x_3 &{}+{}& 2&x_4 &{}={}& -3
            \end{alignat*}
    \end{tasks}
\end{exercise}

\begin{solution}
    \begin{enumerate}[label = \alph*)]
        \item Let
            \[
                \bm{\tilde{A}} = \bm{\tilde{A}}^{(1)} =
                    \begin{pNiceArray}{ S[table-format=1] S[table-format=-1] S[table-format=-1] : S[table-format=1] }
                        0  &   1  &  -2  &  4  \\
                        1  &  -1  &   1  &  6  \\
                        1  &   0  &  -1  &  2  \\
                    \end{pNiceArray}
            \]

            As \(a_{11}^{(1)} = 0\), we need to swap row 1 and 2. Eliminating
            \(x_1\) by these transformation
            \[E_3 \coloneqq E_3 - E_1\]
            gives:
            \[
                \bm{\tilde{A}}^{(2)} =
                \begin{pNiceArray}{ S[table-format=1] S[table-format=-1] S[table-format=-1] : S[table-format=-1] }
                        1  &  -1  &   1  &   6  \\
                        0  &   1  &  -2  &   4  \\
                        0  &   1  &  -2  &  -4  \\
                    \end{pNiceArray}
            \]

            As \(a_{22}^{(2)} = 0\), we have to swap row 2 and 3. Eliminating
            \(x_2\) by these transformation
            \[E_3 \coloneqq E_3 - E_2\]
            gives:
            \[
                \bm{\tilde{A}}^{(3)} =
                    \begin{pNiceArray}{ S[table-format=1] S[table-format=-1] S[table-format=-1] : S[table-format=-1] }
                        1  &  -1  &   1  &   6  \\
                        0  &   1  &  -2  &   4  \\
                        0  &   0  &   0  &  -8  \\
                    \end{pNiceArray}
            \]

            The system has no unique solution.

        \item Let
            \[
                \bm{\tilde{A}} = \bm{\tilde{A}}^{(1)} =
                    \begin{pNiceArray}{ S[table-format=-1] S[table-format=-1.1] S[table-format=-1.1] S[table-format=1] : S[table-format=1] }
                        1  &  -0.5  &   1    &  0  &  4  \\
                        2  &  -1    &  -1    &  1  &  5  \\
                        1  &   1    &   0.5  &  0  &  2  \\
                        1  &  -0.5  &   1    &  1  &  5  \\
                    \end{pNiceArray}
            \]

            Eliminating \(x_1\) by these transformation
            \[E_2 \coloneqq E_2 - 2 E_1; \, E_3 \coloneqq E_3 - E_1; \, E_4 \coloneqq E_4 - E_1\]
            gives:
            \[
                \bm{\tilde{A}}^{(2)} =
                    \begin{pNiceArray}{ S[table-format=-1] S[table-format=-1.1] S[table-format=-1.1] S[table-format=1] : S[table-format=-1] }
                        1  &  -0.5  &   1    &  0  &   4  \\
                        0  &   0    &  -3    &  1  &  -3  \\
                        0  &   1.5  &  -0.5  &  0  &  -2  \\
                        0  &   0    &   0    &  1  &   1  \\
                    \end{pNiceArray}
            \]

            As \(a_{22}^{(2)} = 0\), we need to swap row 2 and 3, effectively
            eliminating \(x_2\) and \(x_3\):
            \[
                \bm{\tilde{A}}^{(3)} =
                    \begin{pNiceArray}{ S[table-format=-1] S[table-format=-1.1] S[table-format=-1.1] S[table-format=1] : S[table-format=-1] }
                        1  &  -0.5  &   1    &  0  &   4  \\
                        0  &   1.5  &  -0.5  &  0  &  -2  \\
                        0  &   0    &  -3    &  1  &  -3  \\
                        0  &   0    &   0    &  1  &   1  \\
                    \end{pNiceArray}
            \]

            The solution is \(x_4 = 1\), \(x_3 \approx \num{1.33333}\), \(x_2
            \approx \num{-0.88889}\), \(x_1 \approx \num{2.22222}\).

        \item Let
            \[
                \bm{\tilde{A}} = \bm{\tilde{A}}^{(1)} =
                    \begin{pNiceArray}{ S[table-format=1] S[table-format=-1] S[table-format=-1] S[table-format=-1] : S[table-format=1] }
                        2  &  -1  &   1  &  -1  &  6  \\
                        0  &   1  &  -1  &   1  &  5  \\
                        0  &   0  &   0  &   1  &  5  \\
                        0  &   0  &   1  &  -1  &  3  \\
                    \end{pNiceArray}
            \]

            \(x_1\) and \(x_2\) are already eliminated. As \(a_{33}^{(3)} = 0\),
            we need to swap row 3 and 4, effectively eliminating \(x_3\):
            \[
                \bm{\tilde{A}}^{(2)} =
                \begin{pNiceArray}{ S[table-format=1] S[table-format=-1] S[table-format=-1] S[table-format=-1] : S[table-format=1] }
                        2  &  -1  &   1  &  -1  &  6  \\
                        0  &   1  &  -1  &   1  &  5  \\
                        0  &   0  &   1  &  -1  &  3  \\
                        0  &   0  &   0  &   1  &  5  \\
                    \end{pNiceArray}
            \]

            The solution is \(x_4 = 5\), \(x_3 = 8\), \(x_2 = 8\), \(x_1 =
            \num{5.5}\).

        \item Let
            \[
                \bm{\tilde{A}} = \bm{\tilde{A}}^{(1)} =
                    \begin{pNiceArray}{ *{4}{ S[table-format=-1] } : S[table-format=-1] }
                        1  &   1  &   0  &   1  &   2  \\
                        2  &   1  &  -1  &   1  &   1  \\
                       -1  &   2  &   3  &  -1  &   4  \\
                        3  &  -1  &  -1  &   2  &  -3  \\
                    \end{pNiceArray}
            \]

            Eliminating \(x_1\) by these transformation
            \[E_2 \coloneqq E_2 - 2E_1; \, E_3 \coloneqq E_3 - (-1) E_1; \, E_4 \coloneqq E_4 - 3E_1\]
            gives:
            \[
                \bm{\tilde{A}}^{(2)} =
                    \begin{pNiceArray}{ *{4}{ S[table-format=-1] } : S[table-format=-1] }
                        1  &   1  &   0  &   1  &   2  \\
                        0  &  -1  &  -1  &  -1  &  -3  \\
                        0  &   3  &   3  &   0  &   6  \\
                        0  &  -4  &  -1  &  -1  &  -9  \\
                    \end{pNiceArray}
            \]

            Eliminating \(x_2\) by these transformation
            \[E_3 \coloneqq E_3 - (-3) E_2; \, E_4 \coloneqq E_4 - 4E_2\]
            gives:
            \[
                \bm{\tilde{A}}^{(3)} =
                    \begin{pNiceArray}{ *{4}{ S[table-format=-1] } : S[table-format=-1] }
                        1  &   1  &   0  &   1  &   2  \\
                        0  &  -1  &  -1  &  -1  &  -3  \\
                        0  &   0  &   0  &  -3  &  -3  \\
                        0  &   0  &   3  &   3  &   3  \\
                    \end{pNiceArray}
            \]

            As \(a_{33}^{(3)} = 0\), we need to swap row 3 and 4, effectively
            eliminating \(x_3\):
            \[
                \bm{\tilde{A}}^{(4)} =
                    \begin{pNiceArray}{ *{4}{ S[table-format=-1] } : S[table-format=-1] }
                        1  &   1  &   0  &   1  &   2  \\
                        0  &  -1  &  -1  &  -1  &  -3  \\
                        0  &   0  &   3  &   3  &   3  \\
                        0  &   0  &   0  &  -3  &  -3  \\
                    \end{pNiceArray}
            \]

            The solution is \(x_4 = 1\), \(x_3 = 0\), \(x_2 = 2\), \(x_1 = -1\).
    \end{enumerate}
\end{solution}

\begin{exercise}\label{exer:3.1.7}
    Use Algorithm 6.1 and Maple with Digits:= 10 to solve the following linear
    systems \ldots
\end{exercise}

\begin{solution}
    Opps, can't help without Maple license.
\end{solution}

\begin{exercise}
    Use Algorithm 6.1 and Maple with Digits:= 10 to solve the following linear
    systems \ldots
\end{exercise}

\begin{solution}
    Opps, can't help without Maple license.
\end{solution}

\begin{exercise}
    Given the linear system

    \begin{alignat*}{5}
               2&x_1 &{}-{}& 6 \alpha& x_2 &{}=      3    \\
        3 \alpha&x_1 &{}-{}&         & x_2 &{}= \num{1.5}
    \end{alignat*}

    \begin{tasks}
        \task Find value(s) of \(\alpha\) for which the system has no solutions.
        \task Find value(s) of \(\alpha\) for which the system has an infinite
            number of solutions.
        \task Assuming a unique solution exists for a given \(\alpha\), find the
            solution.
    \end{tasks}
\end{exercise}

\begin{solution}
    Let
    \[
        \bm{\tilde{A}} = \bm{\tilde{A}}^{(1)} =
            \begin{pNiceArray}{ cc : c }
                2         &  -6 \alpha  &  3          \\
                3 \alpha  &         -1  &  \num{1.5}  \\
            \end{pNiceArray}
    \]

    Eliminating \(x_1\) gives:
    \[
        \bm{\tilde{A}}^{(2)} =
            \begin{pNiceArray}{ cc : c }
                2  &  -6 \alpha       &  3          \\
                0  &  9 \alpha^2 - 1  &  \num{1.5} - \num{4.5} \alpha  \\
            \end{pNiceArray}
    \]

    The system has no unique solution (either no solution or infinite number of
    solutions) if and only if:
    \[9 \alpha^2 - 1 = 0 \iff \alpha = \pm \frac{1}{3}\]

    \begin{enumerate}[label = \alph*)]
        \item The system has no solution if it has no unique solution and
            \[\num{1.5}(1 - 3 \alpha) \neq 0 \iff \alpha = -\frac{1}{3}\]

        \item The system has an infinite number of solution if it has no unique
            solution and
            \[\num{1.5}(1 - 3 \alpha) = 0 \iff \alpha = \frac{1}{3}\]

            In this case, the solution assumes a general form:
            \[x_2 \in \mathbb{R} \text{ and } x_1 = x_2 + \num{1.5}\]

        \item The system has a unique solution if and only if \(\alpha \neq \pm
            \frac{1}{3}\). Then the unique solution is:
            \[x_2 = \frac{\num{-1.5}}{3 \alpha + 1} \text{ and } x_1 = \frac{\num{1.5}}{3 \alpha + 1}\]
    \end{enumerate}
\end{solution}

\begin{exercise}
    Given the linear system

    \begin{alignat*}{5}
                  &x_1 &{}-{}&  & x_2 &{}+{}& \alpha&x_3 &{}= -2 \\
               -{}&x_1 &{}+{}& 2& x_2 &{}-{}& \alpha&x_3 &{}=  3 \\
            \alpha&x_1 &{}+{}&  & x_2 &{}+{}& \alpha&x_3 &{}=  2
    \end{alignat*}

    \begin{tasks}
        \task Find value(s) of \(\alpha\) for which the system has no solutions.
        \task Find value(s) of \(\alpha\) for which the system has an infinite
            number of solutions.
        \task Assuming a unique solution exists for a given \(\alpha\), find the
            solution.
    \end{tasks}
\end{exercise}

\begin{solution}
    Let
    \[
        \bm{\tilde{A}} = \bm{\tilde{A}}^{(1)} =
            \begin{pNiceArray}{ ccc : c }
                    1   &  -1  &   \alpha  &  -2  \\
                   -1   &   2  &  -\alpha  &   3  \\
                \alpha  &   1  &   \alpha  &   2  \\
            \end{pNiceArray}
    \]

    Eliminating \(x_1\) by these transformation
    \[E_2 \coloneqq E_2 - (-1) E_1; \, E_3 \coloneqq E_3 - \alpha E_1\]
    gives:
    \[
        \bm{\tilde{A}}^{(2)} =
            \begin{pNiceArray}{ ccc : c }
                1  &      -1      &        \alpha       &       -2       \\
                0  &       1      &           0         &        1       \\
                0  &  \alpha + 1  &  \alpha - \alpha^2  &  2 \alpha + 2  \\
            \end{pNiceArray}
    \]

    Eliminating \(x_2\) by these transformation
    \[E_3 \coloneqq E_3 - (\alpha + 1) E_2\]
    gives:
    \[
        \bm{\tilde{A}}^{(3)} =
            \begin{pNiceArray}{ ccc : c }
                1  &  -1  &        \alpha       &      -2      \\
                0  &   1  &           0         &       1      \\
                0  &   0  &  \alpha - \alpha^2  &  \alpha + 1  \\
            \end{pNiceArray}
    \]

    The system has no unique solution (either no solution or infinite number of
    solutions) if and only if:
    \[\alpha - \alpha^2 = 0 \iff \alpha \in \{0, 1\}\]

    \begin{enumerate}[label = \alph*)]
        \item The system has no solution if it has no unique solution and
            \[2 \alpha + 2 \neq 0 \iff \alpha \in \{0, 1\}\]

        \item The system has an infinite number of solution if it has no unique
            solution and
            \[2 \alpha + 2 \neq 0 \iff \alpha \in \varnothing\]

        \item The system has a unique solution if and only if \(\alpha \notin
            \{0, 1\}\). Then the unique solution is:
            \[x_3 = \frac{\alpha + 1}{\alpha - \alpha^2} \, \text{, } x_2 = 1 \text{ and } x_1 = \frac{2}{\alpha - 1}\]
    \end{enumerate}
\end{solution}

\begin{exercise}
    Show that the 3 elementary row operations do not change the solution set of
    a linear system.
\end{exercise}

\begin{solution}
    Let \(x_1, x_2, \ldots, x_n\) be the solution of the original system.

    When an elementary row operations is applied on row \(i^{th}\), the original
    solution still satisfies the unchanged rows. We have to proove that it also
    satisfies the changed row.

    \begin{enumerate}[label = \alph*)]
        \item If \(i^{th}\) row is scaled, \(i^{th}\) equation is still
            satisfied by the original solution because both size of it is
            multiplied with a constant.

        \item If a scaled \(j^{th}\) row is added to \(i^{th}\) row, then the
            original solution still satisfies the new row, as
            \begin{itemize}
                \item it satisfies the \(j^{th}\) row, therefore satisfies the
                    scaled \(j^{th}\) row, as proven above, and
                \item it satisfies the original \(i^{th}\) row
            \end{itemize}

        \item If the rows are swapped, the solution does not change, as the set
            of the equation does not change.
    \end{enumerate}
\end{solution}

\begin{exercise}\label{exer:3.1.12}
    Gauss-Jordan Method: This method is described as follows. Use the \(i^{th}\)
    equation to eliminate not only \(x_i\) from the equations \(E_{> i}\) as was
    done in the Gaussian elimination method, but also from \(E_{< i}\). Upon
    reducing \([\bm{A}, \bm{b}]\) to:
    \[\begin{pNiceArray}{cccc:c}
        a_{11}^{(1)}  &                &          &                &  b_1^{(1)}  \\
                      &  a_{22}^{(2)}  &          &                &  b_2^{(2)}  \\
                      &                &  \ddots  &                &  \vdots     \\
                      &                &          &  a_{nn}^{(n)}  &  b_n^{(n)}  \\
    \end{pNiceArray}\]
    the solution can be obtained by
    \[x_i = \frac{b_i^{(i)}}{a_{ii}^{(i)}}\]

    This procedure circumvents the backward substitution in the Gaussian
    elimination. Construct an algorithm for the Gauss-Jordan procedure patterned
    after that of Algorithm 6.1.
\end{exercise}

\begin{solution}
    In Step 4, change \(j\) from \(j > i\) to \(j \neq i\).

    In Step 8, calculate for all \(i\):
    \[x_i = \frac{b_i}{a_{ii}}\]

    Remove Step 9.
\end{solution}

\begin{exercise}
    Use the Gauss-Jordan method and two-digit rounding arithmetic to solve the
    systems in \hyperref[exer:3.1.3]{Exercise 3}.
\end{exercise}

\begin{solution}

    \begin{enumerate}[label = \alph*)]
        \item Let
            \[
                \bm{\tilde{A}} = \bm{\tilde{A}}^{(1)} =
                    \begin{pNiceArray}{ S[table-format=1] S[table-format=-1] S[table-format=1] : S[table-format=2] }
                        4  &  -1  &  1  &   8  \\
                        2  &   5  &  2  &   3  \\
                        1  &   2  &  4  &  11  \\
                    \end{pNiceArray}
            \]

            Eliminating \(x_1\) by these transformation
            \[E_2 \coloneqq E_2 - 0.5 E_1; \, E_3 \coloneqq E_3 - 0.25 E_1\]
            gives:
            \[
                \bm{\tilde{A}}^{(2)} =
                    \begin{pNiceArray}{ S[table-format=1] S[table-format=-1.2] S[table-format=1.2] : S[table-format=-1] }
                        4  &  -1     &  1     &   8  \\
                        0  &   5.5   &  1.5   &  -1  \\
                        0  &   2.25  &  3.75  &   9  \\
                    \end{pNiceArray}
            \]

            Eliminating \(x_2\) by these transformation
            \[E_1 \coloneqq E_1 - (-0.18182) E_2; \, E_3 \coloneqq E_3 - 0.40909 E_2\]
            gives:
            \[
                \bm{\tilde{A}}^{(3)} =
                    \begin{pNiceArray}{ S[table-format=1] S[table-format=1.1] S[table-format=1.5] : S[table-format=-1.5] }
                        4  &  0    &  1.27273  &   7.81818  \\
                        0  &  5.5  &  1.5      &  -1        \\
                        0  &  0    &  3.13636  &   9.40909  \\
                    \end{pNiceArray}
            \]

            Eliminating \(x_3\) by these transformation
            \[E_1 \coloneqq E_1 - 0.40580 E_3; \, E_2 \coloneqq E_2 - 0.47826 E_3\]
            gives:
            \[
                \bm{\tilde{A}}^{(4)} =
                    \begin{pNiceArray}{ S[table-format=1] S[table-format=1.1] S[table-format=1.5] : S[table-format=-1.5] }
                        4  &  0    &  0        &   4        \\
                        0  &  5.5  &  0        &  -5.5      \\
                        0  &  0    &  3.13636  &   9.40909  \\
                    \end{pNiceArray}
            \]

            The solution is \(x_3 \approx 3\), \(x_2 \approx -1\), \(x_1 \approx
            1\).

        \item Let
            \[
                \bm{\tilde{A}} = \bm{\tilde{A}}^{(1)} =
                    \begin{pNiceArray}{ S[table-format=1] S[table-format=1] S[table-format=-1] : S[table-format=-1] }
                        4  &  1  &   2  &   9  \\
                        2  &  4  &  -1  &  -5  \\
                        1  &  1  &  -3  &  -9
                    \end{pNiceArray}
            \]

            Eliminating \(x_1\) by these transformation
            \[E_2 \coloneqq E_2 - 0.50000 E_1; \, E_3 \coloneqq E_3 - 0.25000 E_1\]
            gives:
            \[
                \bm{\tilde{A}}^{(2)} =
                    \begin{pNiceArray}{ S[table-format=1] S[table-format=1.2] S[table-format=-1.1] : S[table-format=-2.2] }
                        4  &  1     &   2    &    9     \\
                        0  &  3.5   &  -2    &   -9.5   \\
                        0  &  0.75  &  -3.5  &  -11.25  \\
                    \end{pNiceArray}
            \]

            Eliminating \(x_2\) by these transformation
            \[E_1 \coloneqq E_1 - 0.28571 E_2; \, E_3 \coloneqq E_3 - 0.21429 E_2\]
            gives:
            \[
                \bm{\tilde{A}}^{(3)} =
                    \begin{pNiceArray}{ S[table-format=1] S[table-format=1.1] S[table-format=-1.5] : S[table-format=-1.5] }
                        4  &  0    &   2.57143  &  11.71429  \\
                        0  &  3.5  &  -2        &  -9.5      \\
                        0  &  0    &  -3.07143  &  -9.21429  \\
                    \end{pNiceArray}
            \]

            Eliminating \(x_3\) by these transformation
            \[E_1 \coloneqq E_1 - (-0.83721) E_3; \, E_2 \coloneqq E_2 - 0.65116 E_3\]
            gives:
            \[
                \bm{\tilde{A}}^{(4)} =
                    \begin{pNiceArray}{ S[table-format=1] S[table-format=1.1] S[table-format=-1.5] : S[table-format=-1.5] }
                        4  &  0    &   0        &   4        \\
                        0  &  3.5  &   0        &  -3.5      \\
                        0  &  0    &  -3.07143  &  -9.21429  \\
                    \end{pNiceArray}
            \]

            The solution is \(x_3 \approx 3\), \(x_2 \approx -1\), \(x_1 \approx
            1\).
    \end{enumerate}
\end{solution}

\begin{exercise}
    Repeat \hyperref[exer:3.1.7]{Exercise 7} using the Gauss-Jordan method.
\end{exercise}

\begin{solution}
    Opps, can't help without Maple license.
\end{solution}

\begin{exercise}
    \begin{tasks}
        \task Show that the Gauss-Jordan method requires
            \[\frac{n^3}{2} + n^2 - \frac{n}{2} \text{ multiplications/divisions}\]
            and
            \[\frac{n^3}{2} - \frac{n}{2} \text{ additions/subtractions}\]

        \task Make a table comparing the required operations for the
            Gauss-Jordan and Gaussian elimination methods for \(n = 3, 10, 50,
            100\). Which method requires less computation?
    \end{tasks}
\end{exercise}

\begin{solution}
    \begin{enumerate}[label = \alph*)]
        \item We have the following analysis:
            \begin{itemize}
                \item In Step 1, \(i\) iterates from \(1\) to \(n\), so there is
                    \(n\) iterations. Inside each iteration:
                    \begin{itemize}
                        \item In Step 4, \(j\) iterates from \(1\) to \(n\) but
                            skips \(i\), so there is \(n - 1\) iterations.
                            Inside each iteration:
                            \begin{itemize}
                                \item In Step 5: \(1\) divisions
                                \item In Step 6: \(n + 1\) multiplications; \(n
                                    + 1\) subtractions.

                                    However, some operations with or known to
                                    results in \(0\) could be skipped,
                                    therefore, Step 6 requires \(n + 1 - i\)
                                    multiplications and \(n + 1 - i\)
                                    subtractions.
                            \end{itemize}

                            Therefore, in each Step 4 iteration, there are \(n - i +
                            2\) multiplications/divisions and \(n - i + 1\)
                            subtractions.
                    \end{itemize}

                    Therefore, in each Step 1 iteration, there are \((n - 1)(n -
                    i + 2)\) multiplications/divisions and \((n - 1)(n - i +
                    1)\) subtractions

                    Therefore, in all Step 1 iterations, there are
                    \begin{align*}
                        \sum_{i = 1}^{n} (n - 1)(n - i + 2) &= (n - 1) \left[n(n + 2) - \sum_{i = 1}^{n} i \right] \\
                                                            &= \frac{n^3 + 2n^2 - 3n}{2} \text{ multiplications/divisions}
                    \end{align*}
                    and
                    \begin{align*}
                        \sum_{i = 1}^{n} (n - 1)(n - i + 1) &= (n - 1) \left[n(n + 1) - \sum_{i = 1}^{n} i \right] \\
                                                            &= \frac{n^3 - n}{2} \text{ subtractions}
                    \end{align*}

                \item In Step 9, \(i\) iterates from \(1\) to \(n\), so there is
                    \(n\) iterations. Inside each iteration, there is only \(1\)
                    divisions. Therefore, in all Step 9 divisions, there are
                    \(n\) divisions.
            \end{itemize}

            We can now conclude that Gauss-Jordan requires
            \[\frac{n^3 + 2n^2 - 3n}{2} + n = \frac{n^3}{2} + n^2 - \frac{n}{2} \text{ multiplications/divisions}\]
            and
            \[\frac{n^3}{2} - \frac{n}{2} \text{ additions/subtractions}\]

            Note that in most simple implementation, the cost of branching code
            to skip operations (for example in Step 6 of this analysis) is
            greater than the save from skipping operations itself. Therefore, a
            well-vectorized implementation, though requiring even more
            computation, turns out to outperform a ``skip'' implementation.

        \item We have the following table:
            \begin{table}[H]
                \centering
                \begin{tabular}{ccccc}
                    \toprule
                    &  \multicolumn{2}{c}{Gauss Elimination}  &  \multicolumn{2}{c}{Gauss-Jordan}  \\
                    \midrule
                    \(n\)  &  M/D  &  A/S  &  M/D  &  A/S  \\
                    \midrule
                    3    &  17      &  11      &  21      &  12      \\
                    10   &  430     &  375     &  595     &  495     \\
                    50   &  44150   &  42875   &  64975   &  62475   \\
                    100  &  343300  &  338250  &  509950  &  499950  \\
                    \bottomrule
                \end{tabular}
            \end{table}

            Obviously, Gauss Elimination requires less computation.
    \end{enumerate}
\end{solution}

\begin{exercise}
    Consider the following Gaussian-elimination-Gauss-Jordan hybrid method for
    solving system of equations. First, apply the Gaussian-elimination technique
    to reduce the system to triangular form. Then use the \(n^{th}\) equation to
    eliminate the coefficients of \(x_n\) in each of the first \(n - 1\) rows.
    After this is completed use the \((n - 1)^{th}\) equation to eliminate the
    coefficients of \(x_{n - 1}\) in the first \(n - 2\) rows, etc. The system
    will eventually appear as the reduced system in
    \hyperref[exer:3.1.12]{Exercise 12}.

    \begin{tasks}
        \task Show that this method requires
            \[\frac{n^3}{3} + \frac{3n^2}{2} - \frac{5n}{6} \text{ multiplications/divisions}\]
            and
            \[\frac{n^3}{2} + \frac{n^2}{2} - \frac{5n}{6} \text{ additions/subtractions}\]

        \task Make a table comparing the required operations for the Gaussian
            elimination, Gauss-Jordan, and hybrid methods, for \(n = 3, 10, 50,
            100\). Which method requires less computation?
    \end{tasks}
\end{exercise}

\begin{solution}
    \begin{enumerate}[label = \alph*)]
        \item We have the following analysis:
            \begin{itemize}
                \item Gauss elimination to upper triangular form: takes
                    \[\frac{n^3}{3} + \frac{n^2}{2} - \frac{5n}{6} \text{ multiplications/divisions}\]
                    and
                    \[\frac{n^3}{3} - \frac{n}{3} \text{additions/subtractions}\]
                \item Use the \(i^{th}\) equation to eliminate \(x_i\) in each
                    of the first \(i - 1\) rows, starting with \(i = n\): Let
                    \(i\) iterates from \(n\) to \(2\), so there is \(n - 1\)
                    iterations.

                    Inside each iteration, \(x_i\) is eliminated from \((i -
                    1)^{th}\) equation to the first one. We only need to update
                    the last column, as most operations with, or results in
                    \(0\) is skipped. So, there is \(1\) division (for
                    multiplier), \(1\) multiplication (scale row, or in fact
                    last element of the row), \(1\) subtraction (elimination).

                    Therefore, in all iterations of this step, there are
                    \[\sum_{i = 2}^{n} 2(i - 1) = 2 \sum_{i = 1}^{n - 1} i = n(n - 1) \text{ multiplications/divisions}\]
                    and
                    \[\sum_{i = 2}^{n} (2i) = \frac{n(n - 1)}{2} \text{ multiplications/divisions}\]
                \item The last step of solving diagonal matrix takes \(n\)
                    divisions
            \end{itemize}

            We can now conclude that the hybrid methods takes
            \[\frac{n^3}{3} + \frac{3n^2}{2} - \frac{5n}{6} \text{ multiplications/divisions}\]
            and
            \[\frac{n^3}{2} + \frac{n^2}{2} - \frac{5n}{6} \text{ additions/subtractions}\]

        \item We have the following table:
            \begin{table}[H]
                \centering
                \begin{tabular}{ccccccc}
                    \toprule
                    &  \multicolumn{2}{c}{Gauss Elimination}  &  \multicolumn{2}{c}{Gauss-Jordan}  &  \multicolumn{2}{c}{Hybrid}  \\
                    \midrule
                    \(n\)  &  M/D  &  A/S  &  M/D  &  A/S  &  M/D  &  A/S  \\
                    \midrule
                    3    &  17      &  11      &  21      &  12      &  20      &  11      \\
                    10   &  430     &  375     &  595     &  495     &  475     &  375     \\
                    50   &  44150   &  42875   &  64975   &  62475   &  45375   &  42875   \\
                    100  &  343300  &  338250  &  509950  &  499950  &  348250  &  338250  \\
                    \bottomrule
                \end{tabular}
            \end{table}
    \end{enumerate}
\end{solution}

\end{document}
