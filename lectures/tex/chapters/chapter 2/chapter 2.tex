\documentclass[../../Lectures.tex]{subfiles}

\begin{document}

\chapter{Giải gần đúng phương trình}

%----------------------------------------------------------------------------------------
%	2.1: Mở đầu
%----------------------------------------------------------------------------------------

\section{Mở đầu}

Sự tăng trưởng của dân số thường có thể được mô hình hóa trong khoảng thời gian
ngắn bằng cách giả định rằng dân số tăng liên tục theo thời gian tỷ lệ thuận với
con số hiện tại vào thời điểm đó. Giả sử \(N(t)\) biểu thị số dân tại thời điểm
\(t\) và \(\lambda\) biểu thị tỷ lệ sinh không đổi của cộng đồng. Khi đó dân số
thỏa mãn phương trình vi phân:

\[\od{N(t)}{t} = \lambda N(t)\]

Nghiệm của phương trình là \(N(t) = N_0 e^{\lambda t}\), ở đây \(N_0\) là dân số
ban đầu.

% TODO: insert chart

Mô hình hàm mũ này chỉ có giá trị khi dân số bị cô lập, không có người nhập cư.
Nếu nhập cư được phép ở tốc độ không đổi \(v\) thì phương trình vi phân trở
thành:

\[\od{N(t)}{t} = \lambda N(t) + v\]

Nghiệm của nó là:

\[N(t) = N_0 e^{\lambda t} + \frac{v}{\lambda} (e^{\lambda t} - 1)\]

Giả sử ban đầu có \(N(0) = \num{1000000}\) người, và có tới \num{435000} người
nhập cư vào cộng đồng trong năm đầu tiên, vậy \(N(1) = \num{1564000}\) người có
mặt vào cuối năm đầu tiên. Để xác định tỷ lệ sinh của cộng đồng dân số này,
chúng ta cần tìm \(\lambda\) trong phương trình:

\[\num{1564000} = \num{1000000} e^\lambda + \frac{\num{435000}}{\lambda} (e^\lambda - 1)\]

Không thể giải một cách chính xác giá trị \(\lambda\) trong phương trình này,
nhưng các phương pháp tính được thảo luận trong chương này có thể được sử dụng
để tính gần đúng nghiệm của các phương trình loại này với độ chính xác cao tùy
ý.


%----------------------------------------------------------------------------------------
%	2.2: Phương pháp chia đôi
%----------------------------------------------------------------------------------------

\section{Phương pháp chia đôi}

Giả sử \(f\) là hàm số xác định và liên tục trên khoảng \(\interval{a}{b}\), với
\(f(a)\) và \(f(b)\) trái dấu.
\emph{\hyperref[thm:intermediate_value_theorem]{Định lý giá trị trung gian}} nói
rằng tồn tại một số \(p \in \interval[open]{a}{b}\) với \(f(p) = 0\).

\begin{theorem}\label{thm:intermediate_value_theorem}
    Định lý giá trị trung gian (Intermediate Value Theorem).

    Nếu \(f\) liên tục trên \(\interval{a}{b}\) và \(K\) nằm giữa \(f(a)\) và
    \(f(b)\), tồn tại \(c \in \interval[open]{a}{b}\) sao cho \(f(c) = K\).
\end{theorem}

Cụ thể hơn, do \(f(a)\) và \(f(b)\) trái dấu, do đó \(0\) nằm giữa \(f(a)\) và
\(f(b)\), do đó tồn tại nghiệm \(p \in \interval[open]{a}{b}\). Trường hợp đặc
biệt này được gọi là \emph{định lý Bolzano}.

Mặc dù có thể tồn tại nhiều hơn một nghiệm trong khoảng
\(\interval[open]{a}{b}\), nhưng để thuận lợi, chúng ta giả thiết chỉ có duy
nhất một nghiệm trong khoảng này. Khi đó, ta có thể dùng phương pháp sau:

\begin{method}
\emph{Phương pháp chia đôi (Bisection method)}

Phương pháp này cho phép tìm nghiệm \(p\) của \(f(p) = 0\) trong khoảng
\(\interval{a}{b}\), với \(f(a)\) và \(f(b)\) trái dấu.

Để bắt đầu, ta đặt \(a_1 = a\) và \(b_1 = b\), và đặt \(p_1\) là điểm giữa của
\(\interval{a}{b}\); nghĩa là:

\[p_1 = a_1 + \frac{b_1 - a_1}{2} = \frac{a_1 + b_1}{2}\]

\begin{itemize}
    \item Nếu \(f(p_1) = 0\) thì \(p = p_1\).
    \item Nếu \(f(p_1) \neq 0\) thì \(f(p_1)\) cùng dấu với \(f(a_1)\) hoặc \(f(b_1)\).
        \begin{itemize}
            \item Nếu \(f(p_1)\) cùng dấu với \(f(a_1)\) thì \(p \in
                \interval{p_1}{b_1}\). Đặt \(a_2 = p_1\), \(b_2 = b_1\).
            \item Nếu \(f(p_1)\) cùng dấu với \(f(b_1)\) thì \(p \in
                \interval{a_1}{p_1}\). Đặt \(a_2 = a_1\), \(b_2 = p_1\).
        \end{itemize}
        sau đó làm tiếp phương pháp trên với khoảng \(\interval{a_2}{b_2}\).
\end{itemize}
\end{method}

Các cách dừng khác (còn gọi là \emph{tiêu chí dừng}) có thể được áp dụng trong
phương pháp trên hoặc trong bất kỳ các kỹ thuật lặp lại trong chương này. Ví dụ,
chúng ta có thể chọn một dung sai \(\varepsilon > 0\) và tạo dãy \(p_1, ...,
p_N\) cho đến khi đáp ứng một trong các điều kiện sau:

\begin{align}
                      \abs{p_N - p_{N - 1}} &< \varepsilon \text{, } \\
    \frac{\abs{p_N - p_{N - 1}}}{\abs{p_N}} &< \varepsilon \text{, } p_N \neq 0 \text{ hoặc} \label{rel_err_stop_criteria} \\
                               \abs{f(p_N)} &< \varepsilon
\end{align}

% TODO: fix reference to the above equation in the individual file.
Không may, khó khăn có thể phát sinh với bất kỳ tiêu chí dừng nào. Ví dụ, có các
chuỗi \(\{p_n\}_{n=1}^\infty\) mà hiệu \(p_n - p_{n - 1}\) hội tụ về \num{0}
trong khi dãy đó lại phân kỳ. Cũng có thể có \(f(p_n)\) gần bằng \num{0} trong
khi \(p_n\) khác đáng kể so với \(p\). Nếu không có kiến thức bổ sung về \(f\)
hoặc \(p\), bất đẳng thức \ref{rel_err_stop_criteria} là tiêu chuẩn dừng tốt
nhất để áp dụng vì nó sát nhất với sai số tương đối.

Khi dùng máy tính để tính xấp xỉ, nên thiết lập một giới hạn trên về số lần lặp
lại. Điều này giúp tránh vòng lặp vô hạn, một tình huống có thể phát sinh khi
chuỗi \(\{p_N\}_{n=0}^\infty\) phân kỳ (và cả khi chương trình sai).

\begin{exmp}
    Chứng minh rằng \(f(x) = x^3 + 4x^2 - 10 = 0\) có nghiệm trong khoảng
    \(\interval{1}{2}\), và dùng phương pháp chia đôi để xác định nghiệm đúng
    đến \(10^{-4}\).

    Vì \(f(1) = -5\) và \(f(2) = 14\), \(f(x) = 0\) chắc chắn có nghiệm trong
    khoảng \(\interval{1}{2}\).

    Ta có bảng sau:

    \begin{tabular}{ r S[table-format=1.9] S[table-format=1.9] S[table-format=1.9] S[table-format=-1.6] }    % - for minus sign
        \\    % Add space between the table and the previous paragraph.
        \toprule
         {\(n\)} &   {\(a_n\)}   &   {\(b_n\)}   &   {\(p_n\)}   & {\(f(p_n)\)} \\
        \midrule
              1  &  1.0          &  2.0          &  1.5          &   2.375      \\
              2  &  1.0          &  1.5          &  1.25         &  -1.79687    \\
              3  &  1.25         &  1.5          &  1.375        &   0.16211    \\
              4  &  1.25         &  1.375        &  1.3125       &  -0.84839    \\
              5  &  1.3125       &  1.375        &  1.34375      &  -0.35098    \\
              6  &  1.34375      &  1.375        &  1.359375     &  -0.09641    \\
              7  &  1.359375     &  1.375        &  1.3671875    &   0.03236    \\
              8  &  1.359375     &  1.3671875    &  1.36328125   &  -0.03215    \\
              9  &  1.36328125   &  1.3671875    &  1.365234375  &   0.000072   \\
             10  &  1.36328125   &  1.365234375  &  1.364257813  &  -0.01605    \\
             11  &  1.364257813  &  1.365234375  &  1.364746094  &  -0.00799    \\
             12  &  1.364746094  &  1.365234375  &  1.364990234  &  -0.00396    \\
             13  &  1.364990234  &  1.365234375  &  1.365112305  &  -0.00194    \\
        \bottomrule
        \\    % Add space between the table and the next paragraph.
    \end{tabular}

    Sau 13 lần lặp, \(p_{13} = \num{1.365112305}\) xấp xỉ nghiệm \(p\) với sai số:

    \[\abs{p - p_{13}} < \abs{b_{14} - a_{14}} = \abs{\num{1.365234375} - \num{1.365112305}} = \num{0.000122070}\]

    Do \(\abs{a_{14}} < \abs{p}\) (khoảng đang xét dương), ta có:

    \[\frac{\abs{p - p_{13}}}{\abs{p}} < \frac{\abs{b_{14} - a_{14}}}{\abs{a_{14}}} \leq \num{9e-5}\]

    Cần chú ý rằng, \(p_9\) thực sự gần với \(p\) hơn kết quả cuối cùng
    \(p_{13}\), tuy nhiên khi thực hiện thuật toán ta không thể biết đều này.
    Hơn nữa, \(\abs{f(p_9)} < \abs{f(p_{13})}\) cũng không liên quan đến việc
    \(p_9\) sát với \(p\) hơn.
\end{exmp}

Phương pháp chia đôi có hai điểm yếu lớn:

\begin{itemize}
    \item Cần số vòng lặp \(N\) lớn
    \item Vô tình bỏ qua các xấp xỉ tốt
\end{itemize}

Dù vậy, phương pháp này lại có một ưu điểm lớn là đảm bảo dãy
\(\{p_N\}_{n=0}^\infty\) hội tụ đến một nghiệm. Do ưu điểm này, phương pháp chia
đôi thường được dùng để tìm điểm bắt đầu cho các phương pháp khác hiệu quả hơn
mà sẽ được giới thiệu sau.

\begin{theorem}
    Cho hàm \(f \in \interval{a}{b}\) và \(f(a) \dot f(b) < 0\). Phương pháp
    chia đôi tạo ra một chuỗi \(\{p_n\}_{n=1}^\infty\) xấp xỉ nghiệm \(p\) của
    \(f\) với sai số như sau:

    \[\abs{p_n - p} \leq \frac{b - a}{2^n} \text{, } n \geq 1\]
\end{theorem}

\begin{proof}
    Với mọi \(n \geq 1\), ta có:

    \[b_n - a_n = \frac{1}{2^{n - 1}} (b - a) \text{ và } p \in \interval[open]{a_n}{b_n}\]

    Do

    \[p_n = \frac{1}{2} (a_n + b_n)\]

    ta suy ra được

    \begin{align*}    % TODO: Make the iff symbol close to the indent.
        &\qquad& \frac{1}{2} (a_n + b_n) - b_n \leq p_n - p &\leq \frac{1}{2} (a_n + b_n) - a_n \\
        \iff&&   \frac{1}{2} (a_n - b_n)       \leq p_n - p &\leq \frac{1}{2} (b_n - a_n)       \\
        \iff&&                                \abs{p_n - p} &\leq \frac{1}{2} (b_n - a_n) = \frac{b - a}{2^n}
    \end{align*}
\end{proof}


%----------------------------------------------------------------------------------------
%	2.3: Phương pháp điểm bất động
%----------------------------------------------------------------------------------------

\section{Phương pháp điểm bất động}

\emph{Điểm bất động (fixed point)} của một hàm là số mà tại đó giá trị của hàm
số bằng đúng giá trị của đối số.

% TODO: count by section
\begin{definition}
    Số \(p\) được gọi là điểm bất động của hàm số \(g\) nếu \(g(p) = p\).
\end{definition}

Trong phần này, chúng ta xét việc đưa bài toán tìm nghiệm về bài toán tìm điểm
bất động và tìm sự liên hệ giữa chúng.

Các bài toán tìm nghiệm và các bài toán tìm điểm cố định là các lớp tương đương
theo nghĩa sau đây:
\begin{itemize}
    \item Từ bài toán tìm nghiệm của phương trình \(f(p) = 0\), ta có thể xác
        định hàm \(g\) với điểm bất động tại \(p\) theo một số cách, ví dụ,

        \[g(x) = x - f(x) \text{, hoặc } g(x) = x + 3f(x)\].

    \item Ngược lại, nếu hàm \(g\) có một điểm bất định tại \(p\), thì hàm \(f\)
        xác định bởi \[f(x) = x - g(x)\] có nghiệm tại \(p\).
\end{itemize}

Mặc dù các bài toán ta muốn giải quyết là dạng tìm nghiệm, nhưng dạng điểm bất
động dễ thực hiện hơn và có một số lựa chọn điểm bất động dẫn tới kỹ thuật tìm
nghiệm rất hiệu quả. Trước hết ta cần đi đến dạng bài toán mới này một cách
thoải mái, và đưa ra quyết định khi nào hàm số có điểm bất động và điểm bất động
được xấp xỉ với độ chính xác bao nhiêu.

Các điểm bất động xuất hiện trong nhiều lĩnh vực toán học khác nhau, và là công
cụ chính của các nhà kinh tế dùng để chứng minh các kết quả liên quan đến tính
cân bằng. Mặc dù ý tưởng đằng sau kỹ thuật là cũ, nhưng thuật ngữ được sử dụng
lần đầu bởi nhà toán học Hà Lan L. E. J. Brouwer (1882 - 1962) trong đầu những
năm 1900.

\begin{exmp}
    Hãy xác định điểm bất động của hàm \(g(x) = x^2 - 2\).

    Điểm bất động \(p\) của \(g\) có tính chất:

    \[p= g(p) \iff p = p^2 - 2\]

    Suy ra

    \[p^2 - p - 2 = (p + 1)(p - 2) = 0\]

    Điểm bất động xảy ra đúng khi khi đồ thị của hàm số \(y = g(x)\) cắt đồ thị
    hàm số \(y = x\), vì vậy \(g\) có 2 điểm bất động là \(-1\) và \(2\). Điều
    này được minh họa bởi hình~\ref{fig:exmp_2.2_fixed_point}.

    \begin{figure}[!h]
        \centering
        \subfile{graphics/exmp_2.2_fixed_point/exmp_2.2_fixed_point.tex}
        \caption{Điểm bất động của \(y = x^2 - 2\)}
        \label{fig:exmp_2.2_fixed_point}
    \end{figure}
\end{exmp}

Định lý sau cho điều kiện đủ để hàm số \emph{có ít nhất một} và \emph{có duy
nhất một} điểm bất động.

\begin{theorem}\label{thm:1}
\phantom\\
\begin{enumerate}
    \item Nếu \(g \in \interval{a}{b}\), và \(g(x) \in \interval{a}{b} \,
        \forall x \in \interval{a}{b}\), khi đó \(g\) có ít nhất một điểm bất
        động trên \(\interval{a}{b}\).
    \item Hơn nữa, nếu \(g'(x)\) tồn tại trên \(\interval[open]{a}{b}\) và
        \(\abs{g'(x)} < 1 \, \forall x \in \interval{a}{b}\), khi đó, tồn tại
        đúng một điểm bất động trên \(\interval{a}{b}\).
\end{enumerate}
\end{theorem}

Trước khi chứng minh định lí trên, ta cần biết \emph{định lí giá trị trung
bình}.

\begin{theorem}\label{thm:mean_value_theorem}
    Định lí giá trị trung bình (Mean Value Theorem).

    Nếu \(f\) liên tục trên \(\interval{a}{b}\) và khả vi trên
    \(\interval[open]{a}{b}\), tồn tại một điểm \(c \in \interval[open]{a}{b}\)
    sao cho tiếp tuyến tại \(c\) song song với cát tuyến qua hai điểm mút \((a,
    f(a))\) và \((b, f(b))\), hay nói cách khác:

    \[f'(c) = \frac{f(b) - f(a)}{b - a}\]
\end{theorem}

\begin{proof}[Chứng minh Định lí~\ref{thm:1}]
\phantom\\
\begin{enumerate}
    \item Nếu \(g(a) = a\) hoặc \(g(b) = b\), \(g\) có điểm bất động \(a\) hoặc
        \(b\). Nếu không, \(g(a) > a\) và đồng thời \(g(b) < b\); ta sẽ xét
        trường hợp này.

        Hàm \(h(x) = g(x) - x\) liên tục trên \(\interval{a}{b}\) với:

        \[h(a) - a > 0 \text{ và } h(b) - b > 0\]

        \hyperref[thm:intermediate_value_theorem]{Định lý giá trị trung gian}
        khẳng định rằng tồn tại \(p \in \interval[open]{a}{b}\) sao cho \(h(p) =
        0\). Điểm \(p\) này là điểm bất động của \(g\) vì:

        \[0 = h(p) = g(p) - p \iff g(p) = p\]

    \item Giả sử \(g\) có hai điểm bất động \(p\), \(q\) trên
        \(\interval{a}{b}\). Không mất tính tổng quát, giả sử \(p < q\). Theo
        \hyperref[thm:mean_value_theorem]{định lí giá trị trung bình}, tồn tại
        \(\xi \in \interval[open]{p}{q}\) sao cho:

        \[g'(\xi) = \frac{g(p) - g(q)}{p - q}\]

        Ta có:

        \[\abs{p - q} = \abs{g(p) - g(q)} = \abs{g'(\xi)} \abs{p - q} < \abs{p - q} \text{ (vô lí)}\]

        Giả thuyết \(g\) có hai điểm bất động trên \(\interval{a}{b}\) sai. Vậy
        với điều kiện ban đầu, chỉ có duy nhất một điểm bất động trên
        \(\interval{a}{b}\).
\end{enumerate}
\end{proof}

\end{document}
