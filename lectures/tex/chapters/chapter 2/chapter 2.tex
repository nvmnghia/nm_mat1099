\documentclass[../../Lectures.tex]{subfiles}

\begin{document}

\chapter{Giải gần đúng phương trình}

\section{Mở đầu}

Sự tăng trưởng của dân số thường có thể được mô hình hóa trong khoảng thời gian
ngắn bằng cách giả định rằng dân số tăng liên tục theo thời gian tỷ lệ thuận với
con số hiện tại vào thời điểm đó. Giả sử \(N(t)\) biểu thị số lượng trong dân số
tại thời điểm \(t\) và \(\lambda\) biểu thị tỷ lệ sinh không 2 đổi của dân số.
Khi đó dân số thỏa mãn phương trình vi phân:

\[\od{N(t)}{t} = \lambda N(t)\]

Nghiệm của phương trình là \(N(t) = N_0 e^{\lambda t}\), ở đây \(N_0\) là dân số
ban đầu.

% TODO: insert chart

Mô hình hàm mũ này chỉ có giá trị khi dân số bị cô lập, không có người nhập cư.
Nếu nhập cư được phép ở tốc độ không đổi \(v\) thì phương trình vi phân trở
thành:

\[\od{N(t)}{t} = \lambda N(t) + v\]

Nghiệm của nó là:

\[N(t) = N_0 e^{\lambda t} + \frac{v}{\lambda} (e^{\lambda t} - 1)\]

Giả sử ban đầu có \(N(0) = \num{1000000}\) người, và có tới \(\num{435000}\)
người nhập cư vào cộng đồng trong năm đầu tiên, vậy \(N(1) = \num{1564000}\)
người có mặt vào cuối năm đầu tiên. Để xác định tỷ lệ sinh của cộng đồng dân số
này, chúng ta cần tìm \(\lambda\) trong phương trình:

\[\num{1564000} = \num{1000000} e^\lambda + \frac{\num{435000}}{\lambda} (e^\lambda - 1)\]

% TODO: Mục 3.3?
Không thể giải một cách chính xác giá trị \(\lambda\) trong phương trình này,
nhưng các phương pháp tính được thảo luận trong chương này có thể được sử dụng
để tính gần đúng nghiệm của các phương trình loại này với độ chính xác cao tùy
ý. Giải pháp cho vấn đề cụ thể này được xem xét trong Bài tập 24 của Mục 3.3.

\section{Phương pháp chia đôi}

Giả sử \(f\) là hàm số xác định và liên tục trên khoảng \(\interval{a}{b}\), với
\(f(a)\) và \(f(b)\) trái dấu. Định lý giá trị trung gian nói rằng tồn tại một
số \(p\) trong \(\interval[open]{a}{b}\) với \(f(p) = 0\).

Mặc dù có thể xảy ra nhiều hơn một nghiệm trong khoảng
\(\interval[open]{a}{b}\), nhưng để thuận lợi trong việc nghiên cứu, chúng ta
giả thiết chỉ có duy nhất một nghiệm trong khoảng này. Khi đó, ta có thể dùng
phương pháp sau:

\begin{method}
\emph{Phương pháp chia đôi (Bisection method)}

Phương pháp này cho phép tìm nghiệm \(p\) của \(f(p) = 0\) trong khoảng
\(\interval{a}{b}\), với \(f(a)\) và \(f(b)\) trái dấu.

Để bắt đầu, ta đặt \(a_1 = a\) và \(b_1 = b\), và đặt \(p_1\) là điểm giữa của
\(\interval{a}{b}\); nghĩa là:

\[p_1 = a_1 + \frac{b_1 - a_1}{2} = \frac{a_1 + b_1}{2}\]

\begin{itemize}
    \item Nếu \(f(p_1) = 0\) thì \(p = p_1\).
    \item Nếu \(f(p_1) \neq 0\) thì \(f(p_1)\) cùng dấu với \(f(a_1)\) hoặc \(f(b_1)\).
        \begin{itemize}
            \item Nếu \(f(p_1)\) cùng dấu với \(f(a_1)\) thì \(p \in \interval{p_1}{b_1}\). Đặt \(a_2 = p_1\), \(b_2 = b_1\).
            \item Nếu \(f(p_1)\) cùng dấu với \(f(b_1)\) thì \(p \in \interval{a_1}{p_1}\). Đặt \(a_2 = a_1\), \(b_2 = p_1\).
        \end{itemize}
        sau đó làm tiếp phương pháp trên với khoảng \(\interval{a_2}{b_2}\).
\end{itemize}
\end{method}

Các cách dừng khác (còn gọi là \emph{tiêu chí dừng}) có thể được áp dụng trong
phương pháp trên hoặc trong bất kỳ các kỹ thuật lặp lại trong chương này. Ví dụ,
chúng ta có thể chọn một dung sai \(\varepsilon > 0\) và tạo dãy \(p_1, ...,
p_N\) cho đến khi đáp ứng một trong các điều kiện sau:

\begin{align}
                      \abs{p_N - p_{N - 1}} &< \varepsilon \text{, } \\
    \frac{\abs{p_N - p_{N - 1}}}{\abs{p_N}} &< \varepsilon \text{, } p_N \neq 0 \text{ hoặc} \label{rel_err_stop_criteria} \\
                               \abs{f(p_N)} &< \varepsilon
\end{align}

% TODO: fix reference to the above equation in the individual file.
Không may, khó khăn có thể phát sinh với bất kỳ tiêu chí dừng nào. Ví dụ, có các
chuỗi \(p_n\) mà hiệu \(p_n - p_{n - 1}\) hội tụ về \num{0} trong khi dãy đó lại
phân kỳ. Cũng có thể có \(f(p_n)\) gần bằng \num{0} trong khi \(p_n\) khác đáng
kể so với \(p\). Nếu không có kiến thức bổ sung về \(f\) hoặc \(p\), bất đẳng
thức \ref{rel_err_stop_criteria} là tiêu chuẩn dừng tốt nhất để áp dụng vì nó
sát nhất với sai số tương đối.

Khi dùng máy tính để tính xấp xỉ, nên thiết lập một giới hạn trên về số lần lặp
lại. Điều này giúp tránh vòng lặp vô hạn, một tình huống có thể phát sinh khi
chuỗi \(p_n\) phân kỳ (và cả khi chương trình sai).

Ta có nhận xét:

\begin{exmp}
    Chứng minh rằng \(f(x) = x^3 + 4x^2 - 10 = 0\) có nghiệm trong khoảng
    \(\interval{1}{2}\), và dùng phương pháp chia đôi để xác định nghiệm đúng
    đến \(10^{-4}\).

    Vì \(f(1) = -5\) và \(f(2) = 14\), \(f(x) = 0\) chắc chắn có nghiệm trong
    khoảng \(\interval{1}{2}\).

    Ta có bảng sau:

    \begin{tabular}{ r S[table-format=1.9] S[table-format=1.9] S[table-format=1.9] S[table-format=-1.6] }    % - for minus sign
        \\
        \toprule
         {\(n\)} &   {\(a_n\)}   &   {\(b_n\)}   &   {\(p_n\)}   & {\(f(p_n)\)} \\
        \midrule
              1  &  1.0          &  2.0          &  1.5          &   2.375      \\
              2  &  1.0          &  1.5          &  1.25         &  -1.79687    \\
              3  &  1.25         &  1.5          &  1.375        &   0.16211    \\
              4  &  1.25         &  1.375        &  1.3125       &  -0.84839    \\
              5  &  1.3125       &  1.375        &  1.34375      &  -0.35098    \\
              6  &  1.34375      &  1.375        &  1.359375     &  -0.09641    \\
              7  &  1.359375     &  1.375        &  1.3671875    &   0.03236    \\
              8  &  1.359375     &  1.3671875    &  1.36328125   &  -0.03215    \\
              9  &  1.36328125   &  1.3671875    &  1.365234375  &   0.000072   \\
             10  &  1.36328125   &  1.365234375  &  1.364257813  &  -0.01605    \\
             11  &  1.364257813  &  1.365234375  &  1.364746094  &  -0.00799    \\
             12  &  1.364746094  &  1.365234375  &  1.364990234  &  -0.00396    \\
             13  &  1.364990234  &  1.365234375  &  1.365112305  &  -0.00194    \\
        \bottomrule                                                             \\    % Without this \\, the next paragraph will be too close to the table.
    \end{tabular}

    Sau 13 lần lặp, \(p_{13} = \num{1.365112305}\) xấp xỉ nghiệm \(p\) với sai số:
    \[\abs{p - p_{13}} < \abs{b_{14} - a_{14}} = \abs{\num{1.365234375} - \num{1.365112305}} = \num{0.000122070}\]

    Do \(\abs{a_{14}} < \abs{p}\) (khoảng đang xét dương), ta có:
    \[\frac{\abs{p - p_{13}}}{\abs{p}} < \frac{\abs{b_{14} - a_{14}}}{\abs{a_{14}}} \leq \num{9e-5}\]

    Cần chú ý rằng, \(p_9\) thực sự gần với \(p\) hơn kết quả cuối cùng
    \(p_{13}\), tuy nhiên khi thực hiện thuật toán ta không thể biết đều này.
    Hơn nữa, \(\abs{f(p_9)} < \abs{f(p_{13})}\) cũng không liên quan đến việc
    \(p_9\) sát với \(p\) hơn.
\end{exmp}

Phương pháp chia đôi có hai điểm yếu lớn:

\begin{itemize}
    \item Cần số vòng lặp \(N\) lớn
    \item Vô tình bỏ qua các xấp xỉ tốt
\end{itemize}

Dù vậy, phương pháp này lại có một ưu điểm lớn là đảm bảo dãy \(p_N\) hội tụ đến
một nghiệm. Do ưu điểm này, phương pháp chia đôi thường được dùng để tìm điểm
bắt đầu cho các phương pháp khác hiệu quả hơn mà sẽ được giới thiệu sau.

\begin{theorem}
    Cho hàm \(f \in \interval{a}{b}\) và \(f(a) \dot f(b) < 0\). Phương pháp
    chia đôi tạo ra một chuỗi \(p_n\) xấp xỉ nghiệm \(p\) của \(f\) với sai số
    như sau:
    \[\abs{p_n - p} \leq \frac{b - a}{2^n} \text{, } n \geq 1\]
\end{theorem}

\begin{proof}
    Với mọi \(n \geq 1\), ta có:
    \[b_n - a_n = \frac{1}{2^{n - 1}} (b - a) \text{ và } p \in \interval[open]{a_n}{b_n}\]

    Do
    \[p_n = \frac{1}{2} (a_n + b_n)\]
    ta suy ra được
    \begin{align*}    % TODO: Make the iff symbol close to the indent.
        &\qquad& \frac{1}{2} (a_n + b_n) - b_n \leq p_n - p &\leq \frac{1}{2} (a_n + b_n) - a_n \\
        \iff&&   \frac{1}{2} (a_n - b_n)       \leq p_n - p &\leq \frac{1}{2} (b_n - a_n)       \\
        \iff&&                                \abs{p_n - p} &\leq \frac{1}{2} (b_n - a_n) = \frac{b - a}{2^n}
    \end{align*}
\end{proof}

\end{document}
