\documentclass[../../Lectures]{subfiles}


\begin{document}

\chapter{Các phương pháp nội suy}


%----------------------------------------------------------------------------------------
%	4.1: Mở đầu & Cơ sở lý thuyết
%----------------------------------------------------------------------------------------

\section{Mở đầu \& Cơ sở lý thuyết}

Chúng ta được biết các giá trị của một hàm số ở những điểm khác nhau \(x_0\),
\(x_1\), \ldots, \(x_n\). Chúng ta mong muốn tính được giá trị xấp xỉ của hàm
tại các điểm \(x\) mới nằm giữa các điểm này mà các giá trị của hàm số được đưa
ra. Quá trình này được gọi là \emph{nội suy}.

Chúng ta nên chú ý đến phần này vì nội suy tạo thành nền móng cho cả hai phần
tính gần đúng đạo hàm và tính gần đúng tích phân. Thật vậy, phép nội suy cho
phép chúng ta phát triển các công thức cho tích phân và đạo hàm bằng số.

Với hàm \(y = f(x)\), ta kí hiệu các giá trị đã biết của \(f\) như sau
\[f_k = f(x_k) \mid k \in \interval{0}{n}\]
hay một cách biểu diễn khác ở dạng cặp
\[(x_k, f_k)\]

Các giá trị hàm số này được lấy từ đâu?
\begin{itemize}
    \item Nó có thể tính được từ một hàm ``toán học'', chẳng hạn hàm logarithm
        hoặc hàm Bessel, hoặc
    \item Nó có thể là số liệu đo được hoặc tự động ghi lại trong một thí nghiệm
        nào đó, chẳng hạn chất lượng không khí của Hà Nội hoặc đo thành phần khí
        thải của ô tô hoặc xe máy khi đi với tốc độ khác nhau
\end{itemize}

Ý tưởng của nội suy là tìm đa thức \(p_n(x)\) có bậc là \(n\) (hoặc nhỏ hơn) thỏa
mãn:
\begin{equation}\label{eq:general_approximation_polynomial}
    p_n(x_k) = x \, \forall k \in \interval{0}{n}
\end{equation}

Chúng ta sẽ gọi \(p_n\) là \emph{đa thức nội suy} và \(x_0\), \(x_1\), \ldots,
\(x_n\) là các điểm mốc. Nếu \(f\) là hàm toán học thì ta gọi \(p_n(x)\) là
\emph{xấp xỉ} (hoặc là xấp xỉ đa thức) của hàm \(f\). Chúng ta sẽ dùng \(p_n\)
để tính giá trị của hàm \(f\) tại các điểm \(x\) khác điểm mốc nằm trong hoặc
ngoài khoảng \(\interval{x_0}{x_n}\).

Lý do đa thức rất thuận tiện để xấp xỉ vì chúng ta có thể dễ dàng lấy đạo hàm và
tích phân chúng, và kết quả cũng là một đa thức. Hơn nữa, \emph{đa thức có thể
xấp xỉ hàm liên tục với bất kỳ độ chính xác tùy ý}. Kết quả này được gọi là
\emph{định lí xấp xỉ Weierstrass}.

\begin{ctheorem}{Định lí xấp xỉ Weierstrass}{weierstrass_approx}
    Với mọi hàm \(f\) xác định trên \(J = \interval{a}{b}\) và một sai số
    \(\beta > 0\) cho trước, luôn tồn tại đa thức \(p_n(x)\) (có bậc \(n\) đủ
    cao) sao cho:
    \[\abs{f(x) - p_n(x)} < \beta \, \forall x \in J\]
\end{ctheorem}

Ta không đi vào chứng minh định lý trên. Ta chú ý đến hai tính chất quan trọng
của các đa thức nội suy \(p_n\):
\begin{itemize}
    \item \(p_n\) tồn tại (đã được khẳng định bởi
        \hyperref[thm:weierstrass_approx]{định lí Weierstrass})
    \item \(p_n\) duy nhất
\end{itemize}

Ta có thể chứng minh tính duy nhất của \(p_n\) như sau:

\begin{proof}[Chứng minh tính duy nhất của đa thức nội suy]\label{proof:polynomial_approx_uniqueness}
    \phantom\\

    Thật vậy, giả sử có đa thức \(q_n\) khác cũng thỏa mãn:
    \begin{gather*}
        q_n(x_k) = x \, \forall k \in \interval{0}{n} \\
        \iff r(x) = p_n(x_k) - q_n(x_k) = 0 \, \forall k \in \interval{0}{n}
    \end{gather*}

    Nói cách khác, \(r\) có đúng \(n + 1\) nghiệm. Nhưng \(r(x)\) là đa thức bậc
    tối đa là \(n\), do đó nó có tối đa \(n\) nghiệm, dẫn đến mâu thuẫn.

    Vậy ta kết luận đa thức nội suy \(p_n\) là duy nhất.
\end{proof}

Sau đây, ta xem xét một số phương pháp tìm \(p_n\). Theo như tính duy nhất đã
chứng minh ở trên, với cùng một số liệu, các phương pháp khác nhau cần phải cho
ra cùng một đa thức. Hơn nữa, các đa thức có thể được biểu diễn dưới các dạng
khác nhau do các mục tiêu khác nhau.


%----------------------------------------------------------------------------------------
%	4.2:  Phương pháp nội suy Lagrange
%----------------------------------------------------------------------------------------

\section{Phương pháp nội suy Lagrange}

\subsection{Phương pháp nội suy Lagrange}

Sau đây ta trình bày việc xây dựng \(p_n\) bằng \emph{phương pháp nội suy
Lagrange}.

\begin{cmethod}{Phương pháp nội suy Lagrange}{lagrange_approx}
    Phương pháp gồm hai bước:
    \begin{enumerate}
        \item Tạo các đa thức \(L_k\) với bậc tối đa là \(n\) theo công thức sau:
            \begin{align*}
                L_k(x) &= \frac{(x - x_0) \ldots (x - x_{k - 1}) (x - x_{k + 1}) \ldots (x - x_n)}{(x_k - x_0) \ldots (x_k - x_{k - 1}) (x_k - x_{k + 1}) \ldots (x_k - x_n)} \\
                       &= \prod_{i = 0 \mid i \neq k}^{n} \frac{x - x_i}{x_k - x_i}
            \end{align*}

            Ý tưởng ở đây là:
            \begin{itemize}
                \item \(L_k(x_k) = 1\)
                \item \(L_k(x_j) = 0 \, \forall j \neq k \in \interval{0}{n} \)
            \end{itemize}

        \item Viết \(p_n\):
            \[p_n(x) = \sum_{k = 0}^{n} L_k(x) \cdot f_k\]

            Ý tưởng ở đây là với mọi \(x_k\), \(p_n\) rút gọn còn \(f_k\), do
            cách xây dựng \(L_k\) ở trên.
    \end{enumerate}
\end{cmethod}

% TODO: examples?

\(p_n\) ở dạng này gọi là \emph{đa thức nội suy Lagrange bậc \(n\)}.

\subsection{Đánh giá sai số}

Ta có định lý sau về sai số của phương pháp nội suy Lagrange, hay nói cách khác,
của \emph{mọi} phương pháp nội suy đa thức:

\begin{ctheorem}{Sai số nội suy đa thức}{polynomial_approx_error}
    Nếu \(f\) khả vi liên tục đến bậc \(n + 1\), thì sai số của mọi phương pháp
    nội suy đa thức xấp xỉ \(f\) được tính theo công thức sau:
    \begin{equation}\label{eq:polynomial_approx_error}
        \epsilon_n(x) = f(x) - p_n(x) = \frac{f^{(n + 1)}(t)}{(n + 1)!} \prod_{i = 0}^{n} (x - x_i)
    \end{equation}
    trong đó, \(t\) là một số nào đó trong khoảng giữa \(x_k\) nhỏ nhất, \(x_k\)
    lớn nhất, và \(x\).
\end{ctheorem}

Ý tưởng của công thức trên đến từ quan sát rằng nếu \(f\) cũng là đa thức, thì
\(f\) chính là \(p_n\), dẫn đến sai số bằng \(0\). Đồng thời, \(f\) lúc này lại
có đạo hàm bậc \(n + 1\) bằng \(0\). Từ hai điều trên, ta kì vọng rằng sai số sẽ
liên quan đến đạo hàm bậc \(n + 1\) của \(f\).

%TODO: proof, example


%----------------------------------------------------------------------------------------
%	4.3:  Phương pháp nội suy Newton
%----------------------------------------------------------------------------------------

\section{Phương pháp nội suy Newton}

Với bộ dữ liệu đã cho, ta đã \hyperref[proof:polynomial_approx_uniqueness]{chứng
minh} đa thức nội suy \(p_n\) là duy nhất. Tuy nhiên, với các yêu cầu khác nhau,
ta có thể dùng \(p_n(x)\) ở các dạng thức khác nhau. Đa thức nội suy
\hyperref[method:lagrange_approx]{dạng Lagrange} thuận lợi cho việc tính gần
đúng giá trị đạo hàm và tích phân số.

Một trường hợp riêng, nhưng rất quan trọng là \emph{đa thức dạng Newton}, dạng
thức này thuận lợi trong việc giải gần đúng phương trình vi phân với giá trị ban
đầu.

% TODO: Ví dụ 3?
Hơn nữa, xây dựng đa thức dạng Newton yêu cầu phải tính toán ít hơn đa thức dạng
Lagrange, đặc biệt khi \(n\) tăng để đạt được độ chính xác cần thiết. Khi đó, đa
thức dạng Newton sử dụng tất cả các kết quả tính toán từ trước đó và cộng thêm
một số hạng khác có thể không được tính từ đa thức dạng Lagrange. Số hạng này
nhận được từ việc ứng dụng Nguyên lý sai số (Được sử dụng trong Ví dụ 3 của đa
thức dạng Lagrange).

Trong hai phần sau, ta sẽ từ từ xây dựng phương pháp nội suy Newton. Phần đầu là
dạng tổng quát (tỉ sai phân), phần sau là một trường hợp đặc biệt hay dùng (sai
phân tiến).

\subsection(Công thức tỉ sai phân)

Giả sử \(p_{n-1}(x)\) là đa thức Newton bậc \((n - 1)\) (dạng của nó ta xác định
sau). Khi đó:
\[p_{n - 1}(x_k) = f_k \, \forall k \in \interval{0}{n - 1}\]

Từ đó, ta có thể viết đa thức Newton bậc \(n\) như sau:
\begin{equation}\label{eq:diff_consecutive_polynomial_approx}
    p_n(x) = p_{n - 1}(x) + g_n(x)
\end{equation}

Do \(p_n\) cũng là một đa thức nội suy, ta có
\[p_{n}(x_k) = f_k \, \forall k \in \interval{0}{n}\]
từ đó ta có một số giá trị của \(g_n\):
\[g_n(x_k) = 0 \, \forall k \in \interval{0}{n - 1}\]

Theo \eqref{eq:diff_consecutive_polynomial_approx}, ta thấy hiển nhiên \(g_n\)
có bậc \(n\).

Từ hai điều trên, ta suy ra được dạng của \(g_n\)
\[g_n(x) = a_n \prod_{i = 0}^{n - 1} (x - x_i)\]

Phần việc còn lại là đi tìm \(a_n\). Ta dễ dàng tính được \(a_n\) khi xét \(x =
x_n\):
\begin{equation}\label{eq:divided_difference_incremental_form}
    a_n = \frac{f_n - p_{n - 1}(x_n)}{\prod_{i = 0}^{n - 1} (x_n - x_i)}
\end{equation}

Tổng quát hóa, ta có công thức cho \(a_k\) như sau:
\[a_k = f[x_0, \ldots, x_k] = \frac{f[x_1, \ldots, x_k] - f[x_0, \ldots, x_{k - 1}]}{x_k - x_0} \, \forall k > 0\]
với
\[a_1 = f[x_0, x_1] = \frac{f_1 - f_0}{x_1 - x_0}\]

Công thức đệ quy trên được gọi là \emph{tỉ sai phân thứ \(k\) (\(k^{th}\)
divided difference)}.

Tổng hợp lại, ta có công thức tỉ sai phân của Newton:
\begin{equation}\label{eq:divided_difference}
    \begin{aligned}
        p_n(x) &= f_0 + a_1 \prod_{i = 0}^{0} (x - x_i) + a_2 \prod_{i = 0}^{1} (x - x_i) + \ldots + a_n \prod_{i = 0}^{n - 1} (x - x_i) \\
               &= f_0 + \sum_{j = 1}^{n}   \left[ a_j \prod_{i = 0}^{j - 1} (x - x_i) \right]
    \end{aligned}
\end{equation}

Khi làm bài tập, ta thường dùng công thức
\eqref{eq:divided_difference_incremental_form}, như trình bày trong ví dụ sau:

%TODO: example...

\subsection{Công thức sai phân tiến}

\emph{Sai phân tiến (forward difference)} là một trường hợp đặc biệt của
\hyperref[eq:divided_difference]{công thức tỉ sai phân}, trong đó các mốc nội
suy cách đều. Ta sẽ thấy các công thức được đơn giản đi đáng kể.

Trước hết ta nêu ra dạng của \(x\):
\[x_n = x_0 + nh\]

Ta định nghĩa công thức cho \(\Delta^1\), tức \emph{sai phân tiến thứ nhất
(\(1^{st}\) forward difference)} của \(f\) tại \(x_j\) như sau:
\[\Delta^1 f_j = f_{j + 1} - f_j\]

Ta tiếp tục có sai phân tiến thứ hai, cũng tại \(x_j\):
\[\Delta^2 f_j = \Delta^1 f_{j + 1} - \Delta^1 f_j\]

Tổng quát, công thức sai phân tiến thứ \(k\) tại \(x_j\) được định nghĩa đệ quy
như sau:
\[\Delta^k f_j = f_{j + 1} - f_j\]

Với cách định nghĩa sai phân tiến như trên, ta có thể chứng minh được, bằng quy
nạp:
\[a_k = f[x_0, \ldots, x_k] = \frac{1}{k! h^k} \Delta^k f_0\]

Áp dụng công thức trên vào \eqref{eq:divided_difference}, ta có:
\[p_n(x) = \sum_{s = 0}^{n} \binom{r}{s} \Delta^s f_0\]
với \(x = x_0 + rh\)

Đồng thời, \eqref{eq:polynomial_approx_error} cũng được đơn giản hóa:
\[\epsilon_n(x) = \frac{h^{n + 1}}{(n + 1)!} r(r - 1) \ldots (r - n) f^{n + 1}(t)\]

\end{document}
