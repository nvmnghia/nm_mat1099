\documentclass[../../Lectures.tex]{subfiles}


\begin{document}

\chapter{Giải tích sai số}

\section{Giới thiệu về phép tính gần đúng}

\subsection{Một số ví dụ về tính toán khoa học và phương pháp tính}

\subsubsection{Lý do nghiên cứu Phương pháp tính}

\begin{itemize}
    \item Làm viêc với các số gần đúng
    \item Giải gần đúng các phương trình và hệ phương trình
    \item Xấp xỉ hàm số: Phương pháp nội suy, phương pháp xấp xỉ hàm số, chuỗi
        Taylor hoặc chuỗi Maclaurin
    \item Số học IEEE    % https://tex.stackexchange.com/questions/55105/when-should-i-use-intersentence-spacing
\end{itemize}

\subsubsection{Các nhiệm vụ}

\begin{itemize}
    \item Tìm hiểu và ứng dụng các thuật toán.
    \item Thể hiện các thuật toán bằng các chương trình.
    \item Tìm các bài toán thực tiễn.
\end{itemize}

Trong thực tế chúng ta thường phải xử lý, tính toán với các đại lượng gần đúng
như các số đo vật lý, các dữ liệu ban đầu, các số làm tròn…với sai số nào đó,
tức là các số gần đúng. Việc ước lượng sai số hợp lý cho phép ta đánh giá được
chất lượng của quá trình tính toán, quyết định số chữ số giữ lại trong các phép
tính trung gian và trong kết quả. Vì vậy, trước tiên ta cần nghiên cứu về các
phép tính gần đúng và sai số.

\section{Số gần đúng, sai số tuyệt đối và tương đối}

\subsection{Sai số tuyệt đối}

Nếu số gần đúng \(a\) có giá trị đúng là \(a_0\) thì ta nói \(a\) xấp xỉ \(a_0\)
hay \(a\) là số gần đúng của \(a_0\). Khi đó sai số của \(a\) là:

\begin{equation} \label{eq:1}
    E_a = a - a_0
\end{equation}

Nhưng giá trị này nói chúng ta không biết được mà chỉ ước lượng được cận trên
của trị tuyệt đối của nó.

\begin{definition}
    Giá trị ước lượng \(\Delta a\) sao cho
    \begin{equation} \label{eq:2}
        \abs{a - a_0} \leq \Delta a
    \end{equation}
    được gọi là \emph{sai số tuyệt đối} của số gần đúng \(a\).
\end{definition}

% TODO: Mục 2.1?
Sai số tuyệt đối nhỏ nhất có thể biết được gọi là sai số tuyệt đối giới hạn của
\(a\). Thông thường ước lượng sai số tuyệt đối giới hạn là khó và nhiều khi
không cần thiết nên người ta chỉ cần ước lượng sai số tuyệt đối đủ nhỏ và dùng
từ 1 đến 3 chữ số có nghĩa (là số chữ số bắt đầu từ chữ số khác không đầu tiên
từ trái sang phải - xem mục 2.1) để biểu diễn sai số tuyệt đối của số gần đúng.

Thay cho~\ref{eq:2} người ta còn dùng cách biểu diễn sau để chỉ sai số tuyệt đối
của \(a\):

\begin{equation} \label{eq:3}
    a_0 = a \pm \Delta a
\end{equation}

Trong thực tế thì sai số \(E_a\) không thể biết được nên khi không có sự hiểu
lầm người ta còn dùng từ \emph{sai số} để chỉ sai số tuyệt đối \(E_a\).

\begin{exmp}\label{exmp:room_example}
    Căn phòng có chiều dài \(d = \SI{5.45}{m}\) và chiều rộng \(r =
    \SI{3.94}{m}\) với sai số \SI{1}{cm}.

    Khi đó ta hiểu là:

    \begin{gather*}
        \Delta d = \SI{0.01}{m} \text{ hay } d = \SI[separate-uncertainty=true]{5.45 \pm 0.01}{m} \\
        \Delta r = \SI{0.01}{m} \text{ hay } r = \SI[separate-uncertainty=true]{3.97 \pm 0.01}{m}
    \end{gather*}

    Như vậy diện tích của phòng được ước lượng bởi:

    \[S = d \cdot r = \num{5.45} \cdot \num{3.94} = \SI{21.473}{m^2}\]

    với cận trên và cận dưới của \(S\) là:

    \[(\num{5.45} - \num{0.01})(\num{3.94} - \num{0.01}) = \num{21.3792} \leq S \leq (\num{5.45} + \num{0.01})(\num{3.94} + \num{0.01}) = \num{21.567}\]

    Vậy ta có ước lượng sai số tuyệt đối của S là:

    \[\abs{S - S_0} \leq \SI{0.094}{m^2}\]
\end{exmp}

\subsection{Sai số tương đối}

Hai số gần đúng có cùng sai số tuyệt đối sẽ có ``mức độ chính xác'' khác nhau nếu
độ lớn của chúng khác nhau: số bé hơn sẽ có độ chính xác kém hơn. Để biểu diễn
độ chính xác này, người ta dùng sai số tương đối.

\begin{definition}
    \emph{Sai số tương đối} của số gần đúng \(a\) là tỷ số giữa sai số tuyệt đối
    và giá trị tuyệt đối của nó, được ký hiệu là \(\delta a\).

    \begin{equation} \label{eq:4}
        \delta a = \frac{\delta a}{\abs{a}}
    \end{equation}
\end{definition}

Thường sai số tương đối được biểu diễn dưới dạng phần trăm với 2 hoặc 3 chữ số.

Từ~\ref{eq:4} ta thấy nếu biết \(\delta a\) thì:

\begin{equation} \label{eq:5}
    \Delta a = \abs{a}\delta a
\end{equation}

nên ta chỉ cần biết một trong hai loại sai số của nó là được.

\begin{exmp}
    Nếu \(a = \num{57}\) và \(\Delta a = \num{0.5}\) thì \(\delta a =
    \num{0.0087719}\) hoặc \SI{0.88}{\percent} (gọn hơn là \SI{0.9}{\percent}).
\end{exmp}

\subsection{Các loại sai số khác}

Để hình dung các loại sai số khác ta xét ví dụ sau:

\begin{exmp}
    Một vật thể rơi từ độ cao \(H_0\) với vận tốc ban đầu \(v_0\) (được đo nhờ
    thiết bị nào đó). Tính độ cao \(H(t)\) của vật thể sau thời gian \(t\). Bài
    toán có thể giải như sau:

    Nếu gọi ngoại lực tác động vào vật thể là \(F(t)\) (gồm lực hút trọng trường
    và lực cản), khối lượng vật thể là \(m\) thì \(H(t)\) là nghiệm của phương
    trình vi phân cấp hai

    \begin{equation} \label{eq:6}
        H''(x) = \frac{-F(t)}{m}
    \end{equation}

    với điều kiện ban đầu \(H(0) = H_0\) và \(H'(0) = -v_0\).

    Ta chọn một phương pháp gần đúng để giải phương trình này, chẳng hạn nếu giả
    thiết \(\dfrac{F(t)}{m}\) không đổi thì

    \[H(t) = H_0 - g\frac{t^2}{2} -v_0 t\]
\end{exmp}

Qua ví dụ trên ta thấy sai số của kết quả nhận được chịu ảnh hưởng của:

\begin{itemize}
    \item các số đo \(H_0\), \(v_0\)
    \item cách lập luận để xác định \(F(t)\)
    \item phương pháp giải phương trình~\ref{eq:6}
    \item và các yếu tố ngẫu nhiên khác
\end{itemize}

Theo các yếu tố ảnh hưởng tới kết quả tính toán ta phân ra các loại sai số sau:

\begin{itemize}
    \item \emph{Sai số dữ liệu}: Còn gọi là sai số của số liệu ban đầu. Trong
        thí dụ trên là sai số khi đo \(H_0\) và \(v_0\).
    \item \emph{Sai số giả thiết}: Sai số này gặp phải khi ta đơn giản hoá bài
        toán thực tiễn để thiết lập mô hình toán học có thể giải được. Trong thí
        dụ trên có thể giả thiết ngoại lực chỉ là trọng lực.
    \item \emph{Sai số phương pháp}: Là sai số của phương pháp giải gần đúng bài
        toán theo mô hình được lập. Trong thí dụ trên là phương pháp giải phương
        trình vi phân~\ref{eq:6}.
    \item \emph{Sai số tính toán}: Là sai số tích luỹ trong quá trình tính toán
        theo phương pháp được chọn.
    \item \emph{Sai số làm tròn}: Khi tính toán ta thường phải làm tròn các số
        nên ảnh hưởng tới kết quả nhiều khi rất đáng kể.
    \item \emph{Sai số ngẫu nhiên}: Là sai số chịu các quy luật chi phối ngẫu
        nhiên không tránh được.
\end{itemize}

Về sau ta  quan tâm tới sai số tính toán và sai số phương pháp.

\section{Sai số tích lũy và các bài toán sai số}

\subsection{Sai số hàm một biến}

Cho hàm số \(y = f(x)\) và \(x\) là số gần đúng của \(x_0\). Ký hiệu \(\Delta
x\) và \(\Delta y\) là sai số tuyệt đối tương ứng của đối số và hàm số. Ta sẽ
xét các bài toán ước lượng sai số của hàm hoặc của đối số khi biết một trong hai
sai số.

\subsubsection{Bài toán thuận}

Bài toán này ta ước lượng \(\Delta y\) khi biết \(x\) và \(\Delta x\).

Theo công thức số gia hữu hạn ta có:

\[\abs{y - y_0} = \abs{f'(c)}\abs{x - x_0}\]

ở đây \(y_0\) là giá trị đúng của \(y\) và \(c\) là điểm thuộc miền \((x, x_0)\)
nếu \(x < x_0\) và thuộc \((x_0, x)\) nếu \(x_0 < x\).

Khi \(\Delta x\) bé, \(x\) gần \(x_0\) ta có ước lượng:

\begin{equation} \label{eq:7}
    \begin{aligned}
                    &\Delta y \approx \abs{f'(x)}\abs{x - x_0} \\
        \text{hay } &\Delta y \leq \abs{f'(x)}\Delta x
    \end{aligned}
\end{equation}

\begin{exmp}
    Cho \(y = \ln{x}\) ta có ước lượng:

    \[\Delta(\ln{x}) = \frac{1}{x} \Delta x = \delta x\]
\end{exmp}

\subsubsection{Bài toán ngược}

Trong bài toán này, ta biết giá trị gần đúng \(x\), ta cần xác định phải tính
\(x\) với \(\Delta x\) là bao nhiêu để đảm bảo \(\Delta y \leq \Delta\). Với giá
trị \(\Delta\) cho trước, từ công thức~\ref{eq:7} ta thấy nếu

\begin{equation} \label{eq:approx_of_relative_error}
    \Delta x \leq \frac{\Delta}{\abs{f'(x)}}
\end{equation}

thì đủ để \(\Delta y \leq \Delta\).

\begin{exmp}
    \(y = e^x\) với \(x \approx 3\) để có \(\Delta y \leq \num{0.01}\) ta tính
    \(x\) với \(\Delta x \leq \dfrac{\num{0.01}}{e^3}\) là đủ.
\end{exmp}

% It should be a subsubsection as per the original doc, but this is more readable.
\subsection{Sai số qua các phép toán số học}

Khi tính toán với các số gần đúng thì sai số sẽ tích luỹ qua các phép toán cơ
bản. Sau đây ta ước lượng sai số khi cộng trừ, nhân chia các số gần đúng.

\subsubsection{Sai số của tổng hoặc hiệu}

\begin{proposition}\label{prop:error_of_sum}
    Sai số tuyệt đối của một tổng hoặc hiệu bằng tổng các sai số tuyệt đối thành
    phần.
\end{proposition}

\begin{proof}
    Để đơn giản ta xét \(u = a \pm b\) với các số \(a\), \(b\) có giá trị đúng
    \(a_0\), \(b_0\) và sai số tuyệt đối \(\Delta a\), \(\Delta b\) tương ứng.
    Trường hợp có nhiều số hạng được xét tương tự.

    Khi đó ta có:

    \[\begin{cases}
        a_0 - \Delta a \leq a \leq a_0 + \Delta a \\
        b_0 - \Delta b \leq b \leq b_0 + \Delta b
    \end{cases}\]

    Do đó ta có:

    \[\begin{cases}
        a_0 + b_0 - (\Delta a + \Delta b) \leq a + b \leq a_0 + b_0 + (\Delta a + \Delta b) \\
        a_0 - b_0 - (\Delta a + \Delta b) \leq a - b \leq a_0 - b_0 + (\Delta a + \Delta b)
    \end{cases}\]

    Nên

    \[a_0 \pm b_0 - (\Delta a + \Delta b) \leq a \pm b \leq a_0 \pm b_0 + (\Delta a + \Delta b)\]
\end{proof}

\begin{exmp}
    Cho \(a = \num{50.5}\), \(b = \num{50.9}\) với \(\Delta a = \Delta b =
    \num{0.05}\) và \(u = a - b\).

    Ta có \(u = \num{0.4}\) với \(\Delta u = \num{0.05} + \num{0.05} =
    \num{0.1}\).

    Vậy \(\delta u = \dfrac{\num{0.1}}{\num{0.4}} = \SI{25}{\percent}\).
\end{exmp}

\subsubsection{Sai số của tích hoặc thương}

\begin{proposition}
    Sai số tương đối của tích hoặc thương bằng tổng các sai số tương đối thành
    phần.
\end{proposition}

\begin{proof}
    Xét

    \[u = \frac{x_1 \ldots x_m}{y_1 \ldots y_n}\]

    Ta có thể giả thiết các \(x_i\) và \(y_j\) đều dương. Khi đó ta có:

    \[\ln{u} = \sum_{i=1}^{m} \ln{x_i} + \sum_{j=1}^{n} \ln{y_j}\]

    Theo mệnh đề~\ref{prop:error_of_sum} ở trên, ta suy ra:

    \[\delta u = \sum_{i=1}^{m} \delta x_i + \sum_{j=1}^{n} \delta y_j\]
\end{proof}

\begin{exmp}
    Xét \(S = d \cdot r\) như ở ví dụ~\ref{exmp:room_example} \(d =
    \num{5.45}\), \(r = \num{3.94}\), \(\Delta d = \Delta r = \num{0.01}\).

    Ta có:

    \begin{align*}
        \delta d &= \num{0.001835} \\
        \delta r &= \num{0.002538} \\
        \delta S &= \num{0.004373} \text{ nên } \Delta S = \num{0.094}
    \end{align*}
\end{exmp}

\subsection{Sai số hàm nhiều biến}

Ta xét hàm nhiều biến \(u = f(x_1, \ldots, x_n)\) với giá trị gần đúng \(x_1,
\ldots, x_n\) và \(y\) đã biết.

\subsubsection{Bài toán thuận}

Trong bài toán này, ta cần ước lượng sai số \(\Delta y\) khi biết \(\delta x_i
\, \forall i \leq n\).    % \, kerning magic. https://tex.stackexchange.com/a/101028/206774

Tương tự hàm một biến, sử dụng công thức số gia hữu hạn ta có ước lượng:
\begin{equation} \label{eq:absolute_error_of_function}
    \Delta u = \sum_{i=1}^{n} \abs{f'_i(x_1, \ldots, x_n)} \Delta x_i
\end{equation}
với \(f'_i\) là đạo hàm riêng của \(u\) theo biến \(x_i\).

\begin{exmp}
    Xét \(u = a^2 b\) với \(a = \num{2.0}\), \(b = \num{25.0}\), \(\Delta a =
    \Delta b = \num{0.1}\).

    Ta có:

    \begin{align*}
               u &= 100 \\
        \Delta u &= 2ab \Delta a + a^2 \Delta b \\
                 &= 2 \cdot \num{2.0} \cdot \num{25.0} \cdot \num{0.1} + \num{2.0}^2 \cdot \num{0.1} \\
                 &= \num{10.4}
    \end{align*}
\end{exmp}

\subsubsection{Bài toán ngược}

Bây giờ ta đã biết các số gần đúng \(x_i\), ta phải tính chúng với sai số tuyệt
đối như thế nào để có \(\Delta y \leq \Delta\); ở đây \(\Delta\) là số cho
trước.

Các phương pháp xử lý bài toán này đều dựa trên công thức~\ref{eq:absolute_error_of_function}
một cách linh hoạt. Sau đây ta xét hai phương pháp thông dụng.

\begin{enumerate}
    \item Sai số của  đối số như nhau.
        Ta xét khi:

        \[\Delta x_k = \Delta x \, \forall k \leq n\]

        Từ~\ref{eq:approx_of_relative_error} ta có:

        \[\Delta u = \sum_{i=1}^{n} \abs{f'_i(x_1, \ldots, x_n)} \Delta x\]

        Vậy để cho \(\Delta u \leq \Delta\) thì chỉ cần:

        \begin{equation}\label{eq:same_error}
            \Delta x \leq \frac{\Delta}{\sum_{i=1}^{n} \abs{f'_i(x_1, \ldots, x_n)}}
        \end{equation}

        là đủ.

    \item Phân bố đều sai số.
        Bây giờ ta xét khi:

        \[\abs{f'_i(x_1, \ldots, x_n)} \Delta x_i = \abs{f'_k(x_1, \ldots, x_n)} \Delta x_k \, \forall i, k \leq n\]

        Khi đó \(\forall j \leq n\), từ~\ref{eq:absolute_error_of_function} ta có:

        \[\Delta u = n \abs{f'_i(x_1, \ldots, x_n)} \Delta x_j\]

        Vậy để cho \(\Delta u \leq \Delta\) thì chỉ cần tính:

        \begin{equation}\label{eq:evenly_distributed_error}
            \Delta x_j \leq \frac{\Delta}{n \abs{f'_i(x_1, \ldots, x_n)}} \, \forall j = 1, \ldots, n
        \end{equation}
\end{enumerate}

\begin{exmp}
    Mảnh vườn có cạnh \(d \approx \SI{45.0}{m}\) và \(r \approx
    \SI{20.0}{m}\). Cần tính \(d\) và \(r\) với \(\Delta d\), \(\Delta r\) như
    thế nào để \(\Delta S \leq \SI{0.1}{m^2}\).

    \emph{Cách 1.}
    Xét \(\Delta d = \Delta r = \Delta x\), ta áp dụng~\ref{eq:same_error}:

    \[\Delta x \leq \frac{\num{0.1}}{45 + 20} = \SI{0.0015}{m}\]

    \emph{Cách 2.}
    Khi đo chiều dài thường có sai số lớn hơn chiều rộng nên ta có thể
    dùng~\ref{eq:evenly_distributed_error}.

    \[
        \Delta d \leq \frac{\num{0.1}}{\num{2.20}} = \SI{0.0025}{m} \, \text{, }
        \Delta r \leq \frac{\num{0.1}}{\num{2.45}} = \SI{0.0010}{m}
    \]

    là đủ để \(\Delta S \leq \SI{0.1}{m^2}\)
\end{exmp}

\section[Sai số quy tròn, quan hệ sai số \& số chữ số chắc]{Sai số qui tròn, quan hệ giữa sai số và số chữ số đáng tin}

Trong mục này ta xét các số được biểu diễn dưới dạng thập phân. Khi các số là
gần đúng, vấn đề đặt ra là nên biểu diễn chúng với bao nhiêu chữ số? Thu gọn
chúng như thế nào?

\subsection{Chữ số có nghĩa}

\begin{definition}
    Trong biểu diễn theo cơ số \(b\) (trường hợp riêng là biểu diễn thập phân):

    \begin{itemize}
        \item các chữ số kể từ chữ số khác \(0\) đầu tiên tính từ trái sang phải
            gọi là \emph{các chữ số có nghĩa},

        \item các chữ số \(0\) bên trái là \emph{không có nghĩa}
    \end{itemize}
\end{definition}

Nếu \(a\) được viết dưới dạng

\begin{equation} \label{eq:number_in_decimal}
    a = \sum_{k=p}^{n} a_k10^k
\end{equation}

thì các chữ số \(0\) bên trái không có ở biểu diễn này, ý nghĩa của các chữ số
\(0\) bên phải liên quan tới cách biểu diễn số gần đúng sẽ xét dưới đây.

\begin{exmp}
    Số \(a = 03.4050\) thì chữ số \(0\) đầu không có nghĩa (người ta có thể điền
    để tránh viết thêm) còn các chữ số \(3\), \(4\), \(0\), \(5\), \(0\) là có
    nghĩa.

    Số \(b = 0.034\) thì các chữ số \(3\), \(4\) là có nghĩa, hai chữ số \(0\)
    bên trái không có nghĩa vì nếu biểu diễn theo dạng~\ref{eq:number_in_decimal}
    thì các chữ số này không cần đến.
\end{exmp}

\subsection{Chữ số chắc}

\begin{definition}
    Xét a có biểu diễn~\ref{eq:number_in_decimal} với sai số \(\Delta a\).

    \begin{itemize}
        \item Nếu \(\Delta a \leq \num{0.5} \cdot 10^m\) thì \(a_k\) là
            \emph{chữ số chắc (đáng tin)} \(\forall k \geq m\) (theo nghĩa hẹp
            dùng trong tính toán).

        \item Nếu \(\num{0.5} \cdot 10^m \leq \Delta a \leq 10^m\) thì \(a_m\)
            là \emph{chắc theo nghĩa rộng}.
    \end{itemize}
\end{definition}

\begin{exmp}
    \(a = \num{21.473}\) và \(\Delta a = \num{0.094} = \num{0.94} \cdot
    10^{-1}\) thì:

    \begin{itemize}
        \item Các chữ số \(2\), \(1\) là chắc theo nghĩa hẹp.
        \item Chữ số \(4\) là chắc theo nghĩa rộng.
        \item Các chữ số \(7\), \(3\) là không đáng tin hay không chắc.
    \end{itemize}
\end{exmp}

Khi cho số gần đúng ta có thể cho theo hai cách:

\begin{itemize}
    \item \emph{Cách 1:} Viết kèm với sai số tuyệt đối.
    \item \emph{Cách 2:} Chỉ viết các chữ số chắc. Nếu ta có số gần đúng mà
        không cho sai số thì luôn ngầm hiểu các chữ số có nghĩa là các chữ số
        chắc. Như vậy các chữ số 0 ở bên phải cho ta biết nó là chữ số chắc.
\end{itemize}

Trong quá trình tính toán, người ta thường để lại vài chữ số không chắc và trong
kết quả thì giữ lại các chữ số chắc theo nghĩa rộng.

\subsection{Số thu gọn}

Khi số \(a\) có nhiều chữ số không chắc hoặc có quá nhiều chữ số có nghĩa thì
người ta thường thu gọn thành số \(\overline{a}\) có ít chữ số có nghĩa hơn. Nếu
\(a\) có biểu diễn~\ref{eq:number_in_decimal} và số thu gọn được giữ lại đến
\(a_m\) (\(m > p\)) thì \(\overline{a}\) có biểu diễn

\begin{equation}
    \overline{a} = \sum_{k=m}^{n} b_k 10^k
\end{equation}

nhờ bỏ đi các chữ số \(a_k\) (\(k < m\)) theo quy tắc sau:

\begin{rule_of_thumb}
Quy tắc chữ số chẵn: Giả sử \(a > 0\) và phần bỏ đi là \mu.

\begin{itemize}
    \item Nếu \(\mu < \num{0.5} \cdot 10^m\) thì

        \begin{equation}\label{eq:removed_part_smaller}
            \overline{a} = \sum_{k=m}^{n} a_k 10^k
        \end{equation}

        nghĩa là ta giữ nguyên các chữ số đến hàng \(m\) tính từ trái sang phải.

    \item Nếu \(\mu > \num{0.5} \cdot 10^m\) thì

        \begin{equation}\label{eq:removed_part_larger}
            \overline{a} = \sum_{k=m}^{n} a_k 10^k + 10^m
        \end{equation}

    \item Nếu \(\mu = \num{0.5} \cdot 10^m\), ta xét tiếp:

        \begin{itemize}
            \item Nếu \(a_m\) chẵn, làm theo~\ref{eq:removed_part_smaller}.
            \item Nếu \(a_m\) lẻ, làm theo~\ref{eq:removed_part_larger}.
        \end{itemize}
\end{itemize}

Khi \(a < 0\) ta thu gọn giá trị tuyệt đối và giữ nguyên dấu.
\end{rule_of_thumb}

Khi thu gọn \(a\) thành \(\overline{a}\) ta có sai số thu gọn \(\Gamma_a \leq
\num{0.5} \cdot 10^m\). Để nó ít ảnh hưởng tới sai số tuyệt đối, ta thu gọn số
và giữ lại một hoặc hai chữ số không chắc.

Nếu a có biểu diễn~\ref{eq:number_in_decimal} và \(a_k\) chắc với \(k \geq m\)
thì \(\Delta a \leq 10^m\) nên

\begin{align*}
    % TODO: Check this carefully.
    \delta_a = \frac{\Delta a}{\abs{a}} \leq \frac{10^m}{\sum\limits_{k=m}\limits^{n} a_k 10^k} &= \frac{1}{\sum\limits_{k=0}\limits^{n} a_{k+m} 10^k}
    % \delta_a = \frac{\Delta a}{\abs{a}} \leq \frac{10^m}{\sum\limits_{k=m}\limits^{n} a_k 10^k} &= \frac{1}{\sum\limits_{k=m}\limits^{n} a_k 10^{k - m}} \\
    %                                                                                             &= \frac{1}{\sum\limits_{k=0}\limits^{n} a_{k + m} 10^k}
\end{align*}

Như vậy sai số tương đối của số gần đúng có thể ước lượng bởi nghịch đảo của số
gồm các chữ số chắc của \(a\) không có dấu phẩy.

\subsection{Dấu phẩy động}

Chúng ta biết rằng trong biểu diễn thập phân, mọi số thực được biểu diễn bởi một
hữu hạn hoặc một dãy vô hạn các chữ số thập phân.

Bây giờ hầu hết các máy tính có hai cách biểu diễn số, được gọi là dấu phẩy tĩnh
và dấu phẩy động.

Trong một biểu diễn \emph{dấu phẩy tĩnh} tất cả các số được được đưa ra với một
số cố định các số thập phân sau dấu thập phân; ví dụ, số được đưa ra với 3 số
thập phân là \num{62.358}, \num{0.014}, \num{1.000}. Trong một văn bản chúng ta
sẽ viết, nói, biểu diễn 3 số thập phân là biểu diễn \emph{3D}.

Biểu diễn dấu phẩy tĩnh có ưu thế:

\begin{itemize}
    \item Thuận tiện trong tính toán hàng ngày,
    \item và trong tính toán với các số gần đơn vị
\end{itemize}

Trong hệ thống \emph{dấu phẩy động}, chúng sẽ ta viết, ví dụ,

\[\num{0.6247} \cdot 10^3; \num{0.1735} \cdot 10^{-13}; \num{-0.2000} \cdot 10^{-1}\]

hay đôi khi ta biểu diễn:

\[\num{6.247} \cdot 10^2; \num{1.735} \cdot 10^{-14}; \num{-2.000} \cdot 10^{-2}\]

Chúng ta thấy rằng trong biểu diễn này số các chữ số có nghĩa được giữ cố định,
trong khi dấu phẩy là ``động''. Ở đây, một chữ số có nghĩa của một số \(c\) là
chữ số bất kỳ của \(c\), ngoại trừ chữ số \(0\) nằm ở bên trái chữ số khác \(0\)
đầu tiên; các chữ số \(0\) này chỉ để xác định vị trí của dấu phẩy (như vậy, bất
kỳ chữ \(0\) khác đều là chữ số có nghĩa của \(c\)).

\begin{exmp}
    \[\num{13600}; \num{1.3600}; \num{0.0013600}\]

    tất cả đều có 5 chữ số có nghĩa.
\end{exmp}

Trong văn bản chúng ta nói rút gọn một số đến 5 chữ số có nghĩa là \emph{5S}.

Việc sử dụng số mũ cho phép chúng ta biểu diễn số rất lớn và rất nhỏ. Thật vậy,
về mặt lý thuyết số khác 0 bất kỳ \(a\) có thể được viết như sau:

\[a = \pm m 10^n \mid 0.1 \leq \abs{m} < 1, n \in \mathbb{Z}\]

\begin{rule_of_thumb}
    Để làm tròn số \(x\) đến \(k\) chữ số sau dấu phẩy, ta cộng vào \(x\) lượng
    \(\num{0.5} \cdot 10^{-(k+1)}\) và bỏ đi từ chữ số thứ \(k+1\) sau dấu phẩy
    trở đi.
\end{rule_of_thumb}

\subsubsection{Số học IEEE}

Trong máy tính hiện đại người ta dùng các số nhị phân, ở đây \(m\) được giới hạn
bởi \(k\) chữ số nhị phân (ví dụ, \(k = 8\)) và \(n\) cũng được giới hạn, ta có
biểu diễn sau (chỉ biểu diễn được một số hữu hạn số):
\begin{equation}
    \overline{a} = \overline{m} \cdot 2^n \mid \overline{m} = 0,d_1d_2\ldots d_k, d_1 > 0
\end{equation}

Các số \(\overline{a}\) ở đây được gọi là số máy nhị phân \(k\) chữ số. Phần sau
dấu phẩy \(m\) (được gọi là \emph{mantissa}), biểu diễn các chữ số có nghĩa của
\(\overline{a}\), \(n\) được gọi là lũy thừa của \(\overline{a}\).

\end{document}
