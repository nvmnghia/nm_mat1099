\documentclass[../../Lectures]{subfiles}


\begin{document}

\chapter{Giải hệ phương trình}

Chúng ta có 2 loại hệ phương trình:

\begin{itemize}
    \item Hệ phương trình tuyến tính
    \item Hệ phương trình phi tuyến
\end{itemize}

Chúng ta đã biết các phương pháp giải trực tiếp và gián tiếp Hệ phương trình
tuyến tính:

\begin{itemize}
    \item Các phương pháp giải trực tiếp:

        \begin{itemize}
            \item Phương pháp Cramer
            \item Phương pháp thế
            \item Phương pháp sử dụng ma trận nghịch đảo
            \item Phương pháp khử Gauss, Gauss-Jordan
        \end{itemize}

    \item Các phương pháp giải gián tiếp

        \begin{itemize}
            \item Phương pháp lặp đơn
            \item Phương pháp lặp Seidel
        \end{itemize}

    \item Các phương pháp tìm trị riêng và véc tơ riêng của ma trận
\end{itemize}


%----------------------------------------------------------------------------------------
%	3.1: Đặt bài toán và phương pháp giải
%----------------------------------------------------------------------------------------

\section{Đặt bài toán và phương pháp giải}

Hệ phương trình tuyến tính \(n\) phương trình, \(n\) ẩn \(x_1\), \(x_2\),
\ldots, \(x_n\) là tập \(n\) phương trình \(E_1\), \(E_2\), \ldots , \(E_n\)
dạng

\[\label{eq:general_system_equations}
    \begin{cases}
        E_1 \text{ : } a_{11} x_1 + a_{12} x_2 + \ldots + a_{1n} x_n = b_1 \\
        E_2 \text{ : } a_{21} x_1 + a_{22} x_2 + \ldots + a_{2n} x_n = b_2 \\
        \ldots \\
        E_n \text{ : } a_{n1} x_1 + a_{n2} x_2 + \ldots + a_{nn} x_n = b_n \\
    \end{cases}
\]

\noindent với các hệ số \(a_{jk}\) và \(b_j\) đã biết. Hệ được gọi là thuần nhất
nếu các \(b_j\) bằng \(0\), trong trường hợp ngược lại được gọi là hệ không
thuần nhất.

Dùng cách biểu diễn ma trận, ta có thể viết hệ
\eqref{eq:general_system_equations} dưới dạng:

\begin{equation}
    \bm{A x} = \bm{b}
\end{equation}

Ở đây, ma trận hệ số \(A = [a_{jk}]\) là ma trận vuông cấp \(n\), \(x\) và \(b\)
là các véc tơ cột:

\[
    \bm{A} =
        \begin{bmatrix}
            a_{11}  &  a_{12}  &  \dots   &  a_{1n}  \\
            a_{21}  &  a_{22}  &  \dots   &  a_{2n}  \\
            \vdots  &  \vdots  &  \ddots  &  \vdots  \\
            a_{n1}  &  a_{n2}  &  \dots   &  a_{nn}  \\
        \end{bmatrix} \quad
    \bm{x} =
        \begin{bmatrix}
            x_1     \\
            x_2     \\
            \vdots  \\
            x_n     \\
        \end{bmatrix} \quad
    \bm{b} =
        \begin{bmatrix}
            b_1     \\
            b_2     \\
            \vdots  \\
            b_n     \\
        \end{bmatrix}
\]

Ma trận \(\bm{\tilde{A}}\) sau được gọi là ma trận mở rộng của hệ
\eqref{eq:general_system_equations}:

\[
    \bm{\tilde{A}} = [\bm{A}, \bm{b}] =
        \begin{bNiceArray}{cccc:c}
            a_{11}  &  a_{12}  &  \dots   &  a_{1n}  &  b_1     \\
            a_{21}  &  a_{22}  &  \dots   &  a_{2n}  &  b_2     \\
            \vdots  &  \vdots  &  \ddots  &  \vdots  &  \vdots  \\
            a_{n1}  &  a_{n2}  &  \dots   &  a_{nn}  &  b_n     \\
        \end{bNiceArray}
\]

Nghiệm của \eqref{eq:general_system_equations} là bộ số \(x_1\), \(x_2\),
\ldots, \(x_n\) thỏa mãn tất cả phương trình của hệ.

Véc tơ nghiệm của \eqref{eq:general_system_equations} là \(\bm{x}\) mà các thành
phần của nó lập thành một nghiệm của \eqref{eq:general_system_equations}.

Hệ phương trình \eqref{eq:general_system_equations} có thể giải được bằng

\begin{itemize}
    \item \emph{phương pháp trực tiếp} (phương pháp khử Gauss, \ldots), hoặc
    \item \emph{phương pháp gián tiếp} hay phương pháp lặp.
\end{itemize}

Chúng ta đã biết phương pháp sử dụng định thức để giải hệ phương trình
\eqref{eq:general_system_equations}, đó là \emph{phương pháp Cramer}. Ở đây ta
xét \hyperref[method:gauss_elimination]{\emph{phương pháp khử Gauss}} để giải hệ
phương trình tuyến tính.

\begin{method}\label{method:gauss_elimination}
    \emph{Phương pháp khử Gauss (Gaussian elimination)}

    Phương pháp khử Gauss là phép giải trực tiếp được thực hiện bằng quá trình
    khử liên tiếp các ẩn để từng bước đưa ma trận mở rộng \(\bm{\tilde{A}}\) về
    dạng tam giác trên (upper triangular matrix, hay dạng bậc thang).

    Phương pháp này gồm hai phần thực hiện lần lượt như sau:

    \begin{enumerate}
        \item \emph{Quá trình thuận (forward elimination)}: Dùng các phép biến
            đổi sơ cấp về hàng để đưa \(\bm{\tilde{A}}\) về dạng tam giác trên:

            \begin{enumerate}
                \item Khử \(x_1\) từ \(E_{\geq 2}\) bằng cách:

                    \[E_i \coloneqq E_i - \frac{a_{i1}}{a_{11}} E_1 \, \forall i \in \interval{2}{n}\]

                    \(a_{11}\) gọi là \emph{phần tử trục xoay (pivot)}, \(E_1\)
                    gọi là phương trình chính (pivot equation).

                \item Khử \(x_i\) từ \(E_{> i}\) bằng cách tương tự như trên,
                    cuối cùng thu được dạng tam giác trên.
            \end{enumerate}

        \item \emph{Quá trình ngược (back substitution)}: Giải \(\bm{x}\) từ
            cuối lên:

            \begin{enumerate}
                \item Giải \(x_n\) từ \(E_n\), giải tiếp được \(x_{n - 1}\) do
                    đã biết \(x_n\).

                \item Tương tự giải được đến \(x_1\), cuối cùng thu được nghiệm
                    \(\bm{x}\).
            \end{enumerate}
    \end{enumerate}
\end{method}

\begin{exmp}
    Hãy giải hệ phương trình:

    \begin{align*}
             8 x_2 + 2 x_3 &= -7 \tag{\(E_1\)} \\
        3x_1 + 5x_2 + 2x_3 &= 8  \tag{\(E_2\)} \\
        6x_1 + 2x_2 + 8x_3 &= 26 \tag{\(E_3\)}
    \end{align*}

    Chúng ta xoay trục từ \(E_1\) , nhưng do \(E_1\) không có ẩn \(x_1\), trong
    khi đó hệ số của \(x_1\) trong phương trình \(E_3\) là lớn nhất. Vì vậy ta
    đổi chỗ \(E_1\) và \(E_3\) cho nhau.

    Tới đây ta có ma trận mở rộng như sau:

    \[
        \bm{\tilde{A}} =
            \begin{bNiceArray}{S[table-format=1] S[table-format=1] S[table-format=1] : S[table-format=-1]}
                6  &  2  &  8  &  26  \\
                3  &  5  &  2  &   8  \\
                   &  8  &  2  &  -7  \\
            \end{bNiceArray}
    \]

    \begin{multicols}{2}
        Khử \(x_1\) được:

        \[
            \begin{bNiceArray}{S[table-format=1] S[table-format=1] S[table-format=-1] : S[table-format=-1]}
                6  &  2  &   8  &  26  \\
                   &  4  &  -2  &  -5  \\
                   &  8  &   2  &  -7  \\
            \end{bNiceArray}
        \]

        Khử \(x_2\) được:

        \[
            \begin{bNiceArray}{S[table-format=1] S[table-format=1] S[table-format=-1] : S[table-format=-1]}
                6  &  2  &   8  &  26  \\
                   &  4  &  -2  &  -5  \\
                   &     &   6  &   3  \\
            \end{bNiceArray}
        \]

    \end{multicols}

    Vậy ta giải được \(x_3 = \num{0.5}\), \(x_2 = -1\), \(x_1 = 4\).
\end{exmp}

\hyperref[method:gauss_elimination]{Phương pháp Gauss} còn có một số biến thể:

\begin{itemize}
    \item Phương pháp Gauss-Jordan: đưa \(\bm{A}\) về dạng đường chéo thay vì
        dạng tam giác trên.

    \item Phương pháp Doolittle, phương pháp Crout, phương pháp Cholesky đều dựa
        trên phân tích \(\bm{LU}\) (\(\bm{LU}\) factorization).
\end{itemize}

\end{document}
