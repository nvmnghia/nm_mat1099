\documentclass[../../Lectures]{subfiles}


\begin{document}

\chapter{Tính gần đúng đạo hàm \& tích phân}

\section{Tính gần đúng đạo hàm}

Vi phân số là tính đạo hàm của hàm số \(f\) dựa vào giá trị của hàm \(f\).

Phép lấy vi phân số nên tránh khi có thể. Trong khi tích phân là một quá trình
làm mịn và không nhạy cảm với các sai số của giá trị hàm số thì đạo hàm có xu
hướng khuyếch đại sai số và nói chung ít chính xác hơn giá trị của \(f\).

Sự khó khăn trong phép lấy đạo hàm gắn với định nghĩa của nó. Đạo hàm là giới
hạn của thương số số gia hàm số với số gia đối số, trong đó thương số thường có
sự sai khác lớn khi lấy một số lớn chia cho một số nhỏ. Điều này dẫn tới sự
không ổn định. Trong khi hiểu rõ điều này, chúng ta vẫn phải phát triển các công
thức tính đạo hàm, vì ta phải sử dụng chúng trong việc giải phương trình vi
phân.



\end{document}
