\documentclass{article}    % TODO: Check other style, e.g. book style.

\usepackage{hyperref}          % Clickable link
\usepackage{indentfirst}

% References. Forced, cuz biber is in the toolchain.
\usepackage{biblatex}
\usepackage[final]{microtype}

\usepackage{mathtools}
\usepackage{icomma}    % TODO: flexible comma-dot.
\usepackage{siunitx}


% Fucking abs.
\DeclarePairedDelimiter\abs{\lvert}{\rvert}%

% Blank page https://tex.stackexchange.com/a/374542/206774
\def\blankpage{%
      \clearpage%
      \thispagestyle{empty}%
      \addtocounter{page}{-1}%
      \null%
      \clearpage}

\newtheorem{definition}{Định nghĩa}[section]
\newtheorem{exmp}{Ví dụ}

\author{Lê Phê Đô \\ \href{mailto:dolp@vnu.edu.vn}{dolp@vnu.edu.vn}}

\date{07/09/2020}

\title{Phương pháp tính MAT1099}

\begin{document}

\maketitle

\blankpage

\section{Giới thiệu về phép tính gần đúng}

\subsection{Một số ví dụ về tính toán khoa học và phương pháp tính}

\subsubsection{Lý do nghiên cứu Phương pháp tính}

\begin{itemize}
    \item Làm viêc với các số gần đúng.
    \item Giải gần đúng các phương trình và hê phương trình.
    \item Xấp xỉ hàm số: Phương pháp nội suy, phương pháp xấp xỉ hàm số, chuỗi
    Taylor hoặc chuỗi Marlorin.
    \item Số học IEEE.
\end{itemize}

\subsubsection{Các nhiệm vụ}

\begin{itemize}
    \item Tìm hiểu và ứng dụng các thuật toán.
    \item Thể hiện các thuật toán bằng các chương trình.
    \item Tìm các bài toán thực tiễn.
\end{itemize}

Trong thực tế chúng ta thường phải xử lý, tính toán với các đại lượng gần đúng
như các số đo vật lý, các dữ liệu ban đầu, các số làm tròn…với sai số nào đó,
tức là các số gần đúng. Việc ước lượng sai số hợp lý cho phép ta đánh giá được
chất lượng của quá trình tính toán, quyết định số chữ số giữ lại trong các phép
tính trung gian và trong kết quả. Vì vậy, trước tiên ta cần nghiên cứu về các
phép tính gần đúng và sai số.

\section{Số gần đúng, sai số tuyệt đối và tương đối}

\subsection{Sai số tuyệt đối và sai số tương đối}

\subsubsection{Sai số tuyệt đối}

Nếu số gần đúng \(a\) có giá trị đúng là \(a_0\) thì ta nói \(a\) xấp xỉ \(a_0\)
hay \(a\) là số gần đúng của \(a_0\). Khi đó sai số của \(a\) là:

\begin{equation} \label{eq:1}
    E_a = a - a_0
\end{equation}

Nhưng giá trị này nói chúng ta không biết được mà chỉ ước lượng được cận trên
của trị tuyệt đối của nó.

\begin{definition}
    Giá trị ước lượng \(\Delta a\) sao cho
    \begin{equation} \label{eq:2}
        \abs{a - a_0} \leq \Delta a
    \end{equation}
    được gọi là \emph{sai số tuyệt đối} của số gần đúng \(a\).
\end{definition}

% TODO: Mục 2.1?
Sai số tuyệt đối nhỏ nhất có thể biết được gọi là sai số tuyệt đối giới hạn của
\(a\). Thông thường ước lượng sai số tuyệt đối giới hạn là khó và nhiều khi
không cần thiết nên người ta chỉ cần ước lượng sai số tuyệt đối đủ nhỏ và dùng
từ 1 đến 3 chữ số có nghĩa (là số chữ số bắt đầu từ chữ số khác không đầu tiên
từ trái sang phải - xem mục 2.1) để biểu diễn sai số tuyệt đối của số gần đúng.

Thay cho \ref{eq:2} người ta còn dùng cách biểu diễn sau để chỉ sai số tuyệt đối
của \(a\):

\begin{equation} \label{eq:3}
    a_0 = a \pm \Delta a
\end{equation}

Trong thực tế thì sai số \(E_a\) không thể biết được nên khi không có sự hiểu
lầm người ta còn dùng từ \emph{sai số} để chỉ sai số tuyệt đối \(E_a\).

\begin{exmp}
    Căn phòng có chiều dài \(d = \SI{5,45}{\m}\) và chiều rộng \(r =
    \SI{3,94}{\m}\) với sai số \(\SI{1}{\cm}\).\\

    Khi đó ta hiểu là:
    \begin{gather*}
        \Delta d = \SI{0,01}{\m} \textrm{ hay } d = 5,45 \pm \SI{0,01}{\m}\\
        \Delta r = \SI{0,01}{\m} \textrm{ hay } r = 3,97 \pm \SI{0.01}{m}
    \end{gather*}

    Như vậy diện tích của phòng được ước lượng bởi:
    \begin{equation*}
        S = d \cdot r = 5,45 \cdot 3,94 = \SI{21,473}{\meter\squared}
    \end{equation*}

    với cận trên và cận dưới của \(S\) là:
    \begin{equation*}
        (5,45 - 0,01)(3,94 - 0,01) = 21,3792 \leq S \leq (5,45 + 0,01)(3,94 + 0,01) = 21,567
    \end{equation*}

    Vậy ta có ước lượng sai số tuyệt đối của S là:
    \begin{equation*}
        \abs{S - S_0} \leq \SI{0,094}{\meter\squared}
    \end{equation*}
\end{exmp}

\subsubsection{Sai số tương đối}

Hai số gần đúng có cùng sai số tuyệt đối sẽ có \emph{"mức độ chính xác"} khác
nhau nếu độ lớn của chúng khác nhau, số bé hơn sẽ có độ chính xác kém hơn. Để
biểu diễn độ chính xác này người ta dùng sai số sai số tương đối.

\begin{definition}
    \emph{Sai số tương đối} của số gần đúng \(a\) là tỷ số giữa sai số tuyệt đối
    và giá trị tuyệt đối của nó, được ký hiệu là \(\delta a\).
    \begin{equation} \label{eq:4}
        \delta a = \frac{\delta a}{\abs{a}}
    \end{equation}
\end{definition}

Thường sai số tương đối được biểu diễn dưới dạng phần trăm với 2 hoặc 3 chữ số.

Từ \ref{eq:4} ta thấy nếu biết \(\delta a\) thì:
\begin{equation} \label{eq:5}
    \Delta a = \abs{a}\delta a
\end{equation}
nên ta chỉ cần biết một trong hai loại sai số của nó là được.

\begin{exmp}
    Nếu \(a = 57\) và \(\Delta a = 0,5\) thì \(\delta a = 0,0087719\) hoặc
    \(0,88\%\) (gọn hơn là \(0,9\%\)).
\end{exmp}

\subsubsection{Các loại sai số khác}

Để hình dung các loại sai số khác ta xét ví dụ sau:

\begin{exmp}
    Một vật thể rơi từ độ cao \(H_0\) với vận tốc ban đầu \(v_0\) (được đo nhờ
    thiết bị nào đó). Tính độ cao \(H(t)\) của vật thể sau thời gian \(t\). Bài
    toán có thể giải như sau:

    Nếu gọi ngoại lực tác động vào vật thể là \(F(t)\) (gồm lực hút trọng trường
    và lực cản), khối lượng vật thể là \(m\) thì \(H(t)\) là nghiệm của phương
    trình vi phân cấp hai
    \begin{equation} \label{eq:6}
        H''(x) = \frac{-F(t)}{m}
    \end{equation}
    với điều kiện ban đầu \(H(0) = H_0\) và \(H'(0) = -v_0\).

    Ta chọn một phương pháp gần đúng để giải phương trình này, chẳng hạn nếu giả
    thiết \(\dfrac{F(t)}{m}\) không đổi thì
    \[H(t) = H_0 - g\frac{t^2}{2} -v_0 t\]
\end{exmp}

Qua ví dụ trên ta thấy sai số của kết quả nhận được chịu ảnh hưởng của:
\begin{itemize}
    \item các số đo\(H_0\), \(v_0\)
    \item cách lập luận để xác định \(F(t)\)
    \item phương pháp giải phương trình \ref{eq:6}
    \item và các yếu tố ngẫu nhiên khác
\end{itemize}

Theo các yếu tố ảnh hưởng tới kết quả tính toán ta phân ra các loại sai số sau:
\begin{itemize}
    \item \emph{Sai số dữ liệu} (còn gọi là sai số của số liệu ban đầu). Trong
        thí dụ trên là sai số khi đo \(H_0\) và \(v_0\).
    \item \emph{Sai số giả thiết}. Sai số này gặp phải khi ta đơn giản hoá bài
        toán thực tiễn để thiết lập mô hình toán học có thể giải được. Trong thí
        dụ trên có thể giả thiết ngoại lực chỉ là trọng lực.
    \item \emph{Sai số phương pháp}. Là sai số của phương pháp giải gần đúng bài
        toán theo mô hình được lập. Trong thí dụ trên là phương pháp giải phương
        trình vi phân \ref{eq:6}.
    \item \emph{Sai số tính toán}. Là sai số tích luỹ trong quá trình tính toán
        theo phương pháp được chọn.
    \item \emph{Sai số làm tròn}. Khi tính toán ta thường phải làm tròn các số
        nên ảnh hưởng tới kết quả nhiều khi rất đáng kể.
    \item \emph{Sai số ngẫu nhiên}. Là sai số chịu các quy luật chi phối ngẫu
        nhiên không tránh được.
\end{itemize}

Về sau ta  quan tâm tới sai số tính toán và sai số phương pháp.

\end{document}
